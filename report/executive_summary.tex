\chapter*{Executive Summary\ifdraft{ (Joe K./Susan) (100\%)}{}}
\label{chapter:executive_summary}

\section*{THE FUTURE OF VOTING: END-TO-END VERIFIABLE INTERNET VOTING
  SPECIFICATION AND FEASIBILITY}

\noindent A PROJECT OF U.S. VOTE FOUNDATION \\
\noindent WRITTEN AND PRODUCED BY GALOIS, INC.\\

\newpage

\noindent \textbf{U.S. VOTE FOUNDATION}\\
\noindent \textbf{SUSAN DZIEDUSZYCKA-SUINAT, PRESIDENT AND CHIEF
  EXECUTIVE OFFICER}\\
\noindent \textbf{JUDY MURRAY, PH.D.}\\

\vspace{1cm}

\noindent \textbf{GALOIS, INC.}\\
\noindent \textbf{JOSEPH R. KINIRY, PH.D. (author and editor)}\\
\noindent \textbf{DANIEL M. ZIMMERMAN, PH.D. (author and editor)}\\
\noindent \textbf{DANIEL WAGNER, PH.D. (chapter author)}\\
\noindent \textbf{ADAM FOLTZER (chapter author)}\\
\noindent \textbf{PHILIP ROBINSON (chapter author)}\\

\section*{ACKNOWLEDGMENTS}

This project and report would not have been possible without the
commitment and tireless hard work of the team at Galois, Inc. Our
special acknowledgment and appreciation goes most especially to
Joseph Kiniry, who brought his decades of knowledge, skill, experience
and leadership to the project, broadened its scope and led the
technical team and writing; and with him, the Galois team members
Daniel Zimmerman, Daniel Wagner, Philip Robinson, Adam Foltzer,
Shpatar Morina and Leah Daniels. We are indebted to Rob Wiltbank for
the Galois engineering contribution, and the long leash he gave to
this project. 

We would also like to thank the research and technical members of the
E2E-VIV Project Team for their contributions to this project from its 
conception to its completion, with special thanks to Josh Benaloh,
Candice Hoke, Keith Instone, David Jefferson, Doug Jones, Aggelos
Kiayias, Judith Murray, Ron Rivest, Barbara Simons, and Poorvi Vora. 

Equally vital and integral to this report were the reflections,
insights and advice from election officials who joined our team, most
especially Lori Augino, Judd Choate, Dana Debeauvoir, Mark Earley,
Stuart Holmes, Dean Logan, Tammy Patrick, Roman Montoya, and Lois
Neuman.

We also thank Sean Beggs, Randall Trzeciak, and Andrew Wasser,
Carnegie Mellon University, Heinz College Master of Information
Systems Management for their support through the CMU ISM Capstone
Program. 

We are grateful for the generous financial support of The Democracy
Fund, as well as their support of collaborative efforts in the realm
of civic tech development. 

For additional information on U.S. Vote Foundation, please visit
www.usvotefoundation.org.   

For additional information on the Overseas Vote Initiative, please
visit www.overseasvote.org. 

For additional information on Galois, Inc., please visit
www.galois.com.

For additional information on the Democracy Fund, please visit
www.democracyfund.org.  

\newpage

\section*{INTRODUCTION}

Societies have conducted elections for thousands of years, but
technologies used to cast and tally votes have varied and evolved
tremendously over that time. In 2015 many of our essential services
have moved online, and some people want elections to follow this
trend. Overseas voters are particularly interested in an online
approach, as their voting processes can require additional effort and
suffer from long delays.

Internet voting systems currently exist, but independent auditing has
shown that these systems do not have the level of security and
transparency needed for mainstream elections. Security experts advise
that end-to-end verifiability---lacking in current systems---is one of
the critical features needed to guarantee the integrity, openness, and
transparency of election systems.

In this report, we examine the future of voting and the possibility of
conducting secure elections online. Specifically, we explore whether
End-to-End Verifiable Internet Voting (E2E-VIV) systems are a viable
and responsible alternative to traditional election systems.

This project combines the experience and knowledge of a diverse group
of experts committed to election integrity. The technical team,
comprised of academic and scientific specialists, has long term,
proven experience in end-to-end verifiable technology, cryptography,
high-assurance systems development, usability, and testing.

\section*{INTERNET VOTING TODAY}

Internet voting was first proposed over thirty years ago. Since then,
many governments and businesses have created Internet voting
technologies that have been used to collect millions of votes in
public elections.

However, computer scientists, cryptographers, and cybersecurity
experts warn that no current Internet voting system is sufficiently
secure and reliable for use in public elections.

Part of the problem is that existing systems do not allow third
parties to observe the election system and independently verify that
the results are correct.  In fact, most companies explicitly forbid
such oversight.

\noindent \textbf{SECRET}

No existing commercial Internet voting system is open to public
review. Independent parties cannot verify that these systems function
and count correctly, nor can they audit and verify election results.

\textbf{INSECURE}

Elections for public office are a matter of national
security. Researchers have shown that every publicly audited,
commercial Internet voting system to date is fundamentally insecure.

\noindent \textbf{NO GUARANTEES}

No existing system guarantees voter privacy or the correct election
outcomes.  Election vendors are rarely held liable for security
failures or election disasters.

\newpage

\section*{END-TO-END VERIFIABILITY}

An end-to-end verifiable voting system allows voters to:
\begin{itemize}
\item check that the system recorded their votes correctly,
\item check that the system included their votes in the final tally,
  and
\item count the recorded votes and double-check the announced outcome
  of the election.
\end{itemize}

An internet voting system that is end-to-end verifiable is an E2E-VIV
system.

The concept of E2E-VIV is decades old. However, most of the required
computer science and engineering techniques were impractical or
impossible before recent advances. Designing and building an E2E-VIV
system in the face of enormous security threats remains a significant
challenge. It is currently unclear whether it is possible to construct
an E2E-VIV system that fulfills the set of requirements contained in
this report.

\section*{INTERNET VOTING REQUIREMENTS}

Internet voting must be end-to-end verifiable. It must also be secure,
usable, and transparent.

\noindent \textbf{SECURE}

Security is a critical requirement for Internet voting, and also one
of the most challenging. An Internet voting system must guarantee the
integrity of election data and keep voters' personal information
safe. The system must resist large-scale coordinated attacks, both on
its own infrastructure and on individual voters' computers. It must
also guarantee vote privacy and allow only eligible voters to vote.

\noindent \textbf{USABLE}

Nearly all E2E-VIV protocols designed to date focus on security at the
expense of usability. Election officials and voters will not adopt a
secure but unusable system. Cryptographers have started to recognize
usability as a primary requirement when designing new protocols, and
usability is a serious challenge that any future work in this area
must address. Any public Internet voting system must be usable and
accessible to voters with disabilities.

\noindent \textbf{TRANSPARENT}

It is not enough for election results to be correct. To be worthy of
public trust, an election process must give voters and observers
compelling evidence that allows them to check for themselves that the
election result is correct and the election was conducted
properly. Open public review of the entire election system and its
operation, including all documentation, source code, and system logs,
is a critical part of that evidence.

\vspace{1cm}

End-to-end verifiability, security, usability, and transparency are
only four of many important requirements. This report contains the
most complete set of requirements to date that must be satisfied by
any Internet voting system for public elections.

\newpage

\section*{RECOMMENDATIONS}

The five key recommendations of this report are:
\begin{itemize}
\item Any public elections conducted over the Internet must be
  end-to-end verifiable.
\item No Internet voting system of any kind should be used for public
  elections before end-to-end verifiable in-person voting systems have
  been widely deployed and experience has been gained from their use.
\item E2E-VIV systems must be designed, constructed, verified,
  certified, operated, and supported according to the most rigorous
  engineering requirements of mission- and safety-critical systems.
\item E2E-VIV systems must be usable and accessible.
\item Many challenges remain in building a usable, reliable, and
  secure E2E-VIV system. They must be overcome before using Internet
  voting in public elections. Research and development efforts toward
  overcoming those challenges should continue.
\end{itemize}

Building a usable, reliable, and secure E2E-VIV system may be
impossible. Solving the remaining challenges, however, would have
enormous impact on the world.

\section*{OUTCOMES}

The full report contains the following:

\noindent \textbf{REQUIREMENTS}

We identify a complete set of requirements for an E2E-VIV system.

\noindent \textbf{ARCHITECTURES}

We review a variety of ways to build, deploy, and run an E2E-VIV
system and the associated engineering issues.

\noindent \textbf{ENGINEERING AND TECHNOLOGY}

We present a set of rigorous engineering methodologies, technologies,
and tools that are fundamental to building a correct and secure
E2E-VIV system.

\noindent \textbf{SECURITY}

We lay the foundation for developing a cryptographic system that
reflects the ideal functionality of an end-to-end verifiable system,
and discuss the technologies that should be used to implement that
system.

\noindent \textbf{USABILITY}

We present the results of an initial usability study showing that
significant effort is needed to develop usable E2E-VIV systems.

\vspace{1cm}

\section*{FULL REPORT}

Download the full report at \url{www.usvotefoundation.org/E2E-VIV}
