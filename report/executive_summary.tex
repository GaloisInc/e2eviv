\chapter{Executive Summary\ifdraft{ (Joe K./Susan) (100\%)}{}}
\label{chapter:executive_summary}

Internet voting was first proposed over thirty years ago. In the
intervening years, a handful of companies and a twenty-odd governments
have created Internet voting technology that has been used in
thousands of elections to collect millions of votes all around the
world except, until very recently, in the U.S.A.

Every month, while one government experiments with a new computing
technology into their elections, another decides that their last
experiment was a failure and they shut down yet another very expensive
failed IT project. All the while, a dedicated group of scientists and
activists---hackers, cryptographers, cyber-security geeks, usability
experts---have fought tooth-and-nail against the use of insecure
voting systems around the world.

Their fight, and those governments' excitement, is particularly
focused on Internet voting. After all, voter turnout is at a seventy
year low, many millennials are politically disengaged, and voting
equipment purchased with HAVA funds is falling apart and there is
little state and federal funding to replace it. This is the a perfect
storm for our democracy, and Internet voting seems to be the silver
bullet.

After all, the Internet permeates our lives. Smart phones. Social
media. Online banking. Streaming music and
movies. Google. Bitcoin. Online gaming. Why not voting?

Imagine a world where inexpensive elections have high participation
rates by a well-informed, engaged public. Imagine elections where the
disabled and abled have equal vote and equal opportunity. Imagine
elections where corrupt electoral authorities or governments have no
ability to manipulate the outcome. Imagine elections that truly
capture the voice of the people and increase their confidence and
trust in their government. Many imagine Internet voting as the
solution that takes us to this democratic utopia.

Then imagine a world where millions have voted, we are about to
tabulate a hundred million digital votes, and suddenly the electoral
authority discovers intruders in their servers.

Or worse yet, imagine that exit polls show that the outcome is razor
thin between candidates “Alice Democrat” and “Bob Republican”, those
responsible for the election push the button to ask their election
server who has won, and the machine says “Charlie Communist”' and
hundreds of politicians, electoral officials, and reporters look at
each other with wide-eyed disbelief.

Today's Internet voting technology guarantees these kinds of horrific
failures. Digital activists self-identifying as Anonymous with a
grudge, Super PAC-contracted blackhat hackers, and enemy nation states
will use their nearly unlimited resources to ensure such.

The only possible solution to this digital democratic dilemma is
End-to-End Verifiable Internet Voting, or E2E-VIV for short. How to
create an E2E-VIV for the American public is the focus of the report
you hold in your hands.

E2E-VIV systems hold the promise of secure, trustworthy
elections. They do so by giving voters, elections officials, and the
general public the ability to observe an election and determine that
it is well-run, not manipulate, and has a trustworthy outcome. 

E2E-VIV system provide these verifiability guarantees by permitting
(1) voters to check that their vote was recorded correctly, (2) voters
to check that their vote was included in the final tally, (3) election
officials to determine if there was any election manipulation or
errors and mitigate such, and (4) the general public to count the vote
and double-check the announced outcome of the election. 

Meanwhile, while providing all of these guarantees, voter privacy is
preserved and vote selling and voter coercion are avoided and attempts
at such can be detected. 

In other words, we maintain the security and privacy of today's
paper-based elections, and yet increase the trustworthiness of our
elections.

While the promise of E2E-VIV is decades old, our ability to design and
build such a system, especially in the face of enormous security
threats, is only recently possible with recent advances in computer
science and cyber-security.  High-assurance software engineering,
software formal verification, formal computer-assisted verification of
advanced cryptographic protocols and algorithms—none of these were
practical, and some where not even possible, as recent as ten years
ago.

Internet voting must be not only end-to-end verifiable, but it must
also be usable, accessible, and open source.  As such, this report
contains over one hundred requirements that must be satisfied by any
E2E-VIV system.

Security and usability are always at odds. Nearly all E2E-VIV
protocols designed to date focus on security at the expense of
usability.  Only recently have cryptographers started to consider
usability as a primary requirement when designing new protocols.  It
is a recognized serious challenge that an future work in this area
must address.

Software is only trusted by security professionals if its description,
including all of its documentation and source code, is publicly
peer-reviewed.  This does not, of course, mean that anyone can change
the code.  It does mean that anyone can read and comment on it,
improving its clarity, elegance, and correctness for the benefit of
all.

This report contains the plan for executing on this difficult but
worthwhile mission.  The cryptographic, architectural, and engineering
foundations on which this edifice must be built are spelled out in
this report.  How must cryptographic protocols be created, evaluated,
and certified?  What kinds of architectures facilitate deploying the
system for different sorts of electoral authorities, including
considerations for their certification constraints, finances, and
existing electoral processes.  What are the best practices in the
design and development of mission-critical systems, as public
elections are obviously nationally critical?

The four key recommendations made in this report are:
\begin{enumerate}
\item Public elections should not be conducted over the Internet using
  systems that are not end-to-end verifiable.

\item  End-to-end verifiable Internet voting systems should not be used
  before end-to-end verifiable poll-site voting systems have been
  widely-deployed and experience has been gained from their use.

\item  E2E-VIV systems must be designed, constructed, verified, certified,
  operated, and supported as high-assurance systems according to the
  most rigorous engineering requirements of mission- and
  safety-critical systems.

\item  E2E-VIV systems must be usable and accessible to the typical set of
  abled and disabled voters.
\end{enumerate}

Living up to these recommendations, while fulfilling the requirements
contained in this report, is an incredibly challenging, but is also an
enormous opportunity to do good for the world.  America is known for
tackling the most difficult problems, as well as for striving to help
other democratic countries.  Moreover, America has most of the best
minds and companies is the world that are capable of fulfilling this
mission.

The future of End-to-End Verifiable Internet Voting is clear: we must
move to a second phase of this project and tackle the recommendations
head-on.  We must create a usable, secure, correct prototype E2E-VIV
system that fulfills the requirements contained in this report so that
electoral authorities countrywide can begin to experiment with
Internet voting, first for their UOCAVA and disabled voters, and
perhaps later, only after those experiments are an unmitigated
success, for the general public.

Now is the time.  We must take this opportunity.  The alternative is a
future where Internet voting is a reality, but we have no confidence
in our elections or democracy.


