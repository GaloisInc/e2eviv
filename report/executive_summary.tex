\chapter{Executive Summary\ifdraft{ (Joe K./Susan) (0\%)}{}}
\label{chapter:executive_summary}

\begin{itemize}
\item framing of project \& methodology
\item who is involved
\item who funded it
\item goals and deliverables
\item feasibility
\item recommendation
\item ``why now'' coupled to history
\item why do existing systems suck
\item ``how'' what is E2E-VIV (illustration)
\item five key mandatory properties: end-to-secure, verifiability,
  100\% accessible, high-assurance, transparent
\item crypto framing
\item architectural framing
\item RSE framing
\item conclusion and future phases
\end{itemize}

% [Susan here - this is my outline - I am visualizing the report in a
% printed or online format]

% Thinking out loud about how to make this piece simple, powerful and
% satisfying...  

% Define the what and how: let’s make it completely unambiguous. With
% an ILLUSTRATION.

% Define the current challenge of IV - that is the basis and reason
% for doing this project - (despite all the hacking, Snowden, Target,
% IRS, Sony) Americans love a challenge - we want to find a way. This
% search will never stop. It is the responsibility of the scientific
% and election communities to team up, research and pursue a real
% answer.

% What is the challenge in the context of Elections in the USA (PCEA,
% modernization, OVR, coupled with lack of certification, diminished
% funding, pressure on LEOs to act, wildly fragmented implementation
% across 10K+ jurisdictions)

% What’s the problem - I mean hey - we can bank online! Come on,
% people!

% So, what is the BFD of this report? I mean it! What has been
% “discovered”. Give it to me straight!

% How will it affect our future of voting? Will we vote online and
% when can we expect to? Come on, people - that is why I am reading
% your Exec Summary!

% What’s in the report, how is it organized and what part of it am I
% interested to read?

% How is life on earth different now that we have this new report -
% this new frame of reference. What might happen now that this
% knowledge is released into the universe?

% E2E-VIV Diagram - perhaps show it as a line, what is the beginning “End” and the final “End”.
%    Voter Intent	 Distribute Ballots	       Record Votes as Cast	  Verify Vote Counted as Cast     Announce Results
% E------||--------||---------||-----------||----------||---------||-----------------||-----------||---------||----E
% 	Set Up Election	    Cast Votes as Intended	Count Votes as Recorded	 Allow Public to Verify

% Starting over here - trying out my Super-Susan-Simple-Outline - a no
% obligations trial!
% 1. End-to-End Verifiable Internet Voting (E2E-VIV): let’s break this down...
% what is End-to-End (E2E)?
% what is Verifiability?
% what is E2E-V when applied to an election?
% what about putting the E2E-V election process onto the Internet to
% create E2E-VIV?
% what challenges are encountered - why, in simple terms is this hard?

% Q. What is End-to-End?
% A. The term End-to-End (E2E) in the context of elections refers to the
% entire process of voting starting with the voter’s intent through to
% the announcement of the election results.
% Q. What is Verifiability? 
% Verifiability (V) is the ability to prove something, to show
% evidence that allows you to confirm or substantiate a fact.
% Q. What is End-to-End Verifiability when applied to an election? 
% End-to-End Verifiability (E2E-V) in an election means you can check
% that your vote was cast as you intended it, recorded as such and
% counted correctly. It also means that anyone can check the outcome
% of the election. In essence, E2E-V is considered a “property” of an
% election.
% Q. Is it beneficial to add E2E-V properties into Internet voting to
% create E2E-VIV?
% Adding E2E-V properties to an election conducted over the Internet
% would bring in security assurances that are lacking in every
% currently available Internet voting system. Elections are, however,
% multifaceted and their implementation involves a diverse set of
% challenges, security being top of the priority list, but closely
% tied with several others.
% Q. What, specifically, are the challenges in developing an E2E-VIV
% System? 
% At the same time that certain security challenges are positively
% addressed with E2E-VIV, new challenges in other aspects of
% elections, specifically, usability, privacy, authentication, 100%
% accessibility, transparency and auditability [add total list] are
% created. 
% Q. Is it possible to overcome these challenges and define an E2E-VIV
% System that we could build, make available to the public and openly
% test? 
% This question is at the heart of the End-to-End Verifiable Internet
% Voting: Specification and Feasibility Assessment Study (E2E-VIV
% Project) that was defined and undertaken by U.S. Vote Foundation
% between September 2013 and July 2015. 
% Q. End-to-End Verifiable Internet Voting: Specification and
% Feasibility Assessment Study (E2E-VIV Project)
% With support from The Democracy Fund, U.S. Vote Foundation (US Vote) was enabled to manage the project and recruit a diverse team of elections specialists including leading election officials, scientists and academics who had studied this topic for decades, usability, testing and auditing experts. In mid-2014, US Vote engaged Galois, Inc. to lead the technical project development. The Galois engineering team brought deep knowledge, skill and experience in the arena of high-assurance cryptographic science - the element at the core of the E2E challenge. In addition, Galois contributed to the project resources with development of key demonstration technology pieces for E2E-VIV.
% The E2E-VIV Project team took on the challenge to examine these difficult election systems  and engineering challenges that we face with Internet voting and put forward a specification and feasibility assessment for how to move forward to begin developing and openly testing such systems. 

% The natural movement to modernize elections

% Elections have been conducted for millennia and are the cornerstone
% of democracy, but the technologies used to cast and tally votes have
% varied and evolved tremendously over that time.  Much of our
% discourse now takes place online, and many have called for elections
% to follow this trend and asked why they haven’t done so already.
 
% The E2E-VIV Research Report

% In this E2E-VIV research report, we examine the future of voting
% and how it might be executed securely online. Specifically, we
% explore whether an end-to-end verifiable Internet voting (E2E-VIV)
% election system can be built that would offer a viable and
% responsible alternative to current systems aimed at support for the
% overseas, military and disabled voter communities. The chief
% deliverable is a System Specification and Documentation for an
% E2E-VIV Election System.
 
% I think this might be covered above. This project combines the
% abilities, knowledge, experience, and expertise of a diverse group
% of technologists, computer scientists, usability and auditing
% specialists, and election officials committed to election
% integrity. The technical team, comprised of academic and scientific
% specialists, has long term, proven experience in E2E-V technology,
% cryptography, usability, and testing.
 
% % Findings and Recommendations Outcomes (frankly I would split these
% into two separate sections. Let’s keep each section short.

% The E2E-VIV Project team has determined that it is technically
% feasible to design, develop and support an open source E2E-VIV
% system. This report puts forth the following:
% ·      A complete set of requirements for an E2E-VIV system.
% Systems which fulfill these requirements and provide evidence of
% such in an open transparent manner, which allows verification and
% certification, may be considered as ian E2E- VIV system
% ·      A collection of alternative architectures for E2E-VIV systems.
% The precise solution architecture will depend upon the threats that
% governments care to address
% ·      A set of rigorous engineering methodologies, technologies,
% and tools. These technologies are fundamental to building an E2E-VIV
% system that is correct and secure.
% ·      The security foundations of an E2E-VIV system
% Lorem ipsum dolor site et.
% ·      The need for  a comprehensive usability study.
% The usability challenges that emerge with the introduction of new
% voting processes that enable and guarantee the integrity of E2E-VIV
% systems, remain formidable, yet we believe,
% surmountable. Significantly higher-level UX design effort is needed
% on actual prototype systems coupled with a comprehensive usability
% study. An iterative process between the designers and usability
% engineers will be required to adequately address the usability
% challenges.  
% ·      Phase II Research and Development.
% The E2E-VIV Project team recommends continuation of this effort
% through a Phase II R&D effort, which would include the development
% of an E2E-VIV prototype, testing, user interface, usability and
% accessibility refinements and testing, --( add some tech soup here -
% the meaty stuff like crypto and rigorous engineering ….)

% What is End-to-End Verifiable Internet Voting (E2E-VIV)

% An end-to-end verifiable Internet voting system guarantees the
% security of the ballot casting process, provides evidence of the
% correctness of both the system and the election, and outputs results
% that are verifiable by independent third parties.

% End-to-End

% Verifiable
% Internet Voting

% A system designed to be secure from threats (for example, hackers,
% insider attacks and administrative errors) from the beginning (the
% voter’s intent) to the end of the voting process (the election
% outcome).

% A system that allows independent third parties to check that [the
% election is being properly conducted?] and that the outcome of the
% election is correct. Current elections do this by …

% A system that transmits voter choice over the Internet, by any
% means, including any aspect of the voting process that is online,
% for example , online voting, voting by phone over VoIP, equipment
% that uploads precinct results to a central tally system, etc.

% How it Works: tell me the tough stuff in simple language

% Characteristics / Requirements  of E2E VIV Systems
% Do we want to say - ?
% End-to-End Verifiability is considered a “property” of an election
% system which comprises and an important set of capabilities.  An
% E2E-VIV system:
% ·      Permits voters to check that their votes are recorded
% properly, without violating their privacy
% ·      Permits voters to check that their votes are included in the
% final tally 

% ·      Permits anyone to check that all of the cast ballots have
% been tallied correctly


% To be able to do this, it must have a set of
% characteristics/functionalities? (I am holding “property” for the
% overarching statement at the top - you may not agree…). A complete
% set of required properties can be found in chapter 5. the most
% important properties it must have are:

% end-to-end secure, verifiable,
% 100% accessible, high-assurance, transparent
 
% Why now?

% Our election systems are in need of modernization and it is a
% natural tendency for election administrators to consider all
% possible technical solutions including online voting. It is the
% responsibility of ….fill in - to offer guidance and  ...

% Many IV implementations exist overseas, however we need our own
% frame of reference in the US and for the US - one which takes into
% account the seriousness of US national elections and structure of
% our elections administration framework and structure

% More and more of what we do is going online

% The internet facilitates a lot of things

% Election officials want the supposed benefits of an online system
% for efficacy

% There are no existing solutions exist that are deemed suitable by
% experts in terms of security and robustness

% In lieu of a good solution, people are using bad ones. Email, closed
% source vendors, etc.

% Recent developments in the field: STAR-Vote? Expert collaboration?
% time is now

% • feasibility
% Designing an E2E VIV System
%  • crypto framing
% 
• architectural framing

% • RSE framing
% • other?
% 

% Recommendations

% Callout: Why open source?

% Lorem ipsum dolor sit amet, consectetur adipiscing elit. Nullam
% efficitur aliquet turpis, vitae eleifend lacus tempor at. Aliquam
% tempor, ex in fermentum tempus, risus ipsum finibus ipsum, ac
% gravida odio augue ut risus. In sollicitudin luctus leo ut
% interdum. Aenean tincidunt gravida pharetra. Donec eu quam eget
% mauris pretium mattis eget et est.
 
% Callout: Why is Internet voting so challenging?

% If we can bank online, why can’t we vote online?
% In banking, the bank...

% Knows who you are
% Knows every transaction you make
% Has a ledger on all transactions in the bank
% The bank and the federal government insure loss

% In elections...

% Voter identity and vote are private
% No ledger of all transactions
% No insurance if something goes wrong
 
%  Lorem ipsum dolor sit amet, consectetur adipiscing elit. Nullam
%  efficitur aliquet turpis, vitae eleifend lacus tempor at. Aliquam
%  tempor, ex in fermentum tempus, risus ipsum finibus ipsum, ac
%  gravida odio augue ut risus.


 

