\chapter{Architecture\ifdraft{ (Joe K./Dan) (30\%)}{}}
\label{chapter:architecture}

The \emph{architecture} of a computing system, akin to the
architecture of a purely physical artifact like a bridge or a
building, is its high-level structure. Just as when designing a
bridge, many choices must be made when designing a computing system;
these choices are driven by the system's requirements, both technical
and non-functional, as well as by external factors such as the
availability of computing hardware or network bandwidth. In this
chapter, we describe the architectural issues associated with E2EVIV
systems, present a model encompassing the various possible
architectural choices for such systems, and explore some architectural
variants based on that model.

It is important to note that we are \emph{not} making a concrete
recommendation for a specific E2EVIV system architecture. The
cryptographic foundations of E2EVIV protocols have been developed to a
point where we have fairly high, though not yet complete, confidence
in their ability to provide the required security and auditability
properties. However, there are many open engineering issues associated
with actually building, running, and maintaining an E2EVIV system that
fulfills its requirements in the face of both routine/expected
failures and a wide range of security threats. Because of these open
engineering issues, architectural experimentation---preferably,
empirical testing of various possible architectures to determine the
most appropriate one(s) to deploy in real-world election
scenarios---is vital to actually implementing a successful E2EVIV
system.

\section{Non-Functional Requirements Forcing Architectural Factors}

Several non-functional requirements of E2EVIV systems force the
inclusion or consideration of specific architectural factors, many of
which are not applicable to other Internet-based computing systems. We
consider each of these in turn.

\subsection{Abstraction}

The software in an E2EVIV system must be high assurance, and must also
undergo a certification process before it can be used in actual
elections. Thus, the software must be designed and implemented with a
level of abstraction that allows convincing evidence of its
correctness to be produced. This requires development techniques that
emphasize formal specification and verification, dividing the software
system into relatively small components with well-defined,
well-constrained interfaces. Each component can then be verified
individually with respect to its specification; moreover, component
behaviors with respect to external communication can be formally
characterized in a way that allows for verification of composed
subsystems.

\subsection{Deployment}

There is a wide spectrum of possible deployment scenarios for an
E2EVIV system, each of which leads to certain decisions about its
architecture. At one extreme, the servers for an E2EVIV system could
be hosted on a bespoke server cluster, built from the ground up
specifically for the system and housed in a facility under the
physical control of electoral authorities or their authorized
representatives. At another extreme, the servers could be hosted on a
commodity cloud computing infrastructure such as Amazon EC2 or
Microsoft Azure where electoral authorities have no physical control
over the servers. A third extreme would see the system implemented in
a purely peer-to-peer fashion, with the system's functionality
distributed among all participating computers with no
specifically-designated servers. The choice within this spectrum has
significant impact on both system availability and system security
requirements.

\todokiniry{devops mentality vs. certification?}

\subsection{Threats}

Mitigating the potential threats to E2EVIV systems also leads to
various architectural choices. The following are some of the threats
that need to be considered. 

Any single point of failure, such as a single server that contains
essential data without which the system can no longer function, is a
tempting target for attack. Therefore, the architecture should attempt
to minimize or eliminate such failure points.

The use of a fixed IP address (or address range) for E2EVIV system
components can open the system to denial of service attacks, limit
deployment flexibility, and make it more difficult to recover from
failures. The architecture should be chosen such that IP addresses
need not be hard-coded. 

\todokiniry{Cloudflare restrictions? I know how Cloudflare works, but
  I'm not sure which restrictions you were thinking of here, and I'm
  not coming up with anything concise to say about it. -dmz}

\todokiniry{non-typical foundations? not sure about what to say about
  the ``threat'' here. -dmz}

\todokiniry{MPC?}

\subsection{Distributing Trust}

The distribution of trust in an E2EVIV system is critical: if the
system is not trustworthy, the election results generated by it are
inherently suspect. Ken Thompson, in his 1984 Turing Award lecture
``Reflections on Trusting Trust'' \cite{Thompson84}, demonstrated a
fundamental problem with trust in computing systems: an attack against
the toolchain used to build a system can silently, and effectively
undetectably, insert a ``back door'' or other corruption into the
system. If this attack is carried out successfully, inspection of the
source code for the toolchain itself and the source code for the
system will show nothing unusual; the corrupted toolchain binary
introduces the corruption when building itself, or when building the
rest of the system, and also corrupts all the tools that can be used
to analyze the system (disassemblers, binary dump tools, etc.) such
that the corruption remains hidden. Thompson himself successfully
carried out such an attack within Bell Labs, and similar attacks have
occurred ``in the wild'' against systems such as the Delphi
development environment for Windows application; with stakes as high
as controlling national election results, it is not a stretch to
believe that such attacks would be attempted against E2EVIV systems.

There are multiple ways to mitigate the possible impact of such an
attack. One is to ensure that the system uses a diverse set of
implementations of key components, all based on the same specification
but with different source code, built with different compilers, and
preferably running on different hardware and OS platforms; corruption
of a single component, or even a small number of them, could then be
detected by the uncorrupted components, and the effort required to
corrupt the system as a whole would be much higher. Another is to
counter the possibility of Thompson-style exploits by using multiple
toolchains in the technique proposed by David A. Wheeler in his
Ph.D. thesis, ``Fully Countering Trusting Trust through Diverse
Double-Compiling'' \cite{Wheeler09}.

Another potential way of distributing trust in an E2EVIV system is to
use secure multiparty computation (MPC), such that no single system
needs to be completely trusted. \todo{flesh this out}

\subsection{Scalability}

An E2EVIV system, particularly at the national level, must be able to
handle a wide range of demand. It is human nature that many voters
will wait until the last day, or even the last hour, of a voting
period to cast their votes. Moreover, it is likely that attacks
against the system---and thus, system activity in general---will
increase in intensity as the end of a voting period approaches. Thus,
while the system may see very little sustained activity for much of an
election period, it must be able to scale to extreme levels of
activity at peak times. The architecture must take this into account,
so that the system can be dynamically deployed on more computing and
network resources as need arises. This might be done either by
utilizing public cloud resources that support elastic demand, or by
using private resources that can be brought on- and offline as
required.

\subsection{Availability}

E2EVIV systems must exhibit high availability;
\autoref{chapter:required_properties} stated an explicit requirement
for 99.9\% uptime during election periods and the ability to recover
from generalized (i.e., not caused by natural disaster or malicious
attack on the system) failures in under 10 minutes, and higher
availability---including in the face of malicious attack---would be
preferable. There are a number of techniques for ensuring high
availablity of systems, including the use of services like those
provided by Cloudflare to handle traffic spikes and distributed denial
of service attacks. The system architecture should be constructed in a
way that does not foreclose the use of such techniques.

\subsection{Usability}

Usability, including accessibility for disabled voters, is of
paramount importance in an E2EVIV system. Especially for the
voter-facing parts of the system, the choice of implementation
technology may have a significant effect on usability. Essentially,
choices may need to be made between using Web technologies, which have
significant advantages in terms of reach (cross-platform, able to be
used on various sizes of device), and native applications, which tend
to exhibit richer interaction design and support more accessibility
features. The architecture might also allow for both types of
implementation, potentially at the cost of additional architectural
complexity.


