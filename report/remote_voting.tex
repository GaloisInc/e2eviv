\chapter{Remote Voting\ifdraft{ (Philip) (15\%)}{}}
\label{chapter:remote_voting}

\section{Rationale}
\subsection{Geographic Dispersion}
\ldots This is refering to residents of sparsley populated\ldots
\subsection{Accessibility}
\ldots 20\% of U.S. adults with disabilities say they have been unable to vote in presidential or congressional elections due to barriers at or getting to the polls.
\subsection{UOCAVA}
In 1986, Congress enacted the Uniformed and Overseas Citizens Absentee Voting Act, stating citizens that are part of the uniformed services, merchant marines, and their families or citizens residing overseas are allowed to register and vode absentee for federal office. 

\ldots In the 2008 Post-Election OUCAVA Survey Report and Analysis 52\% of attempted votes were not counted because the ballots were late or never arrived.
\subsection{Early Voting}
\subsection{Expectations}
\section{History}
\section{Current Practice}
\subsection{Integration with Local Elections}
\section{Shortcomings of Current Practice}
\subsection{Use of Communication/Internet}
\subsection{Accessibility and Usability}
\subsection{Auditing}
\subsubsection{Current Practice}
\subsubsection{Digital vs. Physical}
\begin{itemize}
\item VVPAT provides voters with paper statements
\end{itemize}
\subsubsection{Risk-Limiting Audits}
Risk limiting audits use a public random auditing process to make an argument about the statistical confidence of a particular election result.
