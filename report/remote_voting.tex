\chapter{Remote Voting\ifdraft{ (Philip) (35\%)}{}}
\label{chapter:remote_voting}

\section{Rationale}
\subsection{Geographic Dispersion}
\ldots This is refering to residents of sparsley populated\ldots
\subsection{Accessibility}
Studies issued by the International Center for Disabiliy Information and the National Institude on Disability and Rehabilitation Research indicate that 20\% of Americans live with dissabilities. 

The Help America Vote Act (HAVA) of 2002 requires that all polling places in ellections for federal offce, anyware in the United States have at least one voting sy

\ldots 20\% of U.S. adults with disabilities say they have been unable to vote in presidential or congressional elections due to barriers at or getting to the polls.
\subsection{UOCAVA}
In 1986, Congress enacted the Uniformed and Overseas Citizens Absentee Voting Act, stating citizens that are part of the uniformed services, merchant marines, and their families or citizens residing overseas are allowed to register and vode absentee for federal office. 

\ldots In the 2008 Post-Election OUCAVA Survey Report and Analysis 52\% of attempted votes were not counted because the ballots were late or never arrived.
\subsection{Early Voting}
\subsection{Expectations}
\begin{itemize}
\item APSA 10 voting rights of military
\end{itemize}

\begin{itemize}
\item Privacy
\item Integrity of Cast and Count
\item VVPAT
\item HAVA and ADA Accessability
\item UOCAVA Accessability
\item Multilingual
\item Sufficient ballot transit time
\end{itemize}

\section{History}
Political attitudes and legistlation on absentee voting has been a slow moving effort, due to dominant partisian attitues changing, and regulations being enforced at the state level.

Before the civil war US citicens primarily voted in their places of residence, and many states legally barred the casting of votes from outside state borders. There was little effort from any state to accomidate absentee voting. However, in 1864 with the American Civil War displacing soldiers from their residences, Lincoln's re-election was at risk. With much lobbying on behaf of the republican party (and opposition from the democratic party), nineteen of the union's states adopted absentee voting procedures for military voters on federal elections in time for the election. (It is unknown if these votes played a significant role in Lincoln's victory.) Unfortunatley since the motiviation to passing these laws was securing Lincoln's re-ellection, rather than persistant enfranchisement, many absentee military voter laws were treated as temporary and repealed after the war.

In 1918 America's War Department decided that it was not ready to support the military vote. World War I had displaced such a large number of persons and military units were rarely composed of same state citicens. Not even states in support of military vote were allowed the soldier vote, even on matters at the state level.

As in the Civil War, World War II inspired another push for the military vote in hopes of supporting the re-election of the presidential incumbant. This introduced the Soldier Voting Act (1942) which, although passed too late for the presidential ellection, mandated military personel rights to absentee vote on federal ellections during times of war without subjication to voting tax or postage costs. From this point forth all overseas voting would be regulated at the federal level and implemented at the state level. However, by 1944, the state mandate to allow military absentee voting was amended to a recomendation.

The absentee civilian vote was a bit further behind. In 1896, states began introducing civilian absentee voting legistlation. By 1924 only three states in the union hadn't passed some absentee voting legislation. The Federal Voting Assistance Act of 1955 combined voting rights for overseas civilian goverment employees with military.




\begin{itemize}
\item Civilian vote
\item World Wars I \& II
\item post WWII
\item APSA (1952)
\begin{enumerate}
\item absentee registration
\item no voting tax
\item use of federal postcard application
\item recive calatoral for informed voting
\item ballot transit time %appropriate time to cast ballot
\item vote in times of peace as well as war
\end{enumerate}
\item FVAP
\item HAVA \& ADA
\item MOVE act (2009) % transit time 
\end{itemize}

\subsection{Integration with Local Elections}
\section{Shortcomings of Current Practice}

20\% of Americans with dissabilities have said that they were unable to vote in presidential or congressional ellection

\subsection{Use of Communication/Internet}
\subsection{Accessibility and Usability}
\subsection{Auditing}
\subsubsection{Current Practice}
\subsubsection{Digital vs. Physical}
\begin{itemize}
\item VVPAT provides voters with paper statements
\end{itemize}
\subsubsection{Risk-Limiting Audits}
Risk limiting audits use a public random auditing process to make an argument about the statistical confidence of a particular election result.
