\chapter{Remote Voting\ifdraft{ (Philip) (100\%)}{}}
\label{chapter:remote_voting}

\section{Rationale}
Remote voting is becoming increasingly common, necessitated by the growing
and diverse needs of voters. It is used to enable overseas
citizens and military personnel to participate in elections, reduce
access related discrimination domestically, and decrease expensive
administrative overhead of polling locations. 

In the United States, fewer than 5\% of ballots cast in general
elections during the 1980s were cast before Election Day. By the 2012
general election, 31\% of all ballots were cast early, and 17\% were
cast by mail. The states of Washington, Oregon, and most recently
Colorado have entirely switched over to all-mail voting. For an
election system to fully enfranchise the electorate, it must treat
remote voting as a first-class capability rather than as a backup
system with second-class effectiveness, speed, security, and
integrity.

\subsection{Accessibility}
According to the International Center for Disability Information and
the National Institute on Disability and Rehabilitation Research, 20\%
of Americans live with disabilities. The Voting Accessibility for the
Elderly and Handicapped Act of 1984 mandates that any person with a
disability may vote remotely without having to present medical
documentation, reducing the barriers to remote voting for those with
accessibility needs.

%The Help America Vote Act (HAVA) of 2002 requires that all polling places in elections for federal office, anywhere in the United States have at least one voting system  \ldots
%\ldots 20\% of U.S. adults with disabilities say they have been unable to vote in presidential or congressional elections due to barriers at or getting to the polls.

\subsection{Overseas and Military Voters}
In 1986, Congress enacted the Uniformed and Overseas Citizens Absentee
Voting Act (UOCAVA) to address the needs of citizens in the uniformed
services, merchant marines, and other overseas civilians. UOCAVA
mandates that these overseas and military voters (UOCAVA voters) be
able to register and vote remotely in federal elections. It is
difficult to calculate the exact number of "UOCAVA eligible" voters,
but \autoref{fig:uocava_populations} lists a recent estimate of the
total.

\todoacf{Ask Philip for context on these numbers. Statewise breakdown
  doesn't seem to add much, and the difference between what the two
  are counting is unclear}

\begin{figure}
\begin{center}
\begin{tabular}{l p{.3\textwidth} p{.3\textwidth}} % this can be made prettier
  {\bf State} &
  {\bf Overseas Voting Eligible \newline Population} \newline(McDonald 2009) &
  {\bf Overseas military and federal \newline civilian employees}
  \newline(US Census Bureau 2010)\\\hline\\
Texas & 11.05\% & 11.78\%\\
California & 9.78\% & 8.44\% \\
Florida & 9.09\% & 9.54\% \\
New York & 5.31\% & 4.12\% \\
Pennsylvania & 4.10\% & 3.12\% \\
Illinois & 4.03\% & 3.24\% \\
Ohio & 3.51\% & 3.07\% \\
Michigan & 3.29\% & 2.68\% \\
Georgia & 2.84\% & 3.83\% \\
Washington & 2.78\% & 2.77\% \\
North Carolina & 2.78\% & 2.91\% \\
Tennessee & 2.57\% & 2.81\% \\
Virginia & 2.51\% & 3.52\% \\\hline\\
Estimated Total & 4,972,217 & 1,042,523
\end{tabular}
\end{center}
\caption{Comparisons of American Overseas Population by State}
\label{fig:uocava_populations} % change to table?
\end{figure}

\subsection{Domestic Absentee}
Domestic absentee voters are those who vote early in-person or cast
their votes by mail because they are unable, or do not want, to be
present at polling locations on Election Day; this excludes UOCAVA
voters and voters in states that vote exclusively by mail. Of all
votes cast in the 2012 U.S. general election, 21,853,762 (16.6\%) were
cast by domestic absentee voters~\cite{eac2012survey}. As of this
writing, 27 states allow voters to apply for an absentee ballot
without providing a justification, known as ``no-excuse absentee
voting''.

\todoacf{Take out this figure too?}
% proportions of domestic absentee voters. These states are frequently
% using domestic absentee to address concerns and costs associated with
% administering polling locations in low population density areas.

\begin{figure}
\begin{center}
\begin{tabular}{l c}
{\bf State} & {\bf Percent of Population}\\\hline\\
Colorado & 71.4\%\\
Arizona & 65.9\%\\
Montana & 57.5\%\\
Georgia & 48.8\%\\
Iowa & 43.1\%\\
California & 39.8\%\\
Hawaii & 36\%\\
North Dakota & 28.8\%\\
Florida & 26.8\%\\
Michigan & 26.4\%\\
Wyoming & 26.2\%\\
Maine & 25.5\%\\
Nebraska & 25.4\%\\
Idaho & 24.3\%\\
Ohio & 22.4\%\\
Wisconsin & 21.4\%\\
Vermont & 20.4\%\\
\end{tabular}
\end{center}
\caption{Votes Cast as Domestic Absentee 2012 General Election}
\label{fig:domestic_populations}
\end{figure}

\subsection{Expectations}

In 1952, a study by the American Political Science Association defined
ten voting rights necessary for members of the armed
services~\cite{american1952findings}. Although initially defined for
military voters, these rights have served as the basis for defining
the expectations of all remote voters through UOCAVA and other
subsequent legislation. Among these rights are:

\begin{enumerate}
  \item To vote without registering in person;
  \item To vote without paying a poll tax or having to meet
    unreasonable requirements;
  \item To use the Federal Postcard Application\footnote{Federal
      Postcard Application is the Department of Defense name for the
      official UOCAVA voter registration and absentee ballot request
      form.} both to register and to request a ballot, rather than
    having to use state-specific paperwork;
  \item To receive ballots for primary and general elections in time to vote;
  \item To be protected in the free exercise of their voting rights;
    and 
  \item To receive essential information needed to vote.
\end{enumerate}

These rights first found their way into law in the Federal Voting
Assistance Act of 1955 (FVAA) but, due to partisan struggles, the
rights were watered down from requirements into recommendations; this
left most of the decisions about final implementation to the
states. Over time, subsequent legislation has strengthened these
recommendations into guarantees and requirements and has expanded
rights to include participation by non-English speakers and people
with disabilities.

% \todoacf{See todo above; this seems out of place}

% The Help America Vote Act of 2002 (HAVA) defines
% additional mandatory minimum standards for states that specifically
% address voter access concerns raised in the 2000 general
% election. This act requires that elections support participation by
% non-English speakers and disabled persons by providing the same
% opportunities for access and participation, including voter privacy
% and independence. Additionally, HAVA mandates that voters whose
% eligibility is questioned must be permitted to cast a provisional
% ballot to be reviewed later by election officials.

% Finally it is most important to include the expectation that votes cast by
% registered voters are counted correctly while preserving privacy. As will
% be mentioned later in this chapter, there are many examples where this
% isn't the case.

% HAVA also put stricter auditing requirements in place, requiring
% voting technologies to produce a Verifiable Voter Paper Audit Trail
% (VVPAT); while preserving the privacy of the voter and the secrecy
% of the cast ballot.


\section{History}
\subsection{Armed Forces Voting}
Before the American Civil War, U.S. citizens primarily voted in their
places of residence and many states legally barred the casting of
votes from outside state borders. There was little effort from any
state to accommodate absentee voting. In 1864, however, the Civil War
displaced soldiers from their residences and Lincoln's re-election was
at risk. With much lobbying on behalf of the Republican Party (and
opposition from the Democratic Party), nineteen Union states adopted
absentee voting procedures for military voters in time for the
election. Since the motivation for passing these laws was to secure
Lincoln's re-election rather than permanently expand voting access,
many states treated absentee military voter laws as temporary and
repealed them after the war.

For the 1918 midterm elections, the U.S. War Department decided that
it was not ready to support the military vote; the Department
prohibited individual states from canvassing overseas soldiers serving
in World War I. World War II inspired another push for the military
vote in hopes of supporting the re-election of the presidential
incumbent. This prompted the Soldier Voting Act (1942).  Although it
was passed too late for the 1942 midterm elections, this law gave
military personnel absentee voting rights for federal elections during
times of war without being required to pay a voting tax or postage
costs. The act has continuing significance: it stated that all
overseas voting would be regulated at the federal level and
implemented at the state level, a structure that continues to this
day. By 1944, partisan politics led to the weakening of the state
mandate from a requirement to a recommendation, leading to a 29.1\%
turnout rate vs. the 60\% domestic turnout rate~\cite{smith2015}.

\subsection{Remote Civilian Voting}

Progress for civilian absentee voters lagged behind progress for
military voters. In 1896, states began to introduce civilian absentee
voting legislation. By 1924, only three states had no absentee voting
legislation. State laws, however, were a confusing and inconsistent
patchwork that limited absentee turnout. Major progress for civilian
absentee voters would only come with legislation motivated primarily
by military voters such as the FVAA.

In the 1960s, lobbying from overseas civilian groups led to amendments
to the FVAA. This effort expanded the number of civilians covered by
the law, though the amendments were once again voluntary
recommendations to the states. As lobbying pressure increased further,
the Overseas Citizens Voting Rights Act (OCVRA) passed in 1974.  OCVRA
was the first law to guarantee, rather than only recommend, absentee
voting rights for overseas civilians.

In 1986, legislators passed UOCAVA, which combined and replaced FVAA
and OCVRA. UOCAVA made the rights recommended by the previous acts
into requirements for both military and overseas civilian voters.

\subsection{Disabled Civilian Voting}
The Voting Rights Act (VRA) of 1965 was the first legislation to allow
voters who require assistance to vote---because of blindness,
disability, or inability to read or write---to receive help from a
person of their choice.

The Voting Accessibility for the Elderly and Handicapped Act of 1984
(VAEHA) improved access for disabled and elderly individuals.  Like
FVAA and OCVRA, VAEHA did not specify standards of access; individual
states set their own standards. VAEHA also limited the disabled voters
group to those with {\em physical disabilities}. It did, however,
mandate that ``no notarization of medical certification shall be
required of a voter with a disability with respect to an absentee
ballot or application for such ballot.''

The 1990 Americans with Disabilities Act (ADA) requires that people
with disabilities have access to basic public services, including the
right to vote. ADA does not strictly require that polling locations
are accessible, however it did extend the definition of disability to
\begin{quote}
  ``a person who has a physical or mental impairment that
  substantially limits one or more major life activities, a person who
  has a history or record of such an impairment, or a person who is
  perceived by others as having such an impairment.''
\end{quote}

Over the years, legislators have passed a number of federal laws to
protect voting rights of disabled citizens. The majority of these laws
have struggled to clearly define a representative range of
disabilities. The laws often focus on in-person access to physical
polling locations, which is expensive for states to
implement. State-defined policies often ignore rights to voting
privacy, and exclude people who have multiple disabilities because
technologies that may help them vote are not yet available.


% Unfortunately most states claim to be unable to handle the costs
% associated with providing appropriate voting technologies.  much of
% this cost falls on the state to implement, and is co designate
% restrictions on physical polling locations.  With the passage of the
% American Disabilities Act (ADA) in Voting Accessibility for the
% Elderly and Handicapped Act of 1984 VVPAT
\subsection{Modern Remote Voting}

In 2002, Congress passed the Help America Vote Act (HAVA) in response
to problems found in gathering, counting, and auditing ballots in the
2000 presidential election. HAVA requires that all polling places in
elections for federal office anywhere in the United States have at
least one voting system capable of assisting disabled voters. This
requirement addresses some accessibility concerns.

HAVA was also a response to the large number of rejected ballots in
the 2000 election and an inability to sufficiently audit ballots. HAVA
recommends that election systems produce a Verifiable Voter Paper
Audit Trail (VVPAT) while preserving the privacy of the voter and the
secrecy of the cast ballot. HAVA also created the United States
Election Assistance Commission (EAC) to oversee the development of new
voting machine standards. The National Institute of Standards and
Technology (NIST) released the Voluntary Voting System Guidelines
(VVSG) to aid in this transition.

The Military and Overseas Voter Empowerment Act of 2009 (MOVE)
addresses barriers to overseas voter participation, and specifically
attempts to reduce the number of ballots that are not counted due to
late receipt. MOVE requires states to send absentee ballots at least
45 days before Election Day, make all registration material and blank
ballots available electronically, and remove notarization requirements
on voting applications and ballots.

%\subsection{Integration with Local Elections}

%Every state has their own requirements, deadlines, and transmission
% restrictions which the FVAP documents in a 'Voting Assistance Guide'
% distributed to potential UOCAVA voters.
%%%%%%%%%%%%%%%%%%%%%%%%%%%%%%%%%%%

Because state governments enforce existing voting regulations, these
regulations are hampered by local political attitudes. In 2010, the
Uniform Law Commission oversaw drafting of the Uniform Military
Services and Overseas Civilian Absentee Voters Act (UMOVA). UMOVA is
designed to identify and standardize the important protections and
benefits found in federal legislation like UOCAVA and MOVE in state
and local elections. As of April 2015, fourteen states and the
District of Columbia have enacted UMOVA.

\section{Shortcomings of Current Practice}

Despite years of progressively stronger legislation addressing the
needs of remote voters, many shortcomings still exist in current
election practices. The topics listed below draw from specific
concerns that negatively affect remote voting participants.

\subsection{Use of Communication Technologies}
The majority of remote voting takes place via postal mail, which has
many inherent faults that affect the voting process. The 2008
Post-Election UOCAVA Survey Report and Analysis found that voting
boards did not count 52\% of attempted UOCAVA votes due to problems in
the mail delivery process. In addition, maintaining correct voter
registration information for military voters and others who frequently
change addresses while abroad is expensive and prone to error.

% \begin{quote} {\em \ldots want to make comment about Florida, but concerned that it may be taken as political}\end{quote}

%%%%%%%%%%%%%%%%%%%%%%%%%%%%%%%%%%%
% Entire process for UOCAVA voters generally can take up between 2 weeks and 2.5 months
%%%%%%%%%%%%%%%%%%%%%%%%%%%%%%%%%%%

To address differences between states' absentee registration and
voting practices, the Federal Voting Assistance Program (FVAP)
provides a Voting Assistance Guide (VAG) to support UOCAVA voter
registration and ballot request. The VAG supports military Voting
Action Officers who assist their units with voter registration
instructions. 

Since MOVE was passed in 2009, online registration has expanded
significantly. OVF and the FVAP offer online voter services to UOCAVA
voters through their websites, as do many states and counties. Online
blank ballot transmission is becoming routine, and email
communications between election officials and voters are flourishing.

Unfortunately, the problem of late ballot receipt and rejection has
not been solved.
\todoacf{doesn't the Federal Postcard Application address this?}

%%%%%%%%%%%%%%%%%%%%%%%%%%%%%%%%%%%
%\begin{itemize}
%\item The major motivations for use of internet and communication technologies for UOCAVA has been to address ballot transit time, and simplify voter registration.
%\item OVF's streamlined website for FPCA in states that allow online registration
%\end{itemize}
%%%%%%%%%%%%%%%%%%%%%%%%%%%%%%%%%%%

\subsection{Accessibility and Usability}

In 2007, 20\% of Americans with disabilities said they were unable to
vote in a presidential or congressional election due to difficulty
getting to the polls or barriers at polling
locations~\cite{runyan2007improving}. The voting technologies used,
and the physical locations of polling places, often cause such
problems. In the 2000 presidential election, 56\% of randomly sampled
polling places in the United States had at least one barrier to
disabled voters~\cite{united2001voters}.

Voters with disabilities often forfeit privacy in order to have
someone help them with in-person voting. Polling locations often lack
technology that can help, and the assistive technology that is
available is often too difficult for voters and poll workers to
use. Remote voting still presents obstacles: those with dexterity
impairments often have problems handling and marking paper absentee
ballots.

\subsection{Auditing}

Although voter fraud is fairly uncommon, it is a major concern in a
bipartisan system. Voter fraud is very difficult to detect without
reducing turnout or disenfranchising legitimate voters. Policies
intended to reduce fraud or protect identities, such as increasingly
prevalent and strict voter ID requirements, often lead to a higher
rate of rejected ballots. This is the case even for remote voters. In
the 2012 general election, voting boards rejected over 20\% of
absentee ballots due to non-matching signatures or insufficient
identification~\cite{eac2012survey}.

\subsection{Voter Privacy}

A fair voting system must ensure voter privacy. Privacy promotes voter
independence and helps prevent voter coercion and vote buying. Most
remote voting practices require that voters forfeit independence or
privacy because election officials cannot enforce privacy when voting
takes place outside of polling locations. Several states require
voters to sign a voter privacy waiver when casting a remote
ballot~\cite{smithtime}.

% In several jurisdictions, UOCAVA votes are
% not counted until it is determined that they may sway the election.
% perhaps nice to have cost analysis

%%%%%%%%%%%%%%%%%%%%%%%%%%%%%%%%%%%
%\subsubsection{Digital vs. Physical}
%\subsubsection{Risk-Limiting Audits}
%Risk limiting audits use a public random auditing process to make a strong
%argument about the statistical confidence of a particular election result. 
