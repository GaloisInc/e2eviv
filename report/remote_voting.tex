\chapter{Remote Voting\ifdraft{ (Philip) (n\%)}{}}
\label{chapter:remote_voting}

\section{Rationale}

\begin{itemize}
\item Who is in this group?
\item How many people are in this group?
\item What is something IV could treat
\end{itemize}

\subsection{Geographic Dispersion}
\ldots This is referring to residents of sparsely populated\ldots

\subsection{Accessibility}
Studies issued by the International Center for Disability Information and the National Institute on Disability and Rehabilitation Research indicate that 20\% of Americans live with disabilities.

The Help America Vote Act (HAVA) of 2002 requires that all polling places in elections for federal office, anywhere in the United States have at least one voting \ldots

\ldots 20\% of U.S. adults with disabilities say they have been unable to vote in presidential or congressional elections due to barriers at or getting to the polls.

\subsection{UOCAVA}
In 1986, Congress enacted the Uniformed and Overseas Citizens Absentee Voting Act, stating citizens that are part of the uniformed services, merchant marines, and their families or citizens residing overseas are allowed to register and vote absentee for federal office.

\subsection{Early Voting}
\subsection{Expectations}

\begin{itemize}
\item American Political Science Association study (1952)
  \begin{itemize}
  \item Ability to vote without registering in person
  \item Ability to vote without unreasonable requirements and costs (federal postcard)
  \item Insure enough time to permit ballot transite time
  \end{itemize}
\item HAVA, ADA
\begin{itemize}
  \item Multilingual
  \item Accessible for individuals with disabilities
\end{itemize}
\item Privacy of Vote
\item Integrity of Vote (VVPAT)
\end{itemize}

\section{History}
Political attitudes and legislation on absentee voting has been a slow moving 
effort, due to dominant partisan attitudes changing, and regulations being 
enforced at the state level.

Before the civil war US citizens primarily voted in their places of residence, and many states legally barred the casting of votes from outside state borders. There was little effort from any state to accommodate absentee voting. However, in 1864 with the American Civil War displacing soldiers from their residences, Lincoln's re-election was at risk. With much lobbying on behalf of the republican party (and opposition from the democratic party), nineteen of the union's states adopted absentee voting procedures for military voters on federal elections in time for the election. Unfortunately since the motivation to passing these laws was securing Lincoln's re-election, rather than persistent enfranchisement, many absentee military voter laws were treated as temporary and repealed after the war.

\ldots something missing \ldots

In 1918 America's War Department decided that it was not ready to support the military vote. World War I had displaced such a large number of voting eligible persons and military units were rarely composed of same state citizens. Not even states in support of military vote were allowed the soldier vote, even on matters at the state level.

As in the Civil War, World War II inspired another push for the military vote in hopes of supporting the re-election of the presidential incumbent. This introduced the Soldier Voting Act (1942) which, although passed too late for the presidential election, mandated military personnel rights to absentee vote on federal elections during times of war without subjugation to voting tax or postage costs. From this point forth all overseas voting would be regulated at the federal level and implemented at the state level. However, by 1944, the state mandate to support military absentee voting was amended to a recommendation.

\ldots bit broken \ldots

The absentee civilian vote was a bit further behind. In 1896, states began introducing civilian absentee voting legislation. By 1924 only three states in the union had no absentee voting legislation, but all states had different laws and restrictions. Major progress on this front wasn't made until federal voting laws were passed that combined for the handling of civilians and military votes. The Voting Assistance Act of 1955 was the first to federally combine voting policy recommendations for overseas civilian government employees with military. In 1986 Voting Assistance Act was amended to include individuals temporarily living outside the United States. With lobbying from sympathetic groups and the quickly growing population of overseas civilians, in 1974 Overseas Citizens Voting Rights Act passed extending the recognized vote to citizens regardless of their intentions to return to the United States.

By 1986, combatant attitudes towards overseas votes had finally settled, and the Uniformed and Overseas Citizens Absentee Voting Act (UOCAVA) was passed replacing/combining Overseas Citizens Voting Rights Act and the Federal Voting Assistance Act, and finally made supporting the overseas absentee ballot a requirement.

\begin{itemize}
\item begin history of disabled voters rights
\item HAVA 
\end{itemize}


\begin{itemize}
\item APSA (1952)
\begin{enumerate}
\item absentee registration
\item no voting tax
\item use of federal postcard application
\item receive collateral for informed voting
\item ballot transit time %appropriate time to cast ballot
\item vote in times of peace as well as war
\end{enumerate}
\item FVAP
\item HAVA \& ADA
\item MOVE act (2009) % transit time 
\end{itemize}

\subsection{Integration with Local Elections}

\section{Shortcomings of Current Practice}



In the 2008 Post-Election OUCAVA Survey Report and Analysis 52\% of attempted votes were not counted because the ballots were late or never arrived.

20\% of Americans with disabilities have said that they were unable to vote in presidential or congressional election due to barriers at or getting to the polls.

Absentee ballots 

\subsection{Use of Communication/Internet}
\subsection{Accessibility and Usability}
\subsection{Auditing}
\subsubsection{Current Practice}
\subsubsection{Digital vs. Physical}
\begin{itemize}
\item VVPAT provides voters with paper statements
\end{itemize}
\subsubsection{Risk-Limiting Audits}
Risk limiting audits use a public random auditing process to make an argument about the statistical confidence of a particular election result.
