\chapter{Remote Voting\ifdraft{ (Philip) (85\%)}{}}
\label{chapter:remote_voting}

\section{Rationale}
Remote voting is becoming ever more common, necessitated by a the growing and diverse needs of voters. It used to enfranchise overseas citizens and military personnel, reduce access related discrimination domestically, and decrease expensive administrative overhead of polling locations. Some states have entirely switched over to remote voting. 

In the 1980's less than 5\% of ballots were cast before election day. In the 2012 general election, 31\% of all ballots were cast early, and 17\% were cast by mail.

\subsection{Accessibility}
Studies issued by the International Center for Disability Information and the National Institute on Disability and Rehabilitation Research indicate that 20\% of Americans live with disabilities. With this in mind, the Voting Accessibility for the Elderly and Handycapped Act of 1984 designates that any person with a disability may apply for absentee ballot without need of medical certificate.

%The Help America Vote Act (HAVA) of 2002 requires that all polling places in elections for federal office, anywhere in the United States have at least one voting system  \ldots
%\ldots 20\% of U.S. adults with disabilities say they have been unable to vote in presidential or congressional elections due to barriers at or getting to the polls.

\subsection{UOCAVA}
In 1986, Congress enacted the Uniformed and Overseas Citizens Absentee Voting Act, stating citizens that are part of the uniformed services, merchant marines, and their families or citizens residing overseas are allowed to register and vote absentee for federal office. It is very difficult to calculate the exact number of UOCAVA eligible voters, so several studies have been performed to provide best estimates. According to the two studies referenced, the states most impacted by UOCAVA are listed in \autoref{fig:uocava_populations}; this estimates the percentage of UOCAVA voters that represent a particular state state.

\begin{figure}
\begin{center}
\begin{tabular}{l p{.3\textwidth} p{.3\textwidth}} % this can be made prettier
{\bf State} & {\bf Overseas Voting Eligible \newline Population} \newline(McDonald 2009) & {\bf Overseas military and federal \newline civilian employees} \newline(US Census Bureau 2010)\\\hline\\
Texas & 11.05\% & 11.78\%\\
California & 9.78\% & 8.44\% \\
Florida & 9.09\% & 9.54\% \\
New York & 5.31\% & 4.12\% \\
Pennsylvania & 4.10\% & 3.12\% \\
Illinois & 4.03\% & 3.24\% \\
Ohio & 3.51\% & 3.07\% \\
Michigan & 3.29\% & 2.68\% \\
Georgia & 2.84\% & 3.83\% \\
Washington & 2.78\% & 2.77\% \\
North Carolina & 2.78\% & 2.91\% \\
Tennessee & 2.57\% & 2.81\% \\
Virginia & 2.51\% & 3.52\% \\\hline\\
Estimated Total & 4,972,217 & 1,042,523
\end{tabular}
\end{center}
\caption{Comparisons of American overseas population by state}
\label{fig:uocava_populations}
\end{figure}


\begin{center}
\end{center}

\subsection{Domestic Absentee}
Domestic Absentee are early mail cast votes by persons who are unable to be present at polling locations on election day. The domestic absentee vote count does not include states who are completely vote by mail nor UOCAVA voters. Gathered from the Election Assistance Commission's 2012 General Election Administration and Voting Survey, \autoref{fig:domestic_populations} lists the states most impacted by domestic absentee votes. These states are frequently using domestic absentee to address concerns associated with low population density.

\begin{figure}
\begin{center}
\begin{tabular}{l c}
State & Percent of Population\\\hline\\
Colorado & 71.4\%\\
Arizona & 65.9\%\\
Montana & 57.5\%\\
Georgia & 48.8\%\\
Iowa & 43.1\%\\
California & 39.8\%\\
Hawaii & 36\%\\
North Dakota & 28.8\%\\
Florida & 26.8\%\\
Michigan & 26.4\%\\
Wyoming & 26.2\%\\
Maine & 25.5\%\\
Nebraska & 25.4\%\\
Idaho & 24.3\%\\
Ohio & 22.4\%\\
Wisconsin & 21.4\%\\
Vermont & 20.4\%\\
\end{tabular}
\end{center}
\caption{Domestic Absentee from }
\label{fig:domestic_populations}
\end{figure}

\subsection{Expectations}

In 1952 there was a American Political Science Association special study of voting in the armed forces, ``to be sure that we have a completely effective program for voting in the armed services.'' In this study ten voting rights were clearly defined expectations of overseas voting.

\begin{enumerate}
  \item To vote without registering in person.
  \item To vote without paying a poll tax.
  \item To vote without meeting unreasonable residence requirements.
  \item To vote without meeting unreasonable literacy and educational requirements.
  \item To use the Federal postcard application for a ballot.
  \item To receive ballots for primary and general elections in time to vote.
  \item To be protected in the free exercise of their voting rights.
  \item To receive essential information concerning candidates and issues.
  \item To receive essential information concerning the methods by which the right to vote may be exercised.
  \item To receive essential information on the duty of `citizens in uniform' to defend our democratic institutions by using, rather than ignoring, their voting rights.
\end{enumerate}

The Help America Vote Act (2002) defines new mandatory minimum standards for states to follow, to specifically address access related concerns raised in the 2000 presidential election. This defines the need for multilingual and disabled persons support for elections and their supporting technologies while providing the same opportunity for access and participation (including privacy and independence). Additionally, persons who have questionable voting eligibility must be permitted a `provisional ballot.' 

HAVA also places stricter auditing requirements in place, requiring voting technologies to produce a Verifiable Voter Paper Audit Trail (VVPAT); while preserving the privacy of the voter and the secrecy of the cast ballot.


\section{History}
Political attitudes and legislation on absentee voting has been a slow moving effort, due to dominant partisan attitudes changing, and regulations being enforced at the state level.

Before the civil war US citizens primarily voted in their places of residence, and many states legally barred the casting of votes from outside state borders. There was little effort from any state to accommodate absentee voting. However, in 1864 with the American Civil War displacing soldiers from their residences, Lincoln's re-election was at risk. With much lobbying on behalf of the republican party (and opposition from the democratic party), nineteen of the union's states adopted absentee voting procedures for military voters on federal elections in time for the election. Unfortunately since the motivation to passing these laws was securing Lincoln's re-election, rather than persistent enfranchisement, many absentee military voter laws were treated as temporary and repealed after the war.

In 1918 America's War Department decided that it was not ready to support the military vote. World War I had displaced such a large number of voting eligible persons and military units were rarely composed of same state citizens. Not even states in support of military vote were allowed the soldier vote, even on matters at the state level.

As in the Civil War, World War II inspired another push for the military vote in hopes of supporting the re-election of the presidential incumbent. This introduced the Soldier Voting Act (1942) which, although passed too late for the presidential election, mandated military personnel rights to absentee vote on federal elections during times of war without subjugation to voting tax or postage costs. From this point forth all overseas voting would be regulated at the federal level and implemented at the state level. However, by 1944, the state mandate to support military absentee voting was amended to a recommendation.

Progress with absentee civilian vote was a further behind. In 1896, states began introducing civilian absentee voting legislation. By 1924 only three states in the union had no absentee voting legislation, but all states had different laws and restrictions. Major progress on this front wasn't made until federal voting laws were passed that combined for the handling of civilians and military votes. The Voting Assistance Act of 1955 was the first to federally combine voting policy recommendations for overseas civilian government employees with military. In 1986 Voting Assistance Act was amended to include individuals temporarily living outside the United States. With lobbying from sympathetic groups and the quickly growing population of overseas civilians, in 1974 Overseas Citizens Voting Rights Act passed extending the recognized vote to citizens regardless of their intentions to return to the United States.

By 1986, combatant attitudes towards overseas votes had finally settled, and the Uniformed and Overseas Citizens Absentee Voting Act (UOCAVA) was passed replacing/combining Overseas Citizens Voting Rights Act and the Federal Voting Assistance Act, and finally made supporting the overseas absentee ballot a requirement.

Meanwhile, the Voting Rights Act (VRA) of 1965 was the first legislation to enfranchise voters with disabilities. The VRA granted voters who require assistance to vote by reason of blindness disability or inability to read or write assistance by a person of the voters choice. This also introduced some of the earlier legislation defining a disabled citizen.

The Voting Accessibility for the Elderly and Handicapped Act of 1984 (VAEHA) was passed to improve access for handicapped and elderly individuals. However, states were left to set their own standards of {\em access}, and limited the disabled voters group to those with {\em physical disabilities}. The VAEHA did, however, mandate `no notarization of medical certification shall be required of a voter with a disability with respect to an absentee ballot or application for such ballot.'

The 1990 American Disabilities Act (ADA), although not specific to voting rights, required that people with disabilities have access to basic public services, including the right to vote, however does not strictly require that polling locations are accessible. ADA did however extend the definition of disability to:
\begin{quote}``a person who has a physical or mental impairment that substantially limits one or more major life activities, a person who has a history or record of such an impairment, or a person who is perceived by others as having such an impairment.''
\end{quote}

The majority of federal laws passed to protect disabled voting have struggled to clearly define an representative range of disabilities, and are often focused on access at physical polling locations; which is expensive for states to implement. Additionally, state defined policies often ignore rights to voting privacy, and exclude persons with multiple disabilities sighting that aiding technologies are not yet available.

%Unfortunately most states claim to be unable to handle the costs associated with providing appropriate voting technologies. 
%much of this cost falls on the state to implement, and is co
%designate restrictions on physical polling locations. 
%With the passage of the American Dissabities Act (ADA) in
% Unfortunately, states found the ne
%Voting Accessibility for the Elderly and Handicapped Act of 1984 

\begin{itemize}
\item VVPAT?
\end{itemize}

\subsection{Integration with Local Elections}

Every state has their own requirements, deadlines, and transmission restrictions which the FVAP documents in a 'Voting Assistance Guide' distributed to potential UOCAVA voters. 

\section{Shortcomings of Current Practice}

Presently the majority of remote voting is done `vote by mail'  which is subject to many inherent faults in vote by mail, that only become more frustrating for UOCAVA's overseas voters. Due to ballot delivery related problems, in the 2008 Post-Election UOCAVA Survey Report and Analysis, 52\% of attempted votes were not counted. \begin{quote} {\em \ldots want to make comment about Florida, but concerned that it may be taken as political}\end{quote}

Due to the many states' diverse registration practices, the Federal Voting Assistance Program provides an accompanying Voting Assistance Guide to registration forms. Unfortunately, this is very large and difficult to follow. For states without a streamlined online registration (hosted by OVF), this has resulted in many failed registration attempts.

% Entire process for UOCAVA voters generally can take up between 2 weeks and 2.5 months

In 2007, 20\% of Americans with disabilities said they were unable to vote in presidential or congressional election due to barriers at or getting to the polls. This is frequently a consequence of the voting technologies used and the physical location of polling places. In the 2000 presidential election 56\% of randomly sampled polling places in the united states have one identified impediments for disabled voters. Additionally, disabilities like dexterity impairments can prevent persons from marking paper ballots (at polling places and at home).

% Ten states explicitly require a privacy waver if a voter uses fax or e-mail to return a voted ballot''
Privacy of vote is a strong expectation of a fair voting system, that is often left addressed. Most practices for disabled and UOCAVA voters forfeit independent or private voting. Domestically, this is likely due to insufficient technology support at polling locations. Additionally, several states that allow for vote by e-mail or fax, require a voter privacy waver to be signed.

Although it is believed that voter fraud is fairly uncommon, it is considered very difficult to detect. Often, policies intended to reduce fraud or protect identities result in a higher rate of rejected ballots. According to the 2012 Election Administration and Voting Survey, 17.6\% of absentee ballots were rejected due to non-matching signatures.

\begin{quote} {\em \ldots need to include geographic dispersion stuff}\end{quote}

\subsection{Use of Communication/Internet}

\begin{itemize}
\item The major motivations for use of internet and communication technologies for UOCAVA has been to address ballot transit time, and simplify voter registration. 
\item OVF's streamlined website for FPCA in states that allow online registration
\end{itemize}

\subsection{Accessibility and Usability}
\subsection{Auditing}
\subsubsection{Current Practice}
\subsubsection{Digital vs. Physical}
\begin{itemize}
\item VVPAT provides voters with paper statements
\end{itemize}
\subsubsection{Risk-Limiting Audits}
Risk limiting audits use a public random auditing process to make an argument about the statistical confidence of a particular election result.
