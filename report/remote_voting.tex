\chapter{Remote Voting\ifdraft{ (Philip) (45\%)}{}}
\label{chapter:remote_voting}

\section{Rationale}

For each subsection::
\begin{itemize}
\item Who is in this group?
\item How many people are in this group?
\end{itemize}

\subsection{Geographic Dispersion}
%\ldots This is referring to residents of sparsely populated \ldots

\subsection{Accessibility}
Studies issued by the International Center for Disability Information and the National Institute on Disability and Rehabilitation Research indicate that 20\% of Americans live with disabilities. 

The Help America Vote Act (HAVA) of 2002 requires that all polling places in elections for federal office, anywhere in the United States have at least one voting system  \ldots

%\ldots 20\% of U.S. adults with disabilities say they have been unable to vote in presidential or congressional elections due to barriers at or getting to the polls.

\subsection{UOCAVA}
In 1986, Congress enacted the Uniformed and Overseas Citizens Absentee Voting Act, stating citizens that are part of the uniformed services, merchant marines, and their families or citizens residing overseas are allowed to register and vote absentee for federal office.

\begin{itemize}
\item approximate count
\item Table 2.1 Convenience Voting and Technology
\end{itemize}

\begin{center}
\begin{tabular}{l p{.3\textwidth} p{.3\textwidth}} % this can be made prettier
{\bf State} & {\bf Overseas Voting Eligible Population} (McDonald 2009) & {\bf Overseas military and federal civilian employees} (US Census Bureau 2010)\\\hline
Texas\\
California\\
Florida\\
New York\\
Pennsylvania\\
Illinois\\
Ohio\\
Michigan\\
Georgia\\
Washington\\
North Carolina\\
Tennessee\\
Virginia\\\hline
Total
\end{tabular}
\end{center}
\subsection{Early Voting}
\subsection{Expectations}

\begin{itemize}
\item American Political Science Association study (1952)
  \begin{itemize}
  \item Ability to vote without registering in person
  \item Ability to vote without unreasonable requirements and costs (federal postcard)
  \item Insure enough time is permitted for ballot transit
  \end{itemize}
\item HAVA, ADA
\begin{itemize}
  \item Multilingual
  \item Accessible for individuals with disabilities
\end{itemize}
\item Privacy of Vote
\item Integrity of Vote (VVPAT)
\end{itemize}

\section{History}
Political attitudes and legislation on absentee voting has been a slow moving 
effort, due to dominant partisan attitudes changing, and regulations being 
enforced at the state level.

Before the civil war US citizens primarily voted in their places of residence, and many states legally barred the casting of votes from outside state borders. There was little effort from any state to accommodate absentee voting. However, in 1864 with the American Civil War displacing soldiers from their residences, Lincoln's re-election was at risk. With much lobbying on behalf of the republican party (and opposition from the democratic party), nineteen of the union's states adopted absentee voting procedures for military voters on federal elections in time for the election. Unfortunately since the motivation to passing these laws was securing Lincoln's re-election, rather than persistent enfranchisement, many absentee military voter laws were treated as temporary and repealed after the war.

In 1918 America's War Department decided that it was not ready to support the military vote. World War I had displaced such a large number of voting eligible persons and military units were rarely composed of same state citizens. Not even states in support of military vote were allowed the soldier vote, even on matters at the state level.

As in the Civil War, World War II inspired another push for the military vote in hopes of supporting the re-election of the presidential incumbent. This introduced the Soldier Voting Act (1942) which, although passed too late for the presidential election, mandated military personnel rights to absentee vote on federal elections during times of war without subjugation to voting tax or postage costs. From this point forth all overseas voting would be regulated at the federal level and implemented at the state level. However, by 1944, the state mandate to support military absentee voting was amended to a recommendation.

Progress with absentee civilian vote was a further behind. In 1896, states began introducing civilian absentee voting legislation. By 1924 only three states in the union had no absentee voting legislation, but all states had different laws and restrictions. Major progress on this front wasn't made until federal voting laws were passed that combined for the handling of civilians and military votes. The Voting Assistance Act of 1955 was the first to federally combine voting policy recommendations for overseas civilian government employees with military. In 1986 Voting Assistance Act was amended to include individuals temporarily living outside the United States. With lobbying from sympathetic groups and the quickly growing population of overseas civilians, in 1974 Overseas Citizens Voting Rights Act passed extending the recognized vote to citizens regardless of their intentions to return to the United States.

By 1986, combatant attitudes towards overseas votes had finally settled, and the Uniformed and Overseas Citizens Absentee Voting Act (UOCAVA) was passed replacing/combining Overseas Citizens Voting Rights Act and the Federal Voting Assistance Act, and finally made supporting the overseas absentee ballot a requirement.

\hspace{2em}
\hrule
\hspace{2em}

Voting Rights Act (VRA) of 1965 was the first legistlation to enfranchise voters with disabilities. The VRA granted voters who require assistance to vote by reason of blindness disability or inibility to read or write assistance by a person of the voters choice. This also introduced some of the earlier legistlation defining a disabled citizen.

The Voting Accessibility for the Elderly and Handicapped Act of 1984 (VAEHA) was passed to improve access for handicapped and elderly individuals. However, states were left to set their own standards of {\em access}, and limited the disabled voters group to those with {\em physical dissabilities}. The VAEHA did, however, mandate `no notarization of medical certification shal be required of a voter with a disability with respect to an absentee ballot or application for such ballot.'

The 1990 American Disabilities Act (ADA), although not specific to voting rights, required that people with disabilities have access to basic public services, including the right to vote, however does not strictly require that polling locations are accessible. ADA did however extend the definition of disability to 
\begin{quote}``a person who has a physical or mental impairment that substantially limits one or more major life activities, a person who has a history or record of such an impairment, or a person who is perceived by others as having such an impairment.''
\end{quote}

The majority of federal laws passed to protect disabled voting have struggled to clearly define an representative range of dissabilities, and are often focused on access at physical polling locations; which is often expensive for states to implement. Additionally, state defined polocies often ignore rights to voting privacy, and exclude persons with multiple disabilities sighting that aiding technologies are not yet availible.


%Unfortunately most states claim to be unable to handle the costs associated with providing appropriate voting technologies. 

%much of this cost falls on the state to implement, and is co



%designate restrictions on physical polling locations. 


% federal law passed including disabled 

%American voting laws didn't 

%With the passage of the Amarican Dissabities Act (ADA) in

% Unfortunately, states found the ne

%Voting Accessibility for the Elderly and Handicapped Act of 1984 

\begin{itemize}
\item integrate history of disabled voters rights
\item[$\star$] TODO as necessary: ADA, HAVA, MOVE
\end{itemize}

\subsection{Integration with Local Elections}

Every state has their own requirements, deadlines, and transmission restrictions which the FVAP documents in a 'Voting Assistance Guide' distributed to potential UOCAVA voters. 

\section{Shortcomings of Current Practice}

\begin{itemize}
\item In the 2008 Post-Election UOCAVA Survey Report and Analysis 52\% of attempted votes were not counted because the ballots were late or never arrived.
\item 20\% of Americans with disabilities have said that they were unable to vote in presidential or congressional election due to barriers at or getting to the polls. (as of 2007)
\item ``ten states explicitly require a privacy waver if a voter uses fax or e-mail to return a voted ballot''
\item FVAP's {\em Voting Assistance Guide} is complicated and results in many failed registration attempts.
\item Entire process for UOCAVA voters generally can take up between 2 weeks and 2.5 months
\item Persons with manual dextarity impairments can prevent them from marking paper ballots.
\item Most practices for disabled voters forfit independent private voting places.
\end{itemize}

\subsection{Use of Communication/Internet}

The major motivations for use of internet and communication technologies for UOCAVA has been to address ballot transit time, and simplify voter registration. 
\begin{itemize}
\item OVF's streamlined website for FPCA in states that allow online registration
\end{itemize}

\subsection{Accessibility and Usability}
\subsection{Auditing}
\subsubsection{Current Practice}
\subsubsection{Digital vs. Physical}
\begin{itemize}
\item VVPAT provides voters with paper statements
\end{itemize}
\subsubsection{Risk-Limiting Audits}
Risk limiting audits use a public random auditing process to make an argument about the statistical confidence of a particular election result.
