\chapter{Remote Voting\ifdraft{ (Philip) (100\%)}{}}
\label{chapter:remote_voting}

\section{Rationale}
Remote voting is becoming increasingly common, prompted by the growing
and diverse needs of voters. Remote voting is used to enable overseas
citizens and military personnel to participate in elections, reduce
access related discrimination domestically, and decrease expensive
administrative overhead of polling locations.  In the United States,
fewer than 5\% of ballots cast in general elections during the 1980s
were cast before election day. By the 2012 general election, 31\% of
all ballots were cast early, and 17\% were cast by mail. The states of
Washington, Oregon, and most recently Colorado have entirely switched
over to all-mail voting.

\subsection{Accessibility}
According to the International Center for Disability Information and
the National Institute on Disability and Rehabilitation Research, 20\%
of Americans live with disabilities. The Voting Accessibility for the
Elderly and Handycapped Act of 1984 mandates that any person with a
disability may vote remotely without having to present medical
documentation, reducing the barriers to remote voting for those with
accessibility needs.

%The Help America Vote Act (HAVA) of 2002 requires that all polling places in elections for federal office, anywhere in the United States have at least one voting system  \ldots
%\ldots 20\% of U.S. adults with disabilities say they have been unable to vote in presidential or congressional elections due to barriers at or getting to the polls.

\subsection{UOCAVA}
In 1986, Congress enacted the Uniformed and Overseas Citizens Absentee
Voting Act (UOCAVA) to address the needs of citizens in the uniformed
services, merchant marines, and other overseas civilians. UOCAVA
mandates that these voters be able to register and vote remotely in
federal elections. It is difficult to calculate the exact number of
UOCAVA eligible voters, but \autoref{fig:uocava_populations} lists a
recent estimate of the total.

\todoacf{Ask Philip for context on these numbers. Statewise breakdown
  doesn't seem to add much, and the difference between what the two
  are counting is unclear}

\begin{figure}
\begin{center}
\begin{tabular}{l p{.3\textwidth} p{.3\textwidth}} % this can be made prettier
  {\bf State} &
  {\bf Overseas Voting Eligible \newline Population} \newline(McDonald 2009) &
  {\bf Overseas military and federal \newline civilian employees}
  \newline(US Census Bureau 2010)\\\hline\\
Texas & 11.05\% & 11.78\%\\
California & 9.78\% & 8.44\% \\
Florida & 9.09\% & 9.54\% \\
New York & 5.31\% & 4.12\% \\
Pennsylvania & 4.10\% & 3.12\% \\
Illinois & 4.03\% & 3.24\% \\
Ohio & 3.51\% & 3.07\% \\
Michigan & 3.29\% & 2.68\% \\
Georgia & 2.84\% & 3.83\% \\
Washington & 2.78\% & 2.77\% \\
North Carolina & 2.78\% & 2.91\% \\
Tennessee & 2.57\% & 2.81\% \\
Virginia & 2.51\% & 3.52\% \\\hline\\
Estimated Total & 4,972,217 & 1,042,523
\end{tabular}
\end{center}
\caption{Comparisons of American Overseas Population by State}
\label{fig:uocava_populations} % change to table?
\end{figure}

\subsection{Domestic Absentee}
Domestic absentee voters are those who vote early in-person, or cast
their votes cast by mail because they are unable or do not want to be
present at polling locations on election day (this excludes UOCAVA
voters, and voters in states that vote exclusively by
mail). 21,853,762 or 16.6\% of all votes cast in the 2012 US general
elections were from domestic absentee voters \cite{eac2012survey}. As
of 2015, 27 states allow voters to apply for an absentee ballot
without providing a justification, known as ``no-excuse absentee
voting''.

\todoacf{Take out this figure too?}
% proportions of domestic absentee voters. These states are frequently
% using domestic absentee to address concerns and costs associated with
% administering polling locations in low population density areas.

\begin{figure}
\begin{center}
\begin{tabular}{l c}
{\bf State} & {\bf Percent of Population}\\\hline\\
Colorado & 71.4\%\\
Arizona & 65.9\%\\
Montana & 57.5\%\\
Georgia & 48.8\%\\
Iowa & 43.1\%\\
California & 39.8\%\\
Hawaii & 36\%\\
North Dakota & 28.8\%\\
Florida & 26.8\%\\
Michigan & 26.4\%\\
Wyoming & 26.2\%\\
Maine & 25.5\%\\
Nebraska & 25.4\%\\
Idaho & 24.3\%\\
Ohio & 22.4\%\\
Wisconsin & 21.4\%\\
Vermont & 20.4\%\\
\end{tabular}
\end{center}
\caption{Votes Cast as Domestic Absentee 2012 General Election}
\label{fig:domestic_populations}
\end{figure}

\subsection{Expectations}

In 1952, a study by the American Political Science Association defined
ten voting rights necessary for members of the armed
services. Although initially focused on military voters, these rights
served as the basis for defining the expectations of all remote voters
through UOCAVA and other subsequent legislation. These rights are:

\begin{enumerate}
  \item To vote without registering in person.
  \item To vote without paying a poll tax.
  \item To vote without meeting unreasonable residence requirements.
  \item To vote without meeting unreasonable literacy and educational
    requirements.
  \item To use the Federal postcard application for a ballot.
  \item To receive ballots for primary and general elections in time to vote.
  \item To be protected in the free exercise of their voting rights.
  \item To receive essential information concerning candidates and issues.
  \item To receive essential information concerning the methods by which the
    right to vote may be exercised.
  \item To receive essential information on the duty of `citizens in uniform'
    to defend our democratic institutions by using, rather than ignoring, their
    voting rights.
\end{enumerate}

\todoacf{This list seems anachronistic; is there an up-to-date version
  that's more comprehensive?}

The Help America Vote Act of 2002 (HAVA) defines additional mandatory
minimum standards for states that specifically address voter access
concerns raised in the 2000 general election. This act requires that
elections support participation by non-English speakers and disabled
persons by providing the same opportunities for access and
participation, including voter privacy and independence. Additionally,
HAVA mandates that voters whose eligibility is questioned must be
permitted to cast a provisional ballot to be reviewed later by
election officials.

% \todoacf{See todo above; this seems out of place}
% Finally it is most important to include the expectation that votes cast by
% registered voters are counted correctly while preserving privacy. As will
% be mentioned later in this chapter, there are many examples where this
% isn't the case.

%HAVA also put stricter auditing requirements in place, requiring voting technologies to produce a Verifiable Voter Paper Audit Trail (VVPAT); while preserving the privacy of the voter and the secrecy of the cast ballot.


\section{History}
\subsection{Armed Forces Voting}
Before the American Civil War, US citizens primarily voted in their
places of residence, and many states legally barred the casting of
votes from outside state borders. There was little effort from any
state to accommodate absentee voting. However, in 1864, with the Civil
War displacing soldiers from their residences, Lincoln's re-election
was at risk. With much lobbying on behalf of the Republican Party (and
opposition from the Democratic Party), nineteen Union states adopted
absentee voting procedures for military voters in time for the
election. Since the motivation for passing these laws was to secure
Lincoln's re-election rather than permanently expand voting access,
many absentee military voter laws were treated as temporary and
repealed after the war.

In 1918, the US War Department decided that it was not ready to
support the military vote. World War I had displaced a large number of
eligible voters, and military units were rarely composed of citizens
of the same state. \todoacf{This seems incomplete... let's flesh out
  or remove}

As in the Civil War, World War II inspired another push for the
military vote in hopes of supporting the re-election of the
presidential incumbent. This introduced the Soldier Voting Act (1942)
which, although passed too late for the presidential election, gave
military personnel absentee voting rights for federal elections during
times of war without subjugation to voting tax or postage
costs. According to the act, all overseas voting would be regulated at
the federal level and implemented at the state level. However, by
1944, the state mandate to support military absentee voting was
weakened from a requirement to a recommendation.

\subsection{Remote Civilian Voting}

Progress for civilian absentee voting lagged the armed forces. In 1896, states
began introducing civilian absentee voting legislation. By 1924 only three
states in the union had no absentee voting legislation, but all states had
different laws and restrictions. Major progress on this front wasn't made until
federal voting laws were passed that enabled both civilian military votes.
The Voting Assistance Act of 1955 was the first to federally
combine voting policy recommendations for overseas civilian government
employees with military.

With lobbying from sympathetic groups and a growing population of overseas
civilians, Overseas Citizens Voting Rights Act passed in 1974 to extent the
vote to citizens regardless of their intentions to return to the United
States.

In 1986, the Voting Assistance Act was amended to include individuals
temporarily living outside the United States. By this time, combatant attitudes
towards overseas votes had finally settled, and the Uniformed and Overseas
Citizens Absentee Voting Act (UOCAVA) was passed. This act combined and
replaced the Federal Voting Assistance Act, and finally made supporting the
overseas absentee ballot a requirement.

\subsection{Disabled Civilian Voting}
The Voting Rights Act (VRA) of 1965 was the first legislation to
enfranchise voters with disabilities. The VRA granted voters who require
assistance to vote by reason of blindness disability or inability to read or
write assistance by a person of the voters choice. This also introduced some of
the earlier legislation defining a disabled citizen.

The Voting Accessibility for the Elderly and Handicapped Act of 1984 (VAEHA)
was passed to improve access for handicapped and elderly individuals. However,
states were left to set their own standards of access, and limited the disabled
voters group to those with {\em physical disabilities}. The VAEHA did, however,
mandate `no notarization of medical certification shall be required of a voter
with a disability with respect to an absentee ballot or application for such
ballot.'

The 1990 American Disabilities Act (ADA), required that people with
disabilities have access to basic public services, including the right to vote.
However, this law does not strictly require that polling locations are
accessible. ADA did however extend the definition of disability to:
\begin{quote}
``a person who has a physical or mental impairment that substantially limits
one or more major life activities, a person who has a history or record of such
an impairment, or a person who is perceived by others as having such an
impairment.''
\end{quote}

The majority of federal laws passed to protect voting rights of disabled
citizens have struggled
to clearly define an representative range of disabilities, and are often
focused on access at physical polling locations; which is expensive for states
to implement. Additionally, state defined policies often ignore rights to
voting privacy, and exclude persons with multiple disabilities sighting that
aiding technologies are not yet available.

%Unfortunately most states claim to be unable to handle the costs associated with providing appropriate voting technologies.
%much of this cost falls on the state to implement, and is co
%designate restrictions on physical polling locations.
%With the passage of the American Disabilities Act (ADA) in
%Voting Accessibility for the Elderly and Handicapped Act of 1984
% VVPAT
\subsection{Modern Remote Voting}

In 2002 Congress passed the Help America Vote Act (HAVA) partially as response
to problems found in gathering, counting, and auditing votes in the 2000
presidential election. HAVA required that all polling places in elections for
federal office, anywhere in the United States have at least one voting system
enabling disabled voters, addressing some accessibility concerns.

HAVA also attempted to address concerns raised by the large number of rejected
critical ballots in 2000, due to an inability to sufficiently audit ballots.
This act advises voting technologies to produce a Verifiable Voter Paper Audit
Trail (VVPAT); while preserving the privacy of the voter and the secrecy of the
cast ballot. The Federal Election Assistance Commission (EAC) was created to
oversee the development of new voting machine standards, and released the
Voluntary Voting System Guidelines to aid in this transition.

The Military and Overseas Voter Empowerment (MOVE) Act was introduced in 2009
attempted to address specific barriers to overseas voter participation. The
MOVE Act required states to transmit absentee ballots at least 45 days prior to
an election states to make all registration material available electronically,
UOCAVA voters to register for each election cycle (rather than every two
election cycles), and removed notarisation requirements on all election
material.

%\subsection{Integration with Local Elections}

%Every state has their own requirements, deadlines, and transmission restrictions which the FVAP documents in a 'Voting Assistance Guide' distributed to potential UOCAVA voters.
%%%%%%%%%%%%%%%%%%%%%%%%%%%%%%%%%%%

Legislation on absentee voting is a slow moving event. Existing voter
rights regulations are enforced at a state level and are hampered by local
political attitudes. In 2010 Uniform Law Commission oversaw drafting of the
Uniform Military Services and Overseas Civilian Absentee Voters Act (UMOVA);
designed to identify and standardize the important protections and benefits
found in UOCAVA and MOVE to state and local elections. As of April 2015,
fourteen states and the District of Columbia have enacted UMOVA.

\section{Shortcomings of Current Practice}

There are many concerns with current election practices that an E2EIV system
could help to mitigate. The topics listed below draw from specific concerns
that have a large impact on remote voting participants.

\subsection{Use of Communication Technologies}
The majority of remote voting is performed by `vote-by-mail' which is
subject to many inherent faults that only become more problematic for UOCAVA's
overseas voters. The 2008 Post-Election UOCAVA Survey Report and Analysis
found that 52\% of attempted UOCAVA votes were not counted, due to problems
in the mail delivery process. Additionally, maintaining correct voter
registration information for those frequently changing their geographic
location abroad is expensive and prone to error.
% \begin{quote} {\em \ldots want to make comment about Florida, but concerned that it may be taken as political}\end{quote}

%%%%%%%%%%%%%%%%%%%%%%%%%%%%%%%%%%%
% Entire process for UOCAVA voters generally can take up between 2 weeks and 2.5 months
%%%%%%%%%%%%%%%%%%%%%%%%%%%%%%%%%%%

Due to the many states' diverse registration practices, the Federal Voting
Assistance Program provides an accompanying Voting Assistance Guide to
registration forms. Unfortunately, this is very large and difficult to follow.
For states without a streamlined online registration (hosted by OVF), this has
resulted in many failed voter registration attempts.

%%%%%%%%%%%%%%%%%%%%%%%%%%%%%%%%%%%
%\begin{itemize}
%\item The major motivations for use of internet and communication technologies for UOCAVA has been to address ballot transit time, and simplify voter registration.
%\item OVF's streamlined website for FPCA in states that allow online registration
%\end{itemize}
%%%%%%%%%%%%%%%%%%%%%%%%%%%%%%%%%%%

\subsection{Accessibility and Usability}

In 2007, 20\% of Americans with disabilities said they were unable to vote in
presidential or congressional election due to barriers at or getting to the
polls. This is frequently a consequence of the voting technologies used and the
physical location of polling places. In the 2000 presidential election, 56\% of
randomly sampled polling places in the United States had at least one identified
impediments for disabled voters.

Disabled persons often forfeit privacy to appointed aids,% in person and remote voting
often as result of insufficient technology support at polling locations.
Those with dexterity impairments often have problems with handling and marking
paper ballots (at polling places and at
home).


\subsection{Auditing}
Although it is believed that voter fraud is fairly uncommon, it is a major
concern in a bipartisan system. Unfortunately, it is very difficult to detect
without verification technologies. Often, policies intended to reduce fraud
or protect identities result in a higher rate of rejected ballots. According
to the 2012 Election Administration and Voting Survey, 17.6\% of absentee
ballots were rejected due to non-matching signatures.

% Ten states explicitly require a privacy waver if a voter uses fax or e-mail to return a voted ballot''
Privacy of vote is a strong expectation of a fair voting system, that is often
left unaddressed. Most practices for UOCAVA voters forfeit independent or
private voting, with several states even requiring a voter privacy waver to be
signed for remote ballots. In several jurisdictions, UOCAVA votes are not
counted until it is determined that they may sway the election.
\footnote{citation needed}

% perhaps nice to have cost analysis

%%%%%%%%%%%%%%%%%%%%%%%%%%%%%%%%%%%
%\subsubsection{Digital vs. Physical}
%\subsubsection{Risk-Limiting Audits}
%Risk limiting audits use a public random auditing process to make a strong
%argument about the statistical confidence of a particular election result. 
