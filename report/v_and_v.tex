\chapter{Verification and Validation\ifdraft{ (Joe K./Dan/Adam) (20\%)}{}}
\label{chapter:v_and_v}

\section{Requirements and Scenarios}
\section{Methodology}

\subsection{Engineering Methodology}

\todoacf{Introduce context with NIST, ISO standards for methodology?}

Sound engineering practices are the foundation for building reliable
and secure software of any type. This is particularly true for
critical systems which must be trusted to perform important tasks
correctly, and where the consequences for failure threaten lives,
political system integrity, and property.

This section introduces methodologies that help reduce errors and
improve confidence in software development. We avoid discussion of
particular technologies except to illustrate our recommended
methodologies with examples in pracice.

\subsubsection{Version Control}

Version control systems (VCS) manage changes between the versions of a
project as it evolves during the course of development. Revision
control is the preferred way to share software artifacts across a
team, but all software projects, even those developed by teams as
small as one person should use VCS.

In general, a developer uses the VCS first to ``check out'' the files
comprising a project into her ``working copy''. Then, after making
changes to those files, the developer ``checks in'' or ``commits'' the
changes to the VCS. After committing, those changes are available for
other developers to integrate into their working copies.

When different sets of changes have been made by multiple developers,
the VCS can merge those changes either automatically or by using
developer input to ensure the project remains consistent. The ability
to merge changes is critical for teams of developers who work
concurrently on a single project, and is the reason that file sharing
tools like Dropbox or Google Drive are an inadequate substitute for a
VCS.

VCS is particularly important for projects that must be audited. An
entry to a project log is created every time a developer commits their
work. Any file in the project can be inspected to show its provenance,
even down to the level of which line was committed by which developer
on what date. Some VCS tools also optionally allow for commits to be
cryptographically signed, offering assurance that, for example, the
changes have been audited by a trusted authority before being
integrated into the project~\cite{chacon2014pro}.

Moving from simple file storage to VCS is a tremendous improvement for
development, but poor use of VCS can negate many of the potential
benefits. For example, the log built from the commits of developers is
of less use to auditors if the changes in each commit are not clearly
associated with a particular new feature or bug fix. Likewise if
developers commit changes in a broken state, other developers'
productivity suffers and it becomes more difficult later to isolate
which commit introduced a bug. Each VCS supports multiple workflow
practices that should be adopted in order to limit these problems and
get the most benefit from VCS
use~\cite{atlassianworkflow}\cite{pilato2008version}.

\subsubsection{Issue Tracking}

\subsubsection{Continuous Integration and Testing}

\subsubsection{Code Review}

\subsubsection{Release Management \& Lifecycle}

\subsubsection{First-Class Documentation}

\subsubsection{Automation}

\section{Technologies}
\section{Interpreting Results}
