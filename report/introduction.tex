\chapter{Introduction\ifdraft{ (Joe K./Susan) (99\%)}{}}
\label{chapter:introduction}

Societies have conducted elections for thousands of years, but
technologies used to cast and tally votes have varied and evolved
tremendously over that time. In 2015 much of our communication takes
place online, and many people want elections to follow this
trend. Overseas voters are particularly interested in an online
approach, as their voting process often requires extraordinary effort.

In March 2013, Overseas Vote Foundation’s President and CEO began a
discussion with a small group of election integrity advocates to
explore what they would do if faced with the challenge of defining an
Internet voting system. Despite their concerns about the security of
Internet voting efforts to date, they agreed that addressing this
challenge is extremely important. 

Obstacles to Internet voting range from the risk of hacking to
political and policy considerations. Scientists, federal agencies,
cybersecurity specialists, and certain organized activists in the
U.S. have urged against exposing the ballots of the most powerful
nation on Earth to the seemingly endless range of threats that lurk on
today’s Internet. Election integrity advocates say that secure,
tested, certified remote voting systems are not
available. Cybersecurity specialists do not consider online ballot
return systems secure, and no Internet voting systems have been
certified at the time of this writing. Many election administrators
have turned to email as a method for ballot transmission, although it
does not provide the benefits that a secure, full-featured voting
system would. Email is  not secure, yet election administrators and
voters regularly use it to transmit ballots because no viable
alternatives are available. 

Existing vendors of Internet voting technologies, whose systems are
neither tested nor certified, would like to openly market and sell
their systems within the U.S. without meeting resistance from election
integrity advocates. A lack of agreement on how to proceed, a long
history of mediocre attempts, ongoing animosity among stakeholders,
and a general lack of research on the current questions have soured
many people on the hope of using the Internet for voting.

Even so, election officials often want to consider Internet-based
technologies so they can serve their constituents in modern and
efficient ways. In the current economic climate, investment in
election innovation is rare. Election officials in the U.S. are stuck
with old technology. They devote scarce resources to support outdated
voting systems, and they lack the means to certify new ones that would
modernize the voting process.

In order to serve all voters, including overseas citizens, military
members, and people with disabilities, new and better ways to use
technology for voting are needed.

%~~~~~~~~~~~~~~~~~~~~~~~~~~~~~~~~~~~~~~~~~~~~~~~~~~~~~~~~~~~~~~~~~~~~~
\subsection{A Better Technology}
\label{sec:better-technology}

In 1981, security researcher David Chaum published an influential
paper describing several applications of \emph{public-key encryption},
a new technology at that time. He suggested a way to use public-key
encryption to make a set of ballots anonymous while allowing any
observer to verify the accuracy of the tally. This was the first step
in the development of \emph{end-to-end verifiable} (E2E-V) election
technologies.

Over the following decades, numerous researchers published papers
describing and refining election systems that use E2E-V
technologies. A unique trait of E2E-V election systems is that they
allow voters to verify that their own votes have been accurately
recorded. At the same time, they allow any observer to verify that all
recorded votes have been accurately counted.

Developers designed many E2E-V election systems for a variety of
settings---in-person and remote voting, paper-based and electronic
voting, simple majority and instant run-off, etc. Early designs were
cumbersome, while more recent E2E-V election systems are easier to
use.

Election professionals appreciate the benefits that end-to-end
verifiability can bring to election systems. Some are taking early
steps toward large-scale deployment of E2E-V election technologies for
in-person voting systems. However, it is not yet clear whether the
benefits of E2E-V technologies can adequately address the legitimate
security concerns inherent to Internet voting.

%~~~~~~~~~~~~~~~~~~~~~~~~~~~~~~~~~~~~~~~~~~~~~~~~~~~~~~~~~~~~~~~~~~~~~
\subsection{A Proposed Study and Objectives: The E2E-VIV Project}
\label{sec:proposed-study-and-objectives}

Overseas Vote Foundation (OVF),\footnote{Overseas Vote Foundation has
  since been renamed as U.S. Vote Foundation, and their overseas voter
  program is now referred to as Overseas Vote, an initiative of
  U.S. Vote Foundation. For the purposes of this project and report,
  however, we continue to refer to the organization as Overseas Vote
  Foundation (OVF).} the leading nonpartisan, nonprofit organization
dedicated to overseas and military voter participation, developed and
wrote the proposal for this project. On December 19, 2013, it was
launched as the End-to-End Verifiable Internet Voting: Specification
and Feasibility Assessment Study (E2E-VIV Project). The Democracy
Fund, a Washington D.C.-based philanthropic organization whose stated
objective is to ``…invest in organizations working to ensure that our
political system is responsive to the priorities of the American
public and has the capacity to meet the greatest challenges facing our
country'', funded the project.

The proposed project had these goals:
\begin{itemize}
\item Convene a diverse group of stakeholders to constructively
  address open questions about how to vote securely online;
\item Define a ``whole-product'' specification for a secure End-to-End
  Verifiable Internet Voting (E2E-VIV) system;
\item Define a set of testing specifications to demonstrate E2E-VIV
  security;
\item Provide a set of guidelines for system usability, accessibility, and testing; 
\item Produce a report that examines the feasibility of creating a
  secure E2E-VIV system;
\item Include consideration of legal and administrative challenges,
  and ballot secrecy, privacy, and confidentiality; and
\item Release all information produced to the public. 
\end{itemize}

In addition, an unofficial objective of the project was to ``change
the conversation'' about Internet voting by starting a new,
constructive dialogue among computer scientists, usability experts,
election integrity advocates, and local election officials from key
counties around the U.S.

For the purposes of the E2E-VIV project, \emph{end-to-end
  verifiable} is defined as follows. First, every voter can check
that his or her ballot is cast and recorded as he or she
intended. Second, anyone can check that the system has accurately
tallied all of the recorded ballots.

Some concerns about Internet voting are justified. Internet voting
takes all the problems with current remote voting systems and adds to
them all the security vulnerabilities of the Internet. No participant
in this project discounts these concerns or views E2E-V as a magic fix
that makes the Internet secure. We believe that E2E-V properties,
which apply even when voters use potentially untrusted devices like
personal computers and transmit votes over an untrusted medium like
the Internet, are important for Internet voting. The E2E-VIV Project
does not attempt to make the Internet secure. Instead, it examines
whether E2E-V can effectively counter the risks of Internet voting
while bringing substantial new benefits not found in today’s voting
systems.

The E2E-VIV Project also aims to clear up a misunderstanding in the
U.S. election community. Our country’s scientists are not against
technology advancements, nor are they inherently at odds with election
officials who seek technology improvements to meet their
administrative challenges. Instead, scientists doubt claims of
security regarding existing systems that are not publicly tested or
vetted. The scientific leaders on this project have often pointed out
security vulnerabilities in past systems; however, the E2E-VIV Project
has led them to agree that if Internet voting can happen, it must be
in a system that takes advantage of end-to-end verifiability and
auditability.

%~~~~~~~~~~~~~~~~~~~~~~~~~~~~~~~~~~~~~~~~~~~~~~~~~~~~~~~~~~~~~~~~~~~~~
\subsection{Shared Goals}
\label{sec:shared-goals}

Election officials and scientists involved in elections share the same
overall goals: that voting systems provide accurate results, protect
voters’ privacy, are easy to use for all voters, and are robust
against both accidental and intentional disruptions. Additionally,
election officials must be able to show the public that their voting
systems achieve these goals.

This project shows that many scientists care deeply about addressing
the needs of election officials as they serve remote voters. These
scientists are highly motivated to help when given an opportunity, and
they would like to work with election administrators to examine the
possibilities in this realm. This project also shows that scientists
could improve their understanding of the practical issues that
election officials face and find better ways to collaborate and
communicate with them to develop technical solutions.

%~~~~~~~~~~~~~~~~~~~~~~~~~~~~~~~~~~~~~~~~~~~~~~~~~~~~~~~~~~~~~~~~~~~~~
\subsection{Success}
\label{sec:success}

We hope, that in addition to the concrete outcomes of this project,
our research and testing-based approach to examining Internet voting
will move beyond the current stalemate and stimulate election
development overall. The election industry continues to operate in a
traditional way, with only a few vendors able to survive despite
demand to move away from outdated, expensive, hardware-oriented
solutions. A successful outcome of this project would:

\begin{itemize}
\item Help build a specification for a usable, secure E2E-VIV technology; 
\item Determine the strengths and weaknesses of such a system; and
\item Identify reasons to pursue or not pursue this approach.
\end{itemize}

Ideally, the resulting specification would be:

\begin{itemize}
\item Supported by the vast majority of the contributors, including
  the technical, usability, testing, and research teams; 
\item Endorsed by the vast majority of the E2E-VIV Project advisory
  council; and  
\item Endorsed by the major stakeholders in elections administration
  as represented by the project’s local election officials. 
\end{itemize}

It was acknowledged from the outset that if the project team
determined that current techniques were weak and should not be
pursued, this would also be an outcome with many useful implications. 

The E2E-VIV Project hopes to receive support and endorsement from many
members of the election integrity community, as represented by key
members of the Election Verification Network, the Verified Voting
Foundation and beyond. We intend for the specification to fulfill the
following requirements:

\paragraph{Independent Implementation}
The specification must have sufficient detail and clarity that an
independent party can implement the election system without having
extensive dialogue with project participants.

\paragraph{Independent Validation}
It must be possible for a moderately proficient IT expert to
objectively determine, in a reasonable time frame and at reasonable
cost, whether a constructed election system fulfills the
specification.

\paragraph{Evidence-Based Decisions}
Every decision made in the crafting of the specification must be
objectively justifiable and the evidence for the decision must be
traceable.

%~~~~~~~~~~~~~~~~~~~~~~~~~~~~~~~~~~~~~~~~~~~~~~~~~~~~~~~~~~~~~~~~~~~~~
\subsection{Scope}
\label{sec:scope}

The original project was limited to system specification and testing
only. The project participants did not envision system development in
Phase I beyond mock-ups to help test usability. That changed, however,
when OVF engaged Galois, Inc. to conduct the technical project
management. Galois’s management offered to donate engineering time to
the project to build ``demonstrators'' that would be used to prove the
concepts of E2E-V and to further examine security and usability. This
additional work broadened the scope of the project.\footnote{The
  Galois engineers developed a set of rigorous engineering
  demonstrators that can be refined to fit into a working election
  system. Third parties can perform independent verification and
  validation of these demonstrators.}

The demonstrators developed using Galois research and development
funding are:

\begin{itemize}
\item Developed in a completely transparent and public manner within
  the Galois GitHub Organization;
\item Cross-referenced, and thus traceable to and from, all
  specification aspects (from domain models to behavioral design
  specifications);
\item Replicated into the E2E-VIV GitHub Organization;
\item Licensed under either a mainstream Open Source license with a
  strong community or an alternative license tuned to the elections
  community.
\end{itemize}

%=====================================================================
\section{Project Team}
\label{sec:people}

The E2E-VIV Project provided an opportunity to combine the abilities,
knowledge, experience, and expertise of a diverse group of
technologists, computer scientists, and election officials involved in
election integrity. Technical, usability, testing, and local election
official sub-teams were created to make communication easier. The
project also established an advisory council to open communication
with interested members of the election community.

OVF, as the official grantee, was responsible for the overall project
conception, proposal development, presentations, communications,
management, team recruitment, contractual obligations, public
relations, events, and budgeting. 

Galois, Inc. provided the technical and engineering project
management, wrote much of and edited the final report, and facilitated
the communication and decision-making of the team.\footnote{See the
  \emph{e2eviv} GitHub repository to best understand exactly what
  Galois's contributions are.}

%~~~~~~~~~~~~~~~~~~~~~~~~~~~~~~~~~~~~~~~~~~~~~~~~~~~~~~~~~~~~~~~~~~~~~
\subsection{Team Members}
\label{sec:team-members}

\begin{center}
\emph{Layout and editorial note: We will be turning this section into
  a more attractive illustration to enumerate participants in the
  project. It will denote exactly who participated and in what
  fashion.  No team member will be listed without their consent.}
\end{center}
  
% Joe, please see my email on this topic.
\textbf{Project Manager:} Susan Dzieduszycka-Suinat, Overseas Vote Foundation

\textbf{Lead Technical Project Manager:} Dr. Joseph Kiniry, Galois

\textbf{Technical Team}

Dr. Josh Benaloh
Senior Cryptographer, Microsoft Research
 
Dr. David R. Jefferson
Lawrence Livermore National Laboratory
 
Dr. Doug W. Jones
Associate Professor, Department of Computer Science, University of Iowa
 
Dr. Aggelos Kiayias
Associate Professor, Computer Science and Engineering, University of Connecticut
 
Dr. Olivier Pereira
Professor, Institute of Information and Communication Technologies, Electronics and Applied Mathematics, Ecole Polytechnique de Louvain
 
Dr. Poorvi Vora
Associate Professor, Department of Computer Science, The George Washington University
 
Dr. David Wagner
Professor, EECS Computer Science Division, University of California Berkeley
 
Dr. Dan Wallach
Professor, Department of Computer Science, Rice University
 
\textbf{Usability Team}

\begin{itemize}
\item Keith Instone, User Experience Consultant
\item Morgan Miller, Usability Analyst, Experience Lab
\item Dr. Judith Murray, Research Consultant
\end{itemize}

\textbf{Election Auditing}

Dr. Philip Stark
Professor and Chair of Statistics, University of California Berkeley
 
\textbf{Testing Team}

Dr. Duncan Buell
Professor of Computer Science and Engineering, University of South Carolina
 
Andrew Regenscheid
Mathematician, National Institute of Standards and Technology
 
\textbf{Advisory Council}

Dr. Ben Adida
 
Dr. Michael Clarkson
Assistant Professor of Computer Science, The George Washington University
 
Dr. J. Alex Halderman
Assistant Professor of Computer Science and Engineering, University of Michigan
 
Candice Hoke
Professor of Law, Cleveland State University
 
Dr. Ron Rivest
Vannevar Bush Professor of Computer Science, Massachusetts Institute of Technology
 
Noel Runyan
Primary Consultant, Personal Data Systems
 
Dr. Peter Ryan
Professor in Applied Security, University of Luxembourg
 
Dr. Barbara Simons
Research Staff Member, IBM Research (retired)
 
Dr. Vanessa Teague
Research Fellow, Department of Computing and Information Systems, University of Melbourne
 
John Wack
Voting Systems Standards, National Institute of Standards and Technology
 
Dr. Filip Zagorski
Assistant Professor of Computer Science, Wroclaw University of Technology
 
\textbf{Local Election Officials}

Lori Augina
Director of Elections, Washington State, Secretary of State

Rachel Bohman
Former Hennepin County Elections Manager (Minnesota)

Judd Choate
Director of Elections, Colorado, Secretary of State

Dana Debeauvoir
Travis County Clerk (Texas)
 
Mark Earley
Voting Systems Manager, Leon County (Florida)
 
Dean Logan
Los Angeles Registrar-Recorder/County Clerk (California)

Stuart Holmes
Election Information Systems Supervisor, Office of the Secretary of State (Washington)
 
Dr. Lois H. Neuman
Chair, Board of Supervisors of Elections, City of Rockville (Maryland)
 
Roman Montoya
Deputy County Clerk, Bernalillo County (New Mexico)
 
Tammy Patrick
Senior Advisor to the Democracy Project, Bipartisan Policy Center and Former Federal Compliance Officer Maricopa County (Arizona)
 
\textbf{Overseas Vote Foundation Support Team}

Susan Dzieduszycka-Suinat
President and CEO
 
Paul McGuire
Legal Counsel and Secretary of the Board
 
Richard Vogt
Treasurer and Chief Financial Officer

Capstone Project Team, Carnegie Mellon University, Heinz College,
School of Information Systems \& Management; Master of Information
Systems Management and Master of Science in Information Security
Policy and Management: in early 2014, a Capstone Team was assigned to
the project team to assist on the Comparative Analysis of E2E systems.

%~~~~~~~~~~~~~~~~~~~~~~~~~~~~~~~~~~~~~~~~~~~~~~~~~~~~~~~~~~~~~~~~~~~~~
\subsection{Stakeholder Groups}
\label{sec:stakeholder-groups}

Although not on the official project team, several communities
relevant to the E2E-VIV Project have offered essential input. These
communities include:

\paragraph{Election Verification Advocates}
Election verification advocates are numerous, well-informed, and
strongly connected. They care deeply about election integrity and
verifiability. Internet voting is a high-priority issue for many of
them.

Election verification advocates tend to be skeptical about Internet
voting. This is because several elections vendors have developed
Internet voting products that are proprietary, closed-source, and not
verifiable. Their products have never had a public audit. Many vendors
nonetheless make claims about the security of their products, and
election verification advocates reject these claims. Some vendors
recommend that elections be outsourced entirely to their own
companies---a condition that will never be acceptable to the election
verification community, even for an E2E-VIV system.

Some election integrity advocates are in favor of verifiable Internet
voting, others are adamantly against Internet voting of any kind, but
the bulk of them are undecided. That majority recognizes that
significant scientific and engineering challenges must be met to
design and develop an Internet voting system. They recognize that the
decision to deploy such a system is very much a subjective, political
one. In some contexts, such as deciding the winner on a reality TV
show, some advocates view using a non-verifiable, outsourced election
apparatus as an acceptable choice. However, for government elections
of any value, such an option is unacceptable to virtually every
advocate.

Being fully transparent with---and listening to the feedback
from---the election verification advocate community is mandatory. If
the evidence presented in this report does not sway the bulk of that
community, the pursuit of any next phase in this project will be
contentious.

\paragraph{Standards Bodies} 
Perhaps surprisingly, little national or international election
standardization exists. An effort to standardize data interchange
formats began about a decade ago and eventually collapsed after
agreement on one small standard. This first effort failed for many
reasons. Vendors lobby against, and are uninterested in,
interoperability. The Election Assistance Commission (EAC) issued
Voluntary Voting System Guidelines (VVSG), but these were not geared
toward a component-based approach to system design; since devices
could not be plugged together, there was little cause for defining
interfaces and data file formats. Also, the election research
community did not accept that first effort at standardization.

In 2015, the situation changed with the rebirth of the Institute of
Electrical and Electronics Engineers (IEEE) 1622 committee and its
focus on elections. The IEEE Voting System Standards Committee 1622
(VSSC/1622) creates standards and guidelines around a common data
format for election data. The goal is that future election equipment
used in U.S. elections and abroad can interoperate more easily. The
Voting System Standards Committee intends for their standards and
guidelines to be required in future versions of the VVSG.

Many of the top researchers, election advocates, and election
officials in the world participate in this committee. Representatives
from the major election systems vendors also participate, because they
recognize that interoperability will be mandated by future versions of
the VVSG.

Standards are critical to any future work on E2E-VIV systems for
several reasons. First, given the compositional nature of most E2E-V
systems’ designs, it is likely that different subsystems will be
created and supported by different organizations. Second, in order to
enable arbitrary third party verification, clearly defined common data
formats must be used. Third, at some point in the future, if E2E-VIV
systems are accepted in the mainstream, the EAC must certify them. The
EAC, National Institute of Standards and Technology (NIST), and IEEE
standards must recognize their core capabilities, subsystems,
interfaces, and what constitutes legitimate evidence of correctness
and security. A standard means to document these aspects and evidence
is necessary to establish a sound, accurate, and expedited
certification process.

\paragraph{Vendors}
Two types of vendor are relevant to this project: (1)~existing vendors
of proprietary election systems, and (2)~future vendors that support
open source election systems.

Existing vendors may be interested in the results of this project,
particularly if it moves to a second phase that focuses on the design
and development of an open source system. Those that have an existing
non-E2E-V Internet voting product (e.g., Scytl, Dominion, and Everyone
Counts) could benefit from this work if they take on the tough
challenges of verifiability and usability and tackle them
accordingly. Any increase in attention on—or any hint of support from
the verifiable elections activist community about—Internet voting
aligns with their marketing goals.

Vendors can use the requirements contained in this project in various
ways. Ideally, they will actively and positively absorb the
recommendations. Any investment towards making their commercial
systems end-to-end secure and verifiable, as long as there is publicly
available evidence of such improvements, would be welcomed by the bulk
of the research and activist communities. On the other hand, it is
also possible that vendors will simply use these requirements as a
checklist, claiming that their systems are E2E-V but providing no
evidence. Only time will tell which path each existing vendor chooses
to take.
 
The other type of vendor that is relevant to this project, supporting
open source election systems, does not yet exist. We expect such
commercial support organizations to emerge to support any E2E-VIV
system that occurs as a result of this project. These organizations
would not need to have expertise in the underlying cryptography,
formal methods, or usability and accessibility. They would rely on the
high-assurance development method described in this report, which
produces validation and verification artifacts that generate evidence
of their correctness and security properties. Additionally, modern
formal system specification languages support traceability from
requirements to evidence.

Any future mission-critical E2E-VIV system will be robust to any
integration, customization, or evolution work by these vendors. That
is, systems developed according to the requirements and
recommendations in this report cannot be accidentally or purposefully
broken without third parties detecting a certification failure. Only
these complete, consistent, traceable, evidence-generating artifacts
make it possible for such new support vendors to emerge.

\paragraph{Hackers and Hacktivists} 
Hackers and election hacktivists are an important audience for this
project. Hackers, constructively or destructively, help make open
source systems more secure. Hacktivists catalyze movements of
like-minded technical individuals. Across the world, their attention
has brought to light the flaws of insecure and incorrect electronic
voting systems.

Within these sub-communities, it is an undisputed precept that the
design and development of a secure system used for public good must be
open to critique and improvements from the public. Leveraging that
attention is especially valuable for nationally critical systems like
public elections. Consequently, direct engagement with these groups,
while potentially tempestuous at times, is wise and will have many
benefits.

\paragraph{Election Officials} 
Local elections officials (LEOs) are the core stakeholders in the
design and deployment of E2E-VIV technologies. There are over 10,000
election jurisdictions in the United States. That means there are over
10,000 different ways that elections can be run. Smoothly integrating
with existing election processes is important. Basic issues like
common data formats are simple technical challenges that have
straightforward, though potentially politically delicate, solutions.

The enormous variety of technologies, large and small, at the federal,
state, and local levels make deploying any new election technology
difficult. Traditional IT practices for the design, development, and
maintenance of election software cannot work in this setting. No
vendor can support 10,000 variants of a single code base to satisfy
10,000 clients, each with slightly different requirements. Some of the
software engineering recommendations in this report, particularly
those that touch upon feature modeling and software product lines, are
directly relevant to the elections setting.

More subtle challenges, such as ensuring that election officials who
are uncomfortable with technology can deploy and support a E2E-VIV
system for their overseas, military or disabled voters, have little to
do with technical solutions. These problems and solutions have more
social, political, and psychological roots, and thus require soft
solutions.

Tools such as documentation and tutorials, webcasts and screencasts,
reports, and demonstration software have a limited reach. Open Source
projects often have friendly online forums to welcome and guide
newcomers, as well as regularly scheduled and community-organized
developer and user conferences. We hope that kind of open, active and
supportive community will evolve around E2E-VIV.

\paragraph{Voters} 
Despite the importance of all of the aforementioned stakeholders, the
most important stakeholders---the ones that hold veto power over the
wide-scale deployment of E2E-VIV technologies---are the voters. Many
voters want flexible, comfortable voting that is not tied to a
particular polling place. However, they may not realize the
implications of this desire, particularly with regard to security and
privacy. Educating voters about the challenges of Internet voting is
important, but it is not possible to make every voter aware of the
usefulness of, and critical need for, E2E-V election systems.

Based upon early usability studies of E2E-V election systems, we can
expect only a small fraction of voters to understand and use the
verifiability features of E2E-VIV systems, even if the user experience
of these systems is well-designed. Voters’ trust in their election
apparatus, and how their voting experience affects their trust in
their government, is paramount. Consequently, changes in elections
that may impact trust are difficult to make.

Voters are sensitive to changes that are bluntly visible, like
electronic pollbooks and electronic voting machines. The past decade,
with its introduction of electronic voting machines and the subsequent
outcry for a return to a “paper trail”, has taught some voters not to
trust election technology. A vocal minority may misunderstand changes
and be hostile or alarmed. Public reactions are unpredictable, as
evidenced by the introduction of voter ID laws across the U.S. and the
alarm raised by the use of email for ballot return (even in
extraordinary circumstances).

Finally, and perhaps most critically, there is a delicate balance
between comprehensibility and security---the twin challenges of
digital election systems. People view paper-based elections as
transparent and comprehensible. Digital elections have lost those
properties. However, in the right circumstances, digital elections
benefit from tabulation efficiency, decreases in residual vote count,
and facilitation of independent voting for voters with
disabilities. Internet voting shifts the balance towards voter
convenience, but with a significant loss in general comprehensibility;
to the first significant digit, the percentage of voters that are
cryptographers is zero.

Choosing to design and deploy an election system that is end-to-end
secure and based upon cryptographic principles is a policy decision
that firmly delegates trust to a precious few: on the front lines, the
elected officials and the politicians voting for modernization, and
behind the scenes, the cryptographers, scientists, and engineers
responsible for the system’s design, development, validation, and
verification. Those few assume an enormous responsibility.

%=====================================================================
\section{Methodology}

The technical project management of this first phase of the E2E-VIV
project was organized in a workflow that focused on delivering an
evidence-based report and its complementary Open Source technical
artifacts. The methodology is summarized as follows:

\begin{itemize}
\item Absorb all input from team members.
\item Read all literature on Internet voting.
\item Write baseline business and technical requirements and solicit
  feedback from technical team.
\item Write personas as a foundation for user experience studies.
\item Interview local election officials based upon requirements,
  personas, and their current elections framing.
\item Outline report and solicit text and reflections from team members.
\item Reflect upon the latest advances in cryptography for E2E-VIV.
\item Reflect upon the latest advances for reasoning about
  cryptographic algorithms, protocols, and implementations.
\item Craft a parameterized architecture space that reflects
  underlying requirements and standard cryptographic protocols.
\item Integrate all team member text and reflections and develop a
  final report table of contents.
\item Solicit more text and reflections from team members based upon
  the final report table of contents.
\item Write the remaining unwritten chapters.
\item Solicit input from all team members on Part 1.
\item Solicit input from technical team members on Part 2.
\item Solicit initial reflections from technical team members on the
  feasibility of Part 2 potential recommendations.
\item Gather all input from team members and capture it in appendices,
  citations, and footnotes.
\item Identify useful illustrations and figures for report and
  communicate them to the illustrator.
\item Copy-edit, lay out, polish, and release the report.
\item Clean up and make public the GitHub repository that includes the
  report and all associated artifacts of the project.
\end{itemize}

The outcomes of this process are this report and the reflections of a
body of experts in the domain of verifiable elections, as summarized
in the following section.

%=====================================================================
\section{Outcomes}
\label{sec:outcomes}

This project has produced a high-level system specification in the
form of a set of business and technical requirements. Accompanying
that set of requirements are recommendations about the underlying
means by which those requirements should be fulfilled.

These recommendations focus on the three dimensions necessary to
fulfill the strenuous requirements of any E2E-V system: cryptography,
architecture, and engineering. The cryptographic foundation is a
formal framework in which to evaluate E2E-V cryptographic
protocols. The architecture specification is a formal description of
an architecture space, defined by several parameters, in which
solutions can be designed, built, and evaluated. Finally, the rigorous
engineering necessary to create a high-assurance E2E-VIV product is
specified via a recommended software engineering process, methodology,
and set of technologies. Used properly, these can fulfill the security
and technical requirements while generating the necessary evidence to
substantiate that the system is fit-for-purpose.

Fulfilling the usability and security requirements would not be
sufficient for a positive assessment by the project team. A full
system specification that is usable and secure may, for example, be
too expensive to build, too difficult to deploy and manage, or mandate
too much expertise from election officials to operate. In the end,
social non-functional requirements may trump technical functional
requirements. 

A positive assessment by the majority of the technical team would mean
they agree that the specified election system meets all of the
requirements set forth by the charter of the group and that,
eventually, an open source E2E-VIV system could be developed, tested,
and potentially deployed.

A negative assessment by the majority of the technical team would mean
they do not agree that the specified election system meets all of the
requirements set forth by the charter of the group, and that designing
a usable and secure Internet voting system is still an open
scientific, not engineering, challenge.

The final assessment of the team is found
in~\autoref{chapter:conclusion}.

%~~~~~~~~~~~~~~~~~~~~~~~~~~~~~~~~~~~~~~~~~~~~~~~~~~~~~~~~~~~~~~~~~~~~~
\subsection{User Interface Design}
\label{sec:user-interf-design}

The user interface (UI) of the E2E-VIV election system is a critical
factor in its acceptance. The system must be simultaneously usable,
accessible, and secure. Consequently, a detailed UI design informed by
user experience (UX) and accessibility testing is a mandatory
component of a future detailed system specification.

This report’s system specification is focused on high-level
requirements for any E2E-V election system (Internet-based or not). As
such, a number of requirements focus on UI and UX, and stipulate the
necessary framing for UI/UX designs and studies.

Usability and accessibility studies and testing are key components of
this report, and the outcome of this first study is meant to inform
the UI design of any future E2E-VIV systems. As such, most of the
effort relating to UI and UX within the project has focused on
developing a technical infrastructure and complementary process that
allows efficient definition and execution of usability and
accessibility studies.

Researchers conducted an initial usability study, gathering
qualitative feedback from several dozen voters. The results of that
study, whose full report is included in
\autoref{appendix:usability_study}, led to three conclusions about
voters’ attitudes regarding Internet voting.

\begin{enumerate}
\item \emph{Voters trust the voting system and their election
    officials.} This trust is both a blessing and a curse. It is a
  blessing because it means that election officials and the government
  are working with voters that have a positive attitude; their trust
  can only be lost and does not need to be won. But it is also a curse
  because it means that voters will trust Internet voting systems that
  have no transparency, end-to-end security, or verifiability. As
  such, this opens the door for vendors to sell their technology and
  gather na\"{i}ve voter feedback as ``evidence'' that their systems are
  fit-for-purpose for modern public elections, even if that is not the
  case.

\item \emph{Voters are not interested in learning about verifiability
    or verification.} Fortunately, only a small fraction of voters
  needs to actually verify their votes for most E2E-VIV election
  schemes to generate sufficient independent evidence that the
  election outcome is correct. The requirements and recommendations
  reflect this low level of voter interest in verification and suggest
  several means by which the threshold for election verification may,
  nevertheless, always be achieved.

\item \emph{Voters expect the Internet voting experience to be somehow
    different from a traditional voting experience.} Voters expect
  modern, rich online experiences akin to those they know from popular
  commercial websites and smart phone applications. This raised
  expectation opens the door for novel experimentation to develop a
  ``21st Century'' voter experience.
\end{enumerate}

%%% Local Variables:
%%% mode: latex
%%% TeX-master: "report"
%%% End:
