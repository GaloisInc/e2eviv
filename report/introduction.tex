\chapter{Introduction\ifdraft{ (Joe K./Susan) (100\%)}{}}
\label{chapter:introduction}

%=====================================================================
\section{The E2E-VIV Project}
\label{sec:e2e-viv-project}

% ~~~~~~~~~~~~~~~~~~~~~~~~~~~~~~~~~~~~~~~~~~~~~~~~~~~~~~~~~~~~~~~~~~~~~
% this is just a comment. perhaps it has been done and I don't know
% about it, but we need to introduce the long name of the project and
% its abbreviated name "End-to-End Verifiable Internet Voting:
% Specification and Feasibility Assessment Study (E2E VIV Project) -
% to just make that this section header/title.

Elections have been conducted for millennia, but the technologies used
to cast and tally votes have varied and evolved tremendously over that
time. In 2015, much of our discourse takes place online, and many 
call for elections to follow this trend and ask why they haven't
done so already. One community that is especially desirous of a better
approach is that of "overseas" voters, for whom voting often requires
extraordinary effort.

In March 2013, Overseas Vote Foundation’s President and CEO began a
discussion with a small group of experienced election integrity
technology advocates about how, if faced with having to specify an
Internet voting system, they would respond. Despite their concerns
about the security of Internet voting efforts to date, it was
nonetheless agreed that taking on the question was crucial at the
time.

Gridlock around Internet Voting is not unique to Washington
politics. Across the U.S., the scientific community, federal agencies,
cyber security specialists, and certain organized activists have
strongly advised against exposing the ballots of the most powerful
nation on earth to the seemingly endless range of cyber threats which
run rampant on today’s Internet.

Nevertheless, faced with ongoing challenges to serve their
constituencies in modern and efficient ways, and having experienced
the everyday efficiencies of technology throughout their lives, when
seeking new and improved election systems, election officials often
want to consider Internet-based technologies. In the current climate
of economic austerity, innovation in elections is rare. Our election
officials are trapped in a technology purgatory; they continue to
devote scarce resources to support outdated voting systems, while
lacking the means to certify the new voting systems they would like.

Election integrity advocates assert that secure, tested, certified
remote voting systems that election officials envision are not
available. The scientific community does not consider online ballot
return systems secure, and no such systems had been certified at the
time of this writing. As a result, email has become a common interim
method for moving ballots online, although it does not provide any of
the benefits that a secure, full-featured voting system would provide.
Email is demonstrably insecure, yet election officials and voters
regularly use it to transmit ballots because no viable alternatives are
available. Examination of new and better ways to use technology to
meet specific voting needs, such as those of remote overseas citizens,
military members, and people with disabilities is needed.

Existing vendors of Internet voting technologies, whose systems are
neither tested nor certified, would like to openly market and sell
their systems within the U.S. and not face the resistance of the
election integrity advocates. No agreement on how to proceed, a
years-long history of mediocre attempts, ongoing animosity between
stakeholder parties, and a general lack of research on the current
questions are among the challenges of this situation.

%~~~~~~~~~~~~~~~~~~~~~~~~~~~~~~~~~~~~~~~~~~~~~~~~~~~~~~~~~~~~~~~~~~~~~
\subsection{A Better Technology}
\label{sec:better-technology}

In 1981, security researcher David Chaum published an influential
paper describing a variety of applications of the then-new technology
known as \emph{public-key encryption}. Within that publication, a single
short paragraph suggested how public-key encryption might be used to
anonymize a set of ballots in such a way that enables universal
verification of the accuracy of their tally.

Over the succeeding decades, numerous researchers have published
hundreds of papers describing and refining various systems that employ
what have come to be known as \emph{end-to-end verifiable} election
technologies. These technologies allow voters to verify for themselves
that their votes have been accurately recorded and simultaneously
allow any observers to verify that all recorded votes have been
accurately tallied.

Dozens of end-to-end verifiable (E2E-V) election systems have been
designed for a variety of settings---in-person and remote, paper-based
and electronic, simple majority and instant run-off, etc. While early
designs were cumbersome and difficult to use, more recent E2E-V
election systems are far more fluid and natural.

There is little debate regarding the benefits that end-to-end
verifiability can bring to election systems, and early steps are being
taken toward large-scale deployment of E2E-V elections technologies in
the context of in-person voting systems. The question at hand is
whether the benefits of E2E-V technologies can adequately mitigate the
legitimate security concerns created by Internet voting.

%~~~~~~~~~~~~~~~~~~~~~~~~~~~~~~~~~~~~~~~~~~~~~~~~~~~~~~~~~~~~~~~~~~~~~
\subsection{Definition}
\label{sec:definition}

The term \emph{end-to-end} is often used casually, without a precise
definition in mind. For the purposes of this study, E2E-verifiability
is a property of an election system with important components: first,
that voters can individually check that their ballots are cast and
recorded as they intend; and second, that anyone can check that all of
the recorded ballots have been accurately tallied.\footnote{Definition
  from Dr. Josh Benaloh, Senior Cryptographer at Microsoft Research.}
While systems of this nature have been developed in the past, none
have been broadly used or successfully commercialized; the E2E-VIV
Project has made a concerted effort to be informed by these past
efforts and build upon them as appropriate. Usability factors were
also considered from the outset of the study to address the
significant challenges faced by remote and disabled voters when using
such systems. This study is intended to enable development efforts in
E2E-verifiable systems that are viable with respect to security,
auditability, and usability.

For those concerned with election integrity, there is a justifiably
negative reflex in response to IV: it takes all the problems with
current remote voting systems and adds to them all the problems and
security vulnerabilities of the Internet. The E2E-VIV Project has
sought to make the case that use of the Internet enables and
facilitates the introduction of E2E-verifiability (E2E-V), and that
the benefits of E2E-V may be able to adequately mitigate the
vulnerabilities introduced by using the Internet.

No participant in this project discounts the concerns of voting over
the Internet, nor is E2E-V viewed as a magic fix that makes the
Internet secure. Nevertheless, we believe that E2E-V properties are
quite relevant to IV, since these properties are achieved even when
votes are cast on potentially untrusted devices like personal
computers and transmitted over an untrusted medium such as the
Internet. The E2E-VIV Project does not attempt to make the Internet
secure. Instead, it examines how E2E-V negates many (although not all)
of the risks of voting via the Internet while introducing substantial
new benefits that are not found in currently deployed voting systems.

%=====================================================================
\subsection{A Proposed Study and Objectives}
\label{sec:proposed-study-and-objectives}

Within this context, a project proposal was developed and written by
Susan Dziedusyzcka-Suinat of the Overseas Vote Foundation (OVF)
\footnote{Since the start of the study, the Overseas Vote Foundation
  has been renamed as ``Overseas Vote", an initiative of U.S. Vote
  Foundation, which has broadened its mission to include U.S. domestic
  and absentee voters.}  the leading nonpartisan, nonprofit
organization dedicated to overseas and military voter participation
and one that has maintained an unwaivering commitment to the cause of
election integrity. Known for its work in developing user-oriented
voter services for the full range of overseas and military voter
needs, OVF was ideally suited to conceive of, define and manage this
project. Over the past decade, OVF was at the forefront of introducing
software solutions for this voter base, from its launch of the first
online state-by-state customized voter registration/absentee ballot
wizard for overseas and military voters, the first online Federal
Write-in Absentee Ballot with dynamic candidate lists, the first
Hosted System Solutions for states or organizations who wanted to
offer to better overseas and military voter services to the first
Election Official Directory API. The market space that OVF opened a
decade ago has become a profitable one for many vendors since.

On December 19, 2013, OVF announced the launch of this project as the
End-to-End Verifiable Internet Voting: Specification and Feasibility
Assessment Study (E2E-VIV Project). The project funding was provided
by the Democracy Fund, a Washington D.C.-based philanthropic
organization whose stated objective is to ``\ldots{}invest in
organizations working to ensure that our political system is
responsive to the priorities of the American public and has the
capacity to meet the greatest challenges facing our country.'' The
proposal envisioned a project to examine the future of voting and how
it might be executed securely online; one that approached the question
of Internet-voting from a research perspective, and that sought to
fill in the gaps of the many open questions plaguing the discussion.

The stated aim was to examine whether an end-to-end verifiable
Internet voting system can be built that would offer a viable and
responsible alternative to current systems to support the voting
rights of overseas voters. According to Joe Goldman, Director of the
Democracy Fund, ``The significance of this project will be in its
ability to break open the conversation from its current stalemate and
include all sides in a constructive project to openly examine and
research what is really needed by voters and election officials, and
to determine whether this form of voting can meet those needs and
still guarantee security of the election. Equally important, it will
identify potential tradeoffs and shortcomings that represent the
diverse range of values we hold dear in our elections.'' 

For this study, a unique team of experts in computer science,
usability, and auditing together with a selection of local election
officials from key counties around the U.S. was been assembled. The
focus is to produce a system specification and set of testing
scenarios, which if they meet the requirements for security,
auditability, and usability, would then be placed in the public
domain. At the same time, a strong effort was made to demonstrate that
confidence in a voting system is built on a willingness to verify its
security through testing and transparency.

There is a historical misunderstanding in the U.S. election community
that the E2E-VIV Project has aimed to correct. Our country's foremost
scientists are not against technology advancements, nor are they
inherently at odds with the election officials who seek technology
improvements to meet their administrative challenges. Instead, the
U.S. scientific community has continued to be extremely skeptical of unproven claims of
security regarding existing systems that are not publicly tested or
vetted. This study aimed to break this impasse and reconcile these
concerns. The scientific leaders on this project have often pointed out
security vulnerabilities in past systems; however the E2E-VIV Project
has led them to agree that if Internet Voting (IV) can happen, it must
be in a system that takes advantage of end-to-end verifiability and
auditability.
% however the E2E-VIV Project has led
% us to agree that Internet Voting (IV) can happen, but that it must be
% in a system that takes advantage of end-to-end verifiability and
% auditability. 


%~~~~~~~~~~~~~~~~~~~~~~~~~~~~~~~~~~~~~~~~~~~~~~~~~~~~~~~~~~~~~~~~~~~~~
\subsection{Shared Goals}
\label{sec:shared-goals}

Election officials and scientists involved in elections share the same
overall goals: that voting systems provide accurate results, protect
the privacy of voters, are easily used by all voters, and are robust
against both accidental and intentional disruptions. Additionally, it
should be publically demonstrable that these properties are achieved.

This project showed that the scientific community cares deeply about
addressing the needs and requests of election officials as they serve
remote voters, that they have a great motivation to address these
questions when given a constructive opportunity to do so, and that
they would like to work together with election administrators to examine 
the possibilities in this realm. It also demonstrated that the scientific 
community needs to improve at understanding the practical aspects of 
administration of elections and in collaborating and communicating with 
election officials to develop technical solutions. 

% maybe add some things it shows outside of the scientific community? I invite you to change what I added.
% I added this because of the obvious divide in opinion that was evident on the first workshop and how the two sides 
% treated and responded to one another. Also, based on some roadshow feedback. 

%~~~~~~~~~~~~~~~~~~~~~~~~~~~~~~~~~~~~~~~~~~~~~~~~~~~~~~~~~~~~~~~~~~~~~
\subsection{Project Goal}
\label{sec:project-goal}

\todo{The tense of the entire introduction seems to change here, from
  present (study is intended, project shows that, etc.) to past (goal
  was, presumed that). That needs to be cleaned up but I'm not going
  to do so at the moment. -dmz}
The goal of this project was to specify and define a system and its
testing scenarios for an online voting method that can provide both
security and confidence to voters that their selections are accurately
recorded and counted. The premise is that E2E-V negates many, although 
not all, of the risks of voting via the Internet while introducing 
substantial new benefits that are not found in currently-deployed voting 
systems. The project presumed that E2E-V is a possible answer, or at least 
a step toward one, and explored how well it can meet the needs of many voters 
and election officials. Additionally, any shortcomings found with this approach
would serve as a starting point for future work.

%~~~~~~~~~~~~~~~~~~~~~~~~~~~~~~~~~~~~~~~~~~~~~~~~~~~~~~~~~~~~~~~~~~~~~
\subsection{Additional Objectives}
\label{sec:addit-object}

A secondary objective of this work was the presentation and discussion
of the report with key stakeholders, integration of feedback, and
seeking of broad acceptance of the report's processes and conclusions
(see Section 2.2.6).

%~~~~~~~~~~~~~~~~~~~~~~~~~~~~~~~~~~~~~~~~~~~~~~~~~~~~~~~~~~~~~~~~~~~~~
\subsection{Deliverables}
\label{sec:deliverables}

The main deliverable of the E2E-VIV Project was the development of a
``whole product solution'' specification (or simply ``specification'')
for a trustworthy E2E-VIV election system.

This report presents a system specification for a secure E2E-VIV
system, a set of testing specifications to demonstrate the security,
and a set of guidelines for system usability, accessibility, and
testing. Additional topics and analyses may be considered and
discussed in the report, such as legal and administrative challenges,
and ballot secrecy, privacy, and confidentiality.

%~~~~~~~~~~~~~~~~~~~~~~~~~~~~~~~~~~~~~~~~~~~~~~~~~~~~~~~~~~~~~~~~~~~~~
\subsection{A First Step}
\label{sec:first-step}

This project represented the first step in an examination of whether
one day E2E-VIV might be possible. The plan was to examine the
potential for an E2E-VIV remote voting system together with election
officials, taking into close account their needs and the needs of
disabled voters. If a system can one day be developed based on these
principles, then this would be determined. A viable outcome of this
study with respect to security, auditability, and usability would
enable development efforts to ensue.

%~~~~~~~~~~~~~~~~~~~~~~~~~~~~~~~~~~~~~~~~~~~~~~~~~~~~~~~~~~~~~~~~~~~~~
\subsection{Success}
\label{sec:success}

Beyond the concrete outcomes of this project, the fact that it 
took a research and testing-based approach to a problem that
had been ``in stalemate mode'' would, it was hoped, stimulate election
development overall. The election industry has been operating in a
traditional paradigm with only a few vendors able to survive despite
demand to move away from outdated, expensive, hardware-oriented
solutions. A successful outcome of this project would be the production of a
specification for a usable, secure E2E-verifiable remote voting
technology, to identify its strengths and weaknesses, and to develop
reasons to pursue or not pursue this approach to remote and/or
disabled citizen voting. 

However, from the beginning it was clear that if the project were to
determine that current techniques were weak and should not be pursued,
then this would also be an outcome with many useful implications. A
complete success for this project would be to produce a specification
that would be: 1) supported by the vast majority of the contributors,
including the technical, usability, testing, and research teams; 2)
endorsed by the vast majority of the advisory council; and 3) endorsed
by the major stakeholders in elections administration as represented
by the project's local election officials. Additionally, the E2E-VIV
Project hoped to receive support and endorsement from many members of
the electronic voting activism community, as represented by key
members of the Election Verification Network, the Verified Voting
Foundation and beyond. The specification was intended to fulfill the
following requirements:

\begin{description}
\item[Independent Implementation] The specification must be of
  sufficient detail and clarity that an implementation of the election
  system must be possible by an independent party without extensive
  dialog with participants in the project.
\item[Independent Validation] It must be possible for a moderately
  proficient IT expert to objectively determine, in a reasonable time
  frame with reasonable cost, whether a constructed election system
  fulfills the specification.
\item[Evidence-Based Decisions] Every decision made in the crafting of
  the specification must be objectively justifiable and the evidence
  for the decision must be traceable.
\end{description}

%~~~~~~~~~~~~~~~~~~~~~~~~~~~~~~~~~~~~~~~~~~~~~~~~~~~~~~~~~~~~~~~~~~~~~
\subsection{Scope}
\label{sec:scope}

The original project was tightly limited to involve system
specification and testing only. No system development was envisioned
in Phase I beyond mock-ups to help test usability. However, this
changed early on in the project when Dr.~Joseph Kiniry of Galois, Inc.
was engaged as the technical project manager and introduced to the project
a new engineering team. 

Significantly, Galois is a leader in the process of computing on data
while it remains encrypted, and in the automated generation,
validation and synthesis of high assurance cryptographic solutions.
They excel in multiple areas of cryptographic implementation and
secure-by-construction software, all of which can be applied to the
challenge of developing secure and usable E2E-VIV voting. The
relevance of Galois’s work to the project is clear: applying
cutting-edge computer science and mathematics to solve difficult
technological problems is needed to solve the secure, verifiable
election systems development challenge. Galois’s management agreed to
donate a significant portion of engineering time to the project in
order to build “demonstrators” that would be used to prove the
concepts of E2E-V and to further examine security and usability.

The Galois engineers developed a set of rigorous engineering
artifacts, ``demonstrators'', fit for refinement into a working
election system, and against which third parties can perform
independent validation and verification. Galois defined demonstrators
as technical artifacts from the point of view of definition and
constructions, but non-technical artifacts from the point of view of
demonstration. The demonstrators developed using Galois interent
research and development funding are:
\begin{itemize}
\item developed in a completely transparent and public manner within
  the Galois GitHub Organization,
\item cross-referenced, and thus traceable to and from, all
  specification aspects (from domain models to behavioral design
  specifications),
\item replicated into the E2E-VIV GitHub Organization, and
\item licensed under either a mainstream Open Source license with a
  strong community or an alternative license tuned to the elections
  community.
\end{itemize}

%=====================================================================
\section{People}
\label{sec:people}

The E2E-VIV Project was an opportunity to combine the abilities,
knowledge, experience and expertise of a diverse group of
technologists, computer scientists and election officials involved in
election integrity together to form the overall project
team. Technical, usability, testing and local election official
sub-teams were formed for ease of communication. The technical team
had decades of experience in E2E-V technology, cryptography, usability,
and testing. An Advisory Council was established to broaden the
communication with interested members of the election community.

Overseas Vote Foundation (OVF), as the official grantee, was
responsible for the overall project conception, proposal development,
presentations, communications, management, team recruitment,
contractual obligations, public relations, events and budgeting. Deep
experience in the arena of overseas and military voting, absentee
voting, community building, voter survey research, election reform and
communications gave OVF a unique edge in managing the project.

Galois, Inc. provided the technical and engineering project
management. Named as the technical project manager, Dr.~Joseph Kiniry,
working as a Principal Investigator at Galois, facilitated the
communication and decision-making of the team. He became the main
author and editor of the report and ran all engineering projects and
usability aspects of the study.

%~~~~~~~~~~~~~~~~~~~~~~~~~~~~~~~~~~~~~~~~~~~~~~~~~~~~~~~~~~~~~~~~~~~~~
\subsection{Team Members}
\label{sec:team-members}

\todokiniry{We'll be turning this section into a more attractive
  illustration to enumerate participants in the project. -JRK}
% Joe, please see my email on this topic.
\textbf{Project Manager:} Susan Dzieduszycka-Suinat, Overseas Vote Foundation

\textbf{Lead Technical Project Manager:} Dr. Joseph Kiniry, Galois

\textbf{Technical Team}

Dr. Josh Benaloh
Senior Cryptographer, Microsoft Research
 
Dr. David R. Jefferson
Lawrence Livermore National Laboratory
 
Dr. Doug W. Jones
Associate Professor, Department of Computer Science, University of Iowa
 
Dr. Aggelos Kiayias
Associate Professor, Computer Science and Engineering, University of Connecticut
 
Dr. Olivier Pereira
Professor, Institute of Information and Communication Technologies, Electronics and Applied Mathematics, Ecole Polytechnique de Louvain
 
Dr. Poorvi Vora
Associate Professor, Department of Computer Science, The George Washington University
 
Dr. David Wagner
Professor, EECS Computer Science Division, University of California Berkeley
 
Dr. Dan Wallach
Professor, Department of Computer Science, Rice University
 
\textbf{Usability Team}

\begin{itemize}
\item Keith Instone, User Experience Consultant
\item Morgan Miller, Usability Analyst, Experience Lab
\item Dr. Judith Murray, Research Consultant
\end{itemize}

\textbf{Election Auditing}

Dr. Philip Stark
Professor and Chair of Statistics, University of California Berkeley
 
\textbf{Testing Team}

Dr. Duncan Buell
Professor of Computer Science and Engineering, University of South Carolina
 
Andrew Regenscheid
Mathematician, National Institute of Standards and Technology
 
\textbf{Advisory Council}

Dr. Ben Adida
 
Dr. Michael Clarkson
Assistant Professor of Computer Science, The George Washington University
 
Dr. J. Alex Halderman
Assistant Professor of Computer Science and Engineering, University of Michigan
 
Candice Hoke
Professor of Law, Cleveland State University
 
Dr. Ron Rivest
Vannevar Bush Professor of Computer Science, Massachusetts Institute of Technology
 
Noel Runyan
Primary Consultant, Personal Data Systems
 
Dr. Peter Ryan
Professor in Applied Security, University of Luxembourg
 
Dr. Barbara Simons
Research Staff Member, IBM Research (retired)
 
Dr. Vanessa Teague
Research Fellow, Department of Computing and Information Systems, University of Melbourne
 
John Wack
Voting Systems Standards, National Institute of Standards and Technology
 
Dr. Filip Zagorski
Assistant Professor of Computer Science, Wroclaw University of Technology
 
\textbf{Local Election Officials}

Lori Augina
Director of Elections, Washington State, Secretary of State

Rachel Bohman
Former Hennepin County Elections Manager (Minnesota)

Judd Choate
Director of Elections, Colorado, Secretary of State

Dana Debeauvoir
Travis County Clerk (Texas)
 
Mark Earley
Voting Systems Manager, Leon County (Florida)
 
Dean Logan
Los Angeles Registrar-Recorder/County Clerk (California)

Stuart Holmes
Election Information Systems Supervisor, Office of the Secretary of State (Washington)
 
Dr. Lois H. Neuman
Chair, Board of Supervisors of Elections, City of Rockville (Maryland)
 
Roman Montoya
Deputy County Clerk, Bernalillo County (New Mexico)
 
Tammy Patrick
Senior Advisor to the Democracy Project, Bipartisan Policy Center and Former Federal Compliance Officer Maricopa County (Arizona)
 
\textbf{Overseas Vote Foundation Support Team}

Susan Dzieduszycka-Suinat
President and CEO
 
Paul McGuire
Legal Counsel and Secretary of the Board
 
Richard Vogt
Treasurer and Chief Financial Officer

Capstone Project Team, Carnegie Mellon University, Heinz College,
School of Information Systems \& Management; Master of Information
Systems Management and Master of Science in Information Security
Policy and Management: in early 2014, a Capstone Team was assigned to
the project team to assist on the Comparative Analysis of E2E systems.

%~~~~~~~~~~~~~~~~~~~~~~~~~~~~~~~~~~~~~~~~~~~~~~~~~~~~~~~~~~~~~~~~~~~~~
\subsection{Stakeholder Groups}
\label{sec:stakeholder-groups}

Although not on the official project team, there are several
communities relevant to the E2E-VIV Project outside of those
represented on the project team, and interaction with members of these
communities has been essential. These communities include the
following overlapping cohorts.

\paragraph{Election verification advocates.} Election verification
advocates are plentiful, well-informed, and strongly connected.  They
care deeply about election integrity and verifiability, and
unsurprisingly Internet voting is a high-priority issue for many.

Their skeptical attitude is compounded by the fact that several vendors
have developed Internet voting products which are proprietary,
closed-source, have never seen a public audit, and are
unverifiable. Moreover, many of these vendors make specious claims
about the security of their products---claims which the advocate
community rejects entirely. Finally, many vendors advocate outsourcing
elections entirely to them---a condition that will never be acceptable
to the advocate community, even for an end-to-end verifiable Internet
voting system.

A small number of advocates are \textbf{for} verifiable Internet
voting, a small number are adamantly \textbf{against} Internet voting
of any kind, but the bulk of advocates are on-the-fence. That majority
recognizes that there are significant scientific and engineering
challenges in designing and developing an Internet voting system.
Moreover, they recognize that the decision to deploy such a system is
very much a subjective, political one. In some contexts, it is viewed
as perfectly acceptable to use a non-verifiable, outsourced election
apparatus (such as Everyone Counts' product); for example, in an election
to decide the winner on a reality show. But for government elections
of any value, such an option is unacceptable to virtually every
advocate.

Consequently, being fully transparent with---and listening to the
feedback from---the election verification advocate community is
absolutely mandatory. If the bulk of that community is not swayed by
the evidence presented in this report, pursuing any next phase in this
project will be fraught with turmoil and will be an uphill battle
against many influential actors, all with good intentions.

\paragraph{Standards Bodies.} Perhaps surprisingly, there is little
national or international standardization in the area of elections. A
nascent effort to begin standardizing data interchange formats began
about a decade ago and eventually fizzled after only producing one
small standard.

There are a myriad of reasons why this first effort failed. Vendors
lobby against, and are disinterested in, interoperability. The
Election Assistance Commission's Voluntary Voting System Guidelines
(VVSG) were not geared toward a component-based approach to system
design; since devices could not be plugged together, there was little 
cause for defining interfaces and data file formats. Finally, there
was insufficient acceptance from the election research community.

In 2015, this situation changed with the rebirth of the IEEE~1622
committee focusing on elections. The IEEE Voting System Standards
Committee~1622 (VSSC/1622) is creating standards and guidelines around
a common data format for election data. The aim is that future
election equipment used in U.S. elections and abroad can interoperate
more easily. It is the intention of the VSSC that standards and
guidelines being developed will be required in future versions of the
EAC's VVSG.

Many of the top researchers, election advocates, and election
officials in the world are a part of this committee. Additionally,
representatives from the major election systems vendors are either
participating, or listening in, because they recognize that
interoperability will be mandated by future versions of the
VVSG.\footnote{Recall that all election systems in the U.S.A. must be
  certified at the State or Federal level according to the EAC's
  voting system testing and certification standards, standards which
  mandate compliance with the VVSG.}

Standards are critical to any future work on E2E-VIV systems for
several reasons. First, given the compositional nature of most E2E-V
systems' designs, it is not unreasonable to expect that different
subsystems will be created and supported by different organizations.
Second, in order to effect arbitrary third party verification, clearly
defined common data formats must be used. Third, at some point in the
future, if E2E-VIV systems are accepted in the mainstream, they must
be certified by the EAC. As such, EAC, NIST, and IEEE standards must
recognize their core capabilities, subsystems, interfaces, and what
constitutes legitimate evidence of correctness and security. Moreover,
it is sensible to presume that there should be a standard means by
which these aspects and evidence are documented to expedite a sound,
accurate, and expedited certification process.

\paragraph{Vendors.} Vendors relevant to this project come in two
forms: (1)~existing vendors of proprietary election systems and
(2)~future vendors that support open source election systems.

Existing vendors are significantly interested in the results of this
project, particularly if it moves to a second phase that focuses on
the design and development of an open source system.

Obviously those that have an existing non-E2E-V Internet voting
products (e.g., Scytl, Dominion, and Everyone Counts) will benefit from
this work, whether that attention is justified or not. Any increase in
attention on---or any hint of support from the verifiable elections
activist community about---Internet voting aligns with their marketing
goals.

Additionally, vendors can use the requirements herein in good and bad
ways. We are hopeful that they will actively and positively absorb the
recommendations of this report and engage with any future phase of
this work.

From an optimistic perspective, if vendors read and understand the
report and its recommendations, then they can closely track the work
of the (provisional) next phase of this project. Any investment they
make towards making their commercial systems end-to-end secure and
verifiable---so long as there is publicly available evidence of such
improvements---would be welcome by the bulk of the research and
activist communities.

The pessimist will argue that vendors will simply use these
requirements as a checklist, making specious claims that their systems
are E2E-V but providing no evidence of such.

Time will tell which of these paths each existing vendor will choose
to take.

The other kind of vendor that is relevant to this project does not yet
exist: vendors that support open source election systems in manifold
ways. Within the open source ecospheres---like those surrounding
operating systems like Linux, blog platforms like WordPress, and
content management systems like Drupal---there exists a pluripotent
variety of companies that support these platforms: value-added
resellers, integrators, hosting facilities, hardened software stacks,
etc. Most of the companies provide commercial support for
open source products at a variety of service levels, either via a
subscription service or using a release-based license model.

We expect that a similar flowering of commercial support organizations
must come to exist to support any E2E-VIV system that results with
this project as its genesis. It is important to note that these
organizations need not become experts in the underlying cryptography,
formal methods, or usability and accessibility. The high-assurance
development method proposed in this report results in validation and
verification artifacts that are evidence-generating for correctness
and security properties.  Additionally, modern formal system
specification languages support traceability from requirements to
evidence. 

Consequently, any future mission-critical E2E-VIV system is robust to
integration, customization, or evolution work by these different kinds
of support vendors. That is to say, any future E2E-VIV system
developed according to the requirements and recommendations herein
cannot be accidentally or purposefully broken without third parties
detecting the certification failure. It is only by virtue of having
these complete, consistent, traceable, evidence-generating artifacts
that such a blossoming of support vendors can come to exist.

\paragraph{Hackers and Hacktivists.} Hackers and election hacktivists
are an important audience for this project. On the one hand, hackers,
constructively or destructively, help make open source systems more
secure.  Hacktivists, on the other hand, catalyze movements of
like-minded technical individuals. Across the world, their attention
has brought to light the flaws of insecure and incorrect electronic
voting systems (e.g., in The Netherlands).
% maybe a link or footnote to exacty which project in the Netherlands?
Within these subcommunities, it is an undisputed truth that the design
and development of a secure system used for public good must be open
source and must witness critique and improvements from the public.
Leveraging that attention is especially valuable for nationally
critical systems like public elections.

Consequently, it is our expectation that direct engagement with these
cohorts, from the Chaos Computer Club to Anonymous, while potentially
tempestuous at times, is wise and will have manifold benefits.

\paragraph{Election Officials.} Local elections officials (LEOs) are
the core stakeholders in the design and deployment of E2E-VIV
technologies.  As such, not only must we pay attention to our expert
LEOs on the project, but we must solicit feedback and reflections from
any election official, past or present, who can give voice to their
jurisdictions, and its voters' needs.

There are over 10,000 election jurisdictions in the U.S.A.; and there
are consequently over 10,000 different ways that elections are
run. Smoothly integrating with existing election processes is
important. Base issues like common data formats and API integration
are simple technical challenges that have straightforward, though
potentially politically delicate, solutions.

The enormous plurality of technologies, large and small, at the
federal, state, and local level make for difficult deployment of any
new election technology. It is clear that traditional IT practices for
the design, development, and maintenance of election software cannot
work in this setting. No vendor can support 10,000 forks of a single
code base to support 10,000 clients with slightly different
requirements. This state of affairs leads to some of our technical
recommendations, particularly those that touch upon feature modeling
and software product lines.

More subtle challenges, like ensuring that election officials
uncomfortable with technology can deploy and support a E2E-VIV system
for their overseas, military and/or disabled voters, have little to 
do with technical solutions. These problems and solutions have more 
of a social,political, and psychological root, and thus require 
``soft'' solutions.

Artifacts like documentation and tutorials, webcasts and screencasts,
reports and demonstration software only go so far. An open supportive
community must exist around E2E-VIV.  This community may be realized
in several ways.  In the Open Source world, it is common to see
friendly online forums for newbies spring up and regularly scheduled
and community-organized developer and user conferences.  Moreover, we
expect to see a wellspring of companies ready and willing to support
LEOs, ranging from value-added resellers to integration partners, as
discussed above.\footnote{For several examples of these communities,
  look no further than those that spring up around Linux
  distributions, Content Management Systems, blog platforms, wiki
  platforms, etc.}
  
As we exit Phase I of this project and, speculatively, move to Phase
2 - R\&D, a wide range of elections officials must be actively engaged in
the design and deployment of E2E-VIV technology, and we must foster
the formation of a healthy, active, and supportive community,
including volunteers and commercial IT firms, around it.

\paragraph{Voters.} Despite the import of all of the aforementioned
stakeholders, truly the most important stakeholder---in fact the one
that holds the veto power over the wide-scale deployment of E2E-VIV
technologies---is the voter.

It is clear that many citizens want flexible, comfortable, location
agnostic voting.  It is also clear that they do not realize the
implications of this desire, particularly with regard to the security
and privacy challenges of such a framing.  Educating voters about the
challenges of Internet voting is worthwhile, but it is clearly not
possible to make every overseas and military voter aware of the utility 
of, and consequent critical need for, E2E-V election systems.

In fact, based upon our early usability studies of E2E-V election
systems, it is not even clear that we can expect that a small fraction
of voters understand and use the verifiability features of such
systems, even if such systems see significant refinement and are
truly usable by all voters.

Voters' trust in their election apparatus, and its transitive impact on
their trust in their sitting government, is paramount. Consequently,
changes in elections that may impact trustworthiness are difficult to
make.

Voters are sensitive to changes that are bluntly visible, like
e-pollbooks or e-voting machines. The DRM revolution and VVPAT redux
have taught some voters to not trust technology at face-value.

Reactions to changes that have the potential to be misunderstood by,
or cause alarm with, a vocal minority are unpredictable. Witness the
introduction of voter ID across the U.S.A. over the past decade, as
well as the alarm raised by the use of email ballot return, even in
extraordinary circumstances like post-Hurricane Sandy elections.

Finally, and perhaps most critically, there is a delicate balance
between comprehensibility and security---the Scylla and Charybdis of
digital election systems. Paper-based elections are viewed as being
transparent and comprehensible. Digital elections have lost those
properties but, in the right circumstances, gain others like
tabulation efficiency, decreases in residual vote count, and
facilitation of independent voting for disabled voters. Internet
voting again shifts the balance towards voter convenience, with a
significant loss in general comprehensibility, since to the first
significant digit the percentage of voters that are cryptographers is
zero.

Choosing to design and deploy an election system that is end-to-end
secure, and thus based upon cryptographic principles, is a policy
decision that firmly delegates trust and responsibility into the hands
of a precious few.  Those few---on the front lines, the elected official
and the politician voting for modernization, and behind the scenes,
the cryptographers, scientists, and engineers responsible for the
system's design, development, validation, and verification---have
enormous responsibility in their hands.

%=====================================================================
\section{Methodology}

In order to transparently and rigorously shoulder this burden in this,
in the first phase of the E2E-VIV project, Galois organized the
technical project management in a workflow that focused on delivering
an evidence-based report and its complementary open source technical
artifacts.  This methodology is summarized as follows.

\begin{enumerate}
\item \textbf{Absorb all input from team members.}
\item \textbf{Read all literature on Internet voting.}
\item \textbf{Write baseline business and technical requirements and
    solicit feedback from technical team.}
\item \textbf{Write personas as foundation for UX studies.}
\item \textbf{Interview LEOs based upon requirements and personas;
    include in interview information about their current elections
    framing.}
\item \textbf{Outline report and solicit text and reflections from
    team members.}
\item \textbf{Reflect upon latest advances in cryptography for E2E-VIV.}
\item \textbf{Reflect upon latest advances for reasoning about
    cryptography algorithms, protocols, and implementations.}
\item \textbf{Craft a parameterized architecture space that reflect
    underlying requirements and standard cryptographic protocols.}
\item \textbf{Integrate all team member text and reflections and craft
    a final report table of contents.}
\item \textbf{Solicit more text and reflections from team members
    based upon final report table of contents.}
\item \textbf{Write chapters that were as-of-yet unwritten by team
    members.}
\item \textbf{Solicit input from all team members on Part 1.}
\item \textbf{Solicit input from technical team members on Part 2.}
\item \textbf{Solicit initial reflections from technical team members
    on feasibility of Part 2 potential recommendations.}
\item \textbf{Gather all input from team members (good and bad,
    nuances, disagreements, etc.) and capture all in appendices,
    citations, and footnotes.}
\item \textbf{Identify useful illustrations and figures for report and
    farm out drafting of such to illustrator.}
\item \textbf{Copy-edit, layout, polish, and release report.}
\item \textbf{Cleanup and make public the git repository that includes
  the report and all associated artifacts of the project.}
\end{enumerate}

The outcomes of this process are this report and the reflections of a
body of experts in the domain of verifiable elections, as summarized
in the following section.

%=====================================================================
\section{Outcome}
\label{sec:outcome}

This project has produced a high-level system specification, in the
form of a set of business and technical requirements.  Accompanying
that set of requirements are recommendations about the underlying
means by which those requirements should be fulfilled.  

These recommendations focus on the three dimensions necessary to
fulfill the very strenuous requirements of any E2E-V system:
cryptography, architecture, and engineering.  The cryptographic
foundation is a formal framework in which to evaluate E2E-V
cryptographic protocols.  The architecture specification is a formal
description of a parameterized architecture space in which solutions
can be designed, built, and evaluated.  Finally, the rigorous
engineering necessary to create a high-assurance E2E-VIV product is
spelled out via a recommended software engineering process,
methodology, and set of technologies which, when used properly, can be
used to fulfill the security and  technical requirements
% I removed the word "correctness" in front of technical requirements. Did you want "correct"?
while generating the necessary evidence that the system is
fit-for-purpose.

Given these underlying recommendations, the assessment of the system
by the expert team has had two possible outcomes:
\begin{enumerate}
\item \textbf{Positively:} The majority of the expert team may decide
  that the specified election system meets all of the requirements set
  forth by the charter of the group. This outcome would indicate that
  OVF might potentially move forward to ensure that the election
  system is developed and, potentially, deployed.
\item \textbf{Negatively:} The majority of the expert team may decide
  that the specified election system does not meet all of the
  requirements set forth by the charter of the group. This outcome
  indicates that further funding to design or construct such an
  election system is, for the moment, unwise and that the community
  believes that designing a usable and secure election system is still
  an open scientific, not engineering, challenge.
\end{enumerate}

Fulfilling the usability and security requirements would not be
sufficient for a positive assessment by the expert team. A full system
specification that is usable and secure may be, for example, far too
expensive to build, too difficult to deploy and manage, or mandate too
much expertise from election officials to operate. Social
non-functional requirements may trump technical functional
requirements.

%~~~~~~~~~~~~~~~~~~~~~~~~~~~~~~~~~~~~~~~~~~~~~~~~~~~~~~~~~~~~~~~~~~~~~
\subsection{User Interface Design}
\label{sec:user-interf-design}

The user interface (or UI for short) of the E2E-VIV election system is
a critical factor in ensuring that the system is simultaneously
usable, accessible, and secure. Consequently, a detailed UI design
informed by user experience (UX, for short) and accessibility testing
is a mandatory component of a future detailed system specification.

At the moment, this report's system specification is focused on
high-level requirements for any E2E-V election system (Internet
voting-based or not).  As such, a number of requirements focus on UI
and UX and stipulate the necessary framing for UI/UX designs and
studies.

Usability and accessibility studies and testing are key components of
this report, and the outcome of this first study is meant to inform
the UI design of any future E2E-VIV systems. As such, most of the
effort relating to UI and UX within the project is focused on
developing a technical infrastructure and complementary process for
the efficient definition and execution of qualitative and quantitative
usability and accessibility studies.

The initial usability study that was conducted gathered qualitative
feedback from several dozen voters.  Based upon that study, whose full
report is included in \autoref{appendix:usability_study}, three
conclusions were reached about voters' mental models about Internet
voting.

\begin{enumerate}
\item \emph{Voters intrinsically trust the voting system and their
    election officials.}  This trust is both a blessing and a curse.
  It is a blessing because it means that election officials and the
  government are working with voters that have a positive disposition; 
  their trust can only be lost and does not need to be won. But it is
  also a curse because it means that voters will trust Internet voting
  systems that have no transparency, end-to-end security, or
  verifiability. As such, this opens the door for existing vendors to
  sell their technology and gather na\"ive voter feedback as
  ``evidence'' that their systems are fit-for-purpose for modern
  public elections.
\item \emph{While many voters are supportive of having verifiability
    in their election system, very few voters are interested in
    learning about verifiability and taking the necessary time and
    energy to perform verification.} Fortunately, only a small fraction
  of voters needs to actually verify for most E2E-VIV election schemes
  to generate sufficient independent evidence that the election outcome
  is correct. As such, our requirements and recommendations reflect
  this level of voter engagement and suggest several means by which
  the threshold for election verification is always achieved.
  \todokiniry{Make sure requirements for verifiability and our
    recommendations are strengthened given the UX study. -JRK}
\item \emph{Voters expect the Internet voting experience to be somehow
    different from a traditional voting experience, be it on paper
    ballot or computer-assisted in a supervised setting.} It was not
  uncommon for voters to say, ``Huh, that's it?!'' after completing
  their voting process. Voters' mental models are trained to expect
  modern, rich web surfing experiences a la Facebook, YouTube, and
  Amazon and novel, interactive user experiences on their smart phones
  like Siri, Google Maps, and Uber. Consequently, this raised
  expectation opens the door for novel experimentation for a 21st
  century voter experiences.
\end{enumerate}

%%% Local Variables:
%%% mode: latex
%%% TeX-master: "report"
%%% End:
