\chapter{Introduction\ifdraft{ (Joe K./Susan) (90\%)}{}}
\label{chapter:introduction}

%=====================================================================
\section{The E2E VIV Project}
\label{sec:e2e-viv-project}

%~~~~~~~~~~~~~~~~~~~~~~~~~~~~~~~~~~~~~~~~~~~~~~~~~~~~~~~~~~~~~~~~~~~~~
\subsection{Situation}
\label{sec:situation}

In March 2013, Overseas Vote Foundation’s President and CEO began a
discussion with a small group of experienced, election integrity
technology advocates about how, if faced with having to specify an
Internet voting system, they would respond. Security concerns being
the primary reason for their general opposition to what they deemed
inadequate efforts of others to do so to date, it was nonetheless
agreed that taking on the question was crucial at the time. 

Gridlock is not unique to Washington politics, it is quite well known
around the topic of Internet Voting in the U.S. The scientific
community, federal agencies, cyber security specialists and certain
organized activists have strongly advised against exposing the ballots
of the most powerful nation on earth to the seemingly endless range of
cyber threats, which run rampant on today’s Internet. 

Nevertheless, faced with ongoing challenges to serve their
constituencies in modern and efficient ways, and having experienced
the everyday efficiencies of technology throughout their lives, when
seeking new and improved election systems, election officials often
want to consider Internet-based technologies. In the current climate
of economic austerity, innovation in elections is rare. Our election
officials are trapped in a technology no-man’s land of ongoing support
payments for outdated voting systems, compounded by no means of
certifying new voting systems they would like.

Election integrity advocates cite that secure, tested, certified
remote voting systems that election officials envision aren’t
available. The scientific community does not consider online ballot
return systems secure, nor are they certified. As a result, email has
become the default stopgap method for moving ballots online, although
it does not provide any of the benefits that a secure, full-featured
voting system would provide. Email is demonstrably weak on security,
yet election officials and voters are using it regularly to transmit
ballots because viable alternatives are not available. Examination of
new and better ways to use technology to meet specific voting needs,
for example, that of the remote overseas citizen, military and
disabled is needed. 

Existing vendors of Internet voting technologies, whose systems are
neither tested nor certified, would like to openly market and sell
their systems within the U.S. and not face the resistance of the
election integrity advocates. No agreement on how to proceed, a
years-long history of mediocre attempts, ongoing animosity between
stakeholder parties, and a general lack of research on the current
questions could well-describe the situation.

%~~~~~~~~~~~~~~~~~~~~~~~~~~~~~~~~~~~~~~~~~~~~~~~~~~~~~~~~~~~~~~~~~~~~~
\subsection{A Proposed Solution}
\label{sec:proposed-solution}

Within this climate, a project proposal was written and funding
provided by The Democracy Fund, a Washington D.C. based philanthropic
organization whose stated objective is to “...invest in organizations
working to ensure that our political system is responsive to the
priorities of the American public and has the capacity to meet the
greatest challenges facing our country.” 

A project intent on examining the future of voting and how it might be
executed securely online fit neatly into the strategic purview of the
fund; one that approached the question of Internet-voting from a
research perspective, that sought to fill in the gaps of the many open
questions plaguing the discussion was regarded by the supporting
organization as a positive endeavor. 

According to Joe Goldman, Director of the Democracy Fund, “The
significance of this project will be in its ability to break open the
conversation from its current stalemate and include all sides in a
constructive project to openly examine and research what is really
needed by voters and election officials, and to determine whether this
form of voting can meet those needs and still guarantee security of
the election. Equally important, it will identify potential tradeoffs
and shortcomings that represent the diverse range of values we hold
dear in our elections.”

On December 19, 2013, Overseas Vote Foundation\footnote{Since the
  start of the study, Overseas Vote Foundation has been renamed as
  “Overseas Vote”, an initiative of U.S. Vote Foundation.} (OVF), a
nonpartisan, nonprofit organization dedicated to overseas and military
voter participation announced the launch of the project, which was
called the End-to-End Verifiable Internet Voting: Specification and
Feasibility Assessment Study (E2E VIV Project). Its stated aim was to
examine a form of remote voting that enables a so-called “end-to-end
verifiability” (E2E) property. A unique team of experts in computer
science, usability, and auditing together with a selection of local
election officials from key counties around the U.S. assembled for the
study.

They agreed to focus their efforts to produce a system specification
and set of testing scenarios, which if they meet the requirements for
security, auditability, and usability, would then be placed in the
public domain. At the same time, their intent was to demonstrate that
confidence in a voting system is built on a willingness to verify its
security through testing and transparency. 

There is an historical misunderstanding in the U.S. election community
that the E2E VIV Project aimed to correct: that our country’s best
scientists are not against technology advancements, nor are they
inherently at odds with the election officials who seek technology
improvements to meet their administrative challenges. Rather, that the
U.S. scientific community takes issue with unproven claims of security
regarding existing systems that are not publicly tested or vetted. The
study aimed to recalibrate this situation. 

The group of scientific leaders on the project has often pointed out
security vulnerabilities in past systems, however, in the face of the
E2E VIV Project, they agreed on one thing: that if Internet Voting
(IV) does happen, it should be in a system that takes advantage of
end-to-end verifiability and auditability.

%~~~~~~~~~~~~~~~~~~~~~~~~~~~~~~~~~~~~~~~~~~~~~~~~~~~~~~~~~~~~~~~~~~~~~
\subsection{Definition}
\label{sec:definition}

The term E2E is often used casually without
precision. E2E-verifiability is considered a property of an election
and for the purposes of the study, an E2E-verifiable election has two
important components: first, that voters can individually check that
their ballots are cast as they intend; and second, that anyone can
check that all of the cast ballots have been accurately
tallied\footnote{Definition from Dr.~Josh Benaloh, Senior
  Cryptographer at Microsoft Research.}.

While systems of this nature have been developed in the past, none
have been broadly used or successfully commercialized; the E2E VIV
Project would make a concerted effort to be informed by these past
efforts and build upon them as appropriate. Usability factors were
also considered from the outset of the study to address the
significant challenges faced by remote and disabled voters when using
such systems to participate. A viable outcome of this study with
respect to security, auditability, and usability is intended to enable
development efforts to ensue.

For those concerned with election integrity, there is a justifiably
negative reflex in response to IV: it takes all of the problems with
current remote voting systems and adds all of the problems and
security vulnerabilities of the Internet.  

The E2E VIV Project sought to potentially make the case that use of
the Internet enables and facilitates the introduction of
E2E-verifiability (E2EV) and that the benefits of E2EV may be able to
overcome the vulnerabilities introduced by using the Internet. No
participant on this project discounted the concerns of voting over the
Internet, nor did or do they view E2EV as a magic sauce that makes the
Internet secure. Nevertheless they believe that E2EV warrants
examination in regards to the properties it achieves. These properties
are achieved even when votes are cast on untrusted devices like PCs
and transmitted over an untrusted medium such as the Internet.

The E2E VIV Project does not attempt to make the Internet
secure. Instead, E2EV negates many (although not all) of the risks of
voting via the Internet while introducing substantial new benefits
that are not found in currently deployed voting systems.

%=====================================================================
\section{Goals and Objectives}
\label{sec:goals-objectives}

%~~~~~~~~~~~~~~~~~~~~~~~~~~~~~~~~~~~~~~~~~~~~~~~~~~~~~~~~~~~~~~~~~~~~~
\subsection{Shared Goals}
\label{sec:shared-goals}

Election officials and scientists involved in elections \textbf{share}
the overall goals: that voting systems can be proven secure,
auditable, verifiable, and accessible.

This project is evidence that the needs and requests of election
officials to explore optimum ways of serving remote voters are of deep
concern to the scientific community, that they have a great motivation
to address these questions when given a constructive opportunity to do
so, and that they would like to work together, not at opposite ends,
to examine the possibilities in this realm.

%~~~~~~~~~~~~~~~~~~~~~~~~~~~~~~~~~~~~~~~~~~~~~~~~~~~~~~~~~~~~~~~~~~~~~ 
\subsection{Project Goal}
\label{sec:project-goal}

Our goal is to specify and define a system and its testing scenarios
for an online voting method that can provide both security and
confidence to voters that their selections are accurately recorded and
counted. Our assertion is that E2E-verifiability negates many,
although not all, of the risks of voting via the Internet while
introducing substantial new benefits that are not found in currently
deployed voting systems.

The project team presumes that if E2E VIV is a possible answer, or a
step toward one, we will find out and we will see if it can answer the
needs of many voters and election officials. If not, we will still
gain specific knowledge about the shortcomings, which can be further
acted upon in the future. 

%~~~~~~~~~~~~~~~~~~~~~~~~~~~~~~~~~~~~~~~~~~~~~~~~~~~~~~~~~~~~~~~~~~~~~
\subsection{Additional Objectives}
\label{sec:addit-object}

Presentation and discussion of the report with key stakeholders,
integrating their feedback, and garnering their cceptance of the
report (see below in Success section) are additional core objectives
of the project. 

%~~~~~~~~~~~~~~~~~~~~~~~~~~~~~~~~~~~~~~~~~~~~~~~~~~~~~~~~~~~~~~~~~~~~~
\subsection{Deliverables}
\label{sec:deliverables}

The main deliverable of the E2E VIV Project is the development of a
``whole product solution'' specification (or simply specification for
short) for a trustworthy E2E VIV election system.

We have produced a report presenting a system specification to create
a secure E2E VIV system, a set of testing specifications to
demonstrate the security, a set of guidelines for system usability,
accessibility, and testing. Additional topics and analyses may be
considered and discussed in the report, such as legal and
administrative challenges, and ballot secrecy, privacy, and
confidentiality.

%~~~~~~~~~~~~~~~~~~~~~~~~~~~~~~~~~~~~~~~~~~~~~~~~~~~~~~~~~~~~~~~~~~~~~
\subsection{A First Step}
\label{sec:first-step}

This project represents step one in an examination of whether one day
this might be possible. Our current plan is to examine the potential
for an E2E VIV remote voting system together with election officials,
taking into close account their needs and the needs of disabled
voters. If a system can one day be developed based on these
principles, then we want to know. We need the answers that this
project will bring before we can say whether we, or anyone, will build
any new system. A viable outcome of this study with respect to
security, auditability, and usability will enable development efforts
to ensue.

%~~~~~~~~~~~~~~~~~~~~~~~~~~~~~~~~~~~~~~~~~~~~~~~~~~~~~~~~~~~~~~~~~~~~~
\subsection{Success}
\label{sec:success}

Beyond judging the outcome – the fact that this project takes a
research and testing-based approach to a problem that has been “in
stalemate mode” will result, we believe, in stimulating election
development overall. The election industry is operating in a
traditional paradigm with only a few vendors able to survive, albeit
demand to move away from outdated, expensive, hardware-oriented
solutions.

It would be considered a success to specify a system and testing for a
usable, secure E2E verifiable remote voting technology, to identify
its strengths and weaknesses and reasons to pursue or not pursue this
approach to remote and/or disabled citizen voting.

However, from the beginning is was clear that if the project
determines that the technology is weak and should not be developed, it
would be a different outcome, but also one with many useful
implications. 

Success of the project can only be determined if the specification is
one that is: 1)~supported by the vast majority the expert teams,
including the technical, usability, testing, and research teams;
2)~endorsed by the vast majority of the advisory council, and;
3)~endorsed by the major stakeholders in elections administration as
represented by the project's local election officials.

Additionally, the E2E VIV Project expects to receive support and
endorsement from many members of the electronic voting activism
community, as represented by key members of the Election Verification
Network and the Verified Voting Foundation.

The specification will be of a form with sufficient detail such that
the following requirements are fulfilled:
\begin{description}
\item[Independent Implementation] The specification must be of
  sufficient detail and clarity that an implementation of the election
  system must be possible by an independent party without extensive
  dialog with participants in the project.
\item[Independent Validation] It must be possible for a moderately
  proficient IT expert to objectively determine, in a reasonable time
  frame with reasonable cost, if any election system constructed which
  claims to fulfill the specification.
\item[Evidence-Based Decisions] Every decision made in the crafting of
  the specification must be objectively justifiable and the evidence
  for the decision must be traceable. 
\end{description}

%~~~~~~~~~~~~~~~~~~~~~~~~~~~~~~~~~~~~~~~~~~~~~~~~~~~~~~~~~~~~~~~~~~~~~
\subsection{Scope}
\label{sec:scope}

The original project was tightly limited to involve system
specification and testing only. No system development was envisioned
in this phase beyond mockups to help test usability. However, this
changed early on in the project when Joe Kiniry of Galois, Inc. came
on board to manage the project and its team. 

The Galois engineers brought significant expertise to the project and
set about to develop a set of rigorous engineering artifacts
``demonstrators'' fit for refinement into a working election system, and
against which third parties can perform independent validation and
verification. According to Galois, demonstrators are technical
artifacts from the point of view of definition and constructions, but
non-technical artifacts from the point of view of
demonstration. Galois suggests that all demonstrators developed using
Galois IR\&D funding be: 
\begin{itemize}
\item developed in a completely transparent and public fashion within
  the Galois GitHub Organization, 
\item cross-referenced, and thus traceable to and from, all
  specification aspects (from domain models to behavioral design
  specifications), are replicated into the E2E VIV GitHub
  Organization, and
\item are licensed under either a mainstream Open Source license with
  a strong community or an alternative license tuned to the elections
  community.
\end{itemize}

Significantly, Galois is a leader in the process of computing on data
while it remains encrypted, and in the automated generation,
validation and synthesis of high assurance cryptographic
solutions. They excel in multiple areas of cryptographic
implementation, all of which can be applied to the challenge of
developing secure and usable E2E VIV voting.

Relevance of Galois’ work to the project was explicit: the aim to
apply cutting edge computer science and mathematics to solve difficult
technological problems was a clear definition of what was needed to
solve the secure, verifiable election systems development
challenge. Galois’ management agreed to donate a significant portion
of engineering time to the project in order to build “demonstrators”
that would be used to prove the concepts of E2EV and to further
examine security and usability.

%=====================================================================
\section{People}
\label{sec:people}

Inherent in the E2E VIV Project was the opportunity to combine the
abilities, knowledge, experience and expertise of a diverse group of
technologists, computer scientists and election officials involved in
election integrity together to form the overall project
team. Technical, usability, testing and local election official
sub-teams were formed for ease of communication. The technical team
has decades of experience in E2E technology, cryptography, usability,
and testing. An Advisory Council was established to broaden the
communication with interested members of the election community. 

Overseas Vote Foundation (OVF), as the official grantee, was
responsible the overall project conception, proposal development,
presentations, communications, management, team recruitment,
contractual obligations, public relations, events and budgeting. Deep
experience in the arena of overseas and military voting, absentee
voting, community building, voter survey research, election reform and
communications gave OVF a unique edge in managing the project. 

Galois, Inc. provided the technical and engineering project
management. Named as the technical project manager, Dr.~Joseph Kiniry,
working as a Principal Investigator at Galois, facilitated the
communication and decision-making of the expert teams. He became the
main author and editor of the report and ran all engineering projects
and usability aspects of the study.

%~~~~~~~~~~~~~~~~~~~~~~~~~~~~~~~~~~~~~~~~~~~~~~~~~~~~~~~~~~~~~~~~~~~~~
\subsection{Team Members}
\label{sec:team-members}

\textbf{Project Manager:} Susan Dzieduszycka-Suinat, Overseas Vote Foundation

\textbf{Lead Technical Project Manager:} Dr. Joseph Kiniry, Galois

\textbf{Technical Team}

Dr. Josh Benaloh
Senior Cryptographer, Microsoft Research
 
Dr. David R. Jefferson
Lawrence Livermore National Laboratory
 
Dr. Doug W. Jones
Associate Professor, Department of Computer Science, University of Iowa
 
Dr. Aggelos Kiayias
Associate Professor, Computer Science and Engineering, University of Connecticut
 
Dr. Olivier Pereira
Professor, Institute of Information and Communication Technologies, Electronics and Applied Mathematics, Ecole Polytechnique de Louvain
 
Dr. Poorvi Vora
Associate Professor, Department of Computer Science, The George Washington University
 
Dr. David Wagner
Professor, EECS Computer Science Division, University of California Berkeley
 
Dr. Dan Wallach
Professor, Department of Computer Science, Rice University
 
\textbf{Usability Team}

\begin{itemize}
\item Keith Instone, User Experience Consultant
\item Morgan Miller, Usability Analyst, Experience Lab
\item Dr. Judith Murray, Research Consultant
\end{itemize}

\textbf{Election Auditing}

Dr. Philip Stark
Professor and Chair of Statistics, University of California Berkeley
 
\textbf{Testing Team}

Dr. Duncan Buell
Professor of Computer Science and Engineering, University of South Carolina
 
Andrew Regenscheid
Mathematician, National Institute of Standards and Technology
 
\textbf{Advisory Council}

Dr. Ben Adida
 
Dr. Michael Clarkson
Assistant Professor of Computer Science, The George Washington University
 
Dr. J. Alex Halderman
Assistant Professor of Computer Science and Engineering, University of Michigan
 
Candice Hoke
Professor of Law, Cleveland State University
 
Dr. Ron Rivest
Vannevar Bush Professor of Computer Science, Massachusetts Institute of Technology
 
Noel Runyan
Primary Consultant, Personal Data Systems
 
Dr. Peter Ryan
Professor in Applied Security, University of Luxembourg
 
Dr. Barbara Simons
Research Staff Member, IBM Research (retired)
 
Dr. Vanessa Teague
Research Fellow, Department of Computing and Information Systems, University of Melbourne
 
John Wack
Voting Systems Standards, National Institute of Standards and Technology
 
Dr. Filip Zagorski
Assistant Professor of Computer Science, Wroclaw University of Technology
 
\textbf{Local Election Officials}

Lori Augina
Director of Elections, Washington State, Secretary of State

Rachel Bohman
Former Hennepin County Elections Manager (Minnesota)

Judd Choate
Director of Elections, Colorado, Secretary of State

Dana Debeauvoir
Travis County Clerk (Texas)
 
Mark Earley
Voting Systems Manager, Leon County (Florida)
 
Dean Logan
Los Angeles Registrar-Recorder/County Clerk (California)

Stuart Holmes
Election Information Systems Supervisor, Office of the Secretary of State (Washington)
 
Dr. Lois H. Neuman
Chair, Board of Supervisors of Elections, City of Rockville (Maryland)
 
Roman Montoya
Deputy County Clerk, Bernalillo County (New Mexico)
 
Tammy Patrick
Senior Advisor to the Democracy Project, Bipartisan Policy Center and Former Federal Compliance Officer Maricopa County (Arizona)
 
\textbf{Overseas Vote Foundation Support Team}

Susan Dzieduszycka-Suinat
President and CEO
 
Paul McGuire
Legal Counsel and Secretary of the Board
 
Richard Vogt
Treasurer and Chief Financial Officer

Capstone Project Team, Carnegie Mellon University, Heinz College,
School of Information Systems \& Management; Master of Information
Systems Management and Master of Science in Information Security
Policy and Management: in early 2014, a Capstone Team was assigned to
the project team to assist on the Comparative Analysis of E2E systems.

%~~~~~~~~~~~~~~~~~~~~~~~~~~~~~~~~~~~~~~~~~~~~~~~~~~~~~~~~~~~~~~~~~~~~~
\subsection{Stakeholder Groups}
\label{sec:stakeholder-groups}

Although not on the official project team, it was widely acknowledged
that there were several communities relevant to the E2E VIV Project
outside of those represented on the expert teams and that interaction
with members of these communities was essential. These include: 
\begin{itemize}
\item \textbf{Election verification advocates.} Election verification
  advocates are plentiful, well-informed, and strongly connected.
  They care deeply about election integrity and verifiability, and
  unsurprisingly internet voting is a hot-button issue for many of
  them.

  This negative attitude is compounded by the fact that several
  vendors have developed internet voting products which are
  proprietary, closed-source, have never seen a public audit, and are
  unverifiable.  Moreover, many of these vendors make specious claims
  about the security of their products---claims which the advocate
  community rejects entirely.  Finally, many vendors advocate
  outsourcing elections entirely to them---a condition that will never
  be deemed acceptable to the advocate community, even for an
  end-to-end verifiable internet voting system.

  A small number of advocates are for verifiable internet voting, a
  small number are adamantly against internet voting of any kind, but
  the bulk of advocates are on-the-fence.  That majority recognize
  that there are significant scientific and engineering challenges in
  designing and developing a internet voting system.  Moreover, they
  recognize that the decision to deploy such a system is very much a
  subjective, political one.  For example, in some contexts, it is
  viewed as perfectly acceptible to use a non-verifiable, outsourced
  election apparatus (such as Everyone Counts' product); say for the
  voting of the winner of a reality show.  But for government
  elections of any value, such an option is unacceptable to virtually
  every advocate.

  Consequently, being fully transparent with---and listening to the
  feedback from---the election verification advocate community is
  absolutely mandatory.  If the bulk of that community is not swayed
  by the evidence presented in this report, pursuing any next phase in
  this project will be fraught with turmoil and will be an uphill
  battle against a number of influential actors, all with good
  intentions.
\item \textbf{Standards Bodies.} Perhaps surprisingly, there is little
  national or international standardization in the area of elections.
  A nascent effort to begin standardizing data interchange formats
  began a decade or so ago and eventually fizzled after only
  producting one small standard.  

  There are a myriad of reasons why this first effort failed.  Vendors
  lobby against, and are disinterested in, interoperability.  The
  EAC's Voluntary Voting System Guidelines (VVSG) are not geared
  toward a component-based approach to system design, thus there is
  little cause for defining interfaces and data file formats, since
  devices cannot be plugged together.  Finally, there was unsufficient
  buy-in from the election research community.

  In 2015 though, this situation changed with the rebirth of the IEEE
  1622 committee focusing on elections.  The IEEE Voting System
  Standards Committee 1622 (VSSC/1622) is creating standards and
  guidelines around a common data format for election data.  The aim
  is that that future election equipment used in U.S. elections and
  abroad can interoperate more easily.  It is the intention of the
  VSSC that standards and guidelines being developed will be required
  in future versions of the EAC's VVSG.  

  Many of the top researchers, election advocates, and election
  officials in the world are a part of this committee.  Additionally,
  representatives from the major election systems vendors are either
  participating, or listening in, because they recognize that
  interoperability will be mandated by future versions of the
  VVSG.\footnote{Recall that all election systems in the U.S.A. must
    be certified at the State or Federal level according to the EAC's
    voting system testing and certification standards, standards which
    mandate compliance with the VVSG.}

  - How do standards and E2E VIV relate?  
\item \textbf{Vendors.}
\item \textbf{Hackers and Hacktivists.}
\item \textbf{Election Officials.}
\item \textbf{Citizens}
\end{itemize}

\todokiniry{Joe: Do you want to say anything about these groups?
  Should we remove this last paragraph and this list? -Susan}

%=====================================================================
\section{Methodology}

\todokiniry{Galois to write this section.}

\begin{itemize}
\item absorb all input from experts
\item read all literature on internet voting
\item write requirements and solicit feedback from technical experts
\item write personas as foundation for UX studies
\item interview LEOs based upon requirements and personas; include in
  interview information about their current elections framing
\item outline report and solicit test and input from experts
\item reflect upon latest advances in crypto for E2E VIV
\item reflect upon latest advances for reasoning about crypto
  algorithms, protocols, and implementations
\item craft a (set of) architectures and designs that reflect
  underlying requirements and crypto protocols
\item integrate all expert text and input and craft a final report TOC
\item solicit more text and reflections from experts based upon final
  report TOC
\item write chapters that were as-of-yet unwritten by experts
\item solicit input from all experts on part 1
\item solicit input from technical experts on part 2
\item gather all input from experts (good and bad, nuances,
  disagreements, etc.) and capture all in appendices, citations, and
  footnotes
\item polish and release
\end{itemize}

%=====================================================================
\section{Outcome}
\label{sec:outcome}

The E2E VIV Project produced a System Specification Development and
Documentation (referred to as a technical report for short) including
a Whole Product Solution Specification for an E2E VIV Election System
(referred to as a election system for short). The assessment of the
system by the expert team has had two possible outcomes. 

\begin{enumerate}
\item Positively, the majority of the expert team may decide that the
  specified election system meets all of the requirements set forth by
  the charter of the group. This outcome would indicate that OVF might
  potentially move forward to ensure that the election system is
  developed and, potentially, deployed.
\item Negatively, the majority of the expert team may decide that the
  specified election system does not meet all of the requirements set
  forth by the charter of the group. This outcome indicates that
  further funding to design or construct such an election system is,
  for the moment, unwise and that the community believes that
  designing a usable and secure election system is still an open
  scientific, not engineering, challenge. 
\end{enumerate}

Fulfilling the usability and security requirements would not be
sufficient for a positive assessment by the expert team. A full system
specification that is usable and secure may be, for example, far to
expensive to build, too difficult to deploy and manage, or mandate too
much expertise from election officials to operate. Social
non-functional requirements may trump technical functional
requirements. 

The Whole Product Solution Specification is written in one or more
specification languages that cover the technical needs of the E2E VIV
Project, particularly with regards to third party high-assurance
verification and validation of implementations. Galois recommended
using Alloy [10], RAISE [9], or PVS [18] to codify a formal domain
model, BON [21] to specify the election system's informal domain
model, requirements, architecture, and design, and F [8] and Cryptol
[4] to specify election system protocols. 

%~~~~~~~~~~~~~~~~~~~~~~~~~~~~~~~~~~~~~~~~~~~~~~~~~~~~~~~~~~~~~~~~~~~~~
\subsection{Deliverables Produced}
\label{sec:deliv-prod}

A set of reports and a set of demonstrators were produced. Some
elements and report section where non-technical and others not. 

Galois contends that all project results should be SMART: 
\begin{itemize}
\item Specific: the determination of whether a result is accomplished
  is as objective as possible; 
\item Measurable: major results have a tracking dashboard on the
  project website and are updated and reviewed weekly; 
\item Attainable: E2E VIV Project participants believe they can
  achieve the results they propose; 
\item Relevant: results contribute to the priorities, goals, or
  on-going; operation of the project, and offer clear value to the
  project; and 
\item Trackable: progress toward the achievement of a result is
  monitored, including project budget. 
\end{itemize}

Moreover, each result must have a customer. A customer is an
individual or group who will negotiate a result and who will be
actively engaged. They will truly care that the result is achieved and
work to make us all successful in doing so. The customers of the E2E
VIV Project are the Local and State Election Officials. 

%~~~~~~~~~~~~~~~~~~~~~~~~~~~~~~~~~~~~~~~~~~~~~~~~~~~~~~~~~~~~~~~~~~~~~
\subsection{User Interface Design}
\label{sec:user-interf-design}

The user interface (or UI for short) of the E2E VIV election system is
the critical factor in ensuring that the system is both usable and
secure. Consequently, a detailed UI design informed by usability
testing, is a mandatory component of the system specification.

\todokiniry{Joe: Are you putting forward a “detailed UI design” –
  please adjust this as needed... and finish this section. I cannot
  finish on Outcomes.... -Susan}

%=====================================================================
\section{Next Steps}
\label{sec:next-steps}

\todokiniry{Joe: I don’t know what exactly you envisioned in “next
  steps” in this Introduction, but I am not sure about having it
  covered in this section. I would recommend it at the conclusion of
  the report and as part of the Executive Summary. -Susan}

%%% Local Variables:
%%% mode: latex
%%% TeX-master: "report"
%%% End:
