\chapter{Introduction\ifdraft{ (Joe K./Susan) (95\%)}{}}
\label{chapter:introduction}

\todo{This section needs another pass to rationalize the tense and
  voice. It's too confusing to talk about ourselves and this report in
  the third person past tense, because it sounds like we're talking
  about a different artifact. I've already made some changes to first
  person, but I'd appreciate other eyes on this. -ACF}
\todokiniry{We'll have a copy-editor take care of that starting around
  15 June. -JRK}

%=====================================================================
\section{The E2E VIV Project}
\label{sec:e2e-viv-project}

%~~~~~~~~~~~~~~~~~~~~~~~~~~~~~~~~~~~~~~~~~~~~~~~~~~~~~~~~~~~~~~~~~~~~~
\subsection{Situation}
\label{sec:situation}

In March 2013, Overseas Vote Foundation’s President and CEO began a
discussion with a small group of experienced election integrity
technology advocates about how, if faced with having to specify an
Internet voting system, they would respond. Despite their concerns
about the security of Internet voting efforts to date, it was
nonetheless agreed that taking on the question was crucial at the
time.

Gridlock around Internet Voting is not unique to Washington
politics. Across the U.S., the scientific community, federal agencies,
cyber security specialists, and certain organized activists have
strongly advised against exposing the ballots of the most powerful
nation on earth to the seemingly endless range of cyber threats which
run rampant on today’s Internet.

Nevertheless, faced with ongoing challenges to serve their
constituencies in modern and efficient ways, and having experienced
the everyday efficiencies of technology throughout their lives, when
seeking new and improved election systems, election officials often
want to consider Internet-based technologies. In the current climate
of economic austerity, innovation in elections is rare. Our election
officials are trapped in a technology no-man’s land; they continue to
devote scarce resources to support outdated voting systems, while
lacking the means to certify the new voting systems they would like.

Election integrity advocates cite that secure, tested, certified
remote voting systems that election officials envision aren’t
available. The scientific community does not consider online ballot
return systems secure, nor are they certified. As a result, email has
become the default stopgap method for moving ballots online, although
it does not provide any of the benefits that a secure, full-featured
voting system would provide. Email is demonstrably insecure, yet
election officials and voters are using it regularly to transmit
ballots because viable alternatives are not available. Examination of
new and better ways to use technology to meet specific voting needs,
such as those of remote overseas citizens, military members, and
people with disabilities is needed.

Existing vendors of Internet voting technologies, whose systems are
neither tested nor certified, would like to openly market and sell
their systems within the U.S. and not face the resistance of the
election integrity advocates. No agreement on how to proceed, a
years-long history of mediocre attempts, ongoing animosity between
stakeholder parties, and a general lack of research on the current
questions are among the challenges of this situation.

%~~~~~~~~~~~~~~~~~~~~~~~~~~~~~~~~~~~~~~~~~~~~~~~~~~~~~~~~~~~~~~~~~~~~~
\subsection{A Proposed Solution}
\label{sec:proposed-solution}

Within this climate, a project proposal was written and funding
provided by the Democracy Fund, a Washington D.C.-based philanthropic
organization whose stated objective is to “...invest in organizations
working to ensure that our political system is responsive to the
priorities of the American public and has the capacity to meet the
greatest challenges facing our country.” The proposal envisioned a
project to examine the future of voting and how it might be executed
securely online; one that approached the question of Internet-voting
from a research perspective, and that sought to fill in the gaps of
the many open questions plaguing the discussion.

According to Joe Goldman, Director of the Democracy Fund, “The
significance of this project will be in its ability to break open the
conversation from its current stalemate and include all sides in a
constructive project to openly examine and research what is really
needed by voters and election officials, and to determine whether this
form of voting can meet those needs and still guarantee security of
the election. Equally important, it will identify potential tradeoffs
and shortcomings that represent the diverse range of values we hold
dear in our elections.”

On December 19, 2013, Overseas Vote Foundation\footnote{Since the
  start of the study, Overseas Vote Foundation has been renamed as
  “Overseas Vote”, an initiative of U.S. Vote Foundation.} (OVF), a
nonpartisan, nonprofit organization dedicated to overseas and military
voter participation, announced the launch of this project as the
End-to-End Verifiable Internet Voting: Specification and Feasibility
Assessment Study (E2E VIV Project). Our stated aim is to examine a
form of remote voting that enables a so-called “end-to-end
verifiability” (E2E) property. For this study, we have assembled a
unique team of experts in computer science, usability, and auditing
together with a selection of local election officials from key
counties around the U.S.

We have focused our efforts to produce a system specification and set
of testing scenarios, which if they meet the requirements for
security, auditability, and usability, would then be placed in the
public domain. At the same time, we intend to demonstrate that
confidence in a voting system is built on a willingness to verify its
security through testing and transparency.

There is a historical misunderstanding in the U.S. election community
that the E2E VIV Project aims to correct. Our country’s foremost
scientists are not against technology advancements, nor are they
inherently at odds with the election officials who seek technology
improvements to meet their administrative challenges. Rather, that the
U.S. scientific community is skeptical of unproven claims of security
regarding existing systems that are not publicly tested or vetted. This
study aims to break this impasse and reconcile these concerns.

The scientific leaders on this project have often pointed out security
vulnerabilities in past systems. If Internet Voting (IV) can happen,
it must be in a system that takes advantage of end-to-end
verifiability and auditability.
% however the E2E VIV Project has led
% us to agree that Internet Voting (IV) can happen, but that it must be
% in a system that takes advantage of end-to-end verifiability and
% auditability. 
\todo{I strengthened the claim here to make the point more clearly,
  but would appreciate a fact-check from others -ACF} 
\todo{I'm not sure who the ``us'' is in the statement I commented out,
  because I don't know that the security experts agree that IV can
  happen? I know that it is a stronger statement than I can support at
  this time. I changed it to something I could agree with. -- PLV}
\todokiniry{Agreed Poorvi---we cannot include that latter clause
  unless the bulk of the experts, after reviewing the feasibility
  chapter and proposed recommendations, are in alignment on this
  topic. -JRK}

%~~~~~~~~~~~~~~~~~~~~~~~~~~~~~~~~~~~~~~~~~~~~~~~~~~~~~~~~~~~~~~~~~~~~~
\subsection{Definition}
\label{sec:definition}

The term ``end-to-end'' is often used casually, without a precise
definition in mind. For the purposes of this study, E2E verifiability
is a property of an election system and can be split into two
sub-properties:
% with important components: 
\begin{itemize}
\item voters can individually check that their ballots are cast as they
intend
\item anyone can check that all of the cast ballots have been
  accurately tallied\footnote{Definition from Dr.~Josh Benaloh, Senior
    Cryptographer at Microsoft Research.}.
\end{itemize}
Note that it should be possible to perform the above checks even in
instances where election officials do not follow procedures and there
are errors or malware in voting technology.

While systems of this nature have been developed in the past, none
have been broadly used or successfully commercialized; the E2E VIV
Project has made a concerted effort to be informed by these past
efforts and build upon them as appropriate. Usability factors were
also considered from the outset of the study to address the
significant challenges faced by remote voters and voters with
disabilities when using such systems. This study is intended to enable
development efforts in E2E systems that are viable with respect to
security, auditability, and usability.

For those concerned with election integrity, there is a justifiably
negative reflex in response to IV: it takes all of the problems with
current remote voting systems and adds all of the problems and
security vulnerabilities of the Internet.  The E2E VIV Project has
sought to make the case that
% use of the Internet enables and
% facilitates the introduction of E2E-verifiability (E2EV), and that 
the benefits of E2EV may address some of the vulnerabilities
introduced by the use of the Internet.
\todo{Again, I commented out the overly optimistic note and replaced
  it-PLV}
\todokiniry{I'm ok with that. In fact, I would change this clause to
  be more in alignment with Josh's recommendation that the
  introduction of E2EV elections technology should first witness
  experimentation in a supervised setting, a la STAR-Vote. -JRK}

No participant in this project discounted the concerns of voting over
the Internet, nor do we view E2EV as a magic sauce that makes the
Internet secure. Nevertheless we believe that E2E properties are quite
relevant to IV, since when these properties are achieved, it is
independent of whether votes are cast on untrusted devices like PCs or
transmitted over an untrusted medium such as the Internet.
\todo{Modified the description of E2E properties-PLV}
\todokiniry{Looks good.  For these kinds of accepted edits, I'm going
  to commit a snapshot with both comments and then we'll trim them out
  during copy-editing. Thanks for your contributions, Poorvi! -JRK}

The E2E VIV Project does not attempt to make the Internet
secure. Instead, we examine how E2EV negates many (although not all)
of the risks of voting via the Internet while introducing substantial
new benefits that are not found in currently deployed voting systems.

%=====================================================================
\section{Goals and Objectives}
\label{sec:goals-objectives}

%~~~~~~~~~~~~~~~~~~~~~~~~~~~~~~~~~~~~~~~~~~~~~~~~~~~~~~~~~~~~~~~~~~~~~
\subsection{Shared Goals}
\label{sec:shared-goals}

Election officials and scientists involved in elections \textbf{share}
the overall goals: that voting systems can be proven secure,
auditable, verifiable, and accessible.

This project shows that the the scientific community cares deeply
about addressing the needs and requests of election officials as they
serve remote voters, that they have a great motivation to address
these questions when given a constructive opportunity to do so, and
that they would like to work together, not at opposite ends, to
examine the possibilities in this realm.

%~~~~~~~~~~~~~~~~~~~~~~~~~~~~~~~~~~~~~~~~~~~~~~~~~~~~~~~~~~~~~~~~~~~~~
\subsection{Project Goal}
\label{sec:project-goal}

Our goal is to specify and define a system and its testing scenarios
for an online voting method that can provide both security and
confidence to voters that their selections are accurately recorded and
counted. Our assertion is that E2E-verifiability negates many,
although not all, of the risks of voting via the Internet while
introducing substantial new benefits that are not found in
currently-deployed voting systems.

The project team presumes that if E2E VIV is a possible answer, or a
step toward one, we will find out and we will see if it can answer the
needs of many voters and election officials. If not, we will still
gain specific knowledge about the shortcomings, which will serve as a
starting point for future work.

%~~~~~~~~~~~~~~~~~~~~~~~~~~~~~~~~~~~~~~~~~~~~~~~~~~~~~~~~~~~~~~~~~~~~~
\subsection{Additional Objectives}
\label{sec:addit-object}

Our core objectives also include presentation and discussion of the
report with key stakeholders, integrating their feedback, and
garnering their acceptance of the report (see \autoref{sec:success}).

%~~~~~~~~~~~~~~~~~~~~~~~~~~~~~~~~~~~~~~~~~~~~~~~~~~~~~~~~~~~~~~~~~~~~~
\subsection{Deliverables}
\label{sec:deliverables}

The main deliverable of the E2E VIV Project is the development of a
``whole product solution'' specification (or simply ``specification''
for short) for a trustworthy E2E VIV election system.

We have produced a report presenting a system specification to create
a secure E2E VIV system, a set of testing specifications to
demonstrate the security, a set of guidelines for system usability,
accessibility, and testing. Additional topics and analyses may be
considered and discussed in the report, such as legal and
administrative challenges, and ballot secrecy, privacy, and
confidentiality.

%~~~~~~~~~~~~~~~~~~~~~~~~~~~~~~~~~~~~~~~~~~~~~~~~~~~~~~~~~~~~~~~~~~~~~
\subsection{A First Step}
\label{sec:first-step}

This project represents step one in an examination of whether one day
E2E VIV might be possible. Our current plan is to examine the
potential for an E2E VIV remote voting system together with election
officials, taking into close account their needs and the needs of
disabled voters. If a system can one day be developed based on these
principles, then we want to know. We need the answers that this
project will bring before we can say whether we, or anyone, will build
any new system. A viable outcome of this study with respect to
security, auditability, and usability will enable development efforts
to ensue.

%~~~~~~~~~~~~~~~~~~~~~~~~~~~~~~~~~~~~~~~~~~~~~~~~~~~~~~~~~~~~~~~~~~~~~
\subsection{Success}
\label{sec:success}

Beyond the concrete outcomes of this project, the fact that this
project takes a research and testing-based approach to a problem that
has been “in stalemate mode” will, we hope, stimulate election
development overall. The election industry is operating in a
traditional paradigm with only a few vendors able to survive despite
demand to move away from outdated, expensive, hardware-oriented
solutions.

We would consider it a success to produce a specification for a
usable, secure E2E-verifiable remote voting technology, to identify
its strengths and weaknesses and reasons to pursue or not pursue this
approach to remote and/or disabled citizen voting.

However, from the beginning it has been clear that if the project
determines that current techniques are weak and should not be pursued,
this would be a different outcome, but also one with many useful
implications.

A complete success for this project would be to produce a
specification that is: 1)~supported by the vast majority the expert
teams, including the technical, usability, testing, and research
teams; 2)~endorsed by the vast majority of the advisory council, and;
3)~endorsed by the major stakeholders in elections administration as
represented by the project's local election officials.

Additionally, the E2E VIV Project expects to receive support and
endorsement from many members of the electronic voting activism
community, as represented by key members of the Election Verification
Network and the Verified Voting Foundation.

The specification will be of a form with sufficient detail such that
the following requirements are fulfilled:
\begin{description}
\item[Independent Implementation] The specification must be of
  sufficient detail and clarity that an implementation of the election
  system must be possible by an independent party without extensive
  dialog with participants in the project.
\item[Independent Validation] It must be possible for a moderately
  proficient IT expert to objectively determine, in a reasonable time
  frame with reasonable cost, whether a constructed election system
  fulfills the specification.
\item[Evidence-Based Decisions] Every decision made in the crafting of
  the specification must be objectively justifiable and the evidence
  for the decision must be traceable.
\end{description}

%~~~~~~~~~~~~~~~~~~~~~~~~~~~~~~~~~~~~~~~~~~~~~~~~~~~~~~~~~~~~~~~~~~~~~
\subsection{Scope}
\label{sec:scope}

The original project was tightly limited to involve system
specification and testing only. No system development was envisioned
in this phase beyond mockups to help test usability. However, this
changed early on in the project when Joe Kiniry of Galois, Inc. came
on board to manage the project and its team.

The Galois engineers brought significant expertise to the project and
set about developing a set of rigorous engineering artifacts,
``demonstrators'', fit for refinement into a working election system,
and against which third parties can perform independent validation and
verification. According to Galois, demonstrators are technical
artifacts from the point of view of definition and constructions, but
non-technical artifacts from the point of view of
demonstration. Galois suggests that all demonstrators developed using
Galois IR\&D funding be:
\begin{itemize}
\item developed in a completely transparent and public fashion within
  the Galois GitHub Organization,
\item cross-referenced, and thus traceable to and from, all
  specification aspects (from domain models to behavioral design
  specifications),
\item replicated into the E2E VIV GitHub Organization, and
\item licensed under either a mainstream Open Source license with a
  strong community or an alternative license tuned to the elections
  community.
\end{itemize}

Significantly, Galois is a leader in the process of computing on data
while it remains encrypted, and in the automated generation,
validation and synthesis of high assurance cryptographic
solutions. They excel in multiple areas of cryptographic
implementation and secure-by-construction software, all of which can
be applied to the challenge of developing secure and usable E2E VIV
voting.

The relevance of Galois’ work to the project is clear: applying
cutting-edge computer science and mathematics to solve difficult
technological problems is needed to solve the secure, verifiable
election systems development challenge. Galois’ management agreed to
donate a significant portion of engineering time to the project in
order to build “demonstrators” that would be used to prove the
concepts of E2EV and to further examine security and usability.

%=====================================================================
\section{People}
\label{sec:people}

The E2E VIV Project is an opportunity to combine the abilities,
knowledge, experience and expertise of a diverse group of
technologists, computer scientists and election officials involved in
election integrity together to form the overall project
team. Technical, usability, testing and local election official
sub-teams were formed for ease of communication. The technical team
has decades of experience in E2E technology, cryptography, usability,
and testing. An Advisory Council was established to broaden the
communication with interested members of the election community.

Overseas Vote Foundation (OVF), as the official grantee, was
responsible for the overall project conception, proposal development,
presentations, communications, management, team recruitment,
contractual obligations, public relations, events and budgeting. Deep
experience in the arena of overseas and military voting, absentee
voting, community building, voter survey research, election reform and
communications gave OVF a unique edge in managing the project.

Galois, Inc. provided the technical and engineering project
management. Named as the technical project manager, Dr.~Joseph Kiniry,
working as a Principal Investigator at Galois, facilitated the
communication and decision-making of the expert teams. He became the
main author and editor of the report and ran all engineering projects
and usability aspects of the study.

%~~~~~~~~~~~~~~~~~~~~~~~~~~~~~~~~~~~~~~~~~~~~~~~~~~~~~~~~~~~~~~~~~~~~~
\subsection{Team Members}
\label{sec:team-members}

\todokiniry{We'll be turning this section into a more attractive
  illustration to enumerate participants in the project. -JRK}

\textbf{Project Manager:} Susan Dzieduszycka-Suinat, Overseas Vote Foundation

\textbf{Lead Technical Project Manager:} Dr. Joseph Kiniry, Galois

\textbf{Technical Team}

Dr. Josh Benaloh
Senior Cryptographer, Microsoft Research
 
Dr. David R. Jefferson
Lawrence Livermore National Laboratory
 
Dr. Doug W. Jones
Associate Professor, Department of Computer Science, University of Iowa
 
Dr. Aggelos Kiayias
Associate Professor, Computer Science and Engineering, University of Connecticut
 
Dr. Olivier Pereira
Professor, Institute of Information and Communication Technologies, Electronics and Applied Mathematics, Ecole Polytechnique de Louvain
 
Dr. Poorvi Vora
Associate Professor, Department of Computer Science, The George Washington University
 
Dr. David Wagner
Professor, EECS Computer Science Division, University of California Berkeley
 
Dr. Dan Wallach
Professor, Department of Computer Science, Rice University
 
\textbf{Usability Team}

\begin{itemize}
\item Keith Instone, User Experience Consultant
\item Morgan Miller, Usability Analyst, Experience Lab
\item Dr. Judith Murray, Research Consultant
\end{itemize}

\textbf{Election Auditing}

Dr. Philip Stark
Professor and Chair of Statistics, University of California Berkeley
 
\textbf{Testing Team}

Dr. Duncan Buell
Professor of Computer Science and Engineering, University of South Carolina
 
Andrew Regenscheid
Mathematician, National Institute of Standards and Technology
 
\textbf{Advisory Council}

Dr. Ben Adida
 
Dr. Michael Clarkson
Assistant Professor of Computer Science, The George Washington University
 
Dr. J. Alex Halderman
Assistant Professor of Computer Science and Engineering, University of Michigan
 
Candice Hoke
Professor of Law, Cleveland State University
 
Dr. Ron Rivest
Vannevar Bush Professor of Computer Science, Massachusetts Institute of Technology
 
Noel Runyan
Primary Consultant, Personal Data Systems
 
Dr. Peter Ryan
Professor in Applied Security, University of Luxembourg
 
Dr. Barbara Simons
Research Staff Member, IBM Research (retired)
 
Dr. Vanessa Teague
Research Fellow, Department of Computing and Information Systems, University of Melbourne
 
John Wack
Voting Systems Standards, National Institute of Standards and Technology
 
Dr. Filip Zagorski
Assistant Professor of Computer Science, Wroclaw University of Technology
 
\textbf{Local Election Officials}

Lori Augina
Director of Elections, Washington State, Secretary of State

Rachel Bohman
Former Hennepin County Elections Manager (Minnesota)

Judd Choate
Director of Elections, Colorado, Secretary of State

Dana Debeauvoir
Travis County Clerk (Texas)
 
Mark Earley
Voting Systems Manager, Leon County (Florida)
 
Dean Logan
Los Angeles Registrar-Recorder/County Clerk (California)

Stuart Holmes
Election Information Systems Supervisor, Office of the Secretary of State (Washington)
 
Dr. Lois H. Neuman
Chair, Board of Supervisors of Elections, City of Rockville (Maryland)
 
Roman Montoya
Deputy County Clerk, Bernalillo County (New Mexico)
 
Tammy Patrick
Senior Advisor to the Democracy Project, Bipartisan Policy Center and Former Federal Compliance Officer Maricopa County (Arizona)
 
\textbf{Overseas Vote Foundation Support Team}

Susan Dzieduszycka-Suinat
President and CEO
 
Paul McGuire
Legal Counsel and Secretary of the Board
 
Richard Vogt
Treasurer and Chief Financial Officer

Capstone Project Team, Carnegie Mellon University, Heinz College,
School of Information Systems \& Management; Master of Information
Systems Management and Master of Science in Information Security
Policy and Management: in early 2014, a Capstone Team was assigned to
the project team to assist on the Comparative Analysis of E2E systems.

%~~~~~~~~~~~~~~~~~~~~~~~~~~~~~~~~~~~~~~~~~~~~~~~~~~~~~~~~~~~~~~~~~~~~~
\subsection{Stakeholder Groups}
\label{sec:stakeholder-groups}

Although not on the official project team, there are several
communities relevant to the E2E VIV Project outside of those
represented on the expert teams, and interaction with members of these
communities has been essential. These communities include:
\begin{itemize}
\item \textbf{Election verification advocates.} Election verification
  advocates are plentiful, well-informed, and strongly connected.
  They care deeply about election integrity and verifiability, and
  unsurprisingly Internet voting is a hot-button issue for many of
  them.

  This skeptical attitude is compounded by the fact that several
  vendors have developed Internet voting products which are
  proprietary, closed-source, have never seen a public audit, and are
  unverifiable. Moreover, many of these vendors make specious claims
  about the security of their products---claims which the advocate
  community rejects entirely. Finally, many vendors advocate
  outsourcing elections entirely to them---a condition that will never
  be acceptable to the advocate community, even for an end-to-end
  verifiable Internet voting system.

  A small number of advocates are \textbf{for} verifiable Internet
  voting, a small number are adamantly \textbf{against} Internet
  voting of any kind, but the bulk of advocates are on-the-fence. That
  majority recognize that there are significant scientific and
  engineering challenges in designing and developing a Internet voting
  system. Moreover, they recognize that the decision to deploy such a
  system is very much a subjective, political one. For example, in
  some contexts, it is viewed as perfectly acceptible to use a
  non-verifiable, outsourced election apparatus (such as Everyone
  Counts' product); say for the voting of the winner of a reality
  show. But for government elections of any value, such an option is
  unacceptable to virtually every advocate.

  Consequently, being fully transparent with---and listening to the
  feedback from---the election verification advocate community is
  absolutely mandatory. If the bulk of that community is not swayed
  by the evidence presented in this report, pursuing any next phase in
  this project will be fraught with turmoil and will be an uphill
  battle against a number of influential actors, all with good
  intentions.

\item \textbf{Standards Bodies.} Perhaps surprisingly, there is little
  national or international standardization in the area of elections.
  A nascent effort to begin standardizing data interchange formats
  began a decade or so ago and eventually fizzled after only
  producting one small standard.

  There are a myriad of reasons why this first effort failed. Vendors
  lobby against, and are disinterested in, interoperability. The EAC's
  Voluntary Voting System Guidelines (VVSG) are not geared toward a
  component-based approach to system design, thus there is little
  cause for defining interfaces and data file formats, since devices
  cannot be plugged together. Finally, there was unsufficient buy-in
  from the election research community.

  In 2015 though, this situation changed with the rebirth of the
  IEEE~1622 committee focusing on elections. The IEEE Voting System
  Standards Committee~1622 (VSSC/1622) is creating standards and
  guidelines around a common data format for election data. The aim is
  that future election equipment used in U.S. elections and abroad can
  interoperate more easily. It is the intention of the VSSC that
  standards and guidelines being developed will be required in future
  versions of the EAC's VVSG.

  Many of the top researchers, election advocates, and election
  officials in the world are a part of this committee. Additionally,
  representatives from the major election systems vendors are either
  participating, or listening in, because they recognize that
  interoperability will be mandated by future versions of the
  VVSG.\footnote{Recall that all election systems in the U.S.A. must
    be certified at the State or Federal level according to the EAC's
    voting system testing and certification standards, standards which
    mandate compliance with the VVSG.}

  Standards are critical to any future work on E2E VIV systems for
  several reasons.  First, given the compositional nature of most E2EV
  election systems' designs, it is not unreasonable to expect that
  different subsystems will be created and supported by different
  organizations.  Second, in order to effect arbitrary third party
  verification, clearly defined common data formats must be used.
  Third, at some point in the future, if E2E VIV systems are accepted
  in the mainstream, they must be certified by the EAC.  As such, EAC,
  NIST, and IEEE standards must recognize their core capabilities,
  subsystems, interfaces, and what constitutes legitimate evidence of
  correctness and security. Moreover, it is sensible to presume that
  there should be a standard means by which these aspects and evidence
  are documented to expedite a sound, accurate, and expedited
  certification process.

\todo{NIST has held a couple of workshops on E2E voting systems in an
  effort to understand what might be required of the standardization
  process. Andrew Regenscheid would be a good person to check with re:
  what NIST or EAC/TGDC might have done recently--PLV}

\todokiniry{I'll directly approach Andrew, whom I know and is on the
  external advisory group of this project, for that kind of specific
  input. -JRK}

\item \textbf{Vendors.}

\item \textbf{Hackers and Hacktivists.}

\item \textbf{Election Officials.}

\item \textbf{Citizens}

\end{itemize}

\todokiniry{Joe: Do you want to say anything about these groups?
  Should we remove this last paragraph and this list? -Susan}
\todokiniry{I'm writing this section today. -JRK}

%=====================================================================
\section{Methodology}

\todokiniry{Galois to write this section.}

\begin{itemize}
\item absorb all input from experts
\item read all literature on Internet voting
\item write requirements and solicit feedback from technical experts
\item write personas as foundation for UX studies
\item interview LEOs based upon requirements and personas; include in
  interview information about their current elections framing
\item outline report and solicit test and input from experts
\item reflect upon latest advances in crypto for E2E VIV
\item reflect upon latest advances for reasoning about crypto
  algorithms, protocols, and implementations
\item craft a (set of) architectures and designs that reflect
  underlying requirements and crypto protocols
\item integrate all expert text and input and craft a final report TOC
\item solicit more text and reflections from experts based upon final
  report TOC
\item write chapters that were as-of-yet unwritten by experts
\item solicit input from all experts on part 1
\item solicit input from technical experts on part 2
\item gather all input from experts (good and bad, nuances,
  disagreements, etc.) and capture all in appendices, citations, and
  footnotes
\item polish and release
\end{itemize}

%=====================================================================
\section{Outcome}
\label{sec:outcome}

This project has produced a high-level system specification, in the
form of a set of business and technical requirements.  Accompanying
that set of requirements are recommendations about the underlying
means by which those requirements should be fulfilled.  

These recommendations focus on the three dimensions necessary to
fulfill the very strenuous requirements of any E2EV system:
cryptography, architecture, and engineering.  The cryptographic
foundation is a formal framework in which to evaluate E2EV
cryptographic protocols.  The architecture specification is a formal
description of a parameterized architecture space in which solutions
can be designed, built, and evaluated.  Finally, the rigorous
engineering necessary to create a high-assurance E2E VIV product is
spelled out via a recommended software engineering process,
methodology, and set of technologies which, when used properly, can be
used to fulfill the security and correctness technical requirements
while generating the necessary evidence that the system is
fit-for-purpose.

Given these underlying recommendations, the assessment of the system
by the expert team has had two possible outcomes:
\begin{enumerate}
\item Positively: the majority of the expert team may decide that the
  specified election system meets all of the requirements set forth by
  the charter of the group. This outcome would indicate that OVF might
  potentially move forward to ensure that the election system is
  developed and, potentially, deployed.
\item Negatively: the majority of the expert team may decide that the
  specified election system does not meet all of the requirements set
  forth by the charter of the group. This outcome indicates that
  further funding to design or construct such an election system is,
  for the moment, unwise and that the community believes that
  designing a usable and secure election system is still an open
  scientific, not engineering, challenge.
\end{enumerate}

Fulfilling the usability and security requirements would not be
sufficient for a positive assessment by the expert team. A full system
specification that is usable and secure may be, for example, far too
expensive to build, too difficult to deploy and manage, or mandate too
much expertise from election officials to operate. Social
non-functional requirements may trump technical functional
requirements.

%~~~~~~~~~~~~~~~~~~~~~~~~~~~~~~~~~~~~~~~~~~~~~~~~~~~~~~~~~~~~~~~~~~~~~
\subsection{User Interface Design}
\label{sec:user-interf-design}

The user interface (or UI for short) of the E2E VIV election system is
the critical factor in ensuring that the system is simultaneously
usable, accessible, and secure. Consequently, a detailed UI design
informed by user experience (UX, for short) and accessibility testing
is a mandatory component of a future detailed system specification.

At the moment, this report's system specification is focused on
high-level requirements for any E2EV election system (internet
voting-based or not).  As such, a number of requirements focus on UI
and UX and stipulate the necessary framing for UI/UX designs and
studies.

Usability and accessibility studies and testing is a key component of
this report, and the outcome of this first study is meant to inform
the UI design of any future E2E VIV systems. As such, most of the
effort relating to UI and UX within the project is focused on
developing a technical infrastructure and complementary process for
the efficient definition and execution of qualitative and quantitative
usability and accessibility studies.

% Text to move starts here...

In order to effect these goals, a demonstration system that mimics a
voter's interaction with an E2E VIV system has been developed by
Galois. That system is a variant of the STAR-Vote system designed by
Wallach et al.~\cite{star-vote}. STAR stands for Secure, Transparent,
Auditable, and Reliable. STAR-Vote is an end-to-end verifiable ballot
marking device. As such, it is not designed for, or meant to be used
for, Internet voting. But insofar as its voting process is identical
to that of most of E2E VIV election schemes in the literature, we
decided to use it as a demonstration vehicle for usability and
accessibility experiments.

The Galois STAR-Vote implementation has a web-based UI, thus can be
used and demonstrated remotely for interactive and non-interactive
experiments to gather both qualitative and quantitative feedback.
Several variants of STAR-Vote have been implemented for UX testing.
These variants include simple changes---like different typeface
choices and sizes, background colors, supporting images, help text,
mouse pointer graphics, etc.---as well as more complex changes---like
different voter, challenge, and audit workflows.

In an interactive, qualitative experiment, a facilitator and a voter
communicate using a video chat system such as Skype and the voter
shares their desktop with the facilitator. Optimally, the facilitator
is someone who is deeply familiar with the issues of E2E VIV systems,
is familiar with STAR-Vote, and has expertise in usability and
accessibility. The voter then uses (one of several variants of)
STAR-Vote, voicing their thoughts and feelings about their experience
in real-time. After the voter has completed their participation in the
demonstration election, the facilitator uses a script to query them
about their impressions.

For a non-interactive, quantitative experiment, voters will be
solicited via social media, mailing lists, etc. to experiment with
(variants of) STAR-Vote. Sample voters in these experiments are given
ample information about what kinds of information is being collected
about their behavior so that they can make a fully-informed judgement
about their participation.

Various quantitative measures related to voter participation and
interaction can be measured automatically, both within their web
browsers and on the STAR-Vote server. Most of this data is akin to the
analytics that any professional website collects about its users: How
do voters navigate the site?  Where does a voter pause for a long time
and read?  When does a voter ask for help?  When does a voter hover
over a button a long time before they decide to click it?  How often
do voters challenge ballots or verify their votes?  How often do
voters examine the bulletin board?  Is there a correlation between the
interactive behavior of a voter while voting and their likelihood of
voting, challenging, or auditing correctly?

% ...and ends here.

The initial usability study that was conducted gathered qualitative
feedback from several dozen voters.  Based upon that study, whose full
report is included in \autoref{appendix:usability_study}, three
conclusions were reached about voters' mental models about internet
voting.

\begin{enumerate}
\item \emph{Voters intrinsically trust the voting system and their
    election officials.}  This trust is both a blessing and a curse.
  A blessing because it means that election officials and the
  government are working with voters that have a positive disposition
  and trust can only be lost and does not need to be won. But it is
  also a curse because it means that voters will trust internet voting
  systems that have no transparency, end-to-end security, or
  verifiability. As such, this opens the door for existing vendors to
  sell their technology and gather naive voter feedback as
  ``evidence'' that their systems are fit-for-purpose for modern
  public elections.
\item \emph{While many voters are supportive of having verifiability
    in their election system, very few voters are interested in
    learning about verifiability and taking the necessary time and
    energy to perform verifiability.} Happily, only a small fraction
  of voters needs to actually verify for most E2E VIV election schemes
  to generate sufficient indendent evidence that the election outcome
  is correct. As such, our requirements and recommendations reflect
  this level of voter engagement and suggest several means by which
  the threshold for election verification is always achieved.
  \todokiniry{Make sure requirements for verifiability and our
    recommendations are strengthened given the UX study. -JRK}
\item \emph{Voter expect the internet voting experience to be somehow
    different from a traditional voting experience, be it on paper
    ballot or computer-assisted in a supervised setting.} It was not
  uncommon for voters to say, ``Huh, that's it?!'' after completing
  their voting process. Voters' mental models trained to expect
  modern, rich web surfing experiences a la Facebook, YouTube, and
  Amazon and novel, interactive user experiences on their smart phones
  like Siri, Google Maps, and Uber. Consequently, this raised
  expectation opens the door for novel experimentation for 21st
  century voter experiences.
\end{enumerate}

\todokiniry{Joe: Are you putting forward a “detailed UI design” –
  please adjust this as needed... and finish this section. I cannot
  finish on Outcomes.... -Susan}
\todo{this seems too detailed for Intro -ACF}
\todokiniry{Agreed, I will move the detailed parts of this section
  elsewhere to complement the UX study results that are being dropped
  into this report. I'll do that in the next commit or so. -JRK}

%=====================================================================
\section{Next Steps}
\label{sec:next-steps}

\todokiniry{Joe: I don’t know what exactly you envisioned in “next
  steps” in this Introduction, but I am not sure about having it
  covered in this section. I would recommend it at the conclusion of
  the report and as part of the Executive Summary. -Susan}

%%% Local Variables:
%%% mode: latex
%%% TeX-master: "report"
%%% End:
