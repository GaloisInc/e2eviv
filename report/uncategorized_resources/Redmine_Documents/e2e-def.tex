\documentclass[11pt,letterpaper]{article}
\usepackage{url,graphicx,amssymb,amsfonts}
%\input{psfig.sty}

\newcommand{\ignore}[1]{}



%%%%% Func Environments %%%%%%%
\newenvironment{nffunc}[2][]{
\begin{center}
   \begin{tabular}{|ll|}\hline
     \hspace{.3ex}\begin{minipage}{.99\linewidth}\vspace{0.5ex}
       {\begin{center}{\bf Functionality}
           #2 \end{center}}\vspace{-2ex}%#3
%\begin{itemize}\setlength{\itemsep}{0mm} \setlength{\parskip}{0pt}
% \setlength{\parsep}{0pt}
%\vspace{-1ex}%\vspace{-3ex}
       }{%
 %      \end{itemize}
       \vspace{-1ex}
       \smallskip
     \end{minipage}& \\
     \hline
   \end{tabular}
   \end{center}
}


\newcommand{\func}[1][\relax]{\ensuremath{\mathcal{F}_{\mathsf{#1}}}}

\newcommand{\aknote}[1]{{\color{red} {\sf AK's note:} #1}}


\usepackage{geometry}
\geometry{letterpaper,top=1.5in,bottom=1.5in,left=1in,right=1in}

%Various
\newcommand{\fl}[1]{\mbox{\( \lfloor #1 \rfloor \)}}
\newcommand{\pair}[2]{\mbox{\(\langle #1,#2 \rangle\)}}

\newcommand{\mc}{\mathcal}
\newcommand{\Pro}{\mbox{\( \mathbf{ Prob } \)}}

\def\squareforqed{\hbox{\(\blacksquare\)}}
\def\qed{\ifmmode\squareforqed\else{\unskip\nobreak\hfill\penalty50\hskip1em\null\nobreak\hfil\squareforqed\parfillskip=0pt\finalhyphendemerits=0\endgraf}\fi}

\newtheorem{theorem}{Theorem}[section]
\newtheorem{assumption}{Complexity Assumption}
\newtheorem{lemma}[theorem]{Lemma}
\newtheorem{remark}{Remark}
\newtheorem{claim}{Claim}
\newtheorem{fact}[theorem]{Fact}
\newtheorem{definition}[theorem]{Definition}
\newtheorem{corollary}[theorem]{Corollary}
\newtheorem{proposition}[theorem]{Proposition}
\newenvironment{proof}{\noindent {\em Proof.}}{\medskip}

\def\PPT{{\rm PPT} }
\def\ff{\mathbb{F}}
\def\sbs{\subseteq}
\def\zz{\mathbb{Z}}
\def\E{\mathsf{E}}
\newcommand{\Pro}{\mbox{\( \mathsf{ Prob } \)}}

%Functionalities
\newcommand{\fete}{\func[e2e]}%
\newcommand{\fbb}{\func[bb]}%
\newcommand{\fsauth}{\func[sauth]}%
\newcommand{\fauth}{\func[auth]}%
\newcommand{\fmail}{\func[mail]}%
\newcommand{\frnd}{\func[rnd]}%



%Parties
\def\EA{\mathsf{EA}}
\def\VT{\mathsf{V}}
\def\RB{\mathsf{RB}}
\def\VC{\mathsf{VC}}
\def\AU{\mathsf{AU}}

\def\Exec{\mathsf{Exec}}


%Commands
\def\Create{\mathtt{Create}}
\def\Sample{\mathtt{Sample}}
\def\FakeBallot{\mathtt{FakeBallot}}
\def\Deliver{\mathtt{Deliver}}
\def\Corrupt{\mathtt{Corrupt}}
\def\Verify{\mathtt{Verify}}
\def\Tally{\mathtt{Tally}}
\def\Vote{\mathtt{Vote}}
\def\RecordVote{\mathtt{RecordVote}}
\def\Append{\mathtt{Append}}
\def\Read{\mathtt{Read}}
\def\Send{\mathtt{Send}}
\def\Mail{\mathtt{Mail}}
\def\Audit{\mathtt{Audit}}
\def\VBB{\mathtt{VBB}}
\def\SendInit{\mathtt{SendInit}}
\def\GetRnd{\mathtt{GetRnd}}
\def\Select{\mathtt{Select}}
\def\RecordTally{\mathtt{RecordTally}}
\def\ReadTally{\mathtt{ReadTally}}
\def\Receipt{\mathtt{Receipt}}
\def\Result{\mathtt{Result}}
\def\ElectionFail{\mathtt{ElectionFail}}


\begin{document}

\title{An ideal functionality for End-to-End Verifiable Elections}

%\titlerunning{}

\author{  Aggelos Kiayias\footnote{National and Kapodistrian University of Athens, {\tt aggelos@di.uoa.gr}.}  } 

%\date{Version 0.2}


\maketitle


\begin{abstract}
We define the problem of realizing End-to-End Verifiable elections (E2E) in a simulation-based
sense. 
\end{abstract}




\section{The ideal functionality expressing E2E verifiable elections}

We introduce an ideal functionality capturing the operation of
end-to-end verifiable election systems. The ideal functionality
called $\fete$ recognizes and interacts with  the election authority
$\EA$, the set of eligible voters $V_1,\ldots, V_n$ and the auditor $\AU$.  These are ``ideal-world'' entities.
Note that a  ``real-world'' implementation of $\fete$ may involve more parties 
that will enable the implementation to realize the ideal functionality.

The ideal functionality $\fete$ accepts a number of commands from the election authority $\EA$, 
the voters and the auditors. At the same time it informs the (ideal world) 
adversary of certain actions that take place and also is influenced
by the adversary to perform certain actions. The ideal functionality keeps track
of which parties are corrupted and may act according to their corruption status. 

There are two parameters for the ideal functionality $\fete$: (i) a function $f: (X\cup \{\bot\})^n\rightarrow E$ 
that defines the election function where $X$ defines  the set of all possible ways to vote for a single voter
and $E$ is the set of all possible election results. The notation $X^n$ defines all possible strings of length $n$ 
over the alphabet $X$. The symbol $\bot$ stands for ``undefined.''
The election function $f$ is invariant with respect to $\bot$, i.e.,  $f(\bot, x)  = f(x)$ for all $x$ and so on. 
(ii) a relation $Q$ that defines the level of sensitivity to manipulation that $\fete$ may permit. In particular 
for two possible election results $T, T'$ we say that $Q(T,T')$  holds if and only if $T'$ is sufficiently
close to $T$. For the most strict  version of $\fete$ one will define   $Q$ to be the equality
relation over $E$ (the reader may consider only this case in a first reading). 

The ideal functionality  $\fete^{f,Q}$ captures the following set of security characteristics: 

\begin{itemize}
\item Provided the $\EA$ is not corrupted, the adversary is incapable of extracting  the choices of the
voters.  

\item Provided the $\EA$ is not corrupted, all votes are recorded and tallied according to the election function $f(\cdot)$. 

\item Even in the case that $\EA$ is corrupted, a set of well defined votes are assigned to the voters of the election
(however  such votes may deviate from the original voters' intent). Note that the votes cannot be manipulated when 
the $\EA$ is honest. 

\item Even in  case that $\EA$ is corrupted, the functionality returns consistently 
the same tally result to all parties that request it. Furthermore, the functionality always tests
the reported tally according to the predicate $Q$ and reports the outcome, 
hence any substantial (according to $Q$) deviation from the recorded tally will be  detectable by all
honest parties. 

\item The functionality preserves the voter intent, and in case of vote manipulation, the voter or an auditor
can use the unique receipts provided in the completion of ballot-casting to test whether the original 
voter intent was  manipulated by a corrupt $\EA$. Any party may use those receipts and hence verification is  ``delegatable.'' 
\end{itemize}

Consider now a protocol $\pi$ that is implementing syntactically the ideal functionality
$\fete$ (i.e., has the same I/O characteristics as $\fete$). Following standard
notation and terminology we have the following : 


\begin{definition}
Let $f$ be an election function and $Q$ a predicate over the election results. 
The protocol $\pi$ implements $\fete^{f,Q}$  provided that for all adversaries $\mc{A}$, there
is a simulator $\mc{S}$ so that for all environments $\mc{Z}$ it holds that 
\[ \Exec_{\pi, \mc{A}, \mc{Z}} \approx \Exec^{\fete^{f,Q}}_{\mc{S},\mc{Z}} \]
\end{definition}

Note the  protocols that will be  considered in practice may   utilize other simpler ideal
functionalities. In such case, the protocol $\pi$ implements $\fete$ conditional on the
existence and availability of these other functionalities. Such functionalities include
``authenticated channels'', a ``bulletin board'' etc.

\paragraph{Party corruption.} As stated, the ideal functionality $\fete$ enables the 
adversary $\mc{A}$ to corrupt parties by issuing special $(\Corrupt, P)$ messages. 
Given such a message the ideal functionality $\fete$ will divulge to the adversary
the complete I/O transcript from the interface between $\fete$ and $P$. We distinguish
between static and adaptive corruptions. In the case of static corruptions
all the messages $(\Corrupt,P)$ should be  delivered at the onset of the execution
while for adaptive corruptions they are delivered at any time. 
For brevity we do not include explicitly the actions taken for $\Corrupt$
messages in the description of the functionality. 

In the real-world the corruption of an entity expresses the action that is taken by the adversary that
results in the complete control of  the entity's computing environment. A corrupted voter, specifically, looses
privacy completely and the adversary may take any action on her behalf. Specifically, 
if corruption happens prior to ballot-casting the adversary may vote on her behalf,
while if corruption happens after ballot-casting the adversary will learn 
her choice. On the other hand, if the adversary corrupts the $\EA$, it may try to manipulate
some of the voter's ballots however only up to the extend that is permitted by $\fete$
(no matter how many parties are corrupted $\fete$ always has the ``upper hand''). 


\begin{nffunc}{$\fete^{f,Q}$}
The functionality recognizes and interacts with the following parties: (i) the election authority $\EA$,
 (ii)  the   eligible voters $\mc{V}= \{V_1,\ldots, V_n\}$, 
 (iii) the auditor $\AU$. 
 (iv) the adversary $\mc{A}$.  It is parameterized by the relation $Q$ over $E$ and
 the election function $f:(X\cup\{\bot\})^n \rightarrow E$.

\begin{itemize}
\item Upon receiving an input $(\Create, sid,  B)$ from the $\EA$, 
record the tuple so that $sid$ is the election identifier and 
$B$ is a string defining the ballot of the election.
Send $(\Create, sid, B)$ to the adversary $\mc{A}$. 

\item Upon receiving an input $(\Deliver, sid)$ from $\EA$,
deliver $(B, s_i)$ to each voter $V_i$ where $s_i$ is some voter-specific
information that is provided by the adversary $\mc{A}$.\footnote{In some systems, 
voters may request this information actively and hence  $\fete$ will be passive 
and will not deliver the ballots.
In such  case the adversary will adaptively provide the $s_i$ values.}

\item Upon receiving an input $(\Vote, sid, a)$ from $V_i$, 
select a unique identifier $vid$ and record $(vid, a)$ provided $a\in X$. 
If $\EA$ is honest,
notify the adversary $\mc{A}$ with message $(\Vote,sid, vid)$ while 
if $\EA$ is corrupted send $(\Vote, sid, vid, V_i, a)$ 
to $\mc{A}$.\footnote{In some systems the voter identity $V_i$ is  leaked to
the adversary during ballot-casting.}

\item 
Upon receiving $(\RecordVote, sid, vid, b)$ from $\mc{A}$ verify that 
a symbol $(vid, a )$ has been recorded before and then
record the triple $(V_i, a, b)$ provided
that  
(i) $V_i$ is a voter that has not been assigned any vote previously\footnote{In some systems
the voter is allowed to change his/her mind and hence vote multiple times.}, 
(ii) the value $a$ is a valid choice consistent with the ballot description $B$.
(iii) the value $b$ is unique. 
Finally return $(\Receipt, b)$ to $V_i$. 

\item Upon receiving an input $(\Tally, sid)$ from the $\EA$, 
collect all recorded inputs  
$\{(V_j, a_j, b_j)\}_{j\in \tilde{\mc{V}}}$ where $\tilde{\mc{V}}$ is the set of voters that voted successfully
and set  $a_j = \bot$ for all $j\not\in \tilde{\mc{V}}$. 
Compute $T =f(\langle a_{1}, \ldots, a_n \rangle )$ and return $(\Tally, T)$ to $\mc{A}$.

\item Upon receiving $(\RecordTally, sid, \mc{M}, \hat{T})$ from  $\mc{A}$,
where $\mc{M}$ can be parsed as a polynomial-size circuit,
 set $\langle a'_{1}, \ldots, a'_n \rangle = \mc{M}(a_1,\ldots,a_n)$ and if $\exists j: (a'_j \neq a_j)$ and $\EA$ is honest
then ignore the message. In any other case, the functionality records $(\Result, T', \hat{T})$ 
and $\langle a'_{1}, \ldots, a'_n \rangle$
where $T'$ is the election 
result calculated as $T' = f(\langle a'_{1}, \ldots, a'_n \rangle )$. 

\item Upon receiving $(\ReadTally, sid)$ from any party, 
return $(\Result, \hat{T}, Q(\hat{T},T'))$.

\item Upon receiving $(\Audit, b)$ from  from any party, 
recover the triple $(V_j, a_j,b_j)$ such that $b_j = b$ and return 1 if and only 
if $(a_j' = a_j)$.

\end{itemize}
\end{nffunc}


\newpage

\section{Claims regarding the ideal functionality $\fete$}

\begin{claim}
Assuming the $\EA$ is not corrupted, the ideal functionality $\fete^{f,Q}$ leaks no information about 
how honest voters vote, except for the information that is revealed from  
the partial tally  of the votes of the honest voters (according to $f$). 
\end{claim}

\begin{claim}
The adversary may delay the recording of an honest voter's ballot, however when it is
recorded the voter obtains a receipt that enables her to verify that her vote has been
properly recorded and tallied. 
\end{claim}

\begin{claim}
The receipts the voters obtain after the vote is recorded are
unique and assuming the $\EA$ is honest
they are independent of the way the voters have voted
and hence they can safely be passed around to e.g.,  a third
party auditor $\AU$. 
\end{claim}


\begin{claim}
When the $\EA$ is corrupted it is possible for the adversary to manipulate all the votes 
(via computational manipulation $\mc{M}(a_1,\ldots, a_n)  = (a_1', \ldots, a'_n)$)
and even provide an incorrect tally $\hat{T}$; nevertheless, the ideal functionality ensures
that honest parties are notified about whether 
 the reported tally $\hat{T}$ and the recorded tally $T'$ satisfy the relation $Q$,
i.e., it returns $Q(T',\hat{T})$.
\end{claim}

\section{Security notions not captured by the ideal functionality}

We (intentionally) left out from this rendering of 
 the end-to-end functionality  $\fete$ a number of security aspects. 

\begin{itemize}
\item Denial of service attacks. The ideal functionality as written enables the adversary
to deny voters from completing the ballot-casting protocol and prevent the tally
from becoming available. From a definitional point of view, expressing such level of security
is feasible by assuming certain qualities of the underlying communication and 
message passing mechanisms that are employed in the implementation. One way to extend the functionality 
to capture such a setting is to oblige the adversary to deliver the $(\RecordVote)$
and $(\RecordTally)$ messages by certain deadlines. In order to do this formally, 
a notion
of time will have to be introduced in the model. This may
be achieved by introducing a global clock functionality.

\item Coercion via corrupting voters. Even though $\fete$ does not permit coercion
via the receipts it provides, the adversary may still achieve coercion by corrupting a voter
(e.g., hacking into the voter's PC). If this happens after the ballot-casting protocol, 
$\fete$ reveals the choice of the voter and hence the voter can be vulnerable  to coercion. 
Addressing this in the model is feasible by further restricting the information that is
divulged when voter corruption takes place. Various intermediate levels of corruption
may be considered e.g., is the voter capable of erasing some information? rewriting some information? and so on.

\item Sybil attacks. The set of voters $V_1,\ldots,V_n$ is predetermined and integrated into
the functionality $\fete$. Hence, the adversary cannot manipulate the list of voters.
It follows that $\fete$ is applicable to the setting where
the list of voters is predetermined, assumed to be public and the adversary 
may not tamper with it. 
\end{itemize}


\end{document}