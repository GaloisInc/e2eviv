\chapter{Feasibility\ifdraft{ (\emph{Joe, David, et al.}) (25\%)}{}}
\label{chapter:feasibility}

In the first fifty-odd pages of this report, in Chapters 2 through 5,
we laid out the motivation for, history of, and requirements on a
remote voting system that both experts demand and the public will
trust.  

Then in Chapters 6 through 8 we described the necessary cryptographic,
architecture, and engineering foundations, tools, and techniques
necessary for designing and building a system that fulfills the
criterion set out in Chapter 5's requirements.

But just because it seems \emph{possible} to design and develop such a
system, does not mean that it is \emph{feasible} to do so.

This chapter analyses the question of feasibility from several
dimensions, some of which are \emph{technical} (correctness, security,
usability, availability) and others \emph{non-technical} (law,
politics, fiscal, research, development, operational, and
business). After discussing each of these semi-orthogonal dimensions,
we summarize with an integrated feasibility analysis, focusing on the
question of ``Does it makes sense to practically tackle the problem of
E2E-VIV at this time?''

In what follows feasibility is determined by a principled examination
of the current state-of-affairs based upon the peer-reviewed
literature, extensive conversations with those responsible for
elections, multi-year dialogs with election verification activists,
and decades of experience in the design and development of secure
high-assurance systems.

The final determination of feasibility is not up to us, as authors of
this report. This is a decision, for the most part, that must be made
by those organizations with resources that can be brought to bear on
this problem. The aligners with that decision---primarily the activist
community and legislatures---while important, are not critical at this
point in time, given the deployment velocity of unverifiable internet
voting systems worldwide.

%=====================================================================
\section{Technical Feasibility Analysis}

We first look at technical feasibilty because, at its core, if
designing and constructing a formally verified, secure E2E-VIV system
is not possible, then we need not analyze any other feasibility
dimension.

\subsection{Engineering for Correctness and Security}

As previously mentioned, the state-of-the-art in formal verification
has advanced tremendously over the past fifteen years. High-assurance
or formally verified operating systems and hypervisors such as
seL4~\cite{seL4}, Miraga, and HaLVM were only a pipedream in the year
2000. Likewise, such is the case for formally verified compilers (such
as CompCert~\cite{CompCert}), static analysis tools (such as Verasco),
and verification tools (such as VST~\cite{VST}) and
verification-centric programming languages (such as
Dafny~\cite{Dafny}). All of this technology is supported by incredible
advances in mechanical theorem proving, particularly for SAT, SMT,
constraint solving, and logical frameworks.

The state-of-the-art is the design, development, and analysis of
secure systems has also progressed tremendously. Powerful open source
static analysis tools (such as Uno~\cite{Uno}), fuzzers (such as the
American fuzzy lop~\cite{afl}), protocol specification and reasoning
frameworks (such as EasyCrypt~\cite{EasyCrypt} and F\*~\cite{Fstar})
are all publicly available and can be applied to commercial systems.

The only thing preventing the design and development formally verified
correct and secure evidence-based systems is market pressure. Only a
very small number of organizations have the necessary
resources---primarily in the form of people and knowledge---to tackle
these kinds of challenges. And these companies cannot tackle such
challenges without clients who provide requirements, funding, and
time.

As such, \emph{designing and developing an E2E-VIV system is technical
  feasible}.  In fact, we can estimate---based upon the size and
complexity of the relevant protocols and subsystems---the effort
necessary to accomplish such a goal, as effort relates directly to
fiscal demands. Such an analysis is provided below
in~\autoref{sec:fiscal}. 

Consequently, technical feasibilty is not a limitation of a phase two
of this project.

\subsection{Design and Engineering for Usability}

Recall that security and usability are often at odds with each other.
This conflict is especially prevelent in early E2E-V election systems
such as Helios, Prêt à Voter, and RIES.

Thus, the following question arises: Is it feasible to design and
develop an E2E-VIV system which follows universal design principles?
This question demands that we speculate, as there is little objective
evidence that such a presupposition is valid.

Usability experts claim that universal design of elections systems is
a reasonable and obvious thing to do. Several organizations have
extensive experience in this area, such as those led by Whitney
Quesenbery and Dana Chisnell. And voting systems that are under
development---such as Los Angeles County's VSAP project and Travis
County's STAR-Vote system---mandate universal design. In fact, in the
case of L.A.'s system, several million dollars have already been spent
on such work.

Consequently, while there are open research questions about certain
aspects of usable E2E-VIV system---particularly that of voter ritual
and verifiability---usability experts agree that a universal design
E2E-VIV system is possible in principle. It is agreed that, in order
to achieve such a universal design, a long-term, in-depth, qualitative
and quantitative usability study based based upon a working
demonstration system is necessary.

% Text to move starts here...

In order to effect this goal, a demonstration system that mimics a
voter's interaction with an E2E-VIV system has been developed by
Galois. That system is a variant of the STAR-Vote system designed by
Wallach et al.~\cite{star-vote}. STAR stands for Secure, Transparent,
Auditable, and Reliable. STAR-Vote is an end-to-end verifiable ballot
marking device. As such, it is not designed for, or meant to be used
for, Internet voting. But insofar as its voting process is identical
to that of most of E2E-VIV election schemes in the literature, we
decided to use it as a demonstration vehicle for usability and
accessibility experiments.

The Galois STAR-Vote implementation has a web-based UI, thus can be
used and demonstrated remotely for interactive and non-interactive
experiments to gather both qualitative and quantitative feedback.
Several variants of STAR-Vote have been implemented for UX testing.
These variants include simple changes---like different typeface
choices and sizes, background colors, supporting images, help text,
mouse pointer graphics, etc.---as well as more complex changes---like
different voter, challenge, and audit workflows.

In an interactive, qualitative experiment, a facilitator and a voter
communicate using a video chat system such as Skype and the voter
shares their desktop with the facilitator. Optimally, the facilitator
is someone who is deeply familiar with the issues of E2E-VIV systems,
is familiar with STAR-Vote, and has expertise in usability and
accessibility. The voter then uses (one of several variants of)
STAR-Vote, voicing their thoughts and feelings about their experience
in real-time. After the voter has completed their participation in the
demonstration election, the facilitator uses a script to query them
about their impressions.

For a non-interactive, quantitative experiment, voters will be
solicited via social media, mailing lists, etc. to experiment with
(variants of) STAR-Vote. Sample voters in these experiments are given
ample information about what kinds of information is being collected
about their behavior so that they can make a fully-informed judgement
about their participation.

Various quantitative measures related to voter participation and
interaction can be measured automatically, both within their web
browsers and on the STAR-Vote server. Most of this data is akin to the
analytics that any professional website collects about its users: How
do voters navigate the site?  Where does a voter pause for a long time
and read?  When does a voter ask for help?  When does a voter hover
over a button a long time before they decide to click it?  How often
do voters challenge ballots or verify their votes?  How often do
voters examine the bulletin board?  Is there a correlation between the
interactive behavior of a voter while voting and their likelihood of
voting, challenging, or auditing correctly?

\subsection{Availability}

A system that is correct, secure, and usable is still not useful to
voters if it is unavailable during an election. Many government
websites are notoriously unreliaable in this regards, especially
during a distributed denial-of-service (DDos) attack or just after a
security breach.

But commercial websites for major corporations whose very business
depends upon having a highly available presence on the Internet have
effectively solved this problem. Companies like Amazon, Google, and
Facebook have uptimes comparable to that which would be necessary to
run a public election, even in the face of serious DDoS threats.

The necessary network, server, and security infrastructure---and their
conseuqent cost---to fulfill the availability demands of these
companies and their customers is significant. In fact, they are so
significant that every government that has attempted to build or
colocate a facility dedicated to running Internet elections has spent
many millions of dollars per election on such.

But this situation has evolved. Today we have the widespread
availability of robust, inexpensive public and private cloud computing
platforms built on top of this pre-existing infrastructure. Moreover,
a number of companies exist whose services provide robust, large-scale
DDoS protection, such as CloudFlare. Couple these premises with the
fact that E2E-VIV systems need not run on dedicated, physically secure
hardware so long as a TPM is available, then we suddenly have the
ability to speculatively have highly available elections systems
running on provisioned hardware. 

As such, we believe that a highly available E2E-VIV system can be
deployed and maintained given the current state-of-the-art in
highly-available networked services. This is especially true if
elections run over a reasonable time frame (e.g., many days to a few
weeks) and, speculatively, are constructed using a peer-to-peer
network model.

\subsection{Operational}
\label{sec:operational}

Finally, from a technical standpoint, we have the question of
operational feasibility.  Namely, presume an E2E-VIV system is created
that is correct, secure, usable, and can be deployed in a
highly-available fashion. If that system is too difficult or expensive
to integrate into existing municipalities' or states' election
systems, or is too complex for LEO IT staff to understand and support,
we have failed.

Traditionally, software like an IV system is delivered for deployment
as a bundle of source code with manifold dependencies. That stack of
software must be hand-built, carefully customized, and installed on a
classical LAMP software stack on server.  Because of the complexity of
IV systems, the number of large pieces of separate software on which
the core IV application depends is typically many dozens.  These
dependencies include large subsystems like databases, application
servers, web servers, authenticate servers, etc. and dozens of
libraries for processing configuration files, communicating over
networks, performing cryptography, etc.

Unfortunately, the vast majority of jurisdictions has neither the
expertise nor the resources to deploy such a system. Consequently,
deploying a traditionally designed and developed IV system is clearly
infeasible.

Happily, packaging and delivering complex distributed processing
systems in cloud deployments has become commonplace in recent years.
Complex deployments by non-technical staff is now possible and widely
available due to the creation of new technologies invented
specifically to fulfill this need. 

The key technologies that solve this problem are mentioned in
\autoref{cha:rigor-softw-engin}.  More specifically, the necessary
technologies include continuous integration systems (e.g., Jenkins),
configuration managements tools for development (e.g., maven and
cabal) and deployment (e.g., puppet and docker), and cloud deployment
and management technologies like those available from the major cloud
providers Amazon, Google, Microsoft, and Heroku.

If an E2E-VIV system is constructed using these specific technologies
then point-and-click deployment and management becomes a possibility,
even for IT-poor LEO offices.  Consequently, at least in this
technical setting, the operational challenges of E2E-VIV become
feasible.

%=====================================================================
\section{Non-Technical Feasibility Analysis}

Non-technically, feasibility of E2E-VIV systems boils down to matters
of matters of a decidedly social and psychological nature.  Law,
politics, money, public perception, and business interests are the
primary forces that will either mandate or forbid the use of E2E-VIV
systems. 

In particular, if any one of these actors is fundamentally opposed to
any of the key features of E2E-VIV systems, then deployment of E2E-VIV
systems is infeasible. Exemplars of such concerns are included below.
These examples are based mainly upon past discussions within the
verifiable elections activist community, media reporting about IV, and
and reflections of past activities within legislatures around the
world.

\subsection{Law}

For the most part, the legal framework of elections must be changed in
every jurisdiction that wishes to use IV.  Historically legislatures
are comfortable with opening up the possibility of trials in IV,
particularly for UOCAVA voters.  Likewise, facilitating disabled voters
with technology is commonplace.

Unfortunately, history also shows that legislatures often permit or
mandate the use of new elections technologies with little restriction
on their form, substance, and impact.  Legalizing IV without mandating
E2E-VIV is a recipe for disaster.  Likewise, permitting the unbounded
rollout of IV without attuning its use according to objective,
evidence-based evaluations of IV's use and impact is a frightenly
slippery slope.

Consequently, based upon historical evidence, the gradual evolution of
state and local election law, particularly facilitated due to the
local nature of elections, seems feasible in a 5--10 year time
frame. 

Moreover, we recommend that any phase two of this project must include
the drafting of concrete legal recommendations to state and local
legislators.  These recommendations must ensure that the legal
framework for IV deployment is rational, evidence-based, and legally
mandates the requirements set forth in this report.

\subsection{Politics}

In the main, politicians and elected officials such as County Clerks
want to be perceived as forward-thinking and modern.  Thus, it is not
uncommon for those running for office to support new election
technologies such as IV, so long as they do not have to pick up the
tab.  On the other hand, political parties and powerful political
special interest groups are motivated by other factors.

The hypothetical implications of widespread trustworthy use of
E2E-VIV---particularly that of increased broad-spectrum voter
participation--- are potentially at odds with the agendas of some
political actors.

As a result, it is an open question if E2E-VIV is politically
feasible.  Only time will tell.

\subsection{Fiscal}
\label{sec:fiscal}

The cost of developing and deploying previous non-E2E IV systems is
often not part of the public record.  Evidence indicates that the cost
of each voting system deployed in the U.S.A. (SERVE), The Netherlands
(KOA and RIES), Norway (with Scytl), Estonia, France, Switzerland, and
Australia (iVote in NSW and vVote in VEC) ranges from approximately
1.5 to 10s of millions dollars.  

Consequently, it is reasonable to expect that creating an E2E-VIV
system as stipulated in this report will cost several million dollars.

Given the amount of money that has flowed into elections over the past
fifteen years, one would think that designing and developing an
E2E-VIV system is fiscally feasible.

After all, just over 3 billion dollars has been spent on elections via
HAVA, the cost of non-E2E voting machines from traditional vendors is
several thousand dollars per machine, and the average cost per vote
in today's elections ranges from \$2 to \$10 per vote, depending
upon the jurisdiction.  

Thus, the hypothesis that an open source E2E-VIV system, even if
licensed and supported at reasonable costs by commercial
organizations, will be extremely cost-effective in the medium-to-long
term is reasonable.

Unfortunately, today is exactly the wrong time to look for federal and
state funding for elections technology.  Congress has no interest in
expanding budgets for elections. In fact, congress is even debating
eliminating the EAC whose yearly budget is only just over ten million
dollars. States and local municipalities are also in a budget pinch,
so we cannot expect any single state or municipality to fund any
future phases of the E2E-VIV project.

Accordingly, the fiscal feasibility of E2E-VIV is contingent upon
actors outside of government with both financial resources and an
interest in the speculative impact of widely available E2E-VIV.  These
actors include non-profit foundations, wealthy individuals, and
existing and new vendors willing to invest millions of dollars of R\&D
into E2E-VIV.

\subsection{Research}

Several research challenges are highlighted in the technical chapters
of this report.  Foremost among them is the open question of whether or
not it is possible to develop a bespoke E2E-VIV protocol for
U.S. elections whose requirements includes universal design
properties. 

The general consensus of the cryptography and usability experts is
that this is a feasible goal, though will require a significant amount
of work in mechanical formalization and verification and usability
testing.

As such, the timeline of a phase two of this project is likely to be
dependent upon the usability team's ability to design, run, and
evaluate multiple concurrent usability testing runs.  

After discussing the complexity of these two research challenges, it
is generally agreed that it is feasible to perform this work in a
speculative phase two of the project in a two year time-frame.

\subsection{Integration}

Considering the discussion focusing on operational issues above
in~\autoref{sec:operational}, is it feasible to speculate that
jurisdictions' IT staff or their contractors
(see~\autoref{sec:business} below) are capable of integrating an
E2E-VIV system into their existing technical and election workflow?

Existing IV products witness serious integration challenges given the
plethora of proprietary data formats and protocols in Election
Management Systems and IV products.  This situation is, in part, the
motivation for the interoperability requirements found
in~\autoref{sec:interoperability} that open protocols and data
standards must be used and respected in any E2E-VIV system.

This idea is further strengthened by the velocity of the
standardization work going on under the auspices of the IEEE 1622
working group that focuses on standardizing election data formats and
protocols~\cite{IEEE1622}. As such, existing and up-in-coming vendors
are planning to revise their products to conform to said standards,
especially since the speculation is that a future VSSG revision will
mandate such.

For this reason, integration between E2E-VIV systems and existing
local and state elections systems looks feasible.

\subsection{Business}
\label{sec:business}

Election officials historically are reluctant to develop their own
technology or to rely upon technologies that do not have a significant
commercial support business infrastructure. There is a cornucopia of
business needs that surround any new IT system, namely system
evolution, support, maintenance, integration, training,
etc. 

Consequently, if an open source E2E-VIV system is released and yet
there is no corporate market of Value-Added Resellers (akin to
Unisys), Business Integrators (the likes of IBM and Accenture), or
straightforward SLA-focused companies willing to support an open
source product (e.g., businesses like Red Hat and Hewlett-Packard),
then it is infeasible to believe that E2E-VIV systems will ever be
accepted and deployed.

Is it feasible to conclude that such a market will come to exist?
Moreover, will it be a competitive market so that LEOs will witness
competition in E2E-VIV support RFPs?  

Given recent shifts in the elections marketplace, especially with the
entrance of a new generation of vendors and technologists like
ClearBallot, Everyone Counts, Democracy Works, and the OSET
Foundation, we believe that it is a feasible to conclude that a health
business ecosystem will come to exist over the next decade.

\subsection{Public Acceptance}

The final, and most crucial feasibilty facet of E2E-VIV is the
question of public acceptance.  Independent of any technological or
political decision, if the voting public does not trust their election
system, they do not trust their elected leaders or their democracy.

Our initial usability study, as well case studies in the U.S.A. and
elsewhere, has revealed that the general public is very accomodating
to the introduction of new technology for elections.  In general,
voters presume that election officials know what they are doing and,
if they have chosen to deploy a new elections technology, then the
decision must be a good one.

This baseline of faith, though, is complemented and complicated by the
fragile nature of trust.  IT systems that are responsible for citizen
data or services, especially in the face of a growing public
understanding of, and sensitivity to, the security failures of
government systems, are currently looked upon with distrust.

Government agencies responsible for these systems have an even harder
time with earning and maintaining the trust.  Distrust is prevelent
both in government employees that must use those systems and in
citizens whose private information is housed in government systems.

As such, earning and maintaining public trust in E2E-VIV systems
requires an extraordinary amount of transparency, strategy, and is a
proactive public relations exercise.  

At its core, since E2E-VIV systems' correctness and security rely upon
deep mathematical and computer science foundations, very few citizens
can directly understand them and come to trust them through objective
evidence. Consequently, the measure of trustworthiness that E2E-VIV
systems have will always be a delegated trust in the hands of a few
experts. The feasibility of public acceptance thus comes down to the
trustworthiness of those experts and the evidence that they, and the
E2E-VIV system itself, can produce which bears witness to public
trust.

%=====================================================================
\section{Integrated Feasibility Analysis}

This chapter's feasibility analysis provides us with the components of
an evaluation of the overall feasibility of pursuing a practical
E2E-VIV system for U.S. elections.

All technical dimensions---engineering for correctness and security,
design and engineering for usability, availability, operational---have
been deemed feasible, though difficult.

Non-technical dimensions' feasibility ranges from unknown to wholly
feasible.  The feasible-but-cautious position with respect to law is
contingent upon legislators, election officials, and social pressure
from voters.  Happily, this is complemented by the matters of
financial, research, integration, and business are relatively
straightforward, and thus feasible.

The biggest open question relates to the core challenge of any modern
IT system: how do people and software relate?

The politics and public acceptance of internet voting is an open
question, but we believe that only through the disciplined,
transparent, scientific, and practical pursuit of E2E-VIV can we hope
to influence that dialog and give evidence to the public that we, and
E2E-VIV systems, deserve their trust.

As such, given the context of IV systems in the U.S.A., we deem it
feasible to pursue future phases of this project, and thus make our
final recommendation (``MOVE-FORWARD'') in our concluding chapter.

