\chapter{Feasibility\ifdraft{ (\emph{Joe, David, et al.}) (25\%)}{}}
\label{chapter:feasibility}

In the first fifty-odd pages of this report, in Chapters 2 through 5,
we laid out the motivation for, history of, and requirements on a
remote voting system that both experts demand and the public will
trust.  

Then in Chapters 6 through 8 we described the necessary cryptographic,
architecture, and engineering foundations, tools, and techniques
necessary for designing and building a system that fulfills the
criterion set out in Chapter 5's requirements.

But just because it seems \emph{possible} to design and develop such a
system, does not mean that it is \emph{feasible} to do so.

This chapter analyses the question of feasibility from several
dimensions, some of which are \emph{technical} (correctness, security,
usability, availability) and others \emph{non-technical} (law,
politics, fiscal, research, development, operational, and
business). After discussing each of these semi-orthogonal dimensions,
we summarize with an integrated feasibility analysis, focusing on the
question of ``Does it makes sense to practically tackle the problem of
E2E-VIV at this time?''

In what follows feasibility is determined by a principled examination
of the current state-of-affairs based upon the peer-reviewed
literature, extensive conversations with those responsible for
elections, multi-year dialogs with election verification activists,
and decades of experience in the design and development of secure
high-assurance systems.

The final determination of feasibility is not up to us, as authors of
this report. This is a decision, for the most part, that must be made
by those organizations with resources that can be brought to bear on
this problem. The aligners with that decision---primarily the activist
community and legislatures---while important, are not critical at this
point in time, given the deployment velocity of unverifiable internet
voting systems worldwide.

%=====================================================================
\section{Technical Feasibility Analysis}

We first look at technical feasibilty because, at its core, if
designing and constructing a formally verified, secure E2E-VIV system
is not possible, then we need not analyze any other feasibility
dimension.

\subsection{Engineering for Correctness and Security}

As previously mentioned, the state-of-the-art in formal verification
has advanced tremendously over the past fifteen years. High-assurance
or formally verified operating systems and hypervisors such as
seL4~\cite{seL4}, Miraga, and HaLVM were only a pipedream in the year
2000. Likewise, such is the case for formally verified compilers (such
as CompCert~\cite{CompCert}), static analysis tools (such as Verasco),
and verification tools (such as VST~\cite{VST}) and
verification-centric programming languages (such as
Dafny~\cite{Dafny}). All of this technology is supported by incredible
advances in mechanical theorem proving, particularly for SAT, SMT,
constraint solving, and logical frameworks.

The state-of-the-art is the design, development, and analysis of
secure systems has also progressed tremendously. Powerful open source
static analysis tools (such as Uno~\cite{Uno}), fuzzers (such as the
American fuzzy lop~\cite{afl}), protocol specification and reasoning
frameworks (such as EasyCrypt~\cite{EasyCrypt} and F\*~\cite{Fstar})
are all publicly available and can be applied to commercial systems.

The only thing preventing the design and development formally verified
correct and secure evidence-based systems is market pressure. Only a
very small number of organizations have the necessary
resources---primarily in the form of people and knowledge---to tackle
these kinds of challenges. And these companies cannot tackle such
challenges without clients who provide requirements, funding, and
time.

As such, \emph{designing and developing an E2E-VIV system is technical
  feasible}.  In fact, we can estimate---based upon the size and
complexity of the relevant protocols and subsystems---the effort
necessary to accomplish such a goal, as effort relates directly to
fiscal demands. Such an analysis is provided below
in~\autoref{sec:fiscal}. 

Consequently, technical feasibilty is not a limitation of a phase two
of this project.

\subsection{Design and Engineering for Usability}

Recall that security and usability are often at odds with each other.
This conflict is especially prevelent in early E2E-V election systems
such as Helios, Prêt à Voter, and RIES.

Thus, the following question arises: Is it feasible to design and
develop an E2E-VIV system which follows universal design principles?
This question demands that we speculate, as there is little objective
evidence that such a presupposition is valid.

Usability experts claim that universal design of elections systems is
a reasonable and obvious thing to do. Several organizations have
extensive experience in this area, such as those led by Whitney
Quesenbery and Dana Chisnell. And voting systems that are under
development---such as Los Angeles County's VSAP project and Travis
County's STAR-Vote system---mandate universal design. In fact, in the
case of L.A.'s system, several million dollars have already been spent
on such work.

Consequently, while there are open research questions about certain
aspects of usable E2E-VIV system---particularly that of voter ritual
and verifiability---usability experts agree that a universal design
E2E-VIV system is possible in principle. It is agreed that, in order
to achieve such a universal design, a long-term, in-depth, qualitative
and quantitative usability study based based upon a working
demonstration system is necessary.

% Text to move starts here...

In order to effect this goal, a demonstration system that mimics a
voter's interaction with an E2E-VIV system has been developed by
Galois. That system is a variant of the STAR-Vote system designed by
Wallach et al.~\cite{star-vote}. STAR stands for Secure, Transparent,
Auditable, and Reliable. STAR-Vote is an end-to-end verifiable ballot
marking device. As such, it is not designed for, or meant to be used
for, Internet voting. But insofar as its voting process is identical
to that of most of E2E-VIV election schemes in the literature, we
decided to use it as a demonstration vehicle for usability and
accessibility experiments.

The Galois STAR-Vote implementation has a web-based UI, thus can be
used and demonstrated remotely for interactive and non-interactive
experiments to gather both qualitative and quantitative feedback.
Several variants of STAR-Vote have been implemented for UX testing.
These variants include simple changes---like different typeface
choices and sizes, background colors, supporting images, help text,
mouse pointer graphics, etc.---as well as more complex changes---like
different voter, challenge, and audit workflows.

In an interactive, qualitative experiment, a facilitator and a voter
communicate using a video chat system such as Skype and the voter
shares their desktop with the facilitator. Optimally, the facilitator
is someone who is deeply familiar with the issues of E2E-VIV systems,
is familiar with STAR-Vote, and has expertise in usability and
accessibility. The voter then uses (one of several variants of)
STAR-Vote, voicing their thoughts and feelings about their experience
in real-time. After the voter has completed their participation in the
demonstration election, the facilitator uses a script to query them
about their impressions.

For a non-interactive, quantitative experiment, voters will be
solicited via social media, mailing lists, etc. to experiment with
(variants of) STAR-Vote. Sample voters in these experiments are given
ample information about what kinds of information is being collected
about their behavior so that they can make a fully-informed judgement
about their participation.

Various quantitative measures related to voter participation and
interaction can be measured automatically, both within their web
browsers and on the STAR-Vote server. Most of this data is akin to the
analytics that any professional website collects about its users: How
do voters navigate the site?  Where does a voter pause for a long time
and read?  When does a voter ask for help?  When does a voter hover
over a button a long time before they decide to click it?  How often
do voters challenge ballots or verify their votes?  How often do
voters examine the bulletin board?  Is there a correlation between the
interactive behavior of a voter while voting and their likelihood of
voting, challenging, or auditing correctly?

\subsection{Availability}

A system that is correct, secure, and usable is still not useful to
voters if it is unavailable during an election. Many government
websites are notoriously unreliaable in this regards, especially
during a distributed denial-of-service (DDos) attack or just after a
security breach.

But commercial websites for major corporations whose very business
depends upon having a highly available presence on the Internet have
effectively solved this problem. Companies like Amazon, Google, and
Facebook have uptimes comparable to that which would be necessary to
run a public election, even in the face of serious DDoS threats.

The necessary network, server, and security infrastructure---and their
conseuqent cost---to fulfill the availability demands of these
companies and their customers is significant. In fact, they are so
significant that every government that has attempted to build or
colocate a facility dedicated to running Internet elections has spent
many millions of dollars per election on such.

But this situation has evolved. Today we have the widespread
availability of robust, inexpensive public and private cloud computing
platforms built on top of this pre-existing infrastructure. Moreover,
a number of companies exist whose services provide robust, large-scale
DDoS protection, such as CloudFlare. Couple these premises with the
fact that E2E-VIV systems need not run on dedicated, physically secure
hardware so long as a TPM is available, then we suddenly have the
ability to speculatively have highly available elections systems
running on provisioned hardware. 

As such, we believe that a highly available E2E-VIV system can be
deployed and maintained given the current state-of-the-art in
highly-available networked services. This is especially true if
elections run over a reasonable time frame (e.g., many days to a few
weeks) and, speculatively, are constructed using a peer-to-peer
network model.

\subsection{Operational}

Finally, from a technical standpoint, we have the question of
operational feasibility.  Namely, presume an E2E-VIV system is created
that is correct, secure, usable, and can be deployed in a
highly-available fashion. If that system is too difficult or expensive
to integrate into existing municipalities' or states' election
systems, or is too complex for LEO IT staff to understand and support,
we have failed.

Traditionally, software like an IV system is delivered for deployment
as a bundle of source code with manifold dependencies. That stack of
software must be hand-built, carefully customized, and installed on a
classical LAMP software stack on server.  Because of the complexity of
IV systems, the number of large pieces of separate software on which
the core IV application depends is typically many dozens.  These
dependencies include large subsystems like databases, application
servers, web servers, authenticate servers, etc. and dozens of
libraries for processing configuration files, communicating over
networks, performing cryptography, etc.

Unfortunately, the vast majority of jurisdictions has neither the
expertise nor the resources to deploy such a system. Consequently,
deploying a traditionally designed and developed IV system is clearly
infeasible.

Happily, packaging and delivering complex distributed processing
systems in cloud deployments has become commonplace in recent years.
Complex deployments by non-technical staff is now possible and widely
available due to the creation of new technologies invented
specifically to fulfill this need. 

The key technologies that solve this problem are mentioned in
\autoref{cha:rigor-softw-engin}.  More specifically, the necessary
technologies include continuous integration systems (e.g., Jenkins),
configuration managements tools for development (e.g., maven and
cabal) and deployment (e.g., puppet and docker), and cloud deployment
and management technologies like those available from the major cloud
providers Amazon, Google, Microsoft, and Heroku.

If an E2E-VIV system is constructed using these specific technologies
then point-and-click deployment and management becomes a possibility,
even for IT-poor LEO offices.  Consequently, at least in this
technical setting, the operational challenges of E2E-VIV become
feasible.

%=====================================================================
\section{Non-Technical Feasibility Analysis}

\todokiniry{If E2E-VIV it is technically feasible, and yet the law,
  politics, or boots-on-the-ground deep it infeasible, then it is
  DOA.}

\subsection{Law}

\todokiniry{Ensure we discuss other legal frameworks; e.g., caselaw, SOS
  directives, national law and policy wrt the use of federal funds,
  etc.}

\subsection{Politics}

\todokiniry{In the main, politicians want internet voting come hell or
  highwater.  How does phase 2 and 3 look given that vendors are
  selling product and that politicians do not care about nuances?}

\subsection{Fiscal}
\label{sec:fiscal}

\todokiniry{Reflect upon the cost of previous experiements in developing
  and trialing internet voting systems.  What is the current static
  state-of-affairs wrt election budgets at the local, state, and
  national level. There is little more HAVA money, jurisdictions are
  having to make-do with what they have, and there is little appetite for
  purchasing new equipment from the existing vendors that they
  dislike.  They really want an inexpensive outsourced product that is
  secure and usable.}

\subsection{Research}

\todokiniry{What are the open research challenges?  Crafting a custom E2E
  VIV protocol which pays attention to practical security,
  development, deployment, and usability.  A long-term UX study
  framework for running dozens/hundreds of microstudies to find the
  right story of E2E-VIV for the masses.}

\subsection{Development}

\todokiniry{How feasible is to to design and develop a high-assurance E2E
  VIV using modern tools, technologies, and theory?}

\subsection{Operational}

\todokiniry{How feasible is integration with local election systems and
  processes, especially given how many jursidictions have rolled their
  own EMSs?}

\subsection{Public Acceptance}

\subsection{Business}

\todokiniry{Pay attention particularly to LEO considerations.  They want
  a product that is double-click deployable, integrates with their
  existing EMSs, and is easy and cheap to maintain and deploy,
  primarily through a set of competitive companies that provides
  various SLAs.}

%=====================================================================
\section{Integrated Feasibility Analysis}

\todokiniry{Roll together the above analysis into a final overall
  framework for determining feasibilty and make a recommendation.}

