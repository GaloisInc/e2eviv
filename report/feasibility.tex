\chapter{Feasibility\ifdraft{ (\emph{Joe K., David, Dan, et al.}) (95\%)}{}}
\label{chapter:feasibility}

In Chapters \ref{chapter:introduction} through
\ref{chapter:required_properties} of this report, we described the
motivation for, history of, and requirements on a remote voting system
that experts can approve and the public can trust. In Chapters
\ref{chapter:crypto_spec} through
\ref{cha:rigor-softw-engin}\iftechreport\else\ of the full report\fi, we
described the necessary cryptographic, architectural, and engineering
foundations, tools, and techniques necessary to design and build a
system that fulfills the requirements set forth in
\autoref{chapter:required_properties}. However, the fact that it seems
\emph{possible} to design and develop such a system does not mean that
it is \emph{feasible} to do so.

This chapter analyzes the question of feasibility in several areas,
some of which are \emph{technical} (correctness, security, usability,
availability) and others \emph{non-technical} (law, politics, fiscal,
research, development, operational, and business). After discussing
each of these areas, we summarize with an integrated feasibility
analysis, focusing on the question: ``Is it practical to tackle the
problem of E2E-VIV at this time?''

To determine feasibility, we took multiple approaches. We examined
current knowledge of these systems as discussed in peer-reviewed
literature. We talked extensively with election officials, and had
discussions over multiple years with both election verification
activists and other experts who have decades of experience designing
and developing secure, high-assurance systems.

The feasibility of E2E-VIV will ultimately be determined by those
organizations with the resources to invest in E2E-VIV research and
system development. Given how quickly unverifiable Internet voting
systems are being deployed worldwide, such investments are time
critical.

%=====================================================================
\section{Technical Feasibility Analysis}

We first examine technical feasibility. If designing and constructing
a formally verified, secure E2E-VIV system is not possible, analyzing
any other feasibility area is unnecessary.

Since this section focuses on technical feasibility, it refers back to
the technical chapters of this report.  If you are reading the
non-technical version of this report, or you are uninterested in
technical matters, we recommend skipping
to~\autoref{sec:non-technical-feasibility-analysis}.

\subsection{Protocol}

A secure and usable E2E-V protocol is an essential component of any
E2E-VIV system. However, E2E-V alone does not provide many of the
necessary properties for public election systems, including:

\begin{itemize}
\item \textbf{COERCION RESISTANCE.} It is difficult to design a
  coercion-resistant E2E-V protocol where the voter votes from an
  untrusted computer. Coercion-resistant protocols exist where the
  voter votes from a trusted computer; however, these require the
  voter to vote multiple times, indicating each time whether or not
  the vote should be considered valid. Current coercion-resistant
  protocols pose significant usability and accessibility challenges.

\item \textbf{DISPUTE RESOLUTION.} An E2E-V protocol enables a voter
  to determine whether her vote was correctly recorded. If she
  discovers that it was not, the protocol does not necessarily provide
  her with evidence to convince a third party of the problem. It is
  therefore difficult for election officials, observers and the
  general public to determine the extent of fraud if there are
  multiple complaints. Existing protocols that do provide the voter
  with evidence of fraud require the use of paper or a second
  independent communication channel, as well as the use of physical
  election security procedures. Such procedures cannot be used for
  remote voting. Additionally, the use of paper or a second
  communication channel presents usability and accessibility
  challenges.

\item \textbf{RESISTANCE TO CLIENT MALWARE AND DENIAL OF SERVICE.} If
  a voter follows all the steps for voter verification of an E2E-V
  protocol, she should be able to determine if her vote has been
  manipulated by malware. However, denial of service attacks might
  limit her ability to perform the verification
  procedure. Additionally, if the protocol does not support dispute
  resolution, she would not be able to provide evidence of a
  problem. In particular, client malware could replace her vote with
  another valid vote in many E2E-V designs. She would recognize this
  if she were able to perform voter verification, but she would not be
  able to prove it. The use of multiple channels or paper can provide
  additional protection against these types of attacks, but dilutes the
  usability and accessibility of the system.

\item \textbf{UNIVERSAL DESIGN.} E2E-V protocols with a number of the
  above properties have been developed, but the use of second channels
  (or paper) and multiple complicated steps are difficult to achieve
  in an accessible system. A protocol that cannot be used properly by
  most voters is not necessarily secure, no matter how well it is
  designed. For example, if voters find it difficult to verify their
  votes, voter verification cannot be relied upon to contribute to the
  correctness of the election outcome.
\end{itemize}

Experts are divided on whether a protocol that has some of these
properties would be an improvement over existing vote-by-mail
systems. Further, most experts agree that it is not clear how to
develop a protocol that has all of these properties.  As such, this
remains one of the most uncertain and challenging areas for progress
on Internet voting.

%~~~~~~~~~~~~~~~~~~~~~~~~~~~~~~~~~~~~~~~~~~~~~~~~~~~~~~~~~~~~~~~~~~~~~
\subsection{Engineering for Correctness and Security}

As previously mentioned, formal verification capabilities have
advanced tremendously over the past fifteen years. Scientists were
barely imagining high-assurance or formally verified operating systems
and hypervisors such as seL4~\cite{klein2009sel4},
Mirage~\cite{OpenMirage}, and HaLVM~\cite{HaLVM} in the year 2000. The
same is true of formally verified compilers (such as
CompCert~\cite{CompCert}), static analysis tools (such as
Verasco~\cite{Verasco}), verification tools (such as VST~\cite{VST}),
and verification-centric programming languages (such as
Dafny~\cite{Dafny}). Incredible advances in mechanical theorem
proving, particularly for SAT, SMT, constraint solving, and logical
frameworks, support all of this technology.

Design, development, and analysis of secure systems have also
progressed tremendously. Powerful open source static analysis tools
(such as Uno~\cite{holzmann2002static}), fuzzers (such as
AFLFuzz~\cite{AFLFuzz}), and protocol specification and reasoning
frameworks (such as EasyCrypt~\cite{EasyCrypt} and F\*~\cite{Fstar})
are all publicly available and can be applied to commercial systems.

The only thing preventing the design and development of formally
verified, correct and secure evidence-based systems is market
pressure. Only a very small number of organizations have the necessary
resources---primarily in the form of people and knowledge---to tackle
such challenges, and they cannot do it without clients who provide
requirements, funding, and time.

As such, if an appropriate protocol is developed, \emph{designing and
  developing an E2E-VIV system is technically feasible}.  We can
estimate---based upon the size and complexity of the relevant
protocols and subsystems---the effort necessary to build an E2E-VIV
system.  We can determine cost based on the estimate of effort. This
analysis is provided below in~\autoref{sec:fiscal}.

%~~~~~~~~~~~~~~~~~~~~~~~~~~~~~~~~~~~~~~~~~~~~~~~~~~~~~~~~~~~~~~~~~~~~~
\subsection{Design and Engineering for Usability}

Striving for security and usability often presents conflicts. Design
features that offer security often decrease usability, and features
that offer usability often decrease security. This problem is very
apparent in early E2E-V election systems such as Helios, Prêt à Voter,
and RIES. Because of this conflict, scientists are faced with a
difficult question: Is it feasible to design and develop an E2E-VIV
system that is secure and usable? More specifically, is it feasible to
design and develop an E2E-VIV system that follows universal design
principles?  

Usability experts claim that universal design of election systems is
reasonable and necessary. Several organizations have extensive
experience in this area, such as the Center for Civic Design. Some new
voting systems that are under development, such as Los Angeles
County's VSAP project and Travis County's STAR-Vote system, mandate
universal design. 

While researchers are still studying certain aspects of usable E2E-VIV
systems---particularly those of voter ritual and
verifiability---usability experts agree that a universal design for an
E2E-VIV system is possible in principle. The experts agree that, in
order to achieve such a universal design, it is necessary to conduct a
long-term, in-depth, qualitative and quantitative usability study
based upon a working demonstration system.

\paragraph{Qualitative Experiments.}
In an interactive, \emph{qualitative} experiment, a facilitator and a
voter communicate using a video chat system such as Skype. The voter
shares their desktop with the facilitator. The facilitator should be
very familiar with the issues of E2E-VIV systems. The facilitator
should also have usability and accessibility knowledge. The voter uses
one of several versions of the E2E-VIV system.  While using the
system, the voter shares their thoughts and feelings about the
experience in real-time. After the voter has finished participating in
the demonstration election, the facilitator uses a script to ask the
voter what they thought.

\paragraph{Quantitative Experiments.}
For a non-interactive, \emph{quantitative} experiment, voters are
solicited via social media, mailing lists, etc. to experiment with
(variants of) an E2E-VIV system. Sample voters in these experiments
are given ample information about what kinds of information are being
collected about their behavior, so they can make a fully-informed
judgement about their participation.

Various quantitative measures related to voter participation and
interaction can be measured automatically, both within voters' web
browsers and on the E2E-VIV server. Most of this data is similar to
the analytics that any professional website collects about its users:
How do voters navigate the site?  Where does a voter pause for a long
time to read?  When does a voter ask for help?  When does a voter
hover over a button a long time before they decide to click it?  How
often do voters challenge ballots or verify their votes?  How often do
voters examine the bulletin board?  Is there a correlation between the
interactive behavior of a voter while voting and their likelihood of
voting, challenging, or auditing correctly?

%~~~~~~~~~~~~~~~~~~~~~~~~~~~~~~~~~~~~~~~~~~~~~~~~~~~~~~~~~~~~~~~~~~~~~
\subsection{Availability}

A system that is correct, secure, and usable is still not useful to
voters if it is unavailable during an election. Many government
websites are unreliable, especially during a distributed
denial-of-service (DDoS) attack or just after a security breach.

Many companies whose businesses depend on having highly available and
secure websites have effectively solved this problem. Companies like
Amazon, Google, and Facebook have uptimes comparable to those
necessary to run a public election, even if threatened by DDoS
attacks.

The necessary network, server, and security infrastructure---and the
consequent cost---to fulfill the availability demands of these
companies and their customers is significant. The cost is so
significant that every government that has attempted to build a
facility dedicated to running Internet elections has spent many
millions of dollars per election.

Today, however, there are many robust, inexpensive, public and private
cloud computing platforms built to work with pre-existing
infrastructure. Strong, large-scale DDoS protection services are now
available. Because E2E-VIV systems do not need to run on dedicated,
physically secure hardware---as long as suitable roots of trust are
available---it should be possible to run highly available elections
systems on existing hardware.

We believe a highly available E2E-VIV system can be deployed and
maintained with current highly-available networked services. This is
especially true if elections are run over a reasonable time frame
(such as many days to a few weeks) and are built using a peer-to-peer
network model.

%~~~~~~~~~~~~~~~~~~~~~~~~~~~~~~~~~~~~~~~~~~~~~~~~~~~~~~~~~~~~~~~~~~~~~
\subsection{Operational}
\label{sec:operational}

Operational feasibility presents other challenges. Creating a correct,
secure, usable E2E-VIV system that can be deployed with high
availability is not enough. If that system is too difficult or
expensive to integrate into existing election workflows, or is too
complex for LEO IT staff to understand and support, it will not be
used.

Software such as an Internet voting system is usually delivered for
deployment as a bundle of source code with many dependencies. That
stack of software must be hand-built, carefully customized, and
installed on a server.  Because of the complexity of Internet voting
systems, the core application typically depends on dozens of other
large pieces of software.  These dependencies include databases,
application servers, web servers, authenticatation servers, and dozens
of libraries for processing configuration files, communicating over
networks, and performing cryptography.

The vast majority of jurisdictions has neither the expertise nor the
resources to deploy such a system. Deploying a traditionally designed
and developed Internet voting system is not feasible.

Packaging and delivering complex distributed processing systems in
cloud deployments, however, has become commonplace in recent years.
Complex deployments by non-technical staff are now possible and widely
available due to the creation of new technologies invented
specifically to fulfill this need.

The key technologies that solve this problem are discussed in
\autoref{cha:rigor-softw-engin}\iftechreport\else\ of the full report\fi.
These include continuous integration
systems and configuration management tools for development and
deployment.  They also include cloud deployment and management
technologies, such as those available from Amazon, Google, Microsoft,
Heroku, and other major cloud providers.

If an E2E-VIV system is constructed using these specific technologies,
then point-and-click deployment and management becomes a possibility
even for LEO offices that have limited IT resources. In this technical
setting, the operational aspects of E2E-VIV are feasible.

%=====================================================================
\section{Non-Technical Feasibility Analysis}
\label{sec:non-technical-feasibility-analysis}

Non-technical feasibility of E2E-VIV systems will be primarily decided
in the realms of law, politics, finance, public perception, and
business interests. In particular, if any of these sectors is
fundamentally opposed to any of the key features of E2E-VIV systems,
then deployment of E2E-VIV systems is infeasible. The examples below
are based upon discussions within the election integrity community,
media reporting about Internet voting, and and reflections upon past
activities within legislatures worldwide.

%~~~~~~~~~~~~~~~~~~~~~~~~~~~~~~~~~~~~~~~~~~~~~~~~~~~~~~~~~~~~~~~~~~~~~
\subsection{Law}

In the U.S., legislators must change the legal framework of elections
in nearly every jurisdiction that wishes to use Internet voting.
Historically, legislatures are comfortable with introducing Internet
voting trials, particularly for UOCAVA voters.  Providing technology
that helps disabled voters is commonplace.

However, legislatures often permit or mandate the use of new election
technologies with little restriction on their form, substance, and
impact.  This creates serious problems. Legalizing Internet voting
without mandating end-to-end verifiability, or permitting large-scale
Internet voting without first evaluating the impact and success of
E2E-V in polling places and in small-scale Internet voting trials,
could have disastrous consequences.

Based upon historical evidence, a gradual evolution of state and local
election law---particularly facilitated by the local nature of
elections---seems feasible in a 5--10 year time frame.

A subsequent research and development phase of this project should
include concrete legal recommendations for state and local
legislators.  These recommendations must ensure that the legal
framework for Internet voting deployment is rational, evidence-based,
and legally mandates the requirements set forth in this report.

%~~~~~~~~~~~~~~~~~~~~~~~~~~~~~~~~~~~~~~~~~~~~~~~~~~~~~~~~~~~~~~~~~~~~~
\subsection{Politics}

In the main, politicians and elected officials want to be perceived as
forward-thinking and modern.  Thus, it is not uncommon for those
running for office to support new election technologies such as
Internet voting.  On the other hand, political parties and powerful
political special interest groups are motivated by other factors.

The hypothetical implications of widespread trustworthy use of
E2E-VIV---particularly the possibility of increased broad-spectrum
voter participation---are potentially at odds with the agendas of some
political actors.

As a result, the political feasibility of E2E-VIV is an open question.
Only time will tell.

%~~~~~~~~~~~~~~~~~~~~~~~~~~~~~~~~~~~~~~~~~~~~~~~~~~~~~~~~~~~~~~~~~~~~~
\subsection{Fiscal}
\label{sec:fiscal}

The cost of developing and deploying previous non-E2E Internet voting
systems is often not part of the public record.  Evidence indicates
that the cost of each voting system deployed in the U.S. (SERVE), The
Netherlands (KOA and RIES), Norway (with Scytl), Estonia, France,
Switzerland, and Australia (iVote in New South Wales and vVote in
Victoria) ranges from approximately one and a half million to tens of
millions of dollars. It is reasonable to expect that creating an
E2E-VIV system as stipulated in this report would cost several million
dollars.

Given the continuous flow of investment into elections, and the
comparatively similar development cost of an E2E-VIV system, it could
be considered fiscally feasible to invest in this type of
innovation. Just over 3 billion dollars have been spent on elections
via HAVA, the cost of non-E2E voting machines from traditional vendors
is several thousand dollars per machine, and the average cost per vote
in today's elections, depending upon the jurisdiction, ranges from \$2
to \$10 per vote.

With current election costs, an open source E2E-VIV system---even if
licensed and supported at reasonable costs by commercial
organizations---will likely be extremely cost-effective in the
medium-to-long term. Unfortunately, federal and state funding for
election technology is currently difficult to find.  The U.S. Congress
has no interest in expanding budgets for elections. The debate around
the elimination of the Election Assistance Commission, whose yearly
budget is only just over ten million dollars, illustrates this
fact. States and local municipalities have tight budgets. We cannot
expect any single state or municipality to fund future phases of the
E2E-VIV project.

The fiscal feasibility of E2E-VIV depends, then, on non-governmental
enetities that have both financial resources and an interest in the
speculative impact of widely available E2E-VIV.  These include
non-profit foundations, wealthy individuals, and existing and new
vendors that are willing to invest millions of dollars into the
research and development of E2E-VIV.

%~~~~~~~~~~~~~~~~~~~~~~~~~~~~~~~~~~~~~~~~~~~~~~~~~~~~~~~~~~~~~~~~~~~~~
\subsection{Integration}

Regarding the operational issues discussed
in~\autoref{sec:operational}, an important question is whether
jurisdictions' IT staff or their contractors
(see~\autoref{sec:business}, below) will be capable of integrating an
E2E-VIV system into their existing technical and election workflows.

Existing Internet voting products have serious integration challenges
because of their own proprietary designs and the many proprietary data
formats and protocols in Election Management Systems.  This situation
is, in part, the motivation for the interoperability requirements
of~\autoref{sec:interoperability}, which state that open protocols and
data standards must be used and respected in any E2E-VIV system.

This idea is further strengthened by the rapid progress of the IEEE
1622 working group that focuses on standardizing election data formats
and protocols~\cite{IEEE1622}. Existing and new vendors are planning
to revise their products to conform to the IEEE standards, especially
since it is likely that a future VVSG revision will mandate their use.

For this reason, integrating E2E-VIV systems with existing local and
state elections systems seems feasible.

%~~~~~~~~~~~~~~~~~~~~~~~~~~~~~~~~~~~~~~~~~~~~~~~~~~~~~~~~~~~~~~~~~~~~~
\subsection{Business}
\label{sec:business}

Election officials have historically been reluctant to develop their
own technology or to rely upon technologies that do not have a
significant commercial support business infrastructure. Any new IT
system requires support services to cover system evolution, support,
maintenance, integration, and training.

A healthy systems market will sustain the growth of an IT support
infrastructure and surrounding services, including value-added
resellers, system integrators, and consultants. A distribution channel
and value-added support infrastructure is necessary for E2E-VIV
systems to be widely accepted and deployed.

Whether such a market will develop and, if so, whether it will be a
competitive market is a critical question. Given recent shifts in the
elections marketplace, especially with the entrance of a new
generation of vendors and technologists, we believe it is feasible
that a healthy business ecosystem will emerge over the next decade.

%~~~~~~~~~~~~~~~~~~~~~~~~~~~~~~~~~~~~~~~~~~~~~~~~~~~~~~~~~~~~~~~~~~~~~
\subsection{Public Acceptance}

The final and most crucial feasibility question for E2E-VIV is that of
public acceptance.  Independent of any technological or political
decision, if voters do not trust the election system, they will not
trust their elected leaders or their democracy.

Our initial usability study, as well as case studies in the U.S. and
elsewhere, shows that the general public usually welcomes new election
technology.  In general, voters believe that election officials know
what they are doing and that if they have chosen to deploy a new
election technology, the technology must be a good one.

Trust, however, can easily break. Much of the public currently
distrusts IT systems that are responsible for citizen data or
services, especially since security failures of government systems and
those of private companies are often in the news.

Government agencies responsible for these systems have an even harder
time earning and maintaining trust.  Distrust is prevalent both in
government employees that must use government systems and in citizens
whose private information is stored in government systems.

Earning and maintaining public trust in E2E-VIV systems will require
an extraordinary amount of transparency and strategy, and will take a
long time.

Because E2E-VIV systems' correctness and security rely upon deep
mathematical and computer science foundations, very few citizens can
directly understand them and come to trust them by reading the
literature for themselves. Non-experts will always have to trust in
the work of experts who develop E2E-VIV systems. The feasibility of public
acceptance therefore depends on the trustworthiness of those experts
and the evidence that they, and the E2E-VIV system itself, can
produce.

%=====================================================================
\section{Integrated Feasibility Analysis}

This chapter's analysis provides us with the necessary components to
evaluate the overall feasibility of building and deploying a practical
E2E-VIV system for U.S. elections.

All technical aspects---engineering for correctness and security,
design and engineering for usability, availability, operational---are
feasible, though difficult.

Feasibility of the non-technical aspects ranges from unknown to
entirely possible.  With respect to law, feasibility is contingent
upon legislators, election officials, and social pressure from voters.
The financial, research, integration, and business aspects are
relatively straightforward, and thus feasible.

The most important open question relates to the main challenge of any
modern IT system: how do people and software relate?

The politics and public acceptance of Internet voting are open
questions. We believe that only the disciplined, transparent,
scientific, and practical pursuit of E2E-VIV can convince the public
that E2E-VIV systems deserve their trust.

In summary, it is feasible to pursue future phases of this project. We
discuss our final recommendations in our concluding chapter.
