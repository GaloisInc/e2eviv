\chapter{Feasibility\ifdraft{ (\emph{Joe, David, et al.}) (25\%)}{}}
\label{chapter:feasibility}

In the first fifty-odd pages of this report, in Chapters 2 through 5,
we laid out the motivation for, history of, and requirements on a
remote voting system that both experts demand and the public will
trust.  

Then in Chapters 6 through 8 we described the necessary cryptographic,
architecture, and engineering foundations, tools, and techniques
necessary for designing and building a system that fulfills the
criterion set out in Chapter 5's requirements.

But just because it seems \emph{possible} to design and develop such a
system, does not mean that it is \emph{feasible} to do so.

This chapter analyses the question of feasibility from several
dimensions, some of which are \emph{technical} (correctness, security,
usability, availability) and others \emph{non-technical} (law,
politics, fiscal, research, development, operational, and
business). After discussing each of these semi-orthogonal dimensions,
we summarize with an integrated feasibility analysis, focusing on the
question of ``Does it makes sense to practically tackle the problem of
E2E-VIV at this time?''

In what follows feasibility is determined by a principled examination
of the current state-of-affairs based upon the peer-reviewed
literature, extensive conversations with those responsible for
elections, multi-year dialogs with election verification activists,
and decades of experience in the design and development of secure
high-assurance systems.

The final determination of feasibility is not up to us, as authors of
this report. This is a decision, for the most part, that must be made
by those organizations with resources that can be brought to bear on
this problem. The aligners with that decision---primarily the activist
community and legislatures---while important, are not critical at this
point in time, given the deployment velocity of unverifiable internet
voting systems worldwide.

%=====================================================================
\section{Technical Feasibility Analysis}

\todokiniry{If E2E-VIV is not technically feasible, then we are DOA.}

\todokiniry{Briefly summarize the main outstanding research and
  engineering challenges, as those are what a phase 2 of this project
  focuses upon.  We have little control over what legislators and
  legislators, LEOs, and SOSs decide in the coming years.}

\subsection{Correctness}

\subsection{Security}

\todokiniry{Core security challenges and their prospective solutions.
  Feasibility analysis from an engineering and operational
  standpoint.}

\subsection{Usability}

\todokiniry{Recall that this includes accessibility.  Reflect upon the
  split personality of the Demos protocol for usability for the
  typical set of voters.}

% Text to move starts here...

In order to effect these goals, a demonstration system that mimics a
voter's interaction with an E2E-VIV system has been developed by
Galois. That system is a variant of the STAR-Vote system designed by
Wallach et al.~\cite{star-vote}. STAR stands for Secure, Transparent,
Auditable, and Reliable. STAR-Vote is an end-to-end verifiable ballot
marking device. As such, it is not designed for, or meant to be used
for, Internet voting. But insofar as its voting process is identical
to that of most of E2E-VIV election schemes in the literature, we
decided to use it as a demonstration vehicle for usability and
accessibility experiments.

The Galois STAR-Vote implementation has a web-based UI, thus can be
used and demonstrated remotely for interactive and non-interactive
experiments to gather both qualitative and quantitative feedback.
Several variants of STAR-Vote have been implemented for UX testing.
These variants include simple changes---like different typeface
choices and sizes, background colors, supporting images, help text,
mouse pointer graphics, etc.---as well as more complex changes---like
different voter, challenge, and audit workflows.

In an interactive, qualitative experiment, a facilitator and a voter
communicate using a video chat system such as Skype and the voter
shares their desktop with the facilitator. Optimally, the facilitator
is someone who is deeply familiar with the issues of E2E-VIV systems,
is familiar with STAR-Vote, and has expertise in usability and
accessibility. The voter then uses (one of several variants of)
STAR-Vote, voicing their thoughts and feelings about their experience
in real-time. After the voter has completed their participation in the
demonstration election, the facilitator uses a script to query them
about their impressions.

For a non-interactive, quantitative experiment, voters will be
solicited via social media, mailing lists, etc. to experiment with
(variants of) STAR-Vote. Sample voters in these experiments are given
ample information about what kinds of information is being collected
about their behavior so that they can make a fully-informed judgement
about their participation.

Various quantitative measures related to voter participation and
interaction can be measured automatically, both within their web
browsers and on the STAR-Vote server. Most of this data is akin to the
analytics that any professional website collects about its users: How
do voters navigate the site?  Where does a voter pause for a long time
and read?  When does a voter ask for help?  When does a voter hover
over a button a long time before they decide to click it?  How often
do voters challenge ballots or verify their votes?  How often do
voters examine the bulletin board?  Is there a correlation between the
interactive behavior of a voter while voting and their likelihood of
voting, challenging, or auditing correctly?

% ...and ends here.

\subsection{Availability}

\todokiniry{Reflect upon the current state-of-the-art in providing
  availability for core services on the internet.  How expensive and
  difficult is such a deployment?  Do the more radical architectures
  described earlier provide serious alternatives?}

\subsection{Operational}

\todokiniry{What are the feasibility challenges in operationalizing an
  E2E-VIV product or service?}

%=====================================================================
\section{Non-Technical Feasibility Analysis}

\todokiniry{If E2E-VIV it is technically feasible, and yet the law,
  politics, or boots-on-the-ground deep it infeasible, then it is
  DOA.}

\subsection{Law}

\todokiniry{Ensure we discuss other legal frameworks; e.g., caselaw, SOS
  directives, national law and policy wrt the use of federal funds,
  etc.}

\subsection{Politics}

\todokiniry{In the main, politicians want internet voting come hell or
  highwater.  How does phase 2 and 3 look given that vendors are
  selling product and that politicians do not care about nuances?}

\subsection{Fiscal}

\todokiniry{Reflect upon the cost of previous experiements in developing
  and trialing internet voting systems.  What is the current static
  state-of-affairs wrt election budgets at the local, state, and
  national level. There is little more HAVA money, jurisdictions are
  having to make-do with what they have, and there is little appetite for
  purchasing new equipment from the existing vendors that they
  dislike.  They really want an inexpensive outsourced product that is
  secure and usable.}

\subsection{Research}

\todokiniry{What are the open research challenges?  Crafting a custom E2E
  VIV protocol which pays attention to practical security,
  development, deployment, and usability.  A long-term UX study
  framework for running dozens/hundreds of microstudies to find the
  right story of E2E-VIV for the masses.}

\subsection{Development}

\todokiniry{How feasible is to to design and develop a high-assurance E2E
  VIV using modern tools, technologies, and theory?}

\subsection{Operational}

\todokiniry{How feasible is integration with local election systems and
  processes, especially given how many jursidictions have rolled their
  own EMSs?}

\subsection{Business}

\todokiniry{Pay attention particularly to LEO considerations.  They want
  a product that is double-click deployable, integrates with their
  existing EMSs, and is easy and cheap to maintain and deploy,
  primarily through a set of competitive companies that provides
  various SLAs.}

%=====================================================================
\section{Integrated Feasibility Analysis}

\todokiniry{Roll together the above analysis into a final overall
  framework for determining feasibilty and make a recommendation.}

