\chapter{Feasibility\ifdraft{ (\emph{Joe K., David, Dan, et al.}) (95\%)}{}}
\label{chapter:feasibility}

In Chapters 2 through 5 of this report, we described the motivation
for, history of, and requirements on a remote voting system that
experts can approve and the public can trust. Then, in Chapters 6
through 8, we described the necessary cryptographic, architectural,
and engineering foundations, tools, and techniques necessary to design
and build a system that fulfills the requirements set forth in Chapter
5. However, the fact that it seems \emph{possible} to design and
develop such a system does not mean that it is \emph{feasible} to do
so.

This chapter analyzes the question of feasibility in several
dimensions, some of which are \emph{technical} (correctness, security,
usability, availability) and others \emph{non-technical} (law,
politics, fiscal, research, development, operational, and
business). After discussing each of these dimensions, we summarize
with an integrated feasibility analysis, focusing on the question:
``Is it practical to tackle the problem of E2E-VIV at this time?''

In what follows, feasibility is determined by a principled examination
of the current state of affairs based upon the peer-reviewed
literature, extensive conversations with those responsible for
elections, multi-year dialogues with election verification activists,
and decades of experience in the design and development of secure
high-assurance systems.

The final determination of feasibility is not up to us, as authors of
this report. This is a decision that, for the most part, must be made
by those organizations with resources that can be brought to bear on
this problem. Other stakeholders in that decision---primarily the
activist community and legislatures---while important, are not
critical at this point in time given how quickly unverifiable Internet
voting systems are being deployed worldwide.

%=====================================================================
\section{Technical Feasibility Analysis}

We first examine technical feasibility. If designing and constructing
a formally verified, secure E2E-VIV system is not possible, we need
not analyze any other feasibility dimension.

%~~~~~~~~~~~~~~~~~~~~~~~~~~~~~~~~~~~~~~~~~~~~~~~~~~~~~~~~~~~~~~~~~~~~~
\subsection{Engineering for Correctness and Security}

As previously mentioned, the state of the art in formal verification
has advanced tremendously over the past fifteen years. High-assurance
or formally verified operating systems and hypervisors such as
seL4~\cite{klein2009sel4}, Mirage, and HaLVM were only a pipe dream in the year
2000. The same is true of formally verified compilers (such as
CompCert~\cite{CompCert}), static analysis tools (such as Verasco),
verification tools (such as VST~\cite{VST}), and verification-centric
programming languages (such as Dafny~\cite{Dafny}). All of this
technology is supported by incredible advances in mechanical theorem
proving, particularly for SAT, SMT, constraint solving, and logical
frameworks.

The state of the art in the design, development, and analysis of
secure systems has also progressed tremendously. Powerful open source
static analysis tools (such as Uno~\cite{holzmann2002static}), fuzzers (such as
AFLFuzz~\cite{AFLFuzz}), and protocol specification and reasoning
frameworks (such as EasyCrypt~\cite{EasyCrypt} and F\*~\cite{Fstar})
are all publicly available and can be applied to commercial systems.

The only thing preventing the design and development of
formally-verified correct and secure evidence-based systems is market
pressure. Only a very small number of organizations have the necessary
resources---primarily in the form of people and knowledge---to tackle
such challenges, and they cannot do it without clients who provide
requirements, funding, and time.

As such, \emph{designing and developing an E2E-VIV system is
  technically feasible}.  In fact, we can estimate---based upon the
size and complexity of the relevant protocols and subsystems---the
effort necessary to accomplish such a goal, as effort relates directly
to fiscal demands. Such an analysis is provided below
in~\autoref{sec:fiscal}.

%~~~~~~~~~~~~~~~~~~~~~~~~~~~~~~~~~~~~~~~~~~~~~~~~~~~~~~~~~~~~~~~~~~~~~
\subsection{Design and Engineering for Usability}

Recall that security and usability are often opposed to each other.
This conflict is especially prevelent in early E2E-V election systems
such as Helios, Prêt à Voter, and RIES.

Thus, the following question arises: Is it feasible to design and
develop an E2E-VIV system that follows universal design principles?
This question demands that we speculate, as there is little objective
evidence that such a presupposition is valid.

Usability experts claim that universal design of election systems is a
reasonable and obvious thing to do. Several organizations have
extensive experience in this area, such as those led by Whitney
Quesenbery and Dana Chisnell. Some voting systems that are under
development, such as Los Angeles County's VSAP project and Travis
County's STAR-Vote system, mandate universal design. In fact, in the
case of the Los Angeles system, several million dollars have already
been spent on such work.

Consequently, while there are open research questions about certain
aspects of usable E2E-VIV systems---particularly those of voter ritual
and verifiability---usability experts agree that a universal design
for an E2E-VIV system is possible in principle. It is agreed that, in
order to achieve such a universal design, a long-term, in-depth,
qualitative and quantitative usability study based upon a working
demonstration system is necessary.

% Text to move starts here...

In order to effect this goal, a demonstration system that mimics a
voter's interaction with an E2E-VIV system has been developed by
Galois. That system is a variant of the STAR-Vote system designed by
Wallach et al.~\cite{star-vote}. STAR stands for Secure, Transparent,
Auditable, and Reliable. STAR-Vote is an end-to-end verifiable ballot
marking device. As such, it is not designed for, or meant to be used
for, Internet voting. But insofar as its voting process is identical
to that of most of E2E-VIV election schemes in the literature, we
decided to use it as a demonstration vehicle for usability and
accessibility experiments.

The Galois STAR-Vote implementation has a web-based UI, so it can be
used and demonstrated remotely for interactive and non-interactive
experiments to gather both qualitative and quantitative feedback.
Several variants of STAR-Vote have been implemented for user
experience (UX) testing.  These variants include simple changes---
different typeface choices and sizes, background colors, supporting
images, help text, mouse pointer graphics, etc.---as well as more
complex changes like different voter, challenge, and audit workflows.

In an interactive, qualitative experiment, a facilitator and a voter
communicate using a video chat system such as Skype and the voter
shares their desktop with the facilitator. Optimally, the facilitator
is someone who is deeply familiar with the issues of E2E-VIV systems,
is familiar with STAR-Vote, and has expertise in usability and
accessibility. The voter then uses one of several variants of
STAR-Vote, voicing their thoughts and feelings about the experience
in real-time. After the voter has completed their participation in the
demonstration election, the facilitator uses a script to query them
about their impressions.

For a non-interactive, quantitative experiment, voters will be
solicited via social media, mailing lists, etc. to experiment with
(variants of) STAR-Vote. Sample voters in these experiments will be
given ample information about what kinds of information is being
collected about their behavior so they can make a fully-informed
judgement about their participation.

Various quantitative measures related to voter participation and
interaction can be measured automatically, both within voters' web
browsers and on the STAR-Vote server. Most of this data is similar to
the analytics that any professional website collects about its users:
How do voters navigate the site?  Where does a voter pause for a long
time and read?  When does a voter ask for help?  When does a voter
hover over a button a long time before they decide to click it?  How
often do voters challenge ballots or verify their votes?  How often do
voters examine the bulletin board?  Is there a correlation between the
interactive behavior of a voter while voting and their likelihood of
voting, challenging, or auditing correctly?

%~~~~~~~~~~~~~~~~~~~~~~~~~~~~~~~~~~~~~~~~~~~~~~~~~~~~~~~~~~~~~~~~~~~~~
\subsection{Availability}

A system that is correct, secure, and usable is still not useful to
voters if it is unavailable during an election. Many government
websites are notoriously unreliable in this regard, especially
during a distributed denial-of-service (DDoS) attack or just after a
security breach.

However, commercial websites for major corporations whose very
business depends upon having a highly available presence on the
Internet have effectively solved this problem. Companies like Amazon,
Google, and Facebook have uptimes comparable to that which would be
necessary to run a public election, even in the face of serious DDoS
threats.

The necessary network, server, and security infrastructure---and the
consequent cost---to fulfill the availability demands of these
companies and their customers is significant. In fact, this cost is so
significant that every government that has attempted to build or
colocate a facility dedicated to running Internet elections has spent
many millions of dollars per election.

Today, however, there are many robust, inexpensive, public and private
cloud computing platforms built on top of pre-existing
infrastructure. Moreover, a number of companies, such as CloudFlare,
provide robust, large-scale DDoS protection services. Since E2E-VIV
systems need not run on dedicated, physically secure hardware so long
as suitable roots of trust are available, it should be possible to run
highly available elections systems on already-provisioned hardware.

As such, we believe that a highly available E2E-VIV system can be
deployed and maintained given the current state of the art in
highly-available networked services. This is especially true if
elections are run over a reasonable time frame (e.g., many days to a
few weeks) and, speculatively, are constructed using a peer-to-peer
network model.

%~~~~~~~~~~~~~~~~~~~~~~~~~~~~~~~~~~~~~~~~~~~~~~~~~~~~~~~~~~~~~~~~~~~~~
\subsection{Operational}
\label{sec:operational}

Finally, from a technical standpoint, we must address operational
feasibility. Creating a correct, secure, usable E2E-VIV system that
can be deployed in a highly-available fashion is not enough. If that
system is too difficult or expensive to integrate into existing
election workflows, or is too complex for LEO IT staff to understand
and support, it will not be used.

Traditionally, software like an Internet voting system is delivered
for deployment as a bundle of source code with manifold
dependencies. That stack of software must be hand-built, carefully
customized, and installed on a server.  Because of the complexity of
Internet voting systems, the core application typically depends on
dozens of other large pieces of software.  These dependencies include
databases, application servers, web servers, authenticatation servers,
etc., plus dozens of libraries for processing configuration files,
communicating over networks, performing cryptography, etc.

Unfortunately, the vast majority of jurisdictions has neither the
expertise nor the resources to deploy such a system. Consequently,
deploying a traditionally designed and developed Internet voting
system is clearly infeasible.

However, packaging and delivering complex distributed processing
systems in cloud deployments has become commonplace in recent years.
Complex deployments by non-technical staff are now possible and widely
available due to the creation of new technologies invented
specifically to fulfill this need.

The key technologies that solve this problem are discussed in
\autoref{cha:rigor-softw-engin}.  More specifically, the necessary
technologies include continuous integration systems, configuration
managements tools for development and deployment, and cloud deployment
and management technologies like those available from major cloud
providers such as Amazon, Google, Microsoft, and Heroku.

If an E2E-VIV system is constructed using these specific technologies,
then point-and-click deployment and management becomes a possibility
even for IT-poor LEO offices.  Consequently, at least in this
technical setting, the operational aspects of E2E-VIV are feasible.

%=====================================================================
\section{Non-Technical Feasibility Analysis}

Non-technical feasibility of E2E-VIV systems boils down to matters of
a decidedly social and psychological nature.  Law, politics, money,
public perception, and business interests are the primary forces that
will either mandate or forbid the use of E2E-VIV systems.

In particular, if any one of these actors is fundamentally opposed to
any of the key features of E2E-VIV systems, then deployment of E2E-VIV
systems is infeasible. Examples of such concerns are included below.
These examples are based mainly upon past discussions within the
verifiable elections activist community, media reporting about
Internet voting, and and reflections upon past activities within
legislatures around the world.

%~~~~~~~~~~~~~~~~~~~~~~~~~~~~~~~~~~~~~~~~~~~~~~~~~~~~~~~~~~~~~~~~~~~~~
\subsection{Law}

The legal framework of elections must be changed in nearly every
jurisdiction that wishes to use Internet voting.  Historically,
legislatures are comfortable with opening up the possibility of
Internet voting trials, particularly for UOCAVA voters.  Likewise,
facilitating disabled voters with technology is commonplace.

However, history also shows that legislatures often permit or mandate
the use of new election technologies with little restriction on their
form, substance, and impact.  Legalizing Internet voting without
mandating end-to-end verifiability is a recipe for disaster.
Likewise, permitting the unbounded rollout of Internet voting without
tuning its use according to objective, evidence-based evaluations of
its impact is a frightenly slippery slope.

Consequently, based upon historical evidence, the gradual evolution of
state and local election law---particularly facilitated by the local
nature of elections---seems feasible in a 5--10 year time frame.

Moreover, we recommend that any phase two of this project must include
the drafting of concrete legal recommendations to state and local
legislators.  These recommendations must ensure that the legal
framework for Internet voting deployment is rational, evidence-based,
and legally mandates the requirements set forth in this report.

%~~~~~~~~~~~~~~~~~~~~~~~~~~~~~~~~~~~~~~~~~~~~~~~~~~~~~~~~~~~~~~~~~~~~~
\subsection{Politics}

In the main, politicians and elected officials want to be perceived as
forward-thinking and modern.  Thus, it is not uncommon for those
running for office to support new election technologies such as
Internet voting, so long as they do not have to pick up the tab.  On
the other hand, political parties and powerful political special
interest groups are motivated by other factors.

The hypothetical implications of widespread trustworthy use of
E2E-VIV---particularly the possibility of increased broad-spectrum
voter participation--- are potentially at odds with the agendas of
some political actors.

As a result, the political feasibility of E2E-VIV is an open question.
Only time will tell.

%~~~~~~~~~~~~~~~~~~~~~~~~~~~~~~~~~~~~~~~~~~~~~~~~~~~~~~~~~~~~~~~~~~~~~
\subsection{Fiscal}
\label{sec:fiscal}

The cost of developing and deploying previous non-E2E Internet voting
systems is often not part of the public record.  Evidence indicates
that the cost of each voting system deployed in the U.S. (SERVE), The
Netherlands (KOA and RIES), Norway (with Scytl), Estonia, France,
Switzerland, and Australia (iVote in New South Wales and vVote in
Victoria) ranges from approximately 1.5 million to 10s of millions of
dollars. Consequently, it is reasonable to expect that creating an
E2E-VIV system as stipulated in this report will cost several million
dollars.

Given the amount of money that has flowed into elections over the past
fifteen years, it would seem that designing and developing an E2E-VIV
system is fiscally feasible. After all, just over 3 billion dollars
has been spent on elections via HAVA, the cost of non-E2E voting
machines from traditional vendors is several thousand dollars per
machine, and the average cost per vote in today's elections ranges
from \$2 to \$10 per vote depending upon the jurisdiction.

Given current election costs, the hypothesis that an open source
E2E-VIV system---even if licensed and supported at reasonable costs by
commercial organizations---will be extremely cost-effective in the
medium-to-long term is reasonable. Unfortunately, now is exactly the
wrong time to look for federal and state funding for election
technology.  Congress has no interest in expanding budgets for
elections. In fact, Congress is even debating eliminating the Election
Assistance Commission, whose yearly budget is only just over ten
million dollars. States and local municipalities are also in a budget
pinch, so we cannot expect any single state or municipality to fund
future phases of the E2E-VIV project.

Accordingly, the fiscal feasibility of E2E-VIV is contingent upon
actors outside of government with both financial resources and an
interest in the speculative impact of widely available E2E-VIV.  These
actors include non-profit foundations, wealthy individuals, and
existing and new vendors willing to invest millions of dollars of R\&D
into E2E-VIV.

%~~~~~~~~~~~~~~~~~~~~~~~~~~~~~~~~~~~~~~~~~~~~~~~~~~~~~~~~~~~~~~~~~~~~~
\subsection{Research}

Several research challenges are highlighted in the technical chapters
of this report.  Foremost among them is the open question of whether
or not it is possible to develop a bespoke E2E-VIV protocol for
U.S. elections with universal design properties.

The general consensus of the cryptography and usability experts is
that this is a feasible goal, though it will require a significant
amount of work in mechanical formalization and verification and
usability testing. As such, the timeline of a phase two of this
project is likely to be dependent upon the usability team's ability to
design, run, and evaluate multiple concurrent usability testing runs.

After discussing the complexity of these two research challenges, it
has been generally agreed that it is feasible to perform this work in
a speculative phase two of the project over a two year timeframe.

%~~~~~~~~~~~~~~~~~~~~~~~~~~~~~~~~~~~~~~~~~~~~~~~~~~~~~~~~~~~~~~~~~~~~~
\subsection{Integration}

Considering the discussion focusing on operational issues
in~\autoref{sec:operational}, an important question is whether
jurisdictions' IT staff or their contractors
(see~\autoref{sec:business}, below) will be capable of integrating an
E2E-VIV system into their existing technical and election workflows.

Existing Internet voting products have serious integration challenges
given their own proprietary designs and the many proprietary data
formats and protocols in Election Management Systems.  This situation
is, in part, the motivation for the interoperability requirements
of~\autoref{sec:interoperability}, which state that open protocols and
data standards must be used and respected in any E2E-VIV system.

This idea is further strengthened by the rapid progress of the IEEE
1622 working group that focuses on standardizing election data formats
and protocols~\cite{IEEE1622}. Existing and up-and-coming vendors are
planning to revise their products to conform to the IEEE standards,
especially since it is likely that a future VSSG revision will mandate
their use.

For this reason, integration between E2E-VIV systems and existing
local and state elections systems seems feasible.

%~~~~~~~~~~~~~~~~~~~~~~~~~~~~~~~~~~~~~~~~~~~~~~~~~~~~~~~~~~~~~~~~~~~~~
\subsection{Business}
\label{sec:business}

Election officials have historically been reluctant to develop their
own technology or to rely upon technologies that do not have a
significant commercial support business infrastructure. There are many
business needs that surround any new IT system, including system
evolution, support, maintenance, integration, and training.

Consequently, if an open source E2E-VIV system is released without a
corporate market of Value-Added Resellers (like Unisys), Business
Integrators (like IBM and Accenture), or companies willing to support
an open source product (like Red Hat and Hewlett-Packard), then it is
infeasible that E2E-VIV systems will ever be accepted and deployed.

Thus, the critical questions are whether such a market will come to
exist and, if so, whether it will be a competitive market. Given
recent shifts in the elections marketplace, especially with the
entrance of a new generation of vendors and technologists like
ClearBallot, Everyone Counts, Democracy Works, and the OSET
Foundation, we believe it is feasible that a healthy business
ecosystem will emerge over the next decade.

%~~~~~~~~~~~~~~~~~~~~~~~~~~~~~~~~~~~~~~~~~~~~~~~~~~~~~~~~~~~~~~~~~~~~~
\subsection{Public Acceptance}

The final and most crucial feasibilty question for E2E-VIV is that of
public acceptance.  Independent of any technological or political
decision, if the voting public does not trust their election system
they do not trust their elected leaders or their democracy.

Our initial usability study, as well as case studies in the U.S. and
elsewhere, has revealed that the general public is very accomodating
to the introduction of new election technology.  In general, voters
presume that election officials know what they are doing and that, if
they have chosen to deploy a new election technology, the technology
must be a good one.

This baseline of faith, though, is complemented and complicated by the
fragile nature of trust. Much of the public currently distrusts IT
systems that are responsible for citizen data or services, especially
in the face of a growing understanding of and sensitivity to the
security failures of government systems.

Government agencies responsible for these systems have an even harder
time earning and maintaining trust.  Distrust is prevalent both in
government employees that must use government systems and in citizens
whose private information is stored in government systems.

As such, earning and maintaining public trust in E2E-VIV systems will
require an extraordinary amount of transparency and strategy, and will
be a proactive public relations exercise.

At its core, since E2E-VIV systems' correctness and security rely upon
deep mathematical and computer science foundations, very few citizens
can directly understand them and come to trust them through objective
evidence. Consequently, the measure of trustworthiness that E2E-VIV
systems have will always be a delegated trust in the hands of a few
experts. The feasibility of public acceptance therefore depends on the
trustworthiness of those experts and the evidence that they, and the
E2E-VIV system itself, can produce.

%=====================================================================
\section{Integrated Feasibility Analysis}

This chapter's feasibility analysis provides us with the components of
an evaluation of the overall feasibility of building and deploying a
practical E2E-VIV system for U.S. elections.

All technical dimensions---engineering for correctness and security,
design and engineering for usability, availability, operational---have
been deemed feasible, though difficult.

The feasibility along the non-technical dimensions ranges from unknown
to entirely feasible.  The feasible-but-cautious position with respect
to law is contingent upon legislators, election officials, and social
pressure from voters.  The financial, research, integration, and
business aspects are relatively straightforward, and thus feasible.

The biggest open question relates to the core challenge of any modern
IT system: how do people and software relate?

The politics and public acceptance of Internet voting is an open
question, but we believe that only through the disciplined,
transparent, scientific, and practical pursuit of E2E-VIV can we hope
to influence that dialogue and give evidence to the public that we,
and E2E-VIV systems, deserve their trust.

As such, given the context of Internet voting systems in the U.S., we
deem it feasible to pursue future phases of this project, and thus
make our final recommendation (\textbf{MOVE-FORWARD}) in our
concluding chapter.

