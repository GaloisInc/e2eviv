\chapter{Expert Statements\ifdraft{ (Dan/Joe K.) (0\%)}{}}
\label{appendix:expert_statements}

\section{Josh Beneloh}

The Viability of Responsible Internet Voting

Remote voting\footnote{Remote voting is defined here as voting without
  the benefit of the public monitoring that takes place in a
  traditional poll site.}  entails significant risks above and beyond
those of in-person poll-site voting.  Included among these are risks
to integrity – as remotely-cast ballots may pass through numerous
hands without independent observation – and risks to privacy – as
voting takes place without the benefit of publicly-enforced voter
isolation.

Internet voting substantially exacerbates the risks of remote voting
by making it possible for small problems to be magnified and
replicated on a large scale.  Careless or malicious errors, intrusive
malware, and unforeseen omissions – all of which can be caused by
individuals or very small groups – can cause very large numbers of
votes to be changed and the privacy of large numbers of voters to be
compromised.

The technology known as end-to-end (E2E) verifiability allows
individual voters to verify that their intended votes have been
properly recorded and that all recorded votes have been properly
counted.  When applied to in-person voting, E2E-verifiability provides
new assurances to voters by allowing them to check for themselves that
the results of an election are correct.  When applied to Internet
voting, E2E-verifiability mitigates some of the risks described above
– but does not eliminate them:  voters are able to check that their
ballots are properly recorded and counted, but malware can still
compromise privacy, prevent voters from casting their ballots, and
otherwise hinder voters.

Although E2E-verifiable election technologies have existed for more
than thirty years, their use has thus far been limited to small
demonstration systems and private elections for student governments,
professional societies, and the like.  E2E-verifiable elections
produce new challenges and complications for implementers and
administrators.  They represent a new and different paradigm for
elections – substantially replacing the notion of verification of
election equipment with that of verification of the integrity of
individual elections.  As such, it is important to act deliberately
and gain experience with E2E-verifiability in more manageable
environments before attempting to deploy E2E-verifiable elections in
their most challenging environment:  the Internet.

These realities lead us to two principal conclusions.

1.	Public elections should not be conducted over the Internet
using systems that are not end-to-end verifiable.

2.	End-to-end verifiable Internet voting systems should not be
used before end-to-end verifiable poll-site voting systems have been
widely-deployed and experience has been gained from their use.

The second of these two principles is also necessitated by the fact
that an E2E-verifiable election must have a tally to verify, and if an
E2E-verifiable system is used only for remote voters, then the votes
of these remote voters must be separately tallied and reported.  Few
jurisdictions are willing to segregate and report the tallies of local
and remote voters separately. 

We take no position here as to whether the integrity benefits of
E2E-verifiability and the privacy benefits it makes possible outweigh
the risks of remotely-executed large-scale corruption of an
Internet-based election, but we are agreed upon the conclusions that
“naked” Internet voting is dangerous and irresponsible and that
E2E-verifiability should be deployed in the less risky and more
manageable scenario of in-person poll-site voting before it is
deployed in the wilds of the Internet.

