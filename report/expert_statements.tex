\chapter{Expert Statements\ifdraft{ (Dan/Joe K.) (0\%)}{}}
\label{appendix:expert_statements}

The following expert statements are all included unedited, in their
entirety.

\section{Josh Beneloh}

\textbf{The Viability of Responsible Internet Voting}

Remote voting\footnote{Remote voting is defined here as voting without
  the benefit of the public monitoring that takes place in a
  traditional poll site.}  entails significant risks above and beyond
those of in-person poll-site voting.  Included among these are risks
to integrity – as remotely-cast ballots may pass through numerous
hands without independent observation – and risks to privacy – as
voting takes place without the benefit of publicly-enforced voter
isolation.

Internet voting substantially exacerbates the risks of remote voting
by making it possible for small problems to be magnified and
replicated on a large scale.  Careless or malicious errors, intrusive
malware, and unforeseen omissions – all of which can be caused by
individuals or very small groups – can cause very large numbers of
votes to be changed and the privacy of large numbers of voters to be
compromised.

The technology known as end-to-end (E2E) verifiability allows
individual voters to verify that their intended votes have been
properly recorded and that all recorded votes have been properly
counted.  When applied to in-person voting, E2E-verifiability provides
new assurances to voters by allowing them to check for themselves that
the results of an election are correct.  When applied to Internet
voting, E2E-verifiability mitigates some of the risks described above
– but does not eliminate them:  voters are able to check that their
ballots are properly recorded and counted, but malware can still
compromise privacy, prevent voters from casting their ballots, and
otherwise hinder voters.

Although E2E-verifiable election technologies have existed for more
than thirty years, their use has thus far been limited to small
demonstration systems and private elections for student governments,
professional societies, and the like.  E2E-verifiable elections
produce new challenges and complications for implementers and
administrators.  They represent a new and different paradigm for
elections – substantially replacing the notion of verification of
election equipment with that of verification of the integrity of
individual elections.  As such, it is important to act deliberately
and gain experience with E2E-verifiability in more manageable
environments before attempting to deploy E2E-verifiable elections in
their most challenging environment:  the Internet.

These realities lead us to two principal conclusions.

\begin{itemize}
\item Public elections should not be conducted over the Internet
using systems that are not end-to-end verifiable.

\item End-to-end verifiable Internet voting systems should not be
used before end-to-end verifiable poll-site voting systems have been
widely-deployed and experience has been gained from their use.
\end{itemize}

The second of these two principles is also necessitated by the fact
that an E2E-verifiable election must have a tally to verify, and if an
E2E-verifiable system is used only for remote voters, then the votes
of these remote voters must be separately tallied and reported.  Few
jurisdictions are willing to segregate and report the tallies of local
and remote voters separately. 

We take no position here as to whether the integrity benefits of
E2E-verifiability and the privacy benefits it makes possible outweigh
the risks of remotely-executed large-scale corruption of an
Internet-based election, but we are agreed upon the conclusions that
“naked” Internet voting is dangerous and irresponsible and that
E2E-verifiability should be deployed in the less risky and more
manageable scenario of in-person poll-site voting before it is
deployed in the wilds of the Internet.

\section{David Jefferson (and Barbara Simons, Concurring)}

\subsection{Election security is national security}

In a democracy election security is a key part of national
security. The very legitimacy of government depends on the fact, and
also the public perception, that the outcome of every election fairly
represents the will of the people. We must be assured that no part of
voting process has been unfairly manipulated to produce a different
outcome, that all and only those who are eligible to vote have the
opportunity to do so, that no one votes more than once, and that the
privacy of the ballot has not been compromised.

In this paper we demonstrate that end-to-end verifiable (E2E-V) voting
systems have a great deal to offer, but when they are embedded in an
Internet voting context (E2E-VIV) they still do not provide sufficient
security to prevent remote attackers from silently modifying votes and
changing the outcome of elections undetectably, or from disrupting the
election and disenfranchising thousands or millions of
voters. Fundamental security problems remain with E2E-VIV systems for
which there are no practical solutions in sight for the foreseeable
future.

\subsection{Verifiability}

A key principle upon which election security depends is
\emph{verifiability}. After an election is closed it is essential that
there remain enough evidence so that anyone who doubts the results can
re-examine it and be able to rationally deduce whether or not the
outcome of the election was called correctly. We need to verify that
only eligible voters voted, that no votes were lost, that no votes
were duplicated, no votes were modified, that no phony votes were
inserted, and that all the votes were counted correctly, once and only
once. That evidence trail has to be \emph{end to end}, spanning the
entire data path through which the ballot data travels, from the
voters' heads to the final result. We must be able to reconstruct the
vote totals (or statistically audit them) from the original unmodified
ballots or copies of those ballots that are provably identical to the
originals as intended by the voter.

The traditional approaches to verifiability are all based on
physically secured and indelible paper ballots (or paper cast vote
records) that can be recounted or audited by humans without having to
trust any software or complex machines. The goal of end-to-end
verifiable (E2E-V) systems is to use cryptographic protocols to
achieve for all-electronic voting systems the same (or higher) level
of confidence in the election outcome as is achievable with
paper-based systems. The goal of end-to-end verifiable \emph{Internet
  voting} (E2E-VIV) systems is identical, but specifically extended to
the much more difficult security context of remote voting from private
platforms and devices over the public Internet.

Some approaches to end-to-end ``verification'' at first sound great,
but fall far short. It is not sufficient, for example, to provide a
feature whereby each voter can verify that her own ballot was
correctly transmitted to the server over the Internet. First, it is
difficult to provide verification capability without also making it
possible for the voter to prove to a third party how she voted,
thereby enabling automated vote selling and voter coercion. But even
if that problem were resolved and if every voter verified that her
voted ballot was properly received, it would not be possible to
demonstrate that no phony votes, unassociated with any actual voter,
were inserted.

\subsection{The power of E2E-V systems}

End-to-end verifiable voting systems are a major conceptual and
mathematical step forward from conventional voting systems. Through
advanced cryptographic techniques these systems, while differing in
many ways in their architecture and in the roles of insiders, share
the following fundamental security properties:

\begin{enumerate}[label={\alph*})]
\item \textbf{Integrity:} Once a voter successfully enters her ballot
  into the E2E-V system it cannot be undetectably lost or modified in
  any way, even in the presence of bugs or malicious logic.

\item \textbf{Privacy:} Once a ballot enters the E2E-V system it is
  encrypted, so that there is no way that the privacy of the ballot to
  be violated subsequently.

\item \textbf{Counting accuracy:} The ballots cannot be miscounted
  without that fact being detectable.

\item \textbf{Universal public verifiability:} The systems output and
  publish sufficient verification data that \emph{anyone} can verify
  that no ballots were lost or modified and that the votes were
  properly counted. The verification data provides essentially a
  \emph{mathematical proof} (at least up to the strength of the
  cryptography) that ballot integrity was preserved and the counts are
  correct. Anyone is free to run a verification program over the
  verification data to confirm it. You don't even have to trust the
  official verification program---you can use one from a source you
  trust, or if you have the skill you can write your own.

\item Openness and transparency: The code for E2E-V systems is
  generally open source. The mathematical principles underlying the
  E2E-V security guarantees have been vetted by multiple cryptographic
  experts and are open and public. And the specifications for proof
  checkers are also publically documented so that mutually suspicious
  political groups can hire their own experts whose independent
  election verification programs, if correct, must agree.
\end{enumerate}

These powerful E2E-V security properties are not shared by any
traditional voting system. In a precinct voting context they make
E2E-V systems essentially invulnerable to ordinary software bugs, to
deliberately malicious code, to transient hardware faults, and to most
kinds of insider fraud (at least without a large conspiracy). Any lost
or modified ballots, and any miscounts will be quickly detected when
anyone runs a verification program. This is in strong contrast to
other forms of purely electronic voting systems which are
unverifiable, and in which bugs or malicious logic can cause errors
that are totally undetectable, so that the wrong people may wind up
taking office without anyone knowing that the election results were
incorrect.

For these reasons it is appropriate to consider E2E-V systems in any
electronic voting situation. E2E-V adds truly powerful security
guarantees, particularly in a precinct voting situation where voter
authentication is done in person and where we have good reason to
presume that the certain software in the voting machines are not
malicious.

But as we explain in the next section, these E2E-V security guarantees
\emph{do not fully extend to Internet voting systems}. E2E-V Internet
voting systems (E2E-VIV) have exploitable security holes for which
there are no good solutions today and that preclude them from being
suitable for use in public elections for the foreseeable future.
  
\subsection{Remaining unsolved security issues with E2E-VIV systems}

Once the voters' choices are safely input to a precinct-based E2E-V
system many, but not all, of the security guarantees described in the
last section follow directly. But that is a key qualification: these
guarantees begin once the voters' choices are safely input to the
E2E-V system, but not before. Unfortunately, when an E2E-V system is
embedded into the Internet voting context as an E2E-VIV system, new
security problems appear that the E2E-V guarantees do not address and
that cannot, with current technology, be fixed. The most serious
problems with E2E-VIV systems arise \emph{before the votes even enter
  the system}. In this section we enumerate the remaining difficult
security problems that will have to be solved definitively before we
can consider deploying an E2E-VIV system.

\subsubsection{Voter authentication}

A central issue with all remote, online voting systems is voter
authentication. The voting system must be able to positively identify
the voter in strong a way, so that it is essentially impossible to
avoid the authentication process, and impossible to fool it so that an
ineligible person is allowed to vote or that someone can fraudulently
impersonate another voter.

Voter authentication is not part of an E2E-VIV system, but is a
separate security issue. Unfortunately it is a very difficult and
complex problem that remains unresolved (in the U.S. at least) for the
foreseeable future.

Strong voter authentication is required for several reasons. Any
online voting system must:

\begin{itemize}
\item verify that potential voters are duly registered or eligible to
  vote in the jurisdiction they attempt to vote in;
\item prevent anyone from voting more than once; and
\item resist vote selling, vote coercion, and proxy voting insofar as
  possible in a remote voting situation.
\end{itemize}

In an online voting system it is not sufficient to use the kinds of
authentication commonly used in ecommerce situations, e.g. passwords,
challenge-response systems based on personal information, or
email-confirmations. These very weak authentication systems are more
or less sufficient in commercial situations where secrecy is not so
important, and where fraudulent transactions can be detected
eventually and frequently be reversed or at least can be absorbed as a
cost of doing business. 

In the national security context of an online election such weak
authentication mechanisms will not suffice. If an attacker has the
technical means to impersonate one voter, he can generally automate
and amplify his methods to impersonate thousands of voters with very
little additional effort. Almost every week we hear of huge data
breaches at commercial or government institutions that have already
allowed vast amounts of personal information on tens of millions of
people to fall into the hands of criminals or foreign powers. Thus,
any authentication mechanisms based on merely presenting personal
information (name, address, account number, driver’s license or social
security number, mother’s maiden name, etc.) is hopelessly compromised
already, and way too weak for use in an election. Unfortunately some
states have implemented such embarrassingly weak online authentication
systems and have been forced to strengthen them, though they are still
not sufficiently strong.

The traditional voter authentication method is based on wet ink
signature matching. The voter is required, either in person at the
precinct or on the envelope of a mail-in ballot, to duplicate with a
new ink signature the old signature image on file from the time she
registered to vote. In some states this is augmented with VoterID
requirements at the polls. But there is no way to securely
(unforgeably) input a wet ink signature image to a computer or mobile
device and transmit it over the Internet for authentication with the
ballot. Nor is there yet a way to securely and unforgeably transmit
any of the usual VoterID documents.

Many people have suggested voter authentication systems based on
biometrics such as fingerprints or retinal scans. For very good
reasons too numerous to fully explain here none of these mechanisms is
suitable for online voting. While some mobile phones and tablets have
fingerprint authentication devices built in, such systems authenticate
the user to the device only. They do not authenticate the user to any
remote service over the Internet, nor can they easily be extended to
do so securely.

There are other stronger, more technical authentication methods that
could be considered.  Voters could be issued cryptographic ID cards
such as the CAC cards issued to DoD personnel or like the national ID
card of Estonia. Cryptographic ID cards would in principle enable
voter authentication from any Internet-connected computer or device
that could read them. But no U.S. state issues such IDs to its
citizens or voters, and it seems unlikely that any will do so in the
foreseeable future. Even if the security climate changes and people
are willing to accept such an ID system, the startup and maintenance
costs will be very high.  Voters would have to buy computers or
devices that could read the cards, and they would almost certainly
have to be useful for other online purposes besides just voting in
order to justify the costs involved to both the government and the
voter.

The fact is that the U.S. has no strong, universally deployed online
citizen identification and authentication system, and none is on the
horizon. While voter authentication is not a fundamentally unsolvable
problem, it is an immense practical problem that has to be solved
before we can consider deploying any online voting system, including
E2E-VIV systems.

\subsubsection{Client side malware}

In an E2E-VIV system voters compose and input their vote choices on a
privately owned (hence unsecured) platform, either PC or mobile
device. If the voting platform is infected with malware or spyware, it
is complicated to prevent the votes from being modified, or reported
to a third party, and impossible to prevent them from just being
thrown away by the malware \emph{before the ballot is encrypted and
  enters the E2E-VIV system.}

The malware threat is ubiquitous now and is fundamental to all online
voting systems. No device is safe from malware infection. No software
safeguards such as commercial antivirus systems are very effective
against it. There are hundreds of ways that malware can infect a
voting platform, sometimes by sophisticated technical means and
sometimes by simply tricking users into doing unsafe things. There are
hundreds of places in the huge multilayered software ecosystem of a PC
or mobile device where malware can hide and be launched from. There
are thousands of places in the runtime logic of operating system,
browser, browser plugin, and related software where malware can act to
silently and invisibly subvert the voting process. And there are
thousands of hardware, OS, browser, combinations, and countless
configuration choices---way too many for any possible comprehensive
defense against malware. A modern PC or mobile device can easily
contain software elements from a hundred different companies or open
source development groups, any one of which may either be malicious or
contain critical vulnerabilities that enables malicious code. Such
vulnerabilities are so numerous that more are discovered all the time
and vendors release security update on a regular basis to plug the
more recently discovered holes. E2E-VIV systems have no ability to
prevent or prevent or even detect the actions of malware before the
votes are safely submitted into the E2E-VIV software.

Malware in voters' computers can undermine the election in three
fundamental ways. 

\begin{enumerate}[label={\roman*})]
\item \textbf{Malware modification of votes:} Malware may actually
  modify the voter’s choices surreptitiously, before they are
  submitted to the E2E-VIV system. The techniques for accomplishing
  this without tipping off the voter will depend on the detailed
  architecture of the voting system, e.g. whether the client side is
  packaged as a full-blown application, or as a mobile app, or a
  Javascript script, or a browser plugin, or some other form. But in
  all cases the voter’s choices must be input to the PC or mobile
  device in the clear, and be processed by a large amount of system
  software and application/browser/script software \emph{before} it is
  encrypted and submitted to the E2E-VIV system. Regardless of the
  E2E-VIV architecture, with today’s software tools it is reasonably
  straightforward for malicious code to modify votes undetectably
  before they are encrypted.

  Depending on the design of the E2E-VIV system, it may be possible
  for some voters to discover that different votes were recorded for
  them than the ones they thought they cast. But even so, there will
  generally be no a way to prove to election officials that they did
  not cast the votes recorded for them by mistake, or cast them
  deliberately and just changed their minds. Whether there is a remedy
  for voters who claim their votes are modified by malware and falsely
  recorded is an unresolved question.

  We cannot eradicate the threat of malware. But in the special case
  of elections there are techniques that in theory can prevent malware
  from surreptitiously modifying votes. Unfortunately they all involve
  additional burdens on the voter in the form of code voting, or
  special hardware devices independent of the PC, or a second
  independent communication channel to the election server that does
  not use the Internet (or at least is guaranteed to use an
  independent path from the one the votes travel). All known methods
  of working around client side malware involve some complication in
  the voting process that will be enough of a barrier, at least for
  the time being, to discourage many voters.

\item \textbf{Malware vote privacy violation:} Malware on a voter’s
  computer or mobile device could allow her completed ballot to enter
  the E2EVIV system, but prior to that it could \emph{also} send a
  copy of her votes to a third party. Unless a voter has considerable
  expertise and has special instrumentation running during the voting
  transaction there is no way for her to know whether this
  happened. And if the instrumentation was not in place before voting
  there is no after-the-fact test that can determine whether this
  happened, and certainly no way to reverse the privacy violation. If
  the voter is voting from a mobile device, as opposed to a PC, often
  no such instrumentation even exists today.

  In any remote voting situation there is always the possibility that
  someone can physically look over a voter’s shoulder and watch her
  vote. That is a risk we live with also with paper mail-in
  ballots. But the main concern is not with individual cases of
  privacy violation, but with widespread automated spying on many
  online votes.

  Widespread vote privacy violation can undermine democracy in two
  major ways. First, in situations where some people have power over
  others (e.g. employers, commanding officers, union supervisors,
  parents, nursing home management) revealing who cast which ballot
  can be the basis for coercion or retaliation. This may not (yet) be
  a widespread concern in the U.S., but it certainly is in other
  countries.

  Also, automated privacy violation can enable large scale, automated
  vote buying.  It is easy to imagine a scheme in which many voters
  are induced to sell their voting credentials, or to run a particular
  program while voting for a particular candidate in exchange for
  PayPal dollars or some other crypto currency such as Bitcoin that
  can be transmitted entirely online. The vote buying transaction
  would likely be totally undetectable by authorities. Even if the
  scheme eventually comes to the attention of authorities, the buyers
  may be long gone, or may be on foreign soil out of reach of
  U.S. law.  In any case there would be no way to know how many votes
  were sold or who the sellers were. Technical tricks, such as
  allowing voters to vote multiple times online with only the last
  cast vote actually counting, are not effective when the attacker
  knows how the system works or when the voter cooperates in a vote
  sale.

  As with malware intended to modify ballots surreptitiously, there
  are workarounds that can prevent malware from surreptitiously
  revealing how someone votes to a third party. But again, they
  complicate the process of online voting sufficiently to be a barrier
  that will discourage many voters from voting

\item \textbf{Malware denial of service:} i.	The easiest and most
  intractable malware attack is one that simply prevents the voter
  from successfully voting. That can be done by any number of
  means. The malware could make it appear that there was an error of
  some kind, which might be frustrating but hardly surprising to
  voters who might either blame themselves or their own flaky
  computers or attribute it to just another buggy online
  service. Alternatively, the malware might perfectly mimic a
  completed voting transaction, so that the voter believes she has
  successfully voted, whereas the malware would simply throw the
  ballot away. 

  Such a denial of service attack might not be very politically
  effective if it is applied to a random set of voters.  But if it can
  be applied \emph{selectively} to voters who would be likely to vote
  in a way that the attacker does not like, then it becomes a powerful
  partisan fraud tool.  The malware writer may want to make a good
  guess as to how the voter will likely vote before deciding whether
  of not to block her vote.  Fortunately, there are many clues in a
  voter’s computer or mobile device to indicate at least a likely
  party preference or social class, and that is probably all the
  information the malware would need.

  Some voters may be sophisticated enough to detect that their ballots
  were never included in the count, especially during the
  post-election verification stage when some might discover that there
  is no record that they voted. But any particular voter would find it
  almost impossible to prove to election officials that she actually
  tried to vote online but that malware prevented it. Perhaps she
  simply never really tried to vote, or for some other technical
  reason not related to malware she had been prevented from
  successfully voting. There would be no evidence anywhere accessible
  to the officials that could help them diagnose the situation. Even
  if the voter brought her computer in for forensic examination by
  experts, chances are that the malware would have erased evidence and
  erased itself, leaving no trace.

  And finally, even if an obvious widespread malware denial of service
  attack were discovered, there would be no way to estimate how many
  ballots were lost and how many voters were disenfranchised. The
  E2E-VIV system does nothing to help with such an estimate because
  the ballots are discarded by the malware before they ever enter the
  E2E-VIV system.

  Unfortunately, while there are (at best inconvenient) workarounds
  for malware that aims to surreptitiously modify a ballot or send it
  to a third party, \emph{there is fundamentally no workaround for
    malware designed to just prevent voting}.  Well-designed malware
  would make the voter believe she had successfully voted, and she
  might never discover until it was too late that she did not. Even if
  she did discover it, the only recourse would be to vote from a
  different, uninfected PC or device, \emph{but she almost certainly
    not know that malware was the cause of the problem and would
    likely not know to vote from a different machine!}

\item Privacy violation: Malware on a voter's computer or mobile
  device could allow her completed ballot to enter the E2EVIV system,
  but prior to that it could also send a copy of her votes to a third
  party. Unless a voter has considerable expertise and has special
  instrumentation running during the voting transaction there is no
  way for her to know whether this happened. And if the
  instrumentation was not in place before voting there is no
  after-the-fact test that can determine whether this happened, and
  certainly no way to reverse the privacy violation. If the voter is
  voting from a mobile device, as oppose to a PC, often no such
  instrumentation even exists today.

  In any remote voting situation there is always the possibility that
  someone can physically look over a voter's shoulder and watch her
  vote. That is a risk we live with also with paper mail-in
\end{enumerate}

Malware is a profound, absolutely fundamental problem that has been
with us since the dawn of the PC age or before and will be with us for
as far into the future as we can see. There is fundamentally no way to
totally eradicate client side malware, or totally immunize against it,
or even detect its presence. Malware is getting easier to write
because templates, kits, libraries, and exemplars of successful
malware are widely available to aid attackers, and because the payoff
far exceeds the risks of getting caught. It is estimated that anywhere
from 10 to 30 percent of all PCs in the world are infected with
malware, and even more when spyware is included. There are probably no
reliable estimates as to the fraction of mobile devices similarly
infected. 

And finally, even if a clean, easy to use, accessible workaround for
client side malware is invented that preserves vote integrity and
privacy, it will not be possible to prevent a malware denial of
service attack that just prevents voting. There is no general way to
thwart such an attack, and no way for voters or election officials to
unambiguously recognize one or, even if it is recognized, to estimate
how many voters were affected.

The conclusion therefore, is that client side malware remains a
fundamental threat that E2E-VIV systems cannot fully defend against. 

\subsubsection{Network attacks and Distributed Denial of Service
  (DDoS) attacks}

E2E-VIV systems are client-server protocols that execute on top of
several layers of other software, including the operating system and
browser on the client side, the operating system and server complex on
the server side, and the various levels of TCP/IP stack all through
the Internet, as well as routing protocols, DNS, NTP, DHCP, and
numerous others used in wireless or mobile devices. The E2E-VIV system
cannot work properly unless all of this other software works properly
also. We have already discussed the problem of malicious logic on the
client side. But E2E-VIV software is also attackable from the server
side or from the Internet infrastructure software that the E2E-VIV
software depends on.

There are many ways to attack an E2E-VIV election by maliciously
modifying or configuring the software in the Internet. Such attacks
are called \emph{network attacks}. Any IT person who controls a
router, DNS server, or another element of Internet infrastructure is
in a position to prevent votes from getting to their destinations.  On
the positive side, E2E-VIV systems are partly robust against such
attacks in that they cannot result in votes being falsely injected or
modified without detection. This is a clear advantage that E2E-VIV
systems have over other Internet voting systems. However, there is no
way to prevent a network attack from disrupting the E2E-VIV protocols
in a way that causes ballots to be lost, i.e. undelivered.  While this
will also be detectable, the malicious loss of votes in transit cannot
be prevented by an E2E-VIV protocol, and it may not be possible even
to estimate the number of votes affected.

One especially dangerous form of network attack is the
\emph{distributed denial of service} (DDoS) attack.  In this attack,
an attacker floods the server (or some other subsystem) with so much
traffic or other work that it either crashes the system or else slows
it to such a crawl that it is effectively down. Voters would
experience a DDoS attack on a vote server as either
\emph{extreeeeemly} long waits between steps in the voting process, or
total nonresponsiveness of the voting system. The net effect is that
large numbers of voters would simply be disenfranchised. The attack
can be directed pointedly at the server side, in which case all online
voters would be affected, or it could be selectively directed at
certain parts of the Internet infrastructure that would affect only a
subset of the voters.

We have to be able to defend online elections against DDoS attacks for
two key reasons.  First they are about the easiest of all network
attacks to perpetrate. There are many different kinds of DDoS attack
on different parts of Internet infrastructure and different levels of
software, and there are many kits available on the dark net to allow
anyone from anywhere in the world to perpetrate a DDoS attack against
almost any target on the Internet. In fact, the means of DDoS attack
are so routinized and ubiquitous that there are illegal businesses
online that will conduct an attack to your specifications against any
target you choose for a moderate price. You can ask for, say, a 50
gigabit per second attack for the last 4 hours on Election Day against
the IP address of the (hypothetical) Cook County vote server. That
would probably prevent anyone from voting online in that jurisdiction
during those hours. All of those voters would be disenfranchised, but
none of them would be able to prove that they were among of the
victims, and election officials would not even be able to make a
decent estimate of the number of ballots lost.

The second reason that we have to pay attention to DDoS attacks is
that they have actually been used in real public elections around the
world at least four separate times that have been made
public. (Arizona Democratic Primary, 2000; Ontario NDP, 2003; Hong
Kong people's election, 2012; NDP of Canada 2012).

While there are various tools that can be used to \emph{ameliorate}
some DDoS attacks, there is no \emph{general solution}, and the DDoS
problem is so fundamental that there will probably never be one with
the current architecture of the Internet. Vulnerability to DDoS
attacks is effectively built in to its design. Hence, all E2E-VIV
systems are vulnerable to network attacks that can result in
disenfranchising a large number voters with no way of even measuring
how many were affected. There is no fundamental defense against DDoS
attacks.

\subsection{Conclusion}

E2E-V offers a dramatic improvement in the security of voting
systems. It is necessary for \emph{any} online voting system for
public elections, but it is not sufficient. Once it is embedded in a
larger \emph{Internet voting} context fundamental new security
vulnerabilities appear for which there are no solutions today, and no
prospect of solutions in the foreseeable future. These include
vulnerability to authentication attacks, client side malware attacks,
and DDoS attacks. Unless and until those additional security problems
are satisfactorily and simultaneously solved---and they may never
be---we must not consider any Internet voting system for use in public
elections.
