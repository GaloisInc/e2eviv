\chapter{Conclusion\ifdraft{ (Joe K./Susan) (100\%)}{}}
\label{chapter:conclusion}

There is tremendous pressure to build Internet voting systems and use
them in public elections. Researchers, developers, and election
officials must take the time to understand the requirements for secure
and trustworthy elections so that they can evaluate systems---both
good and bad---and make well-informed decisions.  The use of flawed
election systems in public elections can result in significant and
irrevocable harm.

This report presents the most complete set of requirements to date
that must be satisfied by any Internet voting system for public
elections. This set of requirements, described
in~\autoref{chapter:required_properties} and published in complete
detail in a separate document~\cite{E2EVIVBON}, is useful to
several audiences:
\begin{itemize}
\item \emph{legislators} and their staffs who may craft laws that
  relate to remote elections, particularly Internet elections and
  elections for overseas, military, and disabled voters;
\item \emph{election officials} who may specify, evaluate, or
  purchase Internet voting products or services;
\item \emph{activists} who wish to better understand, and advocate
  for, E2E-VIV systems;
\item \emph{standards bodies} that may standardize various classes of
  Internet voting technologies, and specify the level of rigor for
  certifying Internet voting systems;
\item \emph{testing organizations} that may test Internet voting
  systems for compliance with technical standards;
\item \emph{researchers and engineers} who may continue working
  toward viable E2E-VIV systems.
\end{itemize}

This report\iftechreport\else\ and the expanded technical report\fi\ also
contains additional information useful to a subset of these audiences,
including:
\begin{itemize}
\item a basis for developing the cryptographic foundations with which
  to evaluate and compare various E2E protocols and E2E-VIV
  systems~(\autoref{chapter:crypto_spec}),
\item an analysis of the architecture space of E2E-VIV
  systems~(\autoref{chapter:architecture}),
\item precise recommendations on the state of the art for rigorous
  engineering of E2E-V systems~(\autoref{cha:rigor-softw-engin}),
\item a framing for an ongoing discussion about the feasibility of
  designing, constructing, certifying, legalizing, and deploying
  E2E-VIV systems~(\autoref{chapter:feasibility}), and
\item a reflection upon the outstanding issues that must be addressed
  in future stages of E2E-VIV development, including political, legal,
  research, and engineering challenges~(\autoref{sec:next-steps}).
\end{itemize}

Following are recommendations and possible next steps.

%=====================================================================
\section{Recommendations}

The E2E-VIV Project team does not assert that Internet voting
\emph{must} be pursued, nor does it assert that Internet voting must
\emph{never} be pursued. There is no consensus among the team members
for either of these positions. With this understanding, the project
team recommends the following.

\recommendation{E2E-V}{Any public
  elections conducted over the Internet must be end-to-end
  verifiable.}

The use of Internet voting systems without end-to-end
verifiability---including all Internet voting systems that
jurisdictions are experimenting with and using at the time of this
writing---is irresponsible. Any voting systems used to conduct public
elections over the Internet must be E2E-VIV systems.

\recommendation{SUPERVISED FIRST}{No
  Internet voting system of any kind should be used for public
  elections before end-to-end verifiable in-person voting systems have
  been widely deployed and experience has been gained from their use.}

It is critical to gain experience with E2E-V in the simpler in-person
setting before attempting to deploy it in the vastly more complex
Internet setting. Using E2E-V for in-person elections will also
improve the integrity of existing in-person voting systems.
 
If election officials and election system vendors ignore these first
two recommendations, we expect that deficient, unverifiable Internet
voting systems will be widely used within ten years.  Vendors will
claim to have solved the security problems, and eager officials will
believe these claims.  Elections may be altered with no public
awareness.  If election officials manage to find evidence left by a
careless attacker after altering an election, the damage will have
already been done.
 
In making these first two recommendations, we realize that we have
created a difficult path to follow. We assert that no attempt at
Internet voting should deviate from this path.  Building an in-person
E2E-V system is no small task. Building an E2E-VIV system that
satisfies the requirements in this report is even more ambitious; it
may even be impossible. However, if it is possible, the resulting
system will be far better than the vulnerable Internet voting
alternatives.

\recommendation{HIGH ASSURANCE}{End-to-end verifiable
  systems must be designed, constructed, verified, certified,
  operated, and supported as high-assurance systems according to the
  most rigorous engineering requirements of mission- and
  safety-critical systems.}

A software independent voting system does not rely on high-assurance
software to detect errors in the election outcome. However,
high-assurance software engineering tools and techniques can make such
errors much less likely to occur, and can also reduce the risk of the
following problems:

\begin{enumerate}
\item \textbf{PRIVACY VIOLATIONS.} While E2E-V systems can identify
  and mitigate issues of election integrity, they cannot do the same
  for privacy issues. A poorly-implemented E2E-V system will allow
  observers to detect certain issues with the election (such as votes
  not being counted correctly), but not to detect when voter
  identification details are stolen from an insufficiently secured
  server.
\item \textbf{PROGRAMMING ERRORS.} A low-quality E2E-V system is far
  more likely than a high-quality one to have software engineering
  flaws in design, functionality, security or other areas that trigger
  failures in verification. These will increase the burden on election
  administrators to deal with partial failures. This could also
  significantly impact the voters' trust in the election process, as
  well as the election administrators, apparatus, and outcome.
\item \textbf{SECURITY ISSUES.} Low-quality implementations of any
  type of software system are extremely difficult---and often
  impossible---to secure in the presence of insider or outsider
  attack. Security is not a band-aid to apply to a poorly-implemented
  system; it can only be achieved through a combination of rigorous
  process, method, design, implementation, validation, verification,
  deployment, and operation.
\end{enumerate}

High-assurance software engineering is the only reasonable way to
attempt to implement an E2E-V system that is correct, secure, and does
not have enormous fiscal and trust implications for election officials
after deployment. A less rigorous development approach will almost
certainly lead to costly defects.

\recommendation{UNIVERSAL DESIGN}{E2E-VIV systems must be usable and accessible.}

It is not feasible to make voting easy for voters with the most
extreme disabilities. However, it is essential that we at least serve
voters who have challenges in vision, hearing, comprehension, or
motion yet can still use some kind of computing device. We should use
a qualitative and quantitative testing-based experimentation platform
to assess usability and accessibility, and follow best
practices~\cite{materials-at-elections.itif.org},
recommendations~\cite{WAI,Section508,WAVE}, and
standards~\cite{ADAStandards} in accessible UI design and
implementation. By doing so, we will be able to service nearly every
overseas and military voter and, in the long term, the more than 84
million disabled voters in the U.S.~\cite{CensusData}.

We must also look to, and learn from, the AnywhereBallot and EZ Ballot
experiments~\cite{AnywhereBallot,lee2012ez} of the Accessible Voting
Technology Initiative~\cite{AVTI}.  We must engage with the
researchers and attendees at the annual California State University,
Northridge International Technology and Persons with Disabilities
Conference~\cite{CSUN}. Only through direct engagement with voters,
both abled and disabled, can we have any hope of understanding how to
develop a usable, accessible E2E-VIV system.

\recommendation{MOVE FORWARD}{Many challenges remain
  in building a usable, reliable, and secure E2E-VIV system. They must
  be overcome before using Internet voting in public
  elections. Research and development efforts toward overcoming those
  challenges should continue.}

Building a usable, reliable, and secure Internet voting system may be
impossible. Solving the remaining challenges, however, would have
enormous impact on the world. Continued research and development
efforts must be conducted transparently, with all results and
artifacts open to peer review. Internet voting systems, including
E2E-VIV systems, must not be deployed in public elections before all
the key security problems are resolved.

%=====================================================================
\section{Next Steps}
\label{sec:next-steps}

To carry out these recommendations, legislators, researchers, and
engineers face several challenges.

%~~~~~~~~~~~~~~~~~~~~~~~~~~~~~~~~~~~~~~~~~~~~~~~~~~~~~~~~~~~~~~~~~~~~~
\subsection{Political and Legal Challenges}

The greatest concern voiced by election verification scientists,
election integrity advocates, and E2E-V researchers is that
legislators will mandate the experimentation with---or use
of---Internet voting before a correct, secure, open, usable,
accessible E2E-VIV system exists. Using the current untested and
unverified systems opens the door to wholesale election manipulation
or failure. Aggressive early adoption of election technology must be
tempered by a clear understanding that voters' trust in their
elections is hard-won and easily lost.

Scientists and election integrity advocates are also very concerned
that well-meaning legislators and election officials will push to
deploy Internet voting systems too early and too quickly, based on
misleading information from prospective vendors and other advocates
that is not balanced with \emph{independent} advice from cybersecurity
experts.  They may also misunderstand how the security risks grow with
the scale and significance of the election, and how these risks change
over time as the threat environment changes. Such misjudgments may
sometimes induce legistators and election officials to weigh political
goals more highly than the security risks.

The political and legal challenges---and related
opportunities---should focus on how to legislate the evidence-based
measured introduction of new elections technologies. This includes
Internet voting. \emph{Defining an appropriate pace, milestones, and
  success criteria for the introduction of E2E-VIV systems must be a
  primary focus of any next phase of this project.}

%~~~~~~~~~~~~~~~~~~~~~~~~~~~~~~~~~~~~~~~~~~~~~~~~~~~~~~~~~~~~~~~~~~~~~
\subsection{Research and Engineering Challenges}

Although E2E-V is necessary for a viable Internet voting system, use
of E2E-V does not ensure that an Internet voting system is free from
vulnerabilities. Also, the definition of an E2E-VIV protocol for
U.S. elections is very challenging. In particular, the research
community must determine how to address five key challenges:
\begin{itemize}
\item how to handle large-scale dispute resolution;
\item how to authenticate voters for public elections;
\item how to defend an E2E-VIV system against denial-of-service
  attacks and automated attacks that aim to disrupt large numbers of
  votes;
\item how to make verifiability comprehensible and useful to the
  average voter; and
\item how to avoid voter coercion and vote selling in the context of
  digital observation of voting and verification.
\end{itemize}

The usability facets of E2E-VIV are also challenging. The main issues
with usability are:
\begin{itemize}
\item how to ensure usable vote privacy and vote integrity in the
  presence of client-side malware and
\item how to ensure that verification is usable and accessible to the
  typical set of voters.
\end{itemize}

While some of these issues can be addressed by current technologies,
further research is necessary to determine if all of these concerns
can be adequately addressed, as discussed at length in the preceding
chapters and as codified in our requirements. \emph{Until researchers
  adequately address these challenges, Internet voting systems should
  not be used in public elections.}

%~~~~~~~~~~~~~~~~~~~~~~~~~~~~~~~~~~~~~~~~~~~~~~~~~~~~~~~~~~~~~~~~~~~~~
%\subsection{Engineering Challenges}

The development and deployment of a high-assurance distributed system
of the scale and import of a public E2E-VIV election system has never
been attempted. It involves considerable engineering challenges in
addition to the fundamental research challenges already
mentioned. \emph{If the research issues can be solved, current best
practices for building high-assurance distributed systems should be
sufficient to address the engineering issues.}

%~~~~~~~~~~~~~~~~~~~~~~~~~~~~~~~~~~~~~~~~~~~~~~~~~~~~~~~~~~~~~~~~~~~~~
\section*{Coda}

Many people believe that Internet voting will increase voter
participation, help with voter decision-making and engagement, provide
equal opportunity for voters with disabilities, and decrease election
costs.

Proponents of E2E-V election systems hope that their adoption will
prevent corrupt election officials and governments from manipulating
election outcomes, and will truly capture the voice of the people and
increase confidence and trust in government.

Trustworthy democracy is a worthwhile goal, and we should strive to
achieve it. The only responsible way to make progress is to continue
peer-reviewed research and experimentation.
