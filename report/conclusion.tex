\chapter{Conclusion\ifdraft{ (Joe K./Susan) (100\%)}{}}
\label{chapter:conclusion}

Internet voting is a like an impending wave moving towards voters and
elections officials. By the time it hits we must have a firm
foundation for trust in our elections, particularly in regards to
their correctness, security, usability, and accessibility.

Given the in-depth analysis in this report, the E2E-VIV Project team
has produced the following results, made a concrete set of
recommendations, and highlighted next steps for various related
communities.

%=====================================================================
\section{Results}

As promised in the original project proposal, we have produced several
deliverables.

The primary deliverable is the ``whole product solution''
specification contained in~\autoref{chapter:required_properties}. This
specification, in the form of a set of mandatory requirements, is
useful to several audiences, including:
\begin{itemize}
\item \emph{legislators} and their staffs who are interested in
  crafting laws that relate to remote elections, particularly Internet
  elections and elections for overseas, military, and disabled voters;
\item \emph{election officials} specifying, evaluating, or purchasing
  Internet voting products or services;
\item \emph{activists} interested in obtaining a better understanding
  of, and advocating for, E2E-VIV systems;
\item \emph{testing organizations} interested in certifying Internet voting
  systems;
\item \emph{standards bodies} interested in standardizing what
  constitutes various classes of Internet voting technologies and how
  rigorous certification of Internet voting systems should be
  specified;
\item \emph{technologists} interested in implementing E2E-VIV systems.
\end{itemize}

This report also contains several other deliverables useful to a
subset of these audiences, including:
\begin{itemize}
\item a cryptographic foundation with which to evaluate and compare
  various E2E protocols and E2E-VIV
  systems~(\autoref{chapter:crypto_spec}),
\item an analysis of the architecture space of E2E-VIV
  systems~(\autoref{chapter:architecture}),
\item precise recommendations on the state of the art for rigorous
  engineering of E2E-V systems~(\autoref{cha:rigor-softw-engin}),
\item a framing for an ongoing discussion about the feasibility of
  designing, constructing, certifying, legalizing, and deploying
  E2E-VIV systems~(\autoref{chapter:feasibility}), and
\item an analysis of the outstanding issues that must be tackled in
  future stages of E2E-VIV development, including political, legal,
  research, engineering, and business
  challenges~(\autoref{sec:next-steps}).
\end{itemize}

%=====================================================================
\section{Recommendations}

The E2E-VIV Project team has come to the following conclusion and set
of recommendations, with the caveat that the experts \emph{do not}
assert that Internet voting \emph{must} be pursued, nor do they assert
that Internet voting must \emph{never} be pursued. There is clearly no
consensus for either of these positions.\todokiniry{This is being
  dropped into this document in this form not because it has been
  decided, but simply to expedite editing and layout as we push to
  completion.  I am using an edited version of Josh's proposed text
  and ideas here---they do not reflect Galois's position in these
  matters.}

\vspace{12pt} 
\textbf{\emph{Recommendation E2E-V.} \ All public elections conducted
  over the Internet must be end-to-end verifiable.} 
\todo{this was the original wording: \textbf{\emph{Recommendation
      E2E-V.} Public elections must not be conducted over the Internet
    using systems that are not end-to-end verifiable.}}

\textbf{\emph{Recommendation SUPERVISED-FIRST.} \ End-to-end
  verifiable Internet voting systems must not be used until
  communities have widely deployed end-to-end verifiable poll-site
  voting systems and gained experience from their use.}

Some experts say that public elections should never be conducted over
the Internet. Despite this stance, the experts uniformly support these
recommendations. The use of Internet voting systems without end-to-end
verifiability, which includes all Internet voting systems that
jurisdictions are experimenting with and using at the time of this
writing, is irresponsible. All voting systems used to conduct public
elections over the Internet must be E2E-VIV systems.

It is clearly prudent to gain experience with E2E-V in the simpler
in-person setting before attempting to deploy it in the vastly more
complex Internet setting. In addition, using E2E-V for in-person
elections will improve the integrity of existing in-person voting
systems.
 
The E2E-VIV Project team firmly believe that we will be ignored if we
assert that Internet voting cannot be conducted with adequate security
and assurance. If that happens, we expect that deficient, unverifiable
Internet voting systems will be widely used within ten years.  Vendors
will claim to have solved the security problems, and eager officials
will believe these claims.  Elections may be altered with no public
awareness, and vendors will assert that they were right and we were
wrong.  If election officials manage to find evidence left by a
careless attacker after altering an election, it will be a Pyrrhic
victory.
 
In making these first two recommendations, we realize that we have
created a narrow path to adoption and asserted that no attempt at
Internet voting should deviate from this path.  This path is difficult
to follow.  Building an in-person E2E-V system is no small task.
After several years, there will be few, if any, E2E-VIV systems in
use.  However, any E2E-VIV system is far better than the vulnerable
Internet voting alternatives. 

\vspace{12pt} 
\textbf{\emph{Recommendation HIGH-ASSURANCE.} \ E2E-VIV systems must
  be designed, constructed, verified, certified, operated, and
  supported as high-assurance systems according to the most rigorous
  engineering requirements of mission- and safety-critical systems.}

It is sometimes argued that a \emph{strongly software independent}
voting system need not be implemented in high-quality software. While
theoretically true, this attitude has no practical value. A
low-quality E2E-V implementation will likely exhibit some or all of
the following:

\begin{enumerate}
\item \textbf{Privacy violations.} While E2E-V systems can identify
  and mitigate issues of election integrity, they cannot do the same
  for privacy issues. A poorly-implemented E2E-V system will allow
  observers to detect when something goes ``wrong'' with the election
  (e.g., a bug or a hacker causes a vote to disappear), but not when
  voter details are stolen from an insufficiently secured server.
\item \textbf{Programming errors.} A low-quality E2E-V system is far
  more likely than a high-quality one to have software engineering
  flaws in design, functionality, security or other areas that trigger
  failures in verification. These will increase the burden on election
  administrators to deal with partial failures, and potentially have
  significant impact on the voters' trust in the election, its
  administrators, its apparatus, and its outcome.
\item \textbf{Security issues.} Low-quality implementations of any
  type of software system are extremely difficult---and often
  impossible---to secure in the presence of insider and outsider
  attack. Security is not a band-aid to stick on a poorly-implemented
  system; it can only be achieved through a combination of rigorous
  process, method, design, implementation, validation, verification,
  deployment, and operation.
\end{enumerate}

In summary, the only way to reasonably implement an E2E-V system that
is correct, secure, and does not have enormous post-deployment fiscal
and trust implications for election officials is to do so using
high-assurance software engineering tools and techniques.
% I don't know why, but I don't like the way this is phrased. I don't
% even really understand what it says except that it is like fear
% mongering and then it tells me to do something I can't do. Why do I
% feel like it will piss people off? Maybe we could say it differently
% - for example: In light of the summation of these issues which must
% be contended with in the development of an E2E-V system, a
% responsible implementation will.... etc. -SDS
% I don't know of any alternative or counter-argument to the above
% declaration. If it ticks people off, then it'll probably be due to
% the fact that they have no idea how to actually do high-assurance
% engineering. More concrete suggestions or edits are welcome, so long
% as we maintain the thrust of the justification. -JRK
% I have made some minor such edits. -DMZ

\vspace{12pt} 
\textbf{\emph{Recommendation UNIVERSAL-DESIGN.} \ E2E-VIV
  systems must be usable and accessible to the typical set of able and
  disabled voters.}\todo{I think we need some better wording than
  ``typical set''. -dmz}

It is imprudent to design and build a voting system that is usable and
accessible to every voter that exists. The one-in-one-hundred-thousand
voter who is seriously sight, mobility, and mentally disabled would be
assisted at the expense of denying a useful system to 99.999\% of the
population.

Instead, in addition to fully able-bodied and mentally fit voters, our
target audience should include voters that have challenges in vision,
hearing, comprehension, and motion yet still can use some kind of
computing device. By tackling the low-hanging fruit of universal
design---using a qualitative and quantitative testing-based
experimentation platform to assess usability and accessibility and
following best practices~\cite{materials-at-elections.itif.org},
recommendations~\cite{WAI,Section508,WAVE}, and
standards~\cite{standards} in accessible UI design and
implementation---we will be able to service nearly every overseas and
military voter and, in the long term, the more than 84 million
disabled voters in the U.S.~\cite{Brennen,CensusData}.

We must also look to, and learn from, the AnywhereBallot and EZ Ballot
experiments~\cite{AnywhereBallot,lee2012ez} of the Accessible Voting
Technology Initiative~\cite{AVTI}.  We must engage with the
researchers and attendees at the annual California State University,
Northridge International Technology and Persons with Disabilities
Conference~\cite{CSUN}. Only through direct engagement with voters,
both able and disabled, can we have any hope of understanding how to
develop a usable, accessible E2E-VIV system.

\vspace{12pt} 

\textbf{\emph{Recommendation MOVE-FORWARD.} \ A recognized, dedicated
  group of qualified subject-matter specialists with credentials in
  the appropriate subject areas (e.g., verifiable elections,
  cryptography, cybersecurity, high-assurance systems engineering, and
  the universal design of election systems) should be chosen to design
  and prototype a customized open source E2E-VIV system that fulfills
  the requirements included in this report.}

The resulting prototype system should be objectively evaluated using
appropriate validation and verification tools and techniques against
both the functional properties (correctness and security) and the 
non-functional properties (accessibility, capacity, disaster recovery,
fault tolerance, maintainability, performance, reliability,
resilience, scalability, and usability) discussed in this report.

Only with the public availability and peer review of such a focused
E2E-VIV system can we see a path forward toward the use of Internet
voting for public elections.

%=====================================================================
\section{Next Steps}
\label{sec:next-steps}

To carry out these recommendations, there are several open challenges
and next steps for legislators, researchers, engineers, and
businesses.

%~~~~~~~~~~~~~~~~~~~~~~~~~~~~~~~~~~~~~~~~~~~~~~~~~~~~~~~~~~~~~~~~~~~~~
\subsection{Political/Legal Challenges}

The greatest fear voiced by election verification scientists,
activists, and E2E-V researchers is that legislators will aggressively
mandate the experimentation with---or use of---Internet voting before
a correct, secure, open, usable, accessible E2E-VIV system
exists. Such a mandate will facilitate only existing vendors whose
systems, with their lack of verifiability and transparency, open the
door to wholesale election manipulation or failure. Aggressive early
adoption of election technology must be tempered by a clear
understanding that voters' trust in their elections is hard-won and
easily lost.

Scientists and activists are also highly concerned that election
systems will be deployed at an inappropriate pace due to irrational
motivations. For example, directors of elections, secretaries of
state, or legislators may make promises about election modernization
to win an election or to look better than a sister state or
jurisdiction. Perhaps a technology like Internet voting will be
deployed deployed, seemingly successfully, at the local level for an
inconsequential election and, based upon that ``success'', electoral
authorities will decide to reuse the same technology in the next
federal election. These slides down the slippery slope of adoption are
a serious concern and must be avoided.

As such, the political and legal challenges---and related
opportunities---focus on how to legislate the evidence-based measured
introduction of new elections technologies, Internet voting
included. \emph{The precise formulation of a ``not-too-hot,
  not-too-cold'' pace, milestones, and success criteria for the
  introduction of E2E-VIV systems must be a primary focus of any next
  phase of this project.}

%~~~~~~~~~~~~~~~~~~~~~~~~~~~~~~~~~~~~~~~~~~~~~~~~~~~~~~~~~~~~~~~~~~~~~
\subsection{Research Challenges}

The key research challenges highlighted in this report are twofold,
focusing on the standard balance between security and usability.

First, the critical definition of an E2E-VIV protocol for
U.S. elections is very challenging. In particular, the research
community does not immediately know how to solve four key challenges:
(1) how to handle large-scale dispute resolution; (2) how to make
verifiability comprehensible and useful to the average voter; (3) how
to authenticate voters for public elections; and (4) how to avoid
voter coercion and vote selling in the context of digital observation
of voting and verification.

Second, the usability facets of E2E-VIV are also challenging. The main
issues with usability are (5) how to ensure usable vote privacy in the
presence of client-side malware and (6) how to ensure that
verification is usable and accessible to the typical set of voters.

While each of these six challenges is quite difficult, the community
believes that they are surmountable by a dedicated team in a
speculative next phase of this project.

%~~~~~~~~~~~~~~~~~~~~~~~~~~~~~~~~~~~~~~~~~~~~~~~~~~~~~~~~~~~~~~~~~~~~~
\subsection{Engineering Challenges}

The development and deployment of a high-assurance distributed system
of the scale and import of a public E2E-VIV election system has never
been attempted, and involves considerable engineering
challenges. Addressing these challenges, while not for the faint of
heart, is straightforward---at least to those organizations with
expertise in high-assurance distributed systems.

Given the size, complexity, and nature of verified software systems
currently being deployed in military, intelligence, scientific, and
civilian settings, a public E2E-VIV election system can be built by
the right team with appropriate resources.

%~~~~~~~~~~~~~~~~~~~~~~~~~~~~~~~~~~~~~~~~~~~~~~~~~~~~~~~~~~~~~~~~~~~~~
\subsection{Business Opportunities}

Finally, the business opportunities surrounding an E2E-VIV
system, and the potential positive impact on the world that such a
system could have, are enormous motivators for many.

Imagine a world where inexpensive elections have high participation
rates by a well-informed, engaged public. Imagine elections where the
disabled and able have equal opportunity. Imagine elections where
corrupt electoral authorities or governments have no ability to
manipulate the outcome. Imagine elections that truly capture the voice
of the people and increase their confidence and trust in their
governments.

This opportunity to impact the world for good through trustworthy
democracy is a supremely worthwhile goal and, while considerable
challenges remain, we should strive to achieve it.
