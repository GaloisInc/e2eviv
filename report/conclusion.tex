\chapter{Conclusion\ifdraft{ (Joe K./Susan) (100\%)}{}}
\label{chapter:conclusion}

Internet voting (IV) is a like an impending wave moving towards voters and elections
officials.  Ensuring that, when it hits us, we have a firm foundation
for trust in our elections - particularly in regards to their
correctness, security, usability, and accessibility - is paramount.

Given the in-depth analysis summarized in this report, the E2E-VIV
Project team have produced the following
results, made a concrete set of recommendations, and highlighted next
steps for various related communities.

\section{Results}

As promised in the proposal that created this project,
several deliverables have been produced.  

Primary among them is the ``whole product solution'' specification
contained in~\autoref{chapter:required_properties}. This
specification, which comes in the form as a set of mandatory
requirements, is useful to several audiences, including:
\begin{itemize}
\item \emph{legislators} and their staffs who are interested in
  crafting laws that relate to remote elections, particularly IV
  elections and elections for overseas, military, and disabled voters;
\item \emph{election officials} specifying, evaluating, or purchasing
  IV products or services;
\item \emph{activists} interested in obtaining a better understanding
  of, and advocating for, E2E-VIV systems;
\item \emph{testing organizations} interested in certifying IV
  systems;
\item \emph{standards bodies} interested in standardizing what
  constitutes various classes of IV technologies and how rigorous
  certification of IV systems should be specified;
\item \emph{technologists} interested in implementing E2E-VIV systems.
\end{itemize}

Moreover, this report also includes several other artifacts useful to
a subset of these audiences, including:
\begin{itemize}
\item a cryptographic foundation with which to evaluate and compare
  various E2E protocols and E2E-VIV
  systems~(\autoref{chapter:crypto_spec}),
\item an analysis of the architecture space of E2E-VIV
  systems~(\autoref{chapter:architecture}),
\item precise recommendations on the state-of-the-art for rigorous
  engineering of E2EV systems~(\autoref{cha:rigor-softw-engin}),
\item a framing for an ongoing discussion about the feasibility of
  designing, constructing, certifying, legalizing, and operationalizing
  E2E-VIV systems~(\autoref{chapter:feasibility}), and
\item an analysis of the outstanding issues that must be tackled in
  future stages of this field, including political, legal, research,
  engineering, and business challenges~(\autoref{sec:next-steps}).
\end{itemize}

\section{Recommendations}

Synthesizing all of the above information, the E2E-VIV Project team
has come to the following conclusion and set of recommendations, with
the caveat that the experts \emph{do not} assert that Internet voting
(IV) \emph{must} be pursued, nor do they assert that IV must
\emph{never} be pursued. There is clearly no consensus for either of
these positions.\todokiniry{This is being dropped into this document
  in this form not because it has been decided, but simply to expedite
  editing and layout as we push to completion.  I am using an edited
  version of Josh's proposed text and ideas here---they do not reflect
  Galois's position in these matters.}

\begin{center}
  \textbf{\emph{Recommendation E2EV.} Public elections should not be
    conducted over the Internet using systems that are not end-to-end
    verifiable.}
\end{center}

Some may say that public elections should not be conducted over the
Internet at all, but we have uniform support as a community of the
assertion above, as opposed to the kind of ``naked'', or vulnerable
due their defenselessness, IV that we are beginning to see
today. Non-end-to-end verifiable systems are irresponsible and should
never be used. We all know of the numerous vulnerabilities of Internet
applications, and the idea of voting with an unverifiable Internet
application is alarming, to say the least.

\begin{center}
  \textbf{\emph{Recommendation SUPERVISED-FIRST.} End-to-end
    verifiable Internet voting systems should not be used before
    end-to-end verifiable poll-site voting systems have been
    widely-deployed and experience has been gained from their use.}
\end{center}

Gaining experience with E2E-verifiability in the simpler in-person
setting before attempting the more complex Internet setting seems like
a natural and prudent step, but there is another reason for starting
with in-person systems.  Few jurisdictions are willing to release
separate tallies for their in-person and remote voters, but a
verifiable system needs to publish a tally to be verified.  If only
the remote voters use an E2E-verifiable system, then the sub-tally of
the remote voters must be reported.\footnote{It may be possible to mix
  a small number of not individually-verified remote votes in with
  fully-verified in-person votes in such a way as to produce a single
  universally-verifiable tally, but it is difficult to imagine how the
  reverse might be accomplished in a meaningful way.}
 
The pragmatism in this recommendation is simpler than the technical
justification: The E2E-VIV Project team firmly believe that if we
were to assert that Internet voting cannot be conducted with adequate
security and assurance, then we will be ignored and see widespread use
of deficient, unverifiable Internet voting systems within ten years.
Vendors will claim to have solved the security problems, and eager
officials will buy these systems.  Elections may be altered with no
public awareness, and vendors will assert that they were right and we
were wrong.  If election officials manage to find
evidence left by a careless attacker after altering an election, it
will be a Pyrrhic victory.
 
In making these recommendations, we realize that we have created a
narrow pathway to adoption and assert that no attempt at Internet
voting should deviate from this path.  This path is a difficult one to
follow.  After all, building an in-person E2E-verifiable voting system
is no small task.  After several years, there will be few, if any,
E2E-VIV systems in common use.  But any E2E-VIV system is enormously
better than the vulnerable Internet voting alternatives.  

In addition, we recognize that we get the side-benefit of the
deployment of E2E-verifiable in-person systems, which has the bonus of
improving the integrity of our in-person voting systems.

\begin{center}
  \textbf{\emph{Recommendation HIGH-ASSURANCE:} E2E-VIV systems must
    be designed, constructed, verified, certified, operated, and
    supported as high-assurance systems according to the most rigorous
    engineering requirements of mission- and safety-critical systems.}
\end{center}

It is sometimes argued that a \emph{strongly software independent}
voting system need not be implemented in high-quality software. While
theoretically true, this attitude has no practical value. 

The need for quality, even for software independent systems, is due to
several implications of developing and using a low-quality
implementation of any E2E-V voting system:
\begin{enumerate}
\item \textbf{Privacy violations.} While E2E-V systems can mitigate
  issues of election integrity well, they do little to mitigate
  matters relating to privacy. A poorly implemented E2E-V system will
  be able to identify when something goes ``wrong'' with the election
  (e.g., a bug or a hacker causes a vote to disappear), but that is of
  little remedy if voter privacy has been violated.
\item \textbf{The impact of programming errors.} It is vastly more
  likely that a poorly-implemented E2E-V system will have software
  engineering flaws in design, functionality, security or other areas
  which trigger failures in verification. These will thereby
  increasing the burden on the election administrator to mitigate
  partial failures and potentially have significant impact on the
  voters' trust in the election, its administrators, apparatus, and
  outcome.
\item \textbf{Security mandates quality.} A low-quality implementation
  is vastly more difficult to secure in the presence of insider and
  outsider attack. It may even be impossible to secure it. Security is
  not a band-aid to tape onto a poorly built system - it is achieved
  through a combination of rigorous process, method, design,
  implementation, validation, verification, deployment, and operation.
\end{enumerate}

In summary, the only way to reasonable implement an E2E-V system that
is correct, secure, and does not have enormous post-deployment fiscal
and trust implications on election officials is to do so using
high-assurance software engineering tools and techniques.
% I don't know why, but I don't like the way this is phrased. I don't
% even really understand what it says except that it is like fear
% mongering and then it tells me to do something I can't do. Why do I
% feel like it will piss people off? Maybe we could say it differently
% - for example: In light of the summation of these issues which must
% be contended with in the development of an E2E-V system, a
% responsible implementation will.... etc. -SDS
% I don't know of any alternative or counter-argument to the above
% declaration. If it ticks people off, then it'll probably be due to
% the fact that they have no idea how to actually do high-assurance
% engineering. More concrete suggestions or edits are welcome, so long
% as we maintain the thrust of the justification. -JRK
\begin{center}
  \textbf{\emph{Recommendation UNIVERSAL-DESIGN:} E2E-VIV systems must
    be usable and accessible to the typical set of abled and disabled
    voters.}
\end{center}

Attempting to design and build a voting system that is usable and
accessible to every voter that exists is an imprudent task. The
one-in-one-hundred thousand voter who is seriously sight, mobility,
and mentally disabled would be assisted at the expense of denying a
useful system to 99.9\% of the population.

Instead, in addition to fully able-bodied and mentally fit voters, our
target audience should include voters that have challenges in vision,
hearing, comprehension, and motion yet still can use some kind of
computing device. By tackling the low-hanging fruit of universal
design - via a qualitative and quantitative testing-based
experimentation platform to assess usability and accessibility and
following best practices~\cite{materials-at-elections.itif.org},
recommendations~\cite{WAI,Section508,WAVE}, and
standards~\cite{standards} in accessible UI design and implementation
- we will be able to service nearly every overseas and miltiary voter,
but also in the long term, the upwards of 84M disabled voters in the
U.S.A.~\cite{Brennen,CensusData}.

We must also look to, and learn, from the AnywhereBallot and
EZ Ballot experiments~\cite{AnywhereBallot,EZBallot} of the
AVTI~\cite{AVTI}.  We must engage with the researchers and attendees
at CSUN's annual Annual International Technology and Persons with
Disabilities Conference~\cite{CSUN}. Only by having direct engagement
with our stakeholders - voters, both abled and disabled - and
witnessing their joys and struggles of participating in democracy, can
we have any hope of understanding how to develop usable design in
E2E-VIV.

\begin{center}
  \textbf{\emph{Recommendation MOVE-FORWARD:} A recognized dedicated
    group of qualified subject-matter specialists with credentials in
    the appropriate subject areas (e.g., verifiable elections,
    cryptography, cyber-security, high-assurance systems engineering,
    and the universal design of election systems) should be identified
    and tasked to design a customized open source E2E-VIV system that
    fulfills the requirements included in this report and that design
    should be implemented in a set high-assurance prototypes.}
\end{center}

Moreover, those prototypes should be objectively evaluated using
appropriate validation and verification tools and techniques against
both functional properties (correctness and security) and
non-functional properties (accessibility, capacity, disaster recovery,
fault tolerance, maintainability, performance, reliability,
resilience, scalability, and usability) mentioned herein.

Only with the public availability and peer-review of such a focused
E2E-VIV system for the U.S. public can we see a path forward
toward the use of Internet voting for public elections.

\section{Next Steps}
\label{sec:next-steps}

To fulfill these recommendations, there are several open challenges
and next steps for legislators, researchers, engineers, and
businesses.

\subsection{Political/Legal Challenges}

The greatest fear voiced by election verification scientists,
activists, and E2EV researchers is that legislators will aggressively
mandate the experimentation with---or use of---Internet voting before
a correct, secure, open, usable, accessible E2E-VIV system
exists. Such a mandate will facilitate only existing vendors whose
systems open the door to wholesale election manipulation or failure
with their lack of verifiability and transparency. Aggressive early
adoption of election technology must be tempered by a clear
understanding that voters' trust in their elections is hard won and
easily lost.

Scientists and activists are also highly concerned that election
systems will be deployed with an inappropriate pace due to irrational
motivators. For example, directors of elections, secretaries of state,
or legislator may make promises about election modernization to win an
election or to look better than a sister state or jurisdiction. Or
perhaps a technology like Internet voting is deployed, seemingly
successfully, at the local level for an inconsequential election and,
based upon that ``success'', electoral authorities decide, without any
evidence that it is a wise decision, to reuse the same technology in
the next federal election. These slides down the slippery slope of
adoption are another major concern and must be avoided.

As such, the political and legal challenges---and related
opportunities---focus on how to legislate the evidence-based measured
introduction of new elections technologies, Internet voting
included. \emph{The precise formulation for a not-too-hot,
  not-too-cold pace, milestones, and success criteria for the
  introduction of E2E-VIV systems must be a primary focus on any next
  phase of this project.}

\subsection{Research Challenges}

The key research challenge highlighted in this report are twofold,
focusing on the standard balance between security and usability.

First, the critical definition of the E2E-VIV protocol bespoke for
U.S.A. elections is very challenging. In particular, the research
community does not immediately know how to solve four key challenges:
(i) how to handle large-scale dispute resolution, (ii) how to make
verifiability comprehensible and useful to the average voter, and
(iii) how to authenticate voters for public elections, and (iv) how to
avoid voter coercion and vote selling in the context of digital
observation of voting and verification.

Second, the usability facets of E2E-VIV also still has several
research challenges. Principally, they are (v) how to ensure usable vote
privacy in the presence of client-side malware and (vi) how to ensure
that verification is comprehensible, usable, and accessible to the
typical set of voters.

While each of these six questions is challenging, the community
believes that they are surmountable for a dedicated team in a
speculative next phase of this project.

\subsection{Engineering Challenges}

The engineering challenges are straightforward, at least to those
organizations that have expertise in high-assurance systems. Even so,
this challenge is not for the faint-hearted. After all, deploying a
high-assurance distributed system of the scale and import of a public
E2E-VIV election system has never been attempted. 

That being said, given the size, complexity, and nature of verified
software systems being deployed in military, intelligence, scientific,
and civilian settings, this engineering challenge can be conquered
with the right team with appropriate resources.

\subsection{Business Opportunities}

Lastly, the business opportunities for an E2E-VIV system---and the
potential positive impact on the world that such a system will
have---is an enormous motivator for many. 

Imagine a world where inexpensive elections have high participation
rates by a well-informed, engaged public. Imagine elections where the
disabled and abled have equal vote and equal opportunity. Imagine
elections where corrupt electoral authorities or governments have no
ability to manipulate the outcome. Imagine elections that truly
capture the voice of the people and increase their confidence and
trust in their government. 

This opportunity to impact the world for good through trustworthy
democracy is a supremely worthwhile goal, and while challenges remain,
we should strive toward it.


