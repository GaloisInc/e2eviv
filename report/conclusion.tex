\chapter{Conclusion\ifdraft{ (Joe K./Susan) (100\%)}{}}
\label{chapter:conclusion}

Internet voting is a like a wave moving toward voters and elections
officials. By the time it hits, we must have a firm foundation for
trust in our elections.

Following are results, recommendations, and possible next steps.

%=====================================================================
\section{Results}

The primary deliverable of this project is the ``whole product
solution'' specification contained
in~\autoref{chapter:required_properties}. The specification, in the
form of a set of mandatory requirements, is useful to several
audiences, including:
\begin{itemize}
\item \emph{legislators} and their staffs who will craft laws that
  relate to remote elections, particularly Internet elections and
  elections for overseas, military, and disabled voters;
\item \emph{election officials} who will specify, evaluate, or
  purchase Internet voting products or services;
\item \emph{activists} who wish to better understand, and advocate
  for, E2E-VIV systems;
\item \emph{testing organizations} that will certify Internet voting
  systems;
\item \emph{standards bodies} that will standardize various classes of
  Internet voting technologies, and specify the level of rigor for
  certifying Internet voting systems;
\item \emph{technologists} who will implement E2E-VIV systems.
\end{itemize}

Several other deliverables---useful to a subset of these
audiences---include:
\begin{itemize}
\item a cryptographic foundation with which to evaluate and compare
  various E2E protocols and E2E-VIV
  systems~(\autoref{chapter:crypto_spec}),
\item an analysis of the architecture space of E2E-VIV
  systems~(\autoref{chapter:architecture}),
\item precise recommendations on the state of the art for rigorous
  engineering of E2E-V systems~(\autoref{cha:rigor-softw-engin}),
\item a framing for an ongoing discussion about the feasibility of
  designing, constructing, certifying, legalizing, and deploying
  E2E-VIV systems~(\autoref{chapter:feasibility}), and
\item an analysis of the outstanding issues that must be tackled in
  future stages of E2E-VIV development, including political, legal,
  research, engineering, and business
  challenges~(\autoref{sec:next-steps}).
\end{itemize}

%=====================================================================
\section{Recommendations}

The E2E-VIV Project team \emph{does not} assert that Internet voting
\emph{must} be pursued, nor does it assert that Internet voting must
\emph{never} be pursued. There is no consensus among the team members
for either of these positions. With this understanding, the project
team recommends the following.

\vspace{12pt} 
\textbf{\emph{Recommendation: E2E-V.} \ All public elections conducted
  over the Internet must be end-to-end verifiable.} 

The use of Internet voting systems without end-to-end verifiability---
including all Internet voting systems that jurisdictions are
experimenting with and using at the time of this writing---is
irresponsible. All voting systems used to conduct public elections
over the Internet must be E2E-VIV systems.

\vspace{12pt}
\textbf{\emph{Recommendation: SUPERVISED FIRST.} \ End-to-end
  verifiable Internet voting systems must not be used until
  communities have widely deployed end-to-end verifiable poll-site
  voting systems and gained experience from their use.}

It is clearly prudent to gain experience with E2E-V in the simpler
in-person setting before attempting to deploy it in the vastly more
complex Internet setting. Using E2E-V for in-person elections will
also improve the integrity of existing in-person voting systems.
 
If election officials and election system vendors ignore these first
two recommendations, we expect that deficient, unverifiable Internet
voting systems will be widely used within ten years.  Vendors will
claim to have solved the security problems, and eager officials will
believe these claims.  Elections may be altered with no public
awareness.  If election officials manage to find evidence left by a
careless attacker after altering an election, the damage will have
already been done.
 
In making these first two recommendations, we realize that we have
created a difficult path to follow. We assert that no attempt at
Internet voting should deviate from this path.  Building an in-person
E2E-V system is no small task. Building an E2E-VIV system is even more
ambitious, and it will take considerable time. However, the time and
effort will be worth it: any E2E-VIV system will be far better than
the vulnerable Internet voting alternatives. 

\vspace{12pt} 
\textbf{\emph{Recommendation: HIGH ASSURANCE.} \ E2E-VIV systems must
  be designed, constructed, verified, certified, operated, and
  supported as high-assurance systems according to the most rigorous
  engineering requirements of mission- and safety-critical systems.}

It is sometimes argued that high-assurance software is not necessary
when implementing a strongly software independent voting system. While
theoretically true, this attitude has no practical value. A
low-quality E2E-V implementation will likely exhibit some or all of
the following:

\begin{enumerate}
\item \textbf{Privacy violations.} While E2E-V systems can identify
  and mitigate issues of election integrity, they cannot do the same
  for privacy issues. A poorly-implemented E2E-V system will allow
  observers to detect when something goes wrong with the election
  (such as a vote disappearing because of a bug or a malicious
  attack), but not when voter identification details are stolen from
  an insufficiently secured server.
\item \textbf{Programming errors.} A low-quality E2E-V system is far
  more likely than a high-quality one to have software engineering
  flaws in design, functionality, security or other areas that trigger
  failures in verification. These will increase the burden on election
  administrators to deal with partial failures. This could also
  significantly impact the voters' trust in the election process, as
  well as the election administrators, apparatus, and outcome.
\item \textbf{Security issues.} Low-quality implementations of any
  type of software system are extremely difficult---and often
  impossible---to secure in the presence of insider or outsider
  attack. Security is not a band-aid to apply to a poorly-implemented
  system; it can only be achieved through a combination of rigorous
  process, method, design, implementation, validation, verification,
  deployment, and operation.
\end{enumerate}

The only way to reasonably implement an E2E-V system that is correct,
secure, and does not have enormous fiscal and trust implications for
election officials after deployment, is to use high-assurance software
engineering tools and techniques.
% I don't know why, but I don't like the way this is phrased. I don't
% even really understand what it says except that it is like fear
% mongering and then it tells me to do something I can't do. Why do I
% feel like it will piss people off? Maybe we could say it differently
% - for example: In light of the summation of these issues which must
% be contended with in the development of an E2E-V system, a
% responsible implementation will.... etc. -SDS
% I don't know of any alternative or counter-argument to the above
% declaration. If it ticks people off, then it'll probably be due to
% the fact that they have no idea how to actually do high-assurance
% engineering. More concrete suggestions or edits are welcome, so long
% as we maintain the thrust of the justification. -JRK
% I have made some minor such edits. -DMZ

\vspace{12pt} \textbf{\emph{Recommendation: UNIVERSAL DESIGN} \
  E2E-VIV systems must be usable and accessible to able and disabled
  voters.}

Designing and building a voting system that is usable and accessible
to every voter that exists is not a good choice. The
one-in-one-hundred-thousand voter who has serious sight, mobility, or
mental challenges would be assisted at the expense of creating a more
usable system that can serve 99.999\% of the population.

Instead, our target audience should include voters who have challenges
in vision, hearing, comprehension, and motion yet can still use some
kind of computing device. We should use a qualitative and quantitative
testing-based experimentation platform to assess usability and
accessibility, and follow best
practices~\cite{materials-at-elections.itif.org},
recommendations~\cite{WAI,Section508,WAVE}, and
standards~\cite{ADAStandards} in accessible UI design and
implementation. By doing so, we will be able to service nearly every
overseas and military voter and, in the long term, the more than 84
million disabled voters in the U.S.~\cite{CensusData}.

We must also look to, and learn from, the AnywhereBallot and EZ Ballot
experiments~\cite{AnywhereBallot,lee2012ez} of the Accessible Voting
Technology Initiative~\cite{AVTI}.  We must engage with the
researchers and attendees at the annual California State University,
Northridge International Technology and Persons with Disabilities
Conference~\cite{CSUN}. Only through direct engagement with voters,
both able and disabled, can we have any hope of understanding how to
develop a usable, accessible E2E-VIV system.

\vspace{12pt} 

\textbf{\emph{Recommendation: MOVE FORWARD.} \ A recognized, dedicated
  group of qualified subject matter specialists with credentials in
  the appropriate subject areas (such as verifiable elections,
  cryptography, cybersecurity, high-assurance systems engineering, and
  the universal design of election systems) should be chosen to design
  and prototype a customized open source E2E-VIV system that fulfills
  the requirements included in this report.}

The resulting prototype system should be objectively evaluated, using
appropriate validation and verification tools and techniques, against
the requirements discussed in this report. Only by making an E2E-VIV
system publicly available and open to peer review can we see a path
forward to using Internet voting for public elections.

%=====================================================================
\section{Next Steps}
\label{sec:next-steps}

To carry out these recommendations, legislators, researchers,
engineers, and businesses face several challenges.

%~~~~~~~~~~~~~~~~~~~~~~~~~~~~~~~~~~~~~~~~~~~~~~~~~~~~~~~~~~~~~~~~~~~~~
\subsection{Political/Legal Challenges}

The greatest concern voiced by election verification scientists,
election integrity advocates, and E2E-V researchers is that
legislators will mandate the experimentation with---or use
of---Internet voting before a correct, secure, open, usable,
accessible E2E-VIV system exists. Using the current untested and
unverified systems opens the door to wholesale election manipulation
or failure. Aggressive early adoption of election technology must be
tempered by a clear understanding that voters' trust in their
elections is hard-won and easily lost.

Scientists and election integrity advocates are also very concerned
that election systems will be deployed at an inappropriate pace due to
irrational motivations. For example, directors of elections,
secretaries of state, or legislators may make promises about election
modernization to win an election or to look better than a sister state
or jurisdiction. Perhaps a technology like Internet voting will be
deployed, seemingly successfully, at the local level for an
inconsequential election and, based upon that ``success'', election
officials will decide to reuse the same technology in the next federal
election.

The political and legal challenges---and related
opportunities---should focus on how to legislate the evidence-based
measured introduction of new elections technologies. This includes
Internet voting. \emph{Defining an appropriate pace, milestones, and
  success criteria for the introduction of E2E-VIV systems must be a
  primary focus of any next phase of this project.}

%~~~~~~~~~~~~~~~~~~~~~~~~~~~~~~~~~~~~~~~~~~~~~~~~~~~~~~~~~~~~~~~~~~~~~
\subsection{Research Challenges}

Although E2E-V is necessary for a viable Internet voting system, use
of E2E-V does not ensure that an Internet voting system is free from
vulnerabilities. Also, the definition of an E2E-VIV protocol for
U.S. elections is very challenging. In particular, the research
community must determine how to address five key challenges:

\begin{itemize}
\item how to handle large-scale dispute resolution;
\item how to authenticate voters for public elections;
\item how to defend an E2E-VIV system against denial-of-service
  attacks and automated attacks that aim to disrupt large numbers of
  votes;
\item how to make verifiability comprehensible and useful to the
  average voter; and
\item how to avoid voter coercion and vote selling in the context of
  digital observation of voting and verification.
\end{itemize}

The usability facets of E2E-VIV are also challenging. The main issues
with usability are:

\begin{itemize}
\item how to ensure usable vote privacy in the presence of client-side
  malware and
\item how to ensure that verification is usable and
  accessible to the typical set of voters.
\end{itemize}

While some of these issues can be addressed by current technologies,
further research is necessary to determine if all of these concerns
can be adequately addressed, as discussed at length in the preceding
chapters and as codified in our requirements. Until researchers
adequately address these challenges, Internet voting systems should
not be used in public elections.

%~~~~~~~~~~~~~~~~~~~~~~~~~~~~~~~~~~~~~~~~~~~~~~~~~~~~~~~~~~~~~~~~~~~~~
\subsection{Engineering Challenges}

The development and deployment of a high-assurance distributed system
of the scale and import of a public E2E-VIV election system has never
been attempted, and involves considerable engineering
challenges. Addressing these challenges is straightforward for
organizations with expertise in high-assurance distributed systems.

Given the size, complexity, and nature of verified software systems
currently being deployed in military, intelligence, scientific, and
civilian settings, a public E2E-VIV election system can be built by
the right team with appropriate resources.

%~~~~~~~~~~~~~~~~~~~~~~~~~~~~~~~~~~~~~~~~~~~~~~~~~~~~~~~~~~~~~~~~~~~~~
\subsection{Business Opportunities}

Finally, the business opportunities surrounding an E2E-VIV
system, and the potential positive impact on the world that such a
system could have, are enormous motivators.

Imagine a world where inexpensive elections have high participation
rates by a well-informed, engaged public. Imagine elections where the
disabled and able have equal opportunity. Imagine elections where
corrupt election officials or governments have no ability to
manipulate the outcome. Imagine elections that truly capture the voice
of the people and increase their confidence and trust in their
governments.

Impacting the world for good through trustworthy democracy is a
supremely worthwhile goal. Though considerable challenges remain, we
should strive to achieve it.
