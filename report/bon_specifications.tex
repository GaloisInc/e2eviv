% The U.S. Vote/OVF E2E-VIV Report
% BON Specifications
% Joseph R. Kiniry and Daniel M. Zimmerman

\documentclass[]{report}

\usepackage[margin=1in]{geometry}
\usepackage{xcolor}

% Fourier for math | Utopia (scaled) for rm | Helvetica for ss | Latin Modern for tt
\usepackage{fourier} % math & rm
\usepackage[scaled=0.875]{helvet} % ss
\renewcommand{\ttdefault}{lmtt} %tt

\usepackage[final]{listings} % always include listings, even in draft mode


% custom colors
\colorlet{darkgreen}{green!35!black}
\colorlet{darkred}{red!45!black}

% shades of red and green that are distinguishable even with the most common
% kinds of color blindness
\colorlet{accessiblegreen}{green!55!black!70!blue!40!white}
\colorlet{accessiblered}{red!80!black!80!yellow!30!white}

\ifpdf
\usepackage[draft=false,pdftex,colorlinks=true,urlcolor=blue,linkcolor=darkred,citecolor=darkgreen,bookmarks=false]{hyperref}
\else
\usepackage[dvips]{hyperref}
\fi


% BON listing style
\usepackage{color}
\usepackage{listings}


\newcommand{\comment}[1]{\textcolor{red}{#1}}
\definecolor{keywordcolor}{rgb}{0.5,0,0.33}
\definecolor{identifiercolor}{rgb}{0,0,0.75}
\definecolor{commentcolor}{rgb}{0.25,0.5,0.37}
\definecolor{commentcolor}{rgb}{0.3,0.3,0.3} 

\lstdefinelanguage{bon} {
  morekeywords={class_chart,indexing,explanation,part,query,command,constraint,
  end,deferred,effective,persistent,require,ensure,invariant,feature,class,
  static_diagram,component,old,not,inherit,delta,for_all,such_that,it_holds,Current,
  Lockset,when,monitors_for,max,concurrency,concurrent,guarded,locks,special,
  failure,sequential,atomic},
  morekeywords={[2]BOOLEAN,INTEGER,REAL,SEQUENCE}, 
  morekeywords={[3]},
  morecomment=[l]{--}, morestring=[b]", morestring=[d]'
  }[keywords,comments,strings]

\lstdefinestyle{bon}{language={bon},showstringspaces={false},
  basicstyle={\small\ttfamily\mdseries},
  keywordstyle={\color{keywordcolor}},
  keywordstyle={[2]\color{black}\bfseries},
  keywordstyle={[3]\color{black}\bfseries},
  identifierstyle={\color{identifiercolor}},
  commentstyle={\color{commentcolor}},
  frame=none}

\lstdefinestyle{bonbw}{language={bon},showstringspaces={false},
  basicstyle={\small\ttfamily\mdseries},
  keywordstyle={\color{black}},
  keywordstyle={[2]\color{black}\bfseries},
  keywordstyle={[3]\color{black}\bfseries},
  identifierstyle={\color{black}\mdseries},
  commentstyle={\color{black}},
  frame=none,
  columns=fullflexible,
  breaklines=true}
\lstset{style=bon, columns=fullflexible, keepspaces=true, frame=lines,
  captionpos=b, numbers=none} 

\title{THE FUTURE OF VOTING \\ End-To-End Verifiable Internet Voting \\ \ \\ BON Specifications}
\author{Joseph R. Kiniry and Daniel M. Zimmerman \\ Galois}

\begin{document}

\hypersetup{pageanchor=false}
\maketitle
\hypersetup{pageanchor=true}

\tableofcontents

\chapter*{About This Document}

This document is a part of the full report ``THE FUTURE OF VOTING:
End-To-End Verifiable Internet Voting'', available from
\url{https://www.usvotefoundation.org/E2E-VIV}. It presents a
comprehensive domain model and set of requirements for an E2E-VIV
system, using the Business Object Notation (BON). 

BON, developed by Kim Wald\'{e}n and Jean-Marc Nerson and described in
their 1995 book \emph{Seamless object-oriented software architecture:
  Analysis and design of reliable systems}, is both a language and a
design/refinement method encompassing informal domain analysis and
modeling, formal modeling, and implementation-independent high- and
medium-level specification. BON has a well-defined semantics, is easy
to learn and write (especially the informal models, which are
effectively collections of simple English sentences), and has
equally-expressive textual and graphical notations. BON was originally
developed for use with the Eiffel programming language; however, it
can be used with other specification and implementation languages. We
use BON for domain modeling for several reasons.

First, BON's equivalently expressive textual and graphical notations
are easy to work with and manipulate.

Second, BON's semantics are an integral part of the language and
method and are easily understandable. 

Third, BON explicitly supports (and encourages) \emph{seamlessness}
and \emph{reversibility}. Seamlessness is the property that allows a
BON model to be smoothly (and, in many cases, completely
automatically) refined to lower-level specification languages, and
further to executable implementations. Reversibility is the property
that allows consistency to be maintained between the BON model, which
is an important part of the system documentation, and the resulting
implementation---when the implementation is changed, that change can
be (again, often completely automatically) propagated back up to the
BON model. Seamlessness and reversibility are both useful properties
for ensuring that the final software product accurately reflects the
original domain analysis and architecture design.

Fourth, BON supports high-level domain modeling using natural
language, making it easy to communicate models not only among software
developers but also with other stakeholders in the development
process. The BON representation of the E2E-VIV requirements consists
almost entirely of simple English sentences; it is therefore far more
accessible to a wide audience than an equivalent set of box-and-arrow
diagrams would be.

Finally, BON is \emph{simple}. Its specification is small, and it is
easy to understand.

\chapter{BON Representation of E2E-VIV Requirements}

\section{e2eviv.bon}
\lstinputlisting[style=bon]{bon_requirements_resources/e2eviv.bon}
\section{technical.bon}
\lstinputlisting[style=bon]{bon_requirements_resources/technical.bon}
\section{non-functional.bon}
\lstinputlisting[style=bon]{bon_requirements_resources/non-functional.bon}
\section{accessibility.bon}
\lstinputlisting[style=bon]{bon_requirements_resources/accessibility.bon}
\section{assurance.bon}
\lstinputlisting[style=bon]{bon_requirements_resources/assurance.bon}
\section{auditing.bon}
\lstinputlisting[style=bon]{bon_requirements_resources/auditing.bon}
\section{authentication.bon}
\lstinputlisting[style=bon]{bon_requirements_resources/authentication.bon}
\section{certification.bon}
\lstinputlisting[style=bon]{bon_requirements_resources/certification.bon}
\section{evolvability.bon}
\lstinputlisting[style=bon]{bon_requirements_resources/evolvability.bon}
\section{functional.bon}
\lstinputlisting[style=bon]{bon_requirements_resources/functional.bon}
\section{interoperability.bon}
\lstinputlisting[style=bon]{bon_requirements_resources/interoperability.bon}
\section{legal.bon}
\lstinputlisting[style=bon]{bon_requirements_resources/legal.bon}
\section{maintenance.bon}
\lstinputlisting[style=bon]{bon_requirements_resources/maintenance.bon}
\section{operational.bon}
\lstinputlisting[style=bon]{bon_requirements_resources/operational.bon}
\section{procedural.bon}
\lstinputlisting[style=bon]{bon_requirements_resources/procedural.bon}
\section{system\_operational.bon}
\lstinputlisting[style=bon]{bon_requirements_resources/system_operational.bon}
\section{reliability.bon}
\lstinputlisting[style=bon]{bon_requirements_resources/reliability.bon}
\section{security.bon}
\lstinputlisting[style=bon]{bon_requirements_resources/security.bon}
\section{usability.bon}
\lstinputlisting[style=bon]{bon_requirements_resources/usability.bon}
 %Dan/Joe K.
\chapter{BON Representation of E2E-VIV Domain Model}

\lstinputlisting[style=bon]{bon_domain_model_resources/E2EVIV.bon} %Dan/Joe K.

\end{document}
