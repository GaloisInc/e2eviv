% The OVF E2E VIV Report

% Turning off "draft" mode on the report class turns off to do notes,
% chapter assignments, and completion percentages
\documentclass[draft]{report}

\usepackage{times}

\usepackage{ifpdf}
\usepackage{ifdraft}
\usepackage[utf8]{inputenc}
\usepackage{fnpct}
\usepackage{xcolor}
\usepackage[final]{graphicx} % always include images, even in draft mode
\usepackage{xspace}
\usepackage{colortbl}
\usepackage{longtable}
\usepackage{tabu}
\usepackage[inline]{enumitem}
\usepackage[final]{listings} % always include listings, even in draft mode

% Bibliography equipment for fancier citations of websites etc
\usepackage[%
  backend=bibtex      % biber or bibtex
%,style=authoryear    % Alphabeticalsch
 ,style=numeric-comp  % numerical-compressed
%,sorting=none        % no sorting
 ,sortcites=true      % some other example options ...
 ,block=none
 ,indexing=false
 ,citereset=none
 ,isbn=true
 ,url=true
 ,doi=true            % prints doi
 ,natbib=true         % if you need natbib functions
]{biblatex}
\addbibresource{bibliography.bib}

\setcounter{biburllcpenalty}{7000}
\setcounter{biburlucpenalty}{8000}

% BON listing style
\usepackage{color}
\usepackage{listings}


%\newcommand{\comment}[1]{\textcolor{red}{#1}}
\definecolor{keywordcolor}{rgb}{0.5,0,0.33}
\definecolor{identifiercolor}{rgb}{0,0,0.75}
%\definecolor{commentcolor}{rgb}{0.25,0.5,0.37}
\definecolor{commentcolor}{rgb}{0.3,0.3,0.3} 

\lstdefinelanguage{bon} {
  morekeywords={class_chart,indexing,explanation,part,query,command,constraint,
  end,deferred,effective,persistent,require,ensure,invariant,feature,class,
  static_diagram,component,old,not,inherit,delta,for_all,such_that,it_holds,Current,
  Lockset,when,monitors_for,max,concurrency,concurrent,guarded,locks,special,
  failure,sequential,atomic},
  morekeywords={[2]BOOLEAN,INTEGER,REAL,SEQUENCE}, 
  morekeywords={[3]},
  morecomment=[l]{--}, morestring=[b]", morestring=[d]'
  }[keywords,comments,strings]

\lstdefinestyle{bon}{language={bon},showstringspaces={false},
  basicstyle={\small\ttfamily\mdseries},
  keywordstyle={\color{keywordcolor}},
  keywordstyle={[2]\color{black}\bfseries},
  keywordstyle={[3]\color{black}\bfseries},
  identifierstyle={\color{identifiercolor}},
  commentstyle={\color{commentcolor}},
  frame=none}

\lstdefinestyle{bonbw}{language={bon},showstringspaces={false},
  basicstyle={\small\ttfamily\mdseries},
  keywordstyle={\color{black}},
  keywordstyle={[2]\color{black}\bfseries},
  keywordstyle={[3]\color{black}\bfseries},
  identifierstyle={\color{black}\mdseries},
  commentstyle={\color{black}},
  frame=none,
  columns=fullflexible,
  breaklines=true}
\lstset{style=bon, columns=fullflexible, keepspaces=true, frame=lines,
  captionpos=b, numbers=none} 

% custom colors
\colorlet{darkgreen}{green!35!black}
\colorlet{darkred}{red!45!black}

% shades of red and green that are distinguishable even with the most common
% kinds of color blindness
\colorlet{accessiblegreen}{green!55!black!70!blue!40!white}
\colorlet{accessiblered}{red!80!black!80!yellow!30!white}

\usepackage[margin=1in]{geometry}
\usepackage[defaultlines=4,all]{nowidow}
\usepackage[parfill]{parskip}

%% To Do Notes
\usepackage[obeyDraft, colorinlistoftodos, textwidth=\marginparwidth]{todonotes}

% individual colors for To Dos, feel free to change yours
\colorlet{tododmz}{green!50}
\colorlet{todokiniry}{red!50}
\colorlet{tododmwit}{orange!50}
\colorlet{todojkr}{blue!50}
\colorlet{todoprobinson}{purple!50}
\colorlet{todoacf}{cyan!50}
\colorlet{todogeneric}{yellow!50}

\newcounter{todocounter}
\newcommand{\todocount}[2][]{\stepcounter{todocounter}\todo[#1]{\thetodocounter:
    #2}}

% individual commands for To Dos
% usage: \todo<username>{Something to do.}
\newcommand{\tododmz}[1]{\todocount[color=tododmz]{#1}}
\newcommand{\todokiniry}[1]{\todocount[color=todokiniry]{#1}}
\newcommand{\tododmwit}[1]{\todocount[color=tododmwit]{#1}}
\newcommand{\todojkr}[1]{\todocount[color=todojkr]{#1}}
\newcommand{\todoprobinson}[1]{\todocount[color=todoprobinson]{#1}}
\newcommand{\todoacf}[1]{\todocount[color=todoacf]{#1}}
\newcommand{\todogeneric}[1]{\todocount[color=todogeneric]{#1}}

\ifpdf
\usepackage[draft=false,pdftex,colorlinks=true,urlcolor=blue,linkcolor=darkred,citecolor=darkgreen,bookmarks=false]{hyperref}
\else
\usepackage[dvips]{hyperref}
\fi


% define a macro \Autoref to allow multiple references to be passed to
% \autoref
\makeatletter
\newcommand\Autoref[1]{\@first@ref#1,@}
\def\@throw@dot#1.#2@{#1}% discard everything after the dot
\def\@set@refname#1{%    % set \@refname to autoefname+s using \getrefbykeydefault
    \edef\@tmp{\getrefbykeydefault{#1}{anchor}{}}%
    \def\@refname{\@nameuse{\expandafter\@throw@dot\@tmp.@autorefname}s}%
}
\def\@first@ref#1,#2{%
  \ifx#2@\autoref{#1}\let\@nextref\@gobble% only one ref, revert to normal \autoref
  \else%
    \@set@refname{#1}%  set \@refname to autoref name
    \@refname~\ref{#1}% add autoefname and first reference
    \let\@nextref\@next@ref% push processing to \@next@ref
  \fi%
  \@nextref#2%
}
\def\@next@ref#1,#2{%
   \ifx#2@ and~\ref{#1}\let\@nextref\@gobble% at end: print and+\ref and stop
   \else, \ref{#1}% print  ,+\ref and continue
   \fi%
   \@nextref#2%
}
\makeatother

\usepackage{lipsum}
\usepackage{soul}

% Modify the autoref names to be capitalized and not sub-sub-subby.
\renewcommand*{\chapterautorefname}{Chapter}
\renewcommand*{\sectionautorefname}{Section}
\renewcommand*{\subsectionautorefname}{Section}
\renewcommand*{\subsubsectionautorefname}{Section}

% Various definitions, from Aggelos E2E spec

\newcommand{\func}[1][\relax]{\ensuremath{\mathcal{F}_{\mathsf{#1}}}}
\newcommand{\fl}[1]{\mbox{\( \lfloor #1 \rfloor \)}}
\newcommand{\pair}[2]{\mbox{\(\langle #1,#2 \rangle\)}}

\newcommand{\mc}{\mathcal}
\newcommand{\Pro}{\mbox{\( \mathbf{ Prob } \)}}

\def\squareforqed{\hbox{\(\blacksquare\)}}
\def\qed{\ifmmode\squareforqed\else{\unskip\nobreak\hfill\penalty50\hskip1em\null\nobreak\hfil\squareforqed\parfillskip=0pt\finalhyphendemerits=0\endgraf}\fi}

\newtheorem{theorem}{Theorem}[section]
\newtheorem{assumption}{Complexity Assumption}
\newtheorem{lemma}[theorem]{Lemma}
\newtheorem{remark}{Remark}
\newtheorem{claim}{Claim}
\newtheorem{fact}[theorem]{Fact}
\newtheorem{definition}[theorem]{Definition}
\newtheorem{corollary}[theorem]{Corollary}
\newtheorem{proposition}[theorem]{Proposition}
\newenvironment{proof}{\noindent {\em Proof.}}{\medskip}

\def\PPT{{\rm PPT} }
\def\ff{\mathbb{F}}
\def\sbs{\subseteq}
\def\zz{\mathbb{Z}}
\def\E{\mathsf{E}}

%Functionalities
\newcommand{\fete}{\func[e2e]}%
\newcommand{\fbb}{\func[bb]}%
\newcommand{\fsauth}{\func[sauth]}%
\newcommand{\fauth}{\func[auth]}%
\newcommand{\fmail}{\func[mail]}%
\newcommand{\frnd}{\func[rnd]}%

%Parties
\def\EA{\mathsf{EA}}
\def\VT{\mathsf{V}}
\def\RB{\mathsf{RB}}
\def\VC{\mathsf{VC}}
\def\AU{\mathsf{AU}}

\def\Exec{\mathsf{Exec}}


%Commands
\def\Create{\mathtt{Create}}
\def\Sample{\mathtt{Sample}}
\def\FakeBallot{\mathtt{FakeBallot}}
\def\Deliver{\mathtt{Deliver}}
\def\Corrupt{\mathtt{Corrupt}}
\def\Verify{\mathtt{Verify}}
\def\Tally{\mathtt{Tally}}
\def\Vote{\mathtt{Vote}}
\def\RecordVote{\mathtt{RecordVote}}
\def\Append{\mathtt{Append}}
\def\Read{\mathtt{Read}}
\def\Send{\mathtt{Send}}
\def\Mail{\mathtt{Mail}}
\def\Audit{\mathtt{Audit}}
\def\VBB{\mathtt{VBB}}
\def\SendInit{\mathtt{SendInit}}
\def\GetRnd{\mathtt{GetRnd}}
\def\Select{\mathtt{Select}}
\def\RecordTally{\mathtt{RecordTally}}
\def\ReadTally{\mathtt{ReadTally}}
\def\Receipt{\mathtt{Receipt}}
\def\Result{\mathtt{Result}}
\def\ElectionFail{\mathtt{ElectionFail}}

% End of preamble

\title{The E2EVIV Report}
\author{Many People}

\begin{document}

% The title page has no visible page number, and the report resets page
% numbering on the next page so that the visible page numbers start from 1.
% One unfortunate side effect of this is that hyperref sees two different
% pages both named "1" and gets confused. See also
% http://tex.stackexchange.com/q/18924/16779
\hypersetup{pageanchor=false}
\maketitle
\hypersetup{pageanchor=true}

\tableofcontents

\clearpage
\ifdraft{\addcontentsline{toc}{chapter}{\ \ \ \ \hl{Note: Names following
    chapter titles are the currently-assigned writers; percentages
    following writer names are very rough estimates of the
    approximate percentage of completion. Some material factored 
    into the percentages may not yet appear in the generated report 
    because it needs to be brought in from external sources.} \\ \ \\
  List of To Do Items}
% temporarily here, so we have a centralized list of all the todo
% items in the document
\listoftodos[List of To Do Items]
}{}
\chapter{Executive Summary\ifdraft{ (Joe K./Susan) (0\%)}{}}
\label{chapter:executive_summary}

\begin{itemize}
\item framing of project \& methodology
\item who is involved
\item who funded it
\item goals and deliverables
\item feasibility
\item recommendation
\item ``why now'' coupled to history
\item why do existing systems suck
\item ``how'' what is E2E-VIV (illustration)
\item five key mandatory properties: end-to-secure, verifiability,
  100\% accessible, high-assurance, transparent
\item crypto framing
\item architectural framing
\item RSE framing
\item conclusion and future phases
\end{itemize}

% [Susan here - this is my outline - I am visualizing the report in a
% printed or online format]

% Thinking out loud about how to make this piece simple, powerful and
% satisfying...  

% Define the what and how: let’s make it completely unambiguous. With
% an ILLUSTRATION.

% Define the current challenge of IV - that is the basis and reason
% for doing this project - (despite all the hacking, Snowden, Target,
% IRS, Sony) Americans love a challenge - we want to find a way. This
% search will never stop. It is the responsibility of the scientific
% and election communities to team up, research and pursue a real
% answer.

% What is the challenge in the context of Elections in the USA (PCEA,
% modernization, OVR, coupled with lack of certification, diminished
% funding, pressure on LEOs to act, wildly fragmented implementation
% across 10K+ jurisdictions)

% What’s the problem - I mean hey - we can bank online! Come on,
% people!

% So, what is the BFD of this report? I mean it! What has been
% “discovered”. Give it to me straight!

% How will it affect our future of voting? Will we vote online and
% when can we expect to? Come on, people - that is why I am reading
% your Exec Summary!

% What’s in the report, how is it organized and what part of it am I
% interested to read?

% How is life on earth different now that we have this new report -
% this new frame of reference. What might happen now that this
% knowledge is released into the universe?

% E2E-VIV Diagram - perhaps show it as a line, what is the beginning “End” and the final “End”.
%    Voter Intent	 Distribute Ballots	       Record Votes as Cast	  Verify Vote Counted as Cast     Announce Results
% E------||--------||---------||-----------||----------||---------||-----------------||-----------||---------||----E
% 	Set Up Election	    Cast Votes as Intended	Count Votes as Recorded	 Allow Public to Verify

% Starting over here - trying out my Super-Susan-Simple-Outline - a no
% obligations trial!
% 1. End-to-End Verifiable Internet Voting (E2E-VIV): let’s break this down...
% what is End-to-End (E2E)?
% what is Verifiability?
% what is E2E-V when applied to an election?
% what about putting the E2E-V election process onto the Internet to
% create E2E-VIV?
% what challenges are encountered - why, in simple terms is this hard?

% Q. What is End-to-End?
% A. The term End-to-End (E2E) in the context of elections refers to the
% entire process of voting starting with the voter’s intent through to
% the announcement of the election results.
% Q. What is Verifiability? 
% Verifiability (V) is the ability to prove something, to show
% evidence that allows you to confirm or substantiate a fact.
% Q. What is End-to-End Verifiability when applied to an election? 
% End-to-End Verifiability (E2E-V) in an election means you can check
% that your vote was cast as you intended it, recorded as such and
% counted correctly. It also means that anyone can check the outcome
% of the election. In essence, E2E-V is considered a “property” of an
% election.
% Q. Is it beneficial to add E2E-V properties into Internet voting to
% create E2E-VIV?
% Adding E2E-V properties to an election conducted over the Internet
% would bring in security assurances that are lacking in every
% currently available Internet voting system. Elections are, however,
% multifaceted and their implementation involves a diverse set of
% challenges, security being top of the priority list, but closely
% tied with several others.
% Q. What, specifically, are the challenges in developing an E2E-VIV
% System? 
% At the same time that certain security challenges are positively
% addressed with E2E-VIV, new challenges in other aspects of
% elections, specifically, usability, privacy, authentication, 100%
% accessibility, transparency and auditability [add total list] are
% created. 
% Q. Is it possible to overcome these challenges and define an E2E-VIV
% System that we could build, make available to the public and openly
% test? 
% This question is at the heart of the End-to-End Verifiable Internet
% Voting: Specification and Feasibility Assessment Study (E2E-VIV
% Project) that was defined and undertaken by U.S. Vote Foundation
% between September 2013 and July 2015. 
% Q. End-to-End Verifiable Internet Voting: Specification and
% Feasibility Assessment Study (E2E-VIV Project)
% With support from The Democracy Fund, U.S. Vote Foundation (US Vote) was enabled to manage the project and recruit a diverse team of elections specialists including leading election officials, scientists and academics who had studied this topic for decades, usability, testing and auditing experts. In mid-2014, US Vote engaged Galois, Inc. to lead the technical project development. The Galois engineering team brought deep knowledge, skill and experience in the arena of high-assurance cryptographic science - the element at the core of the E2E challenge. In addition, Galois contributed to the project resources with development of key demonstration technology pieces for E2E-VIV.
% The E2E-VIV Project team took on the challenge to examine these difficult election systems  and engineering challenges that we face with Internet voting and put forward a specification and feasibility assessment for how to move forward to begin developing and openly testing such systems. 

% The natural movement to modernize elections

% Elections have been conducted for millennia and are the cornerstone
% of democracy, but the technologies used to cast and tally votes have
% varied and evolved tremendously over that time.  Much of our
% discourse now takes place online, and many have called for elections
% to follow this trend and asked why they haven’t done so already.
 
% The E2E-VIV Research Report

% In this E2E-VIV research report, we examine the future of voting
% and how it might be executed securely online. Specifically, we
% explore whether an end-to-end verifiable Internet voting (E2E-VIV)
% election system can be built that would offer a viable and
% responsible alternative to current systems aimed at support for the
% overseas, military and disabled voter communities. The chief
% deliverable is a System Specification and Documentation for an
% E2E-VIV Election System.
 
% I think this might be covered above. This project combines the
% abilities, knowledge, experience, and expertise of a diverse group
% of technologists, computer scientists, usability and auditing
% specialists, and election officials committed to election
% integrity. The technical team, comprised of academic and scientific
% specialists, has long term, proven experience in E2E-V technology,
% cryptography, usability, and testing.
 
% % Findings and Recommendations Outcomes (frankly I would split these
% into two separate sections. Let’s keep each section short.

% The E2E-VIV Project team has determined that it is technically
% feasible to design, develop and support an open source E2E-VIV
% system. This report puts forth the following:
% ·      A complete set of requirements for an E2E-VIV system.
% Systems which fulfill these requirements and provide evidence of
% such in an open transparent manner, which allows verification and
% certification, may be considered as ian E2E- VIV system
% ·      A collection of alternative architectures for E2E-VIV systems.
% The precise solution architecture will depend upon the threats that
% governments care to address
% ·      A set of rigorous engineering methodologies, technologies,
% and tools. These technologies are fundamental to building an E2E-VIV
% system that is correct and secure.
% ·      The security foundations of an E2E-VIV system
% Lorem ipsum dolor site et.
% ·      The need for  a comprehensive usability study.
% The usability challenges that emerge with the introduction of new
% voting processes that enable and guarantee the integrity of E2E-VIV
% systems, remain formidable, yet we believe,
% surmountable. Significantly higher-level UX design effort is needed
% on actual prototype systems coupled with a comprehensive usability
% study. An iterative process between the designers and usability
% engineers will be required to adequately address the usability
% challenges.  
% ·      Phase II Research and Development.
% The E2E-VIV Project team recommends continuation of this effort
% through a Phase II R&D effort, which would include the development
% of an E2E-VIV prototype, testing, user interface, usability and
% accessibility refinements and testing, --( add some tech soup here -
% the meaty stuff like crypto and rigorous engineering ….)

% What is End-to-End Verifiable Internet Voting (E2E-VIV)

% An end-to-end verifiable Internet voting system guarantees the
% security of the ballot casting process, provides evidence of the
% correctness of both the system and the election, and outputs results
% that are verifiable by independent third parties.

% End-to-End

% Verifiable
% Internet Voting

% A system designed to be secure from threats (for example, hackers,
% insider attacks and administrative errors) from the beginning (the
% voter’s intent) to the end of the voting process (the election
% outcome).

% A system that allows independent third parties to check that [the
% election is being properly conducted?] and that the outcome of the
% election is correct. Current elections do this by …

% A system that transmits voter choice over the Internet, by any
% means, including any aspect of the voting process that is online,
% for example , online voting, voting by phone over VoIP, equipment
% that uploads precinct results to a central tally system, etc.

% How it Works: tell me the tough stuff in simple language

% Characteristics / Requirements  of E2E VIV Systems
% Do we want to say - ?
% End-to-End Verifiability is considered a “property” of an election
% system which comprises and an important set of capabilities.  An
% E2E-VIV system:
% ·      Permits voters to check that their votes are recorded
% properly, without violating their privacy
% ·      Permits voters to check that their votes are included in the
% final tally 

% ·      Permits anyone to check that all of the cast ballots have
% been tallied correctly


% To be able to do this, it must have a set of
% characteristics/functionalities? (I am holding “property” for the
% overarching statement at the top - you may not agree…). A complete
% set of required properties can be found in chapter 5. the most
% important properties it must have are:

% end-to-end secure, verifiable,
% 100% accessible, high-assurance, transparent
 
% Why now?

% Our election systems are in need of modernization and it is a
% natural tendency for election administrators to consider all
% possible technical solutions including online voting. It is the
% responsibility of ….fill in - to offer guidance and  ...

% Many IV implementations exist overseas, however we need our own
% frame of reference in the US and for the US - one which takes into
% account the seriousness of US national elections and structure of
% our elections administration framework and structure

% More and more of what we do is going online

% The internet facilitates a lot of things

% Election officials want the supposed benefits of an online system
% for efficacy

% There are no existing solutions exist that are deemed suitable by
% experts in terms of security and robustness

% In lieu of a good solution, people are using bad ones. Email, closed
% source vendors, etc.

% Recent developments in the field: STAR-Vote? Expert collaboration?
% time is now

% • feasibility
% Designing an E2E VIV System
%  • crypto framing
% 
• architectural framing

% • RSE framing
% • other?
% 

% Recommendations

% Callout: Why open source?

% Lorem ipsum dolor sit amet, consectetur adipiscing elit. Nullam
% efficitur aliquet turpis, vitae eleifend lacus tempor at. Aliquam
% tempor, ex in fermentum tempus, risus ipsum finibus ipsum, ac
% gravida odio augue ut risus. In sollicitudin luctus leo ut
% interdum. Aenean tincidunt gravida pharetra. Donec eu quam eget
% mauris pretium mattis eget et est.
 
% Callout: Why is Internet voting so challenging?

% If we can bank online, why can’t we vote online?
% In banking, the bank...

% Knows who you are
% Knows every transaction you make
% Has a ledger on all transactions in the bank
% The bank and the federal government insure loss

% In elections...

% Voter identity and vote are private
% No ledger of all transactions
% No insurance if something goes wrong
 
%  Lorem ipsum dolor sit amet, consectetur adipiscing elit. Nullam
%  efficitur aliquet turpis, vitae eleifend lacus tempor at. Aliquam
%  tempor, ex in fermentum tempus, risus ipsum finibus ipsum, ac
%  gravida odio augue ut risus.


 

 % Joe K./Susan
\chapter{Introduction\ifdraft{ (Joe K./Susan) (30\%)}{}}
\label{chapter:introduction}

\section{The E2E VIV Project}
\section{Goals}
\section{People}
\section{Methodology}
\section{Outcome}
\section{Next Steps}
 % Joe K./Susan
\chapter{Remote Voting\ifdraft{ (Philip) (45\%)}{}}
\label{chapter:remote_voting}

\section{Rationale}

For each subsection::
\begin{itemize}
\item Who is in this group?
\item How many people are in this group?
\end{itemize}

\subsection{Geographic Dispersion}
%\ldots This is referring to residents of sparsely populated \ldots

\subsection{Accessibility}
Studies issued by the International Center for Disability Information and the National Institute on Disability and Rehabilitation Research indicate that 20\% of Americans live with disabilities. 

The Help America Vote Act (HAVA) of 2002 requires that all polling places in elections for federal office, anywhere in the United States have at least one voting system  \ldots

%\ldots 20\% of U.S. adults with disabilities say they have been unable to vote in presidential or congressional elections due to barriers at or getting to the polls.

\subsection{UOCAVA}
In 1986, Congress enacted the Uniformed and Overseas Citizens Absentee Voting Act, stating citizens that are part of the uniformed services, merchant marines, and their families or citizens residing overseas are allowed to register and vote absentee for federal office.

\begin{itemize}
\item approximate count
\item Table 2.1 Convenience Voting and Technology
\end{itemize}

\begin{center}
\begin{tabular}{l p{.3\textwidth} p{.3\textwidth}} % this can be made prettier
{\bf State} & {\bf Overseas Voting Eligible Population} (McDonald 2009) & {\bf Overseas military and federal civilian employees} (US Census Bureau 2010)\\\hline
Texas\\
California\\
Florida\\
New York\\
Pennsylvania\\
Illinois\\
Ohio\\
Michigan\\
Georgia\\
Washington\\
North Carolina\\
Tennessee\\
Virginia\\\hline
Total
\end{tabular}
\end{center}
\subsection{Early Voting}
\subsection{Expectations}

\begin{itemize}
\item American Political Science Association study (1952)
  \begin{itemize}
  \item Ability to vote without registering in person
  \item Ability to vote without unreasonable requirements and costs (federal postcard)
  \item Insure enough time is permitted for ballot transit
  \end{itemize}
\item HAVA, ADA
\begin{itemize}
  \item Multilingual
  \item Accessible for individuals with disabilities
\end{itemize}
\item Privacy of Vote
\item Integrity of Vote (VVPAT)
\end{itemize}

\section{History}
Political attitudes and legislation on absentee voting has been a slow moving 
effort, due to dominant partisan attitudes changing, and regulations being 
enforced at the state level.

Before the civil war US citizens primarily voted in their places of residence, and many states legally barred the casting of votes from outside state borders. There was little effort from any state to accommodate absentee voting. However, in 1864 with the American Civil War displacing soldiers from their residences, Lincoln's re-election was at risk. With much lobbying on behalf of the republican party (and opposition from the democratic party), nineteen of the union's states adopted absentee voting procedures for military voters on federal elections in time for the election. Unfortunately since the motivation to passing these laws was securing Lincoln's re-election, rather than persistent enfranchisement, many absentee military voter laws were treated as temporary and repealed after the war.

In 1918 America's War Department decided that it was not ready to support the military vote. World War I had displaced such a large number of voting eligible persons and military units were rarely composed of same state citizens. Not even states in support of military vote were allowed the soldier vote, even on matters at the state level.

As in the Civil War, World War II inspired another push for the military vote in hopes of supporting the re-election of the presidential incumbent. This introduced the Soldier Voting Act (1942) which, although passed too late for the presidential election, mandated military personnel rights to absentee vote on federal elections during times of war without subjugation to voting tax or postage costs. From this point forth all overseas voting would be regulated at the federal level and implemented at the state level. However, by 1944, the state mandate to support military absentee voting was amended to a recommendation.

Progress with absentee civilian vote was a further behind. In 1896, states began introducing civilian absentee voting legislation. By 1924 only three states in the union had no absentee voting legislation, but all states had different laws and restrictions. Major progress on this front wasn't made until federal voting laws were passed that combined for the handling of civilians and military votes. The Voting Assistance Act of 1955 was the first to federally combine voting policy recommendations for overseas civilian government employees with military. In 1986 Voting Assistance Act was amended to include individuals temporarily living outside the United States. With lobbying from sympathetic groups and the quickly growing population of overseas civilians, in 1974 Overseas Citizens Voting Rights Act passed extending the recognized vote to citizens regardless of their intentions to return to the United States.

By 1986, combatant attitudes towards overseas votes had finally settled, and the Uniformed and Overseas Citizens Absentee Voting Act (UOCAVA) was passed replacing/combining Overseas Citizens Voting Rights Act and the Federal Voting Assistance Act, and finally made supporting the overseas absentee ballot a requirement.

\begin{itemize}
\item integrate history of disabled voters rights
\item[$\star$] TODO as necessary: ADA, HAVA, MOVE
\end{itemize}

\subsection{Integration with Local Elections}

Every state has their own requirements, deadlines, and transmission restrictions which the FVAP documents in a 'Voting Assistance Guide' distributed to potential UOCAVA voters. 

\section{Shortcomings of Current Practice}

\begin{itemize}
\item In the 2008 Post-Election UOCAVA Survey Report and Analysis 52\% of attempted votes were not counted because the ballots were late or never arrived.
\item 20\% of Americans with disabilities have said that they were unable to vote in presidential or congressional election due to barriers at or getting to the polls. (as of 2007)
\item ``ten states explicitly require a privacy waver if a voter uses fax or e-mail to return a voted ballot''
\item FVAP's {\em Voting Assistance Guide} is complicated and results in many failed registration attempts.
\item Entire process for UOCAVA voters generally can take up between 2 weeks and 2.5 months
\end{itemize}

\subsection{Use of Communication/Internet}

The major motivations for use of internet and communication technologies for UOCAVA has been to address ballot transit time, and simplify voter registration. 
\begin{itemize}
\item OVF's streamlined website for FPCA in states that allow online registration
\end{itemize}

\subsection{Accessibility and Usability}
\subsection{Auditing}
\subsubsection{Current Practice}
\subsubsection{Digital vs. Physical}
\begin{itemize}
\item VVPAT provides voters with paper statements
\end{itemize}
\subsubsection{Risk-Limiting Audits}
Risk limiting audits use a public random auditing process to make an argument about the statistical confidence of a particular election result.
 % Philip
\chapter{E2E VIV Explained\ifdraft{ (Philip/Daniel/Adam) (45\%)}{}}
\label{chapter:e2e_viv_explained}

As discussed in \autoref{chapter:remote_voting}, there are several
difficulties with current voting processes: voters with disabilities cannot
vote unassisted, communication channels with remote voters are slow and
unreliable, vote tallying is labor-intensive and error-prone, and
election audits are costly. Additionally, there is little visibility into
the election process, meaning that individual voters and, in some cases,
even auditors, must trust the reports of election officials and voting
hardware vendors on election outcomes and processes. Election officials are
naturally seeking technology that can mitigate these problems, such as
automated, computer-based vote tallying to reduce tallying and auditing
costs, computerized ballot completion with accessibility technology to
assist disabled voters, Internet-based vote submission to increase the speed
and reliability of communicating with remote voters, and cryptographic
techniques for providing visibility without violating voter privacy. Below,
we give an overview of the goals of these technologies, discuss their
success at these goals, and highlight some common pitfalls.

\section{IV, VIV, E2E}

The core idea of Internet voting is deceptively simple: take a system that
allows remote voting via postal mail---which is already done in many
places---and replace some or all uses of postal mail in that process with
Internet-based delivery mechanisms. The draw of this idea is clear: messages
can be sent between election officials and voters in seconds, no matter how
distant, instead of weeks, and messages in typical Internet communication
methods are rarely lost, while up to half of overseas postal messages get
lost.

The general process is something like this:

\newcommand\stepref[1]{\hyperref[#1]{step~\ref*{#1}}}
\begin{enumerate}
  \item\label{enum:setup} Well before the election, officials compile a list
    of registered voters. They use this to populate a central database with
    contact information, details about what each voter's ballot should
    contain, and so forth. General instructions and other information about
    the election may be broadcast at or near this time---for example, via a
    central website.
  \item\label{enum:share secrets} Election officials contact each voter
    individually to set up some shared, but secret, information specific to
    that voter. For example, this might include an empty ballot with secret
    numbers associated with each candidate; or a cryptographic key-pair that
    the voter can use during communications with the officials. This is
    sometimes done by traditional means (like postal mail) and sometimes not
    done at all.
  \item\label{enum:send blank ballot} Once a means of communication is
    established in \stepref{enum:share secrets}, the election officials send
    the voter a blank ballot.
  \item\label{enum:complete ballot} The voter fills out the blank ballot on
    their computer, perhaps using custom software.
  \item\label{enum:send completed ballot} Again using the communication
    method established in \stepref{enum:share secrets}, the voter sends the
    completed ballot back to the election officials. Many systems have an
    additional step in which the election officials confirm receipt of the
    ballot.
  \item\label{enum:tally} After the election is over, all ballots sent in
    are tallied, and the outcome of the election is announced. In some
    systems, other supplemental information is often announced, such as
    whose votes were counted.
\end{enumerate}

Unfortunately, simply switching to Internet-based communication does not
solve all the problems discussed above, and introduces new problems of its
own.

\tododmwit{expand each of these}
\begin{description}
  \item[Secure transport] For the integrity of the vote, it is important
    that the empty ballot sent in \stepref{enum:send blank ballot} and the
    completed ballot sent in \stepref{enum:send completed ballot} arrive at
    their destination unmodified. The Internet includes a lot of hardware
    that is controlled by neither the election officials nor the voter, so
    ensuring this property can be quite a challenge. Some systems use a
    different communication channel in \stepref{enum:share secrets} to give
    greater confidence in this property, using information communicated by
    mail to cross-check information communicated over the Internet.
  \item[Private transport] Normally, votes are anonymous---there is no
    connection between a cast vote and the person who cast it. However,
    extant Internet communication protocols cannot support this privacy
    guarantee (and vote-by-mail systems typically sacrifice this property as
    well).
  \item[Attack surfaces] The voter is presumably using his own computer.
    It is possible that his computer has been taken over by somebody else.
    This kind of problem is completely fresh compared to paper voting.
  \item[Verifiability] Voters should be able to independently verify that
    their votes were cast and counted correctly, without needing to place
    trust in the communication medium or the election officials. Auditors
    should be able to check that the election was executed correctly without
    trusting the voting hardware and software vendors. Paper ballots support
    auditing well, but make individual voter confidence difficult.
\end{description}

\section{E2E Election Rituals}
\subsection{Pre-Election Phase}
\subsection{Voting}
\subsection{Post-Election Phase}
\section{Shortcomings and Expectations of E2EVIV}
\subsection{Access to Communication/Internet}
\subsection{Accessibility}
\subsection{Usability}
\section{E2E VIV in Practice}
\tododmwit{This section should be filled in with material from
  history.tex and comparative\_analysis.tex, and then history.tex and
  comparative\_analysis.tex should be ``retired'' and their resources
  moved into e2e\_viv\_explained\_resources.}
\section{Limitations of Existing Systems}
 % Philip/Daniel
\chapter{Required Properties of E2E Systems\ifdraft{ (Dan) (35\%)}{}}
\label{chapter:required_properties}

We now describe the required properties that E2E VIV systems must have
in order to be considered for use in real elections. These
requirements can be broadly divided into two groups: \emph{technical
  requirements} and \emph{non-functional requirements}. Technical
requirements are those that can be directly addressed by the design
and implementation of the system, such as authentication requirements
for voters and election officials. Non-functional requirements are
those that are imposed on the system by external entities or where the
system depends on external behaviors outside its control, such as
specific election certification guidelines and operational
procedures. Each of these groups is itself divided into several
categories, and \autoref{fig:e2eviv_requirements_hierarchy} gives a
high-level overview of these.

\begin{figure}
\begin{center}
\includegraphics[width=6in]{required_properties_resources/hierarchy}
\end{center}
\caption{The hierarchy of requirements for E2E VIV systems.}
\label{fig:e2eviv_requirements_hierarchy}
\end{figure}

The following is a high-level description of the categories and the
requirements within each; \autoref{appendix:bon_requirements} is a
complete listing of the requirements expressed in the Business Object
Notation.\tododmz{is \autoref{appendix:bon_requirements} actually
  necessary, or will everything basically be described here?}

\section{Technical Requirements}
There are ten categories of technical requirements for E2E VIV
systems: functional, accessibility, usability, security,
authentication, auditing, system operational, reliability,
certification, and interoperability.

\subsection{Functional} 
\todogeneric{Should this really be ``functional''? Would ``core'' or
  something similar work better? It seems that some other
  requirements, such as those dealing with auditing and counting, are
  also functional...}  The functional requirements of an E2E VIV
system deal primarily with the casting and recording of ballots and
associated voter records. One important requirement is that there must
be a correspondence between the recorded ballots and the voters that
are listed as having voted; a ballot cannot be recorded without a
voter casting it, and a voter cannot be listed as having voted without
casting a ballot. Similarly, if a voter is informed by the system that
her ballot has been successfully cast, the system must correctly
retain the record of her having voted and her cast ballot information
even in the event of server failures.

Another functional requirement is the property of \emph{receipt
  freedom}: it must be impossible for a voter to prove to anybody any
information regarding how they voted their ballot, beyond what can be
mathematically deduced from the final distribution of votes. For
example, if a referendum passes with 100\% of the vote, there is no
way to hide the fact that every voter approved of the referendum;
however, if the result is mixed, it must be impossible for any
individual voter to prove how they voted.

In some elections voters are allowed to cast multiple ballots with
only the last cast ballot counting toward the final election tally,
while in others voters are prohibited from casting multiple
ballots. The system must accommodate both of these election formats,
ensuring that only the last cast ballot is counted for each voter when
multiple ballots are allowed and ensuring that each voter casts at
most one ballot otherwise.

Maintaining voter anonymity is critical, so it must be impossible
after the election to reconstruct a link between a cast ballot and any
identifying information about the voter who cast it. However, in
systems that support the casting of multiple ballots, it is important
to maintain links between voters and their ballots \emph{during} the
election to ensure that later ballots replace the correct earlier
ballots. To balance these concerns, it is a functional requirement of
the system that once it is determined that a ballot will be counted
toward the final tally, any link between that ballot and the voter who
cast it must be irrevocably broken.

Finally, because the voter should be able to focus on the voting
process without undue distractions or external influences, the voting
system must not display or permit the display of any advertising or
commercial logos during a voting session; the exception to this rule
is that an election jurisdiction may display its own logo to the voter
during the voting process. Along the same lines, the voting system
must not display any links to other Internet sites outside of the
voting system, except to provide help with the actual mechanics of
voting.

\subsection{Usability}

The usability of an E2E VIV system is critical to its successful
adoption and use. Since the user experience is so important, many of
the requirements of the system have some relation to usability even
though they may be categorized under other headings. There are,
however, two requirements that are exclusively related to the
usability of the system with respect to vote casting and one general
usability requirement that applies to the system as a whole.

The first vote casting requirement is that, if a voter receives a
final vote confirmation (e.g., ``Thank you for voting!'' or a similar
notice) from the system, she can be certain that her ballot was
recorded correctly. This is the usability counterpart to the
functional requirement that ballot records and voter records must be
maintained correctly even in the event of server failures.

The second vote casting requirement is that, if a voter is uncertain
whether or not her ballot was recorded (e.g., she clicked a ``submit''
button but never got a response from the system), she must be free
to attempt to vote again.

Finally, usability testing must be performed on any E2E VIV system
before it is deployed. The reports of the usability testing must be
made public, and the system must achieve satisfactory test results
before being used in a binding election.


\subsection{Accessibility}

Accessibility---the property of being usable by and useful to the
disabled---is one of the main goals of an E2E VIV system. It is
closely related to usability, but there are several requirements
associated specifically with accessibility that go beyond typical
usability requirements. 

Users must be involved in the design of the system to identify
accessibility constraints at each stage of the development
process. Consideration must be given to the system's compatibility
with existing technologies designed to help disabled individuals; for
example, the system should be developed in a way that allows assistive
input devices such as switches and eye trackers to be used in addition
to keyboards, mice and touchscreens. Similarly, the system's
presentation of voting options should be optimized to voters' needs by
providing alternative display fonts, audio representations, braille
representations, and other representations as appropriate.

All possible measures must be taken to ensure that the system can be
used by all voters and, if that is not possible in all circumstances,
to provide access to alternative methods of voting for those voters
who cannot use the system.

Finally, accessibility testing must be performed in addition to the
previously-mentioned mandatory usability testing. The reports of the
accessibility testing must be made public, and the system must achieve
satisfactory test results before being used in a binding election.

\subsection{Security}

Security is the largest category of technical requirements, comprised
both of requirements on the system infrastructure (data storage,
communications, etc.) and of requirements on the voting process
enabled by the infrastructure (voter authorization, voter privacy,
accuracy of the final tally, etc.).

With respect to the system infrastructure, it is critical that data
integrity be ensured throughout the system. This means that measures
must be taken to ensure that no data can be permanently lost in the
event of a breakdown or fault affecting the system; that the system
must maintain the integrity of the voters' register, lists of
candidates, ballot information, cast ballots, and other critical
information, in addition to authenticating the original source(s) of
that information and tracking provenance where appropriate; that all
data communications within the system must have associated integrity
checks; that system equipment under the control of the electoral
authority must be protected against influences that could modify the
election results; and that the integrity of the election results must
not depend in any way upon the security of system equipment not under
control of the electoral authority.

The tracking of precise timing information is a critical aspect of
system security. The system must maintain reliable synchronized time
sources, with sufficient accuracy to maintain timing data for audit
trails, election observation data, and time limits for various aspects
of the election process. It must be possible to determine, using the
timing information stored by the system, whether nominations (and, if
required, acceptance thereof by the candidate or electoral authority),
voter registration, and vote casting have occurred within the
prescribed time limits for those actions.

The system must ensure that each voter and candidate has a unique
identification, so that there is no possibility of mistaking one voter
or candidate for another. In addition, the system shall restrict
access to its services to those users that are authorized to access them.

The electoral authority shall have overall responsibility for
compliance with these security requirements, and such compliance shall
be assessed by independent bodies as appropriate.

\subsection{Authentication}
\lipsum[6]
\subsection{Auditing}
\lipsum[7]
\subsection{System Operational}
\lipsum[8]
\subsection{Reliability}
\lipsum[9]
\subsection{Certification}
\lipsum[10]
\subsection{Interoperability}
\lipsum[11]

\section{Non-functional Requirements}
There are six categories of non-functional requirements for E2E VIV
systems: operational, procedural, legal, assurance, maintenance, and
evolvability.

\subsection{Operational}
\lipsum[13]

\subsection{Procedural}
\lipsum[14]

\subsection{Legal}
\lipsum[15]

\subsection{Assurance}
\lipsum[16]

\subsection{Maintenance}
\lipsum[17]

\subsection{Evolvability}
\lipsum[18]
 % Dan
\chapter{Crypto Specification (Joe R.)}

 % Aggelos
\chapter{Architecture\ifdraft{ (Joe K./Dan) (30\%)}{}}
\label{chapter:architecture}

The \emph{architecture} of a computing system, akin to the
architecture of a purely physical artifact like a bridge or a
building, is its high-level structure. Just as when designing a
bridge, many choices must be made when designing a computing system;
these choices are driven by the system's requirements, both technical
and non-functional, as well as by external factors such as the
availability or affordability of computing hardware or network
bandwidth. In this chapter we describe the architectural issues
associated with E2EVIV systems, present a model encompassing the
various possible architectural choices for such systems, and briefly
explore some architectural variants.

It is important to note that we are \emph{not} making a concrete
recommendation for a specific E2EVIV system architecture. The
cryptographic foundations of E2EVIV protocols have been developed to a
point where we have fairly high confidence in their ability to
eventually provide the required security and auditability
properties. However, there are many open engineering issues associated
with actually building, running, and maintaining an E2EVIV system that
fulfills its requirements in the face of both routine/expected
failures and a wide range of security threats. Because of these open
engineering issues, architectural experimentation---preferably,
empirical testing of various possible architectures to determine the
most appropriate one(s) to deploy in real-world election
scenarios---is vital to actually implementing a successful E2EVIV
system.

\section{Non-Functional Requirements Forcing Architectural Factors}

Several non-functional requirements of E2EVIV systems force the
inclusion or consideration of specific architectural factors. We
consider each of these in turn.

\subsection{Abstraction}

The software in an E2EVIV system must be high assurance, and must also
undergo a certification process before it can be used in actual
elections. Thus, the software must be designed and implemented with a
level of abstraction that enables the generation of convincing
evidence for its correctness. This requires development techniques
that emphasize formal specification and verification, which divide the
software system into relatively small components with well-defined,
well-constrained interfaces. Each component can then be verified
individually with respect to its specification; moreover, component
behaviors with respect to external communication can be formally
characterized in ways that allow for verification of composed
subsystems.

\tododmz{NIST component-based certification}
\todokiniry{Are we talking here about the VVSG (co-developed by NIST),
  or something else?}

\subsection{Deployment}

There is a wide spectrum of possible deployment scenarios for an
E2EVIV system, each of which leads to certain decisions about its
architecture. At one extreme, the servers for an E2EVIV system could
be hosted on a bespoke server cluster, built from the ground up
specifically for the system and housed in a facility under the
physical control of electoral authorities or their authorized
representatives. At another extreme, the servers could be hosted on a
commodity cloud computing infrastructure such as Amazon EC2 or
Microsoft Azure where electoral authorities have no physical control
over the servers. A third extreme would see the system implemented in
a purely peer-to-peer fashion, with the system's functionality
distributed among all participating computers and no
specifically-designated servers. The choice within this spectrum has
significant impact on both system availability and system security
requirements.

\tododmz{agile and devops mentality vs. certification?}

\subsection{Threats}

Mitigating the potential threats to E2EVIV systems also leads to
various architectural choices. The following are some of the threat
vectors and mitigation strategies that need to be considered.

\paragraph {Single Point of Failure.} Any single point of failure,
such as a single server that contains essential data without which the
system can no longer function, is a tempting target for
attack. Therefore, the architecture should attempt to minimize or
eliminate such failure points.

\paragraph{Easy DDoS Targets.} The use of fixed IP addresses (or
address ranges) for E2EVIV system components can open the system to
denial of service attacks, limit deployment flexibility, and make it
more difficult to recover from failures. The architecture should be
chosen such that IP addresses need not be hard-coded; preferably, they
should be changeable be on-the-fly (i.e., during a live election
scenario) if necessary to protect or restore system integrity.

\paragraph{DDoS Mitigation.} Content distribution and network
protection services such as CloudFlare \cite{CloudFlare}, which route
traffic for their clients through their own high-bandwidth global
content distribution networks, can detect and protect against DDoS
attacks. They can also provide other benefits, such as higher
availability and improved response time. Such services could easily be
used to deliver static data to clients in an E2EVIV system; however,
although they support technologies such as WebSockets for dynamic
data, they cannot necessarily protect systems that have complex
back-end architectures and client-server interactions. It is unclear
whether such services can successfully be used for all interactions in
E2EVIV systems without compromising privacy or other requirements;
even if it is possible, doing so would certainly require specific
considerations in the design of the system architecture.

\paragraph{Non-Standard Foundations.} One way of countering threats is
to build the system using non-standard foundational techologies. For
example, instead of being built atop general-purpose operating systems
like Linux or NetBSD, E2EVIV system components could be implemented as
\emph{unikernels}~\cite{Madhavapeddy13}, effectively single-purpose
application/operating system combinations that run directly atop
hypervisors such as Xen~\cite{Xen}. Unikernels, written using
technologies such as Mirage~\cite{OpenMirage} and HalVM~\cite{HalVM},
have considerably less code than general purpose operating systems;
each one performs a specific task, and they communicate amongst
themselves in the same manner as machines in a distributed
system. This implementation style can improve security in general, by
reducing the effort required to demonstrate the security of each
component and by minimizing potential attack surfaces once the
components are deployed.

\paragraph{Multi-Party Computation (MPC).} \tododmz{How might an E2EVIV
  system look that built from the ground-up using MPC?
  MPC+peer-to-peer at one extreme; }

\subsection{Distributing Trust}

The distribution of trust in an E2EVIV system is critical: if the
system is not trustworthy, the election results generated by it are
inherently suspect. The following are descriptions of the various
aspects of the system that must be trusted, and some possible ways to
establish that trust.

\paragraph{Trusting the Electoral Authority.} The public must be able
to trust that the electoral authority, including elected or
appointment government officials and potentially others employed or
contracted by them to carry out election tasks, will not compromise
the integrity of the election process or its results. In the event of
a completely corrupt electoral authority, the entirety of the election
process, all evidence presented to the public, and all announced
results can be fabricated; thus, we need both mechanisms that minimize
the possibility of a completely corrupt electoral authority and
mechanisms that allow for the detection of corruption if the electoral
authority includes at least one honest official.

One mechanism to help accomplish this is threshold cryptography, where
the cryptographic keys that allow for results to be generated by the
election system are divided into some number of shares, a smaller
threshold number of which are required to actually generate
results. For example, the keys might be divided into 10 shares, each
of which is entrusted to a different official and at least 7 of which
are required to generate the tally. This allows the election to
proceed in situations where some of the election officials are unable
(due to accident, illness, etc.) or unwilling (due to corruption) to
produce the keys, as well as providing a check against corruption by
requiring at least 7 of the 10 officials to collaborate in generating
the tally. This check against corruption can be strengthened even
further with public ceremonies surrounding key generation, share
distribution, and tally generation, so that the specific electoral
officials involved are publicly known and can be held accountable for
their actions.

Another mechanism to increase trust in the electoral authority is
access restriction; if the election system allows physical access only
via computing systems at specific, publicly-known and well-secured
locations, and only during specified time frames, it is far more
difficult for a corrupt official (or group of officials) to manipulate
the election results without being detected.

Finally, the use of an append-only public bulletin board to record
encrypted ballot information in a tamper-evident fashion, combined
with publication of sufficient information about the election protocol
to allow members of the public to validate that their own ballots were
cast as intended and counted as cast, also acts as a significant check
against corruption by the electoral authority.

\paragraph{Trusting the Software.} The software in an E2EVIV system
must be trusted to fulfill the system's requirements. Making all
development artifacts freely available as open source is an important
step toward such trust, as it allows for independent verification that
the source code has been designed and implemented
appropriately. However, the mere fact that it is open source is not
enough to make a piece of software trustworthy; there must be evidence
that the running software actually corresponds to the open source
code, has passed all required testing, and has not been corrupted
after installation.

Several mechanisms are available for increasing the trustworthiness of
deployed software. One is code signing; the system vendor can
digitally sign the final binary distribution of the software and make
the signature public, allowing any observer with sufficient access (in
practice, this would likely be the electoral authority) to validate
the signature against the actual deployed binaries. Self-certification
is a variant on code signing, where the software computes a signature
on on itself at startup time and compares it against a known value;
this can detect accidental corruption (data storage errors, spurious
bit flips, etc.) but generally not malicious corruption, because an
attacker that successfully corrupts any part of the software could
also corrupt its self-certification mechanism.

Proof-carrying code~\cite{Necula02} allows software to be shipped with
accompanying formal correctness proofs that can be easily verified by
a theorem prover or certifying compiler at runtime. Such proofs can
help increase the trustworthiness of software and can easily be
included in small critical sections of an E2EVIV system, but are
impractical for verifying the correctness of the entire system.

Another way of increasing trust in the software is to ensure that all
implemented protocols are open, with publicly available
specifications, and that multiple redundant implementations of the
protocols are used throughout the system. Ideally, the redundant
implementations should have different low-level designs and be
implemented using different programming languages. Thus, in order to
corrupt the system, it would not be sufficient to corrupt a single
protocol implementation; instead, all the implementations would need
to be corrupted in a consistent way that could not be detected by
observing differences in their behaviors.

\paragraph{Trusting the Servers.} \tododmz{minimize TCB; use of public
  cloud infrastructure thereby intentionally loosing control over
  which system is used for which purpose; presumption in protocol and
  system design that no system, software, or person is to be trusted;
  full peer-to-peer robust architecture in the presence of active
  malfeasance and bad software; software independence; MPC; mobile
  root of trust}

Another potential way of distributing trust in an E2EVIV system is to
use secure multiparty computation (MPC), such that no single system
needs to be completely trusted. \todo{flesh this out}

\paragraph{Trusting the Network.} The Internet is inherently insecure,
so measures must be taken to guarantee some level of trust in the
system's Internet-based communications. The first, and most obvious,
of these is that all communication, among any components in the
system, should use the TLS protocol. The certificates used to
establish TLS connections should be \emph{pinned}, meaning that when a
client wants to connect to a server, the client already has
information about the server's security certificate (or information
about a set of certificates, if more than one is acceptable). If an
unexpected certificate is provided by the server, even if the hostname
on the certificate matches the hostname of the server, the connection
fails. 

Certificates used in E2EVIV systems should have signature chains that
involve not only a certificate authority (like Comodo, Symantec, etc.)
but also the electoral authority or other appropriate government
entity; requiring the electoral authority to be involved in the
signature chain ensures that certificates cannot be issued for
malicious purposes by a rogue certificate authority and used to
compromise a running E2EVIV system.

Finally, all communications of critical information (ballot
information, voter information, logs, audit data, etc.) in the
system---even over TLS-encrypted connections---should be carried out
using custom cryptographic protocols, so that even in the unlikely
event of a TLS compromise the data is protected.

\paragraph{Trusting the Voting Client.} \tododmz{client cannot be
  trusted; application certification and attestation; challenges in
  web client infrastructure v-v plethora of Javascript engines;
  Javascript JIT complexity; poor Javascript language design; tools
  like JSCert\footnote{\url{http://jscert.org/}}, Anders Moller's
  tools\footnote{\url{http://casa.au.dk/software-tools/}}, and MSR's
  CVK\footnote{\url{http://research.microsoft.com/en-us/projects/cvk/}}
  only take us a little bit of the way there; still research to do a
  la SAW for asm.js}

\paragraph{Trusting the Data.} \tododmz{data-in-transit discussed
  above; use of mainstream DB technology inappropriate given the
  integrity, confidentiality, and provenence requirements; use of
  novel systems crypto like
  CryptDB\footnote{\url{https://css.csail.mit.edu/cryptdb/}}, MPC via
  frameworks like ShareMonad, (partial) homomorphic crypto}

\paragraph{Trusting the Voter.} \tododmz{physical and digital process
  of distributing credentials to the voter; false claims from, or
  mistaken memory of, voters wrt challenging votes; inability to put
  any trust in voter to keep their client up-to-date, check SSL
  certificates, etc.}

\paragraph{Trusting the Cryptography.} \tododmz{on-paper proofs
  vs. mechanized proofs; verification of implementation against
  specification; synthesis of implementation from specification; use
  of NSA/NIST-blessed cryptography; improper sources of randomness;
  use of Intel-blessed hardware crypto}

\paragraph{Trusting the Toolchain.} Ken Thompson, in his 1984 Turing
Award lecture ``Reflections on Trusting Trust'' \cite{Thompson84},
demonstrated a fundamental problem with trust in computing systems: an
attack against the toolchain (compilers, assemblers, linkers) used to
build a system can silently, and effectively undetectably, insert a
``back door'' or other corruption into the system. If this attack is
carried out successfully, inspection of the source code for the
toolchain itself and the source code for the system will show nothing
unusual; the corrupted toolchain binary introduces the corruption when
building itself, or when building the rest of the system, and also
corrupts all the tools that can be used to analyze the system
(disassemblers, binary dump tools, etc.) such that the corruption
remains hidden. Thompson himself successfully carried out such an
attack within Bell Labs, and similar attacks have occurred ``in the
wild'' against systems such as the Delphi development environment for
Windows application; with stakes as high as controlling national
election results, it is not a stretch to believe that such attacks
would be attempted against E2EVIV systems.

There are multiple ways to mitigate the possible impact of such an
attack. One is to ensure that the system uses a diverse set of
implementations of key components, all based on the same specification
but with different source code, built with different compilers, and
preferably running on different hardware and OS platforms; corruption
of a single component, or even a small number of them, could then be
detected by the uncorrupted components, and the effort required to
corrupt the system as a whole would be much higher. Another is to
counter the possibility of Thompson-style exploits by using multiple
toolchains in the technique proposed by David A. Wheeler in his
Ph.D. thesis, ``Fully Countering Trusting Trust through Diverse
Double-Compiling'' \cite{Wheeler09}.

\subsection{Scalability}

An E2EVIV system, particularly at the national level, must be able to
handle a wide range of demand. It is human nature that many voters
will wait until the last day, or even the last hour, of a voting
period to cast their votes. Moreover, it is likely that attacks
against the system---and thus, system activity in general---will
increase in intensity as the end of a voting period approaches. Thus,
while the system may see very little sustained activity for much of an
election period, it must be able to scale to extreme levels of
activity at peak times. The architecture must take this into account,
so that the system can be dynamically deployed on more computing and
network resources as need arises. This might be done either by
utilizing public cloud resources that support elastic demand or by
using private resources that can be brought on- and offline as
required.

\subsection{Availability}

E2EVIV systems must exhibit high availability;
\autoref{chapter:required_properties} stated an explicit requirement
for 99.9\% uptime during election periods and the ability to recover
from generalized (i.e., not caused by natural disaster or malicious
attack on the system) failures in under 10 minutes, and higher
availability---including in the face of malicious attack---would be
preferable. There are a number of techniques for ensuring high
availablity of systems, including the use of services like those
provided by Cloudflare to handle traffic spikes and distributed denial
of service attacks. The system architecture should be constructed in a
way that does not foreclose the use of such techniques.

\subsection{Usability}

Usability, including accessibility for disabled voters, is of
paramount importance in an E2EVIV system. Especially for the
voter-facing parts of the system, the choice of implementation
technology may have a significant effect on usability. Essentially,
choices may need to be made between using Web technologies, which have
significant advantages in terms of reach (cross-platform, able to be
used on various sizes of device), and native applications, which tend
to exhibit richer interaction design and support more accessibility
features. The architecture might also allow for both types of
implementation, potentially at the cost of additional architectural
complexity.

\section{Architectural Feature Model}

As we have seen, there are many considerations to take into account
when making architectural decisions about an E2EVIV system. Here, we
model the various architectural dimensions, and the (possibly wide)
range of choices within each, to give a sense of the potential
solution space for a workable E2EVIV system architecture.

The Business Object Notation diagram in \autoref{figure:arch-choices}
shows the architectural choices that need to be made; for each
attribute of the architecture, a list of possible choices is
provided. There are seven dimensions, each of which can take on a set
of values. The values are chosen from 2-element sets for five of the
dimensions, and from 3-element and 4-element sets for the remining
two. Since each dimension must have at least one selection (the empty
set is disallowed), this allows for
$(2^2-1)^5\times(2^3-1)\times(2^4-1)=25,\!515$ possible architectural
variants.

\begin{figure}
\begin{center}
\lstinputlisting[style=bon,frame=lines]{architecture_resources/architecture-dimensions.bon}
\end{center}
\caption{A specification of the possible variants for an E2EVIV system.}
\label{figure:arch-choices}
\end{figure}

The architectural dimensions we have identified are the following:

\begin{itemize}
\item \textbf{Distribution of Authority} \ Authority in the
  system---that is, the ``official'' set of data stored in the system
  and control over access to and manipulation of that data---can be
  centralized or distributed. Centralized authority eliminates
  concerns about data consistency, as data can only be manipulated by
  one entity, but may cause issues related to system responsiveness,
  availability, and reliability. Distributed authority eliminates a
  single point of failure at the expense of needing to ensure data
  consistency and integrity. It is also possible to implement a hybrid
  authority model, where authority is concentrated in a small set of
  entities relative to the system as a whole; this is technically
  distributed authority because it is spread across entities, but
  behaves like centralized authority from the perspective of most of
  the entities in the system.

\item \textbf{Cryptography} \ The set of cryptographic algorithms and
  protocols used to protect voter privacy and insure ballot integrity
  is a critical component in any E2EVIV system. Security
  characterizations of individual cryptographic algorithms (e.g.,
  block ciphers such as AES, standard public key cryptosystems such as
  RSA, threshold cryptosystems such as ElGamal) can be found in the
  large body of cryptography literature, and the selection of
  individual algorithms is generally a matter of picking an
  appropriate algorithm and security strength for a given
  task. However, since novel cryptographic protocols such as those
  required to implement E2EVIV systems are not widely used or studied,
  evidence must be provided for the security of any such protocols
  used in a deployed system. Such evidence can be provided in two
  basic ways: through ``paper'' proofs that are carefully checked by
  multiple experts, or, if the protocols are mechanized in a formal
  specification language, through proofs that are automatically
  generated by cryptography protocol verifiers such as ProVerif
  \cite{ProVerif}. In general, it would be preferable for all
  cryptographic protocols in the system to be mechanized, as evidence
  could be generated repeatably and easily regenerated in the event of
  minor protocol changes; however, there may be cases where this is
  impractical. Thus, the cryptographic protocols of the system may
  have ``paper'' specifications, be mechanized in a formal system, or
  some combination thereof.

\item \textbf{Evidence of Correctness} \ The development of a high
  assurance system such as an E2EVIV system proceeds from a
  specification, at some level of formality, to an implementation that
  is intended to fulfill the specification. Assurance that the
  implementation actually does fulfill the specification can generally
  be obtained in two ways. First, the implementation can be developed
  in a way such that it is mechanically tied to the specification; for
  example, code generation techniques and refinement techniques can be
  used to mechanically generate correct implementations from
  specifications. Second, the implementation can be developed ``by
  hand'' with a set of included assertions that are meant to establish
  that the specification is being faithfully implemented; these
  assertions may then be checked by code analysis tools when building
  the system, or may be automatically compiled into testing code that
  validates the assertions when running the system. In practice, while
  it is clearly desirable for as much of the implementation as
  possible to be mechanically generated from the specification, most
  high assurance systems use some combination of these two techniques.

\item \textbf{Implementation Type} \ \todokiniry{Isn't there a
    requirement that the protocols and specs be open? Given that, and
    the previous dimension, why do we have this dimension?  i.e., what
    am I missing? -dmz} Regardless of the choices made along any of
  the other dimensions, a reference for what constitutes a ``correct''
  implementation must be provided. This can come in the form of a
  ``golden im\-ple\-men\-ta\-tion''---if a given implementation
  behaves identically to the golden implementation in all
  circumstances, it is a correct implementation. Alternatively, it can
  come in the form of open protocols and specifications, whereby any
  implemention can be checked for conformance and declared correct
  based on the results of that check.

  \tododmz{I was thinking here more about the critical choices of
    platform and programming languages. The golden implementation
    aspect still holds, since one might synthesize or build such an
    implementation via a correct-by-construction approach, but that
    version will not fulfill the performance requirements, e.g. -jrk}

\item \textbf{Key Distribution Method} \ {public ceremony; threshold
    crypto; PKI vs. web-of-trust}

\item \textbf{Deployment Style} \ The E2EVIV software can be deployed
  in one of three ways: (1) servers run entirely on trusted servers
  managed by the electoral authority or its designated
  representatives, and client applications access the servers through
  well-defined, well-controlled interfaces; (2) servers run, in whole
  or in part, on public cloud infrastructure, while client applications
  still access them through well-defined interfaces; (3) the system is
  structured in a peer-to-peer fashion, where ``server'' functionality
  is distributed across all entities in the system and at least some
  of them run in an uncontrolled environment (i.e., voters'
  computers). 

\item \textbf{Client Technology} \ Implementation of the client
  software used by voters, as well as the administration software used
  by the electoral authority, can be done in two basic ways: (1)
  develop custom applications for the various hardware/OS platforms
  that will be used by voters and the electoral authority; or (2) use
  Web application technologies to develop a single Web-based
  application that will be accessible from all (reasonable)
  platforms. It is also possible to choose both implementation
  strategies for all applications (i.e., both voters and the electoral
  authority can access the system through either native applications
  or a Web application, as they choose) or make different choices for
  different applications (e.g., the voter application is implemented
  as a Web application while the electoral authority's administration
  application is implemented as a native application).

\end{itemize}

\section{Primary Architectural Variants}

Given the many dimensions of the architectural feature model, and
number of choices in each, the number of possible architectural
variants for the system is large. Here, we briefly describe a few of
the primary system variants that can be described by the feature
model. Since we are only describing them at a high level, some of the
variants described correspond to multiple possible feature selections
in the feature model (for example, they might have any of the
different types of correctness evidence or any of the different
implementation types).

\subsection{Mirrored Servers}

One possible architecture, which features centralized authority as its
primary defining characteristic, is the ``mirrored servers''
architecture depicted in \autoref{figure:arch-mirrored-servers}. The
double arrows in this diagram (and later diagrams in this chapter)
denote \emph{client} relationships (one entity making use of another's
services), while the thick double-ended arrows denote \emph{mirroring}
relationships (entities, or groups of entities, ensuring that their
states accurately reflect each other for redundancy or availability).

\begin{figure}[t]
\begin{center}
\includegraphics[width=6.5in]{architecture_resources/mirrored-servers.pdf}
\end{center}
\caption{An architecture with centralized authority and mirrored
  servers.}
\label{figure:arch-mirrored-servers}
\end{figure}

In this architecture the Web/App Server (to which voters, using either
a web-based interface or custom applications, connect to cast their
ballots) is a client of the Database, which stores all information
relevant to the operation of the E2EVIV system (ballot styles, cast
and spoiled ballots, etc.). In an actual implementation, the
monolithic Database would likely be split into multiple databases
since the access patterns and performance needs for data such as
ballot styles, cast and spoiled ballots, voter lists, etc., are likely
to be quite different. There might also be more components within each
mirror (for example, separate servers for dealing with native
applications vs. web access in a system that supports both).

Regardless of the number of servers within each mirror, the mirroring
in this architecture is done primarily for availability and
reliability; it ensures that, as long as at least one set of mirrored
servers is running, the system can remain operational (albeit perhaps
at a degraded level of responsiveness). Authority is centralized in
the sense that each mirror has a complete set of data for the system
and behaves accordingly; one mirror is designated as the
\emph{primary} mirror and is considered the authoritative source of
information in the event of inconsistency. Voters and the electoral
authority access the system by interacting with an individual
(typically, the primary) mirror, and the entire set of mirrors appears
logically as a single server-side system.

\subsection{Large Fixed Set of Servers}

Another possible architecture, which introduces the potential for
distributed authority but still has the logical presentation of a
single server-side system, is a large fixed set of servers. This
architecture, an example of which is depicted in
\autoref{figure:arch-large-fixed-set-of-servers}, still features
mirroring for redundancy and availability; however, it allows for
flexible allocation of resources. For example, there might be twice as
many Web/App Server instances as there are Database instances, or
there might be more Database instances dealing with dynamic cast and
spoiled ballot data than dealing with fixed election definition data
such as ballot styles.

\begin{figure}[p]
\begin{center}
\includegraphics[width=6.5in]{architecture_resources/large-fixed-set-of-servers.pdf}
\end{center}
\caption{An architecture with a large fixed set of servers.}
\label{figure:arch-large-fixed-set-of-servers}
\end{figure}

\begin{figure}
\begin{center}
\includegraphics[width=6.5in]{architecture_resources/large-fixed-set-modified.pdf}
\end{center}
\caption{The same fixed set of servers as in
  \autoref{figure:arch-large-fixed-set-of-servers} performing a different
  allocation of tasks.}
\label{figure:arch-large-fixed-set-modified}
\end{figure}

The servers within this architecture could, amongst themselves, behave
as a peer-to-peer system, a set of client-server systems, or a set of
mirrors of various sizes for the purposes of providing high
availability and redundant storage and ensuring data consistency. A
key aspect of this architecture is that the number of servers, while
large, is fixed; this allows the topology of the servers and the
communications amongst them to be known at all times, making it
straightforward to monitor the system's health and performance and to
quickly detect any issues that arise.

As an example, \autoref{figure:arch-large-fixed-set-of-servers}---only
one of many possible server topologies in such an architecture---has
two separate mirrored Databases (one with two mirrors, and one with
three) being accessed by three separate Web/App Servers. If it is
determined that the Databases are underloaded and the Web/App Servers
are overloaded, one of the servers running Database B could easily be
repurposed to run an additional Web/App Server
(\autoref{figure:arch-large-fixed-set-modified}) without changing the
actual set of servers in the architecture and without compromising the
redundancy of data storage in the system.

While one possible deployment of this architecture would see every
server containing the full authoritative data set, it is far more
likely that each would contain only part of it and that the authority
in the system would, therefore, follow either a hybrid or a distributed
model. 

\subsection{Dynamic Cloud}

The two previous architectural variants involved the deployment of a
fixed set of servers, either as a collection of mirrors or in other
topologies. The next variant departs from these by deploying services
not across a fixed set of servers, but instead within a dynamic cloud
infrastructure, while still presenting itself as a single server-side
system for external interactions. Such an infrastructure allows for
the addition and removal of computing resources as necessary during
the operation of the system, using various distributed communication
and consistency protocols to deal with resource changes in a way that
is effectively invisible to the system's users while maintaining data
integrity and service
availability. \Autoref{figure:arch-dynamic-cloud-small,
  figure:arch-dynamic-cloud-large} show snapshots of a dynamic cloud
deployment at times when it has five and eleven running servers,
respectively. Note, in particular, that the client relationships among
the servers in the cloud may evolve over time as well; for example, in
\autoref{figure:arch-dynamic-cloud-small}, the server at the ``top''
of the cloud could establish direct communication with the server at
the ``bottom left'' of the cloud if necessary.

\begin{figure}[p]
\begin{center}
\includegraphics[width=5.5in]{architecture_resources/dynamic-cloud-small.pdf}
\end{center}
\caption{A dynamic cloud architecture with a small number of servers.}
\label{figure:arch-dynamic-cloud-small}
\end{figure}

\begin{figure}
\begin{center}
\includegraphics[width=6.5in]{architecture_resources/dynamic-cloud-large.pdf}
\end{center}
\caption{A dynamic cloud architecture with a larger number of servers.}
\label{figure:arch-dynamic-cloud-large}
\end{figure}

Effectively, a dynamic cloud deployment behaves similarly to a
deployment with a large fixed number of servers; the main difference
is that the number of servers is variable. This allows for the system
to initially consume minimal resources, expanding or contracting as
necessary (within the bounds of the dynamic cloud) to maintain
acceptable response time and availability in the face of elastic
demand.

Despite the use of the word ``cloud'', a dynamic cloud architecture
need not actually be deployed on a public cloud infrastructure;
private cloud infrastructures consisting of only trusted servers may
be built as necessary to support the system.\footnote{In the E2EVIV
  context, however, public cloud infrastructures are likely preferable
  for economic reasons; it is virtually inconceivable that an
  electoral authority or its suppliers could build a cloud
  infrastructure with scale and reliability comparable to existing
  public cloud infrastructures at reasonable cost.}  Regardless of
whether the system is deployed on a trusted or public infrastructure,
authority in a dynamic cloud architecture follows either a fully
distributed or a hybrid model; some servers in the cloud may have
authority over others, or they may interact using consensus protocols
or similar mechanisms.

\begin{figure}[t!]
\begin{center}
\includegraphics[width=6.5in]{architecture_resources/peer-to-peer.pdf}
\end{center}
\caption{A peer-to-peer architecture.}
\label{figure:arch-peer-to-peer}
\end{figure}

\subsection{Peer-to-Peer}

In all the architectural variants described so far, the system
presents itself as a single ``server'' regardless of its ``internal''
network topology. In a \emph{peer-to-peer} implementation, the
computational work of the system is distributed across all the
participants and there is no clearly defined distinction between
``client'' and ``server''. For example,
\autoref{figure:arch-peer-to-peer} depicts a peer-to-peer system with
a number of peers belonging to individual voters, some belonging to
political parties (A, B and C), and some belonging to the electoral
authority. The double-headed arrows in the figure represent
communication links among the peers; for example, if the upper-left
peer belonging to Party A needs to communicate with the lower-right
peer belonging to the electoral authority, it must send a message that
travels across at least 7 communication links. The communication links
in a peer-to-peer network typically change over time, based on each
peer's knowledge about its network environment and the locations of
other peers.

Authority in a peer-to-peer architecture is fully distributed. In the
case of an E2EVIV system, the electoral authority would set up and
maintain some trusted peers as a way to ``bootstrap'' the peer-to-peer
network, and political organizations (parties, lobbying groups, etc.)
might also choose to maintain peers, perhaps with their own
implementations of the election software in a system designed with
open protocols and specifications, as a way of participating in the
electoral process and strengthening trust in the results. Individual
voters running the software on their own machines would also be peers
for the duration of their voting sessions (or longer, if they chose to
contribute to the management of the election by leaving the software
running); effectively, a peer-to-peer architecture is a way of
``crowdsourcing'' the resources required to run election system. 

A peer-to-peer architecture raises significant security concerns that
differ from those of the other architectures we have described. While
some of the computer systems controlled by the electoral authority
might be trusted, the vast majority of systems belonging to individual
voters or political organzations will certainly not be; it is
impossible to run a peer-to-peer E2EVIV system using only trusted
computing resources.\footnote{This is true of ``pure'' peer-to-peer
  architectures of the type we are discussing in this section; an
  architecture where only the \emph{servers} interact in a
  peer-to-peer fashion while presenting a single interface or set of
  interfaces to clients can, as previously noted, be implemented on a
  fixed set of servers or in a dynamic cloud.} It is therefore
important to ensure that no corrupt peer, or set of corrupt peers, can
undetectably compromise election results, violate voter privacy, or
otherwise violate the E2EVIV system requirements.

One way to address this problem is to employ a \emph{blockchain}, like
that used in Bitcoin and other cryptocurrencies, to log critical
election information (cast and spoiled ballots, the fact that a given
voter has voted in the election, etc.). A blockchain is a public
write-only ledger, collectively maintained by the peers in the system,
that records a sequence of events. The mechanism by which this
recording is done ensures that the peers reach a consensus about the
events that have occurred and their ordering, and that once an event
(such as the casting of an encrypted ballot) has been placed in the
ledger it can be neither modified nor reordered with respect to other
events. As long as more than half of the peer computing power in the
network is ``honest'' and follows the correct protocol, the integrity
of the ledger is guaranteed. At any given time, it is likely that the
computing power contributed by the electoral authority and
high-profile political organizations---which can be hosted on trusted,
closely-monitored computing systems---will vastly outweigh the
computing power contributed by individual voters during their ballot
casting sessions; moreover, the situation where more than half the
peer computing power is dishonest can be detected (by the honest part
of the network, or by external observers) and dealt with in various
ways. Thus, maintaining the integrity of a blockchain should be
reasonably straightforward in an E2EVIV system. However, other aspects
of implementing a peer-to-peer architecture---such as distribution of
the computing client to voters and organizations, achieving sufficient
ease of use and performance, etc.---may prove more difficult.

\section{Summary}

As can be seen from the many architectural dimensions we have
described and the primary architectural variants we have briefly
discussed, there are many different ways in which an E2EVIV system
could be designed and implemented. It is not clear which of the
primary variants would be the ``best'' option, nor is it clear exactly
what criteria would be used to make that determination among multiple
architectures that fulfill all the E2EVIV requirements. Further
research and experimentation is therefore necessary to determine a
suitable path forward for E2EVIV implementation and deployment.

 % Joe K./Dan
\chapter{Verification and Validation\ifdraft{ (Joe K./Dan/Adam) (20\%)}{}}
\label{chapter:v_and_v}

\section{Requirements and Scenarios}
\section{Methodology}

\subsection{Engineering Methodology}

\todoacf{Introduce context with NIST, ISO standards for methodology?}

Sound engineering practices are the foundation for building reliable
and secure software of any type. This is particularly true for
critical systems which must be trusted to perform important tasks
correctly, and where the consequences for failure threaten lives,
political system integrity, and property.

This section introduces methodologies that help reduce errors and
improve confidence in software development. We avoid discussion of
particular technologies except to illustrate our recommended
methodologies with examples in pracice.

\subsubsection{Version Control}

Version control systems (VCS) manage changes between the versions of a
project as it evolves during the course of development. Revision
control is the preferred way to share software artifacts across a
team, but all software projects, even those developed by teams as
small as one person should use VCS.

In general, a developer uses the VCS first to ``check out'' the files
comprising a project into her ``working copy''. Then, after making
changes to those files, the developer ``checks in'' or ``commits'' the
changes to the VCS. After committing, those changes are available for
other developers to integrate into their working copies.

When different sets of changes have been made by multiple developers,
the VCS can merge those changes either automatically or by using
developer input to ensure the project remains consistent. The ability
to merge changes is critical for teams of developers who work
concurrently on a single project, and is the reason that file sharing
tools like Dropbox or Google Drive are an inadequate substitute for a
VCS.

VCS is particularly important for projects that must be audited. An
entry to a project log is created every time a developer commits their
work. Any file in the project can be inspected to show its provenance,
even down to the level of which line was committed by which developer
on what date. Some VCS tools also optionally allow for commits to be
cryptographically signed, offering assurance that, for example, the
changes have been audited by a trusted authority before being
integrated into the project~\cite{chacon2014pro}.

Moving from simple file storage to VCS is a tremendous improvement for
development, but poor use of VCS can negate many of the potential
benefits. For example, the log built from the commits of developers is
of less use to auditors if the changes in each commit are not clearly
associated with a particular new feature or bug fix. Likewise if
developers commit changes in a broken state, other developers'
productivity suffers and it becomes more difficult later to isolate
which commit introduced a bug. Each VCS supports multiple workflow
practices that should be adopted in order to limit these problems and
get the most benefit from VCS
use~\cite{atlassianworkflow}\cite{pilato2008version}.

\subsubsection{Issue Tracking}

\subsubsection{Continuous Integration and Testing}

\subsubsection{Code Review}

\subsubsection{Release Management \& Lifecycle}

\subsubsection{First-Class Documentation}

\subsubsection{Automation}

\section{Technologies}
\section{Interpreting Results}
 % Joe K./Dan
\chapter{Feasibility (???)}
\label{chapter:feasibility}

\section{Threats and Security Risks}
\section{Availability}
\todogeneric{Should this be ``accessibility''?}
\section{Usability}
\section{Legal Frameworks and Politics}
\section{LEO Considerations}
\section{Cost}
\subsection{Design and Development}
\subsection{Operational}
\subsection{Integration with Local Election Systems and Processes} % Poorvi/David J.
\chapter{Conclusion\ifdraft{ (Joe K./Susan) (100\%)}{}}
\label{chapter:conclusion}

There is tremendous pressure to build Internet voting systems and use
them in public elections. Researchers, developers, and election
officials must take the time to understand the requirements for secure
and trustworthy elections so that they can evaluate systems---both
good and bad---and make well-informed decisions.  The use of flawed
election systems in public elections can result in significant and
irrevocable harm.

This report presents the most complete set of requirements to date
that must be satisfied by any Internet voting system for public
elections. This set of requirements, described
in~\autoref{chapter:required_properties} and published in complete
detail in a separate document~\cite{E2EVIVBON}, is useful to
several audiences:
\begin{itemize}
\item \emph{legislators} and their staffs who may craft laws that
  relate to remote elections, particularly Internet elections and
  elections for overseas, military, and disabled voters;
\item \emph{election officials} who may specify, evaluate, or
  purchase Internet voting products or services;
\item \emph{activists} who wish to better understand, and advocate
  for, E2E-VIV systems;
\item \emph{standards bodies} that may standardize various classes of
  Internet voting technologies, and specify the level of rigor for
  certifying Internet voting systems;
\item \emph{testing organizations} that may test Internet voting
  systems for compliance with technical standards;
\item \emph{researchers and engineers} who may continue working
  toward viable E2E-VIV systems.
\end{itemize}

This report also contains additional information useful to a subset of
these audiences, including:
\begin{itemize}
\item a basis for developing the cryptographic foundations with which
  to evaluate and compare various E2E protocols and E2E-VIV
  systems~(\autoref{chapter:crypto_spec}),
\item an analysis of the architecture space of E2E-VIV
  systems~(\autoref{chapter:architecture}),
\item precise recommendations on the state of the art for rigorous
  engineering of E2E-V systems~(\autoref{cha:rigor-softw-engin}),
\item a framing for an ongoing discussion about the feasibility of
  designing, constructing, certifying, legalizing, and deploying
  E2E-VIV systems~(\autoref{chapter:feasibility}), and
\item a reflection upon the outstanding issues that must be addressed
  in future stages of E2E-VIV development, including political, legal,
  research, and engineering challenges~(\autoref{sec:next-steps}).
\end{itemize}

Following are recommendations and possible next steps.

%=====================================================================
\section{Recommendations}

The E2E-VIV Project team does not assert that Internet voting
\emph{must} be pursued, nor does it assert that Internet voting must
\emph{never} be pursued. There is no consensus among the team members
for either of these positions. With this understanding, the project
team recommends the following.

\recommendation{E2E-V}{Any public
  elections conducted over the Internet must be end-to-end
  verifiable.}

The use of Internet voting systems without end-to-end
verifiability---including all Internet voting systems that
jurisdictions are experimenting with and using at the time of this
writing---is irresponsible. Any voting systems used to conduct public
elections over the Internet must be E2E-VIV systems.

\recommendation{SUPERVISED FIRST}{No
  Internet voting system of any kind should be used for public
  elections before end-to-end verifiable in-person voting systems have
  been widely deployed and experience has been gained from their use.}

It is critical to gain experience with E2E-V in the simpler in-person
setting before attempting to deploy it in the vastly more complex
Internet setting. Using E2E-V for in-person elections will also
improve the integrity of existing in-person voting systems.
 
If election officials and election system vendors ignore these first
two recommendations, we expect that deficient, unverifiable Internet
voting systems will be widely used within ten years.  Vendors will
claim to have solved the security problems, and eager officials will
believe these claims.  Elections may be altered with no public
awareness.  If election officials manage to find evidence left by a
careless attacker after altering an election, the damage will have
already been done.
 
In making these first two recommendations, we realize that we have
created a difficult path to follow. We assert that no attempt at
Internet voting should deviate from this path.  Building an in-person
E2E-V system is no small task. Building an E2E-VIV system that
satisfies the requirements in this report is even more ambitious; it
may even be impossible. However, if it is possible, the resulting
system will be far better than the vulnerable Internet voting
alternatives.

\recommendation{HIGH ASSURANCE}{End-to-end verifiable
  systems must be designed, constructed, verified, certified,
  operated, and supported as high-assurance systems according to the
  most rigorous engineering requirements of mission- and
  safety-critical systems.}

A software independent voting system does not rely on high-assurance
software to detect errors in the election outcome. However,
high-assurance software engineering tools and techniques can make such
errors much less likely to occur, and can also reduce the risk of the
following problems:

\begin{enumerate}
\item \textbf{PRIVACY VIOLATIONS.} While E2E-V systems can identify
  and mitigate issues of election integrity, they cannot do the same
  for privacy issues. A poorly-implemented E2E-V system will allow
  observers to detect certain issues with the election (such as votes
  not being counted correctly), but not to detect when voter
  identification details are stolen from an insufficiently secured
  server.
\item \textbf{PROGRAMMING ERRORS.} A low-quality E2E-V system is far
  more likely than a high-quality one to have software engineering
  flaws in design, functionality, security or other areas that trigger
  failures in verification. These will increase the burden on election
  administrators to deal with partial failures. This could also
  significantly impact the voters' trust in the election process, as
  well as the election administrators, apparatus, and outcome.
\item \textbf{SECURITY ISSUES.} Low-quality implementations of any
  type of software system are extremely difficult---and often
  impossible---to secure in the presence of insider or outsider
  attack. Security is not a band-aid to apply to a poorly-implemented
  system; it can only be achieved through a combination of rigorous
  process, method, design, implementation, validation, verification,
  deployment, and operation.
\end{enumerate}

High-assurance software engineering is the only reasonable way to
attempt to implement an E2E-V system that is correct, secure, and does
not have enormous fiscal and trust implications for election officials
after deployment. A less rigorous development approach will almost
certainly lead to costly defects.

\recommendation{UNIVERSAL DESIGN}{E2E-VIV systems must be usable and accessible.}

It is not feasible to make voting easy for voters with the most
extreme disabilities. However, it is essential that we at least serve
voters who have challenges in vision, hearing, comprehension, or
motion yet can still use some kind of computing device. We should use
a qualitative and quantitative testing-based experimentation platform
to assess usability and accessibility, and follow best
practices~\cite{materials-at-elections.itif.org},
recommendations~\cite{WAI,Section508,WAVE}, and
standards~\cite{ADAStandards} in accessible UI design and
implementation. By doing so, we will be able to service nearly every
overseas and military voter and, in the long term, the more than 84
million disabled voters in the U.S.~\cite{CensusData}.

We must also look to, and learn from, the AnywhereBallot and EZ Ballot
experiments~\cite{AnywhereBallot,lee2012ez} of the Accessible Voting
Technology Initiative~\cite{AVTI}.  We must engage with the
researchers and attendees at the annual California State University,
Northridge International Technology and Persons with Disabilities
Conference~\cite{CSUN}. Only through direct engagement with voters,
both abled and disabled, can we have any hope of understanding how to
develop a usable, accessible E2E-VIV system.

\recommendation{MOVE FORWARD}{Many challenges remain
  in building a usable, reliable, and secure E2E-VIV system. They must
  be overcome before using Internet voting in public
  elections. Research and development efforts toward overcoming those
  challenges should continue.}

Building a usable, reliable, and secure Internet voting system may be
impossible. Solving the remaining challenges, however, would have
enormous impact on the world. Continued research and development
efforts must be conducted transparently, with all results and
artifacts open to peer review. Internet voting systems, including
E2E-VIV systems, must not be deployed in public elections before all
the key security problems are resolved.

%=====================================================================
\section{Next Steps}
\label{sec:next-steps}

To carry out these recommendations, legislators, researchers, and
engineers face several challenges.

%~~~~~~~~~~~~~~~~~~~~~~~~~~~~~~~~~~~~~~~~~~~~~~~~~~~~~~~~~~~~~~~~~~~~~
\subsection{Political and Legal Challenges}

The greatest concern voiced by election verification scientists,
election integrity advocates, and E2E-V researchers is that
legislators will mandate the experimentation with---or use
of---Internet voting before a correct, secure, open, usable,
accessible E2E-VIV system exists. Using the current untested and
unverified systems opens the door to wholesale election manipulation
or failure. Aggressive early adoption of election technology must be
tempered by a clear understanding that voters' trust in their
elections is hard-won and easily lost.

Scientists and election integrity advocates are also very concerned
that well-meaning legislators and election officials will push to
deploy Internet voting systems too early and too quickly, based on
misleading information from prospective vendors and other advocates
that is not balanced with \emph{independent} advice from cybersecurity
experts.  They may also misunderstand how the security risks grow with
the scale and significance of the election, and how these risks change
over time as the threat environment changes. Such misjudgments may
sometimes induce legistators and election officials to weigh political
goals more highly than the security risks.

The political and legal challenges---and related
opportunities---should focus on how to legislate the evidence-based
measured introduction of new elections technologies. This includes
Internet voting. \emph{Defining an appropriate pace, milestones, and
  success criteria for the introduction of E2E-VIV systems must be a
  primary focus of any next phase of this project.}

%~~~~~~~~~~~~~~~~~~~~~~~~~~~~~~~~~~~~~~~~~~~~~~~~~~~~~~~~~~~~~~~~~~~~~
\subsection{Research and Engineering Challenges}

Although E2E-V is necessary for a viable Internet voting system, use
of E2E-V does not ensure that an Internet voting system is free from
vulnerabilities. Also, the definition of an E2E-VIV protocol for
U.S. elections is very challenging. In particular, the research
community must determine how to address five key challenges:
\begin{itemize}
\item how to handle large-scale dispute resolution;
\item how to authenticate voters for public elections;
\item how to defend an E2E-VIV system against denial-of-service
  attacks and automated attacks that aim to disrupt large numbers of
  votes;
\item how to make verifiability comprehensible and useful to the
  average voter; and
\item how to avoid voter coercion and vote selling in the context of
  digital observation of voting and verification.
\end{itemize}

The usability facets of E2E-VIV are also challenging. The main issues
with usability are:
\begin{itemize}
\item how to ensure usable vote privacy and vote integrity in the
  presence of client-side malware and
\item how to ensure that verification is usable and accessible to the
  typical set of voters.
\end{itemize}

While some of these issues can be addressed by current technologies,
further research is necessary to determine if all of these concerns
can be adequately addressed, as discussed at length in the preceding
chapters and as codified in our requirements. \emph{Until researchers
  adequately address these challenges, Internet voting systems should
  not be used in public elections.}

%~~~~~~~~~~~~~~~~~~~~~~~~~~~~~~~~~~~~~~~~~~~~~~~~~~~~~~~~~~~~~~~~~~~~~
%\subsection{Engineering Challenges}

The development and deployment of a high-assurance distributed system
of the scale and import of a public E2E-VIV election system has never
been attempted. It involves considerable engineering challenges in
addition to the fundamental research challenges already
mentioned. \emph{If the research issues can be solved, current best
practices for building high-assurance distributed systems should be
sufficient to address the engineering issues.}

%~~~~~~~~~~~~~~~~~~~~~~~~~~~~~~~~~~~~~~~~~~~~~~~~~~~~~~~~~~~~~~~~~~~~~
\section*{Coda}

Many people believe that Internet voting will increase voter
participation, help with voter decision-making and engagement, provide
equal opportunity for voters with disabilities, and decrease election
costs.

Proponents of E2E-V election systems hope that their adoption will
prevent corrupt election officials and governments from manipulating
election outcomes, and will truly capture the voice of the people and
increase confidence and trust in government.

Trustworthy democracy is a worthwhile goal, and we should strive to
achieve it. The only responsible way to make progress is to continue
peer-reviewed research and experimentation.
 % Joe K./Susan
\appendix
\chapter{BON Representation of E2E VIV Requirements\ifdraft{ (Dan/Joe K.) (50\%)}{}}
\label{appendix:bon_requirements}

BON (from external files) will appear here. Currently it is just
dumped in a somewhat reasonable order, but it will be cleaned up and
brought up to date.

\lstinputlisting[style=bonbw]{bon_requirements_resources/e2eviv.bon}
\lstinputlisting[style=bonbw]{bon_requirements_resources/technical.bon}
\lstinputlisting[style=bonbw]{bon_requirements_resources/non-functional.bon}
\lstinputlisting[style=bonbw]{bon_requirements_resources/accessibility.bon}
\lstinputlisting[style=bonbw]{bon_requirements_resources/assurance.bon}
\lstinputlisting[style=bonbw]{bon_requirements_resources/auditing.bon}
\lstinputlisting[style=bonbw]{bon_requirements_resources/authentication.bon}
\lstinputlisting[style=bonbw]{bon_requirements_resources/certification.bon}
\lstinputlisting[style=bonbw]{bon_requirements_resources/evolvability.bon}
\lstinputlisting[style=bonbw]{bon_requirements_resources/functional.bon}
\lstinputlisting[style=bonbw]{bon_requirements_resources/interoperability.bon}
\lstinputlisting[style=bonbw]{bon_requirements_resources/legal.bon}
\lstinputlisting[style=bonbw]{bon_requirements_resources/maintenance.bon}
\lstinputlisting[style=bonbw]{bon_requirements_resources/operational.bon}
\lstinputlisting[style=bonbw]{bon_requirements_resources/procedural.bon}
\lstinputlisting[style=bonbw]{bon_requirements_resources/system_operational.bon}
\lstinputlisting[style=bonbw]{bon_requirements_resources/reliability.bon}
\lstinputlisting[style=bonbw]{bon_requirements_resources/security.bon}
\lstinputlisting[style=bonbw]{bon_requirements_resources/usability.bon}
 %Dan/Joe K.

\printbibliography

\end{document}

%%% Local Variables:
%%% mode: latex
%%% TeX-master: "report"
%%% End:
