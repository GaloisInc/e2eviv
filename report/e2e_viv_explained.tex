\chapter{E2E VIV Explained\ifdraft{ (Philip/Daniel) (20\%)}{}}
\label{chapter:e2e_viv_explained}

\section{IV, VIV, E2E}
\section{E2E Election Rituals}
\subsection{Pre-Election Phase}
\subsection{Voting}
\subsection{Post-Election Phase}
\section{Shortcomings and Expectations of E2EVIV}
\subsection{Access to Communication/Internet}
\subsection{Accessibility}
\subsection{Usability}
\section{E2E VIV in Practice}
Building on these ideas, a number of practical voting systems have been
developed.  Each of the following systems has actually been used in a real
election or in a pilot:

\subsection{RIES~\cite{hubbers2004}}

RIES, the Rijnland Internet Election System, was developed to support
elections to the Rijnland water management board, supplementing the system
of postal voting used by the water board.  It weakens the receipt-freeness
requirement generally accepted for E2E voting systems while providing
universal verifiability and a degree of individual verifiability.  These
compromises are based on the fact that remote voting generally, including
postal voting, has weak coercion resistance, and adding universal
verifiability and any degree of individual verifiability is a distinct
improvement.

Before the election, credentials are mailed to every voter in the form of a
very long number (encoded in Crockford's Base 32)
(see~\cite{crockford2002}).  The same mailing also includes instructions.
Voters log into an election web site that includes a client-side voting
application written in Javascript.  The algorithms used by this script are
public, and each voter, havign access to all of the inputs and outputs, may
(in principle) check the computations.  This is weaker than the desired
individual verifiability, but nonetheless, far stronger than conventional
voting systems.

The client-side application encrypts the vote by passing the voter
authorization code and the public ID of the candidate through a one-way
function to create the encrypted vote.  The encrypted vote is then placed on
the public bulletin board that serves as a ballot box.  At the close of the
polls, the election authority releases the codebook containing the
encryptions of all valid credentials with all candidate IDs. If a voter
discloses her credential or her encrypted vote, the published codebook may
be used to violate ballot secrecy. The developers of RIES judged this
violation to be no more severe than the threats to ballot secrecy inherent
in postal voting.

RIES was used for Rijnland Water Board elections starting in 2004, and in
2006, it was used to allow expatriate voters to participate in the Dutch
parliamentary elections, see~\cite{gonggrijp2009}. The OSCE sent an election
assessment team to observe the use of the system in 2006.  Their report
contains observations of critical security features of the system that could
not be observed; see~\cite{osce2007}.

One of the more important practical contributions of RIES involves the
recognition that, when voter authorizations are distributed by post or more
generally, distributed long in advance of the election, a mechanism must be
provided allowing voters to obtain replacement credentials, for example, by
telephone or e-mail, and a mechanism is needed to invalidate lost
credentials.  Adding these mechanisms adds significant complexity to the
system and is a source of some of the problems reported in the OSCE report.
A second feature of RIES is that it allows parallel testing during the
election, where pre-invalidated test ballots are deliberately added to the
bulletin board in order to test the network path from selected internet
clients to the server.  This offers the possibility of detecting a variety
of spoofing and denial of service attacks.

\subsection{Prêt à Voter~\cite{chaum2005}}

The Prêt à Voter system uses two-part paper ballots with the candidate names
on one part and the voting targets plus a ballot ID number on the other.
Typically, the two parts are printed as a single sheet, with a perforation
along which the sheet can be divided after voting.  The order of the
candidate names appears to be random, from the voters' perspective.  After
voting, the half of the ballot containing candidate names is shredded. The
voted half forms the cast ballot, and the voter may take a copy home.

A curious voter may inspect the public record of any cast ballot by
searching for it by the ballot ID number in the public database of cast
ballots.  That database shows the positions that were marked on that ballot,
but crucially, it does not show the identifying letters or candidate names.
The voter may therefore verify that the positions marked at the polling
place were correctly recorded by the jurisdiction, but because the voter
only has half of the ballot and there is no public display of the linking of
candidate names to ballot positions, the voter cannot prove to anyone else
how the ballot was voted.

The key to tabulating the votes is that there is a cryptographically secure
mapping from the ballot serial numbers to the apparent random order of the
candidate voting positions.  The decoding of a cast ballot to a
corresponding plain-text ballot, where the voted positions and candidate
names are in a canonical order (alphabetical order, for example) is
performed in multiple steps, by multiple custodians, for example, using a
mixnet. As with the mixnet, no one, even if colluding with a subset of the
custodians, can connect a decoded ballot to the corresponding cast ballot
(and, hence, to the voter).  Unvoted ballots may be audited before, during
and after the election to ensure that the decoding of cast ballots is being
correctly performed (audited ballots may not be used for voting purposes,
however, because the connection between the voting positions and candidate
names is publicly revealed during the audit). Randomly selected stages in
the decoding can be challenged to prove the integrity of the count, and
decoded ballots are easily counted.  Anyone may therefore check the count.

A version of Prêt à Voter is planned for use in a governmental election by
the state of Victoria in Australia in November 2014, see~\cite{burton2012}.
Previously, an attempt was made to use Prêt à Voter in a student election at
the University of Surrey in February 2007, see~\cite{bismark2007} -- this
attempt illustrates many of the perils of working with an election
authority.

\subsection{Punchscan~\cite{popoveniuc2006,popoveniuc2010punchscan}}

The Punchscan system used a two-part paper ballot where the top part had
candidate names and candidate numbers (or letters) and the bottom part had
the numbered (or lettered) voting targets.  The top part had holes punched
in it to expose the voting targets below.  The order of the voting targets
for each race appears random to the voter.  Both halves of the ballot bear
an identical serial number.  After marking the ballot with a bingo dauber,
the two halves are separated.  Prior to separation, the voter can easily see
that the correct target has been marked.  To vote, the voter separates the
two sides.  Either side can be scanned (since the bingo dauber marked both
through the hole and around it) as the cast ballot. The other side is
destroyed. A copy of the cast side may be retained by the voter.

A curious voter may inspect the public record of any cast ballot exactly as
with Prêt à Voter. It does not matter which half of the ballot the voter
retained, because there is no public display of the numbers that link
candidate names to voting positions; only the position that was marked is
displayed.  Again, individual ballots may be audited, and the key to
tabulating the votes is that there is a cryptographically secure mapping
from the ballot serial numbers to the apparent random order of the candidate
voting positions.

Punchscan was used for the graduate student association elections of the
University of Ottawa in 2007, see~\cite{essex2007}. It is likely the first
end-to-end voting system with ballot privacy used in a binding election.

\subsection{Scantegrity II~\cite{chaum2008,chaum2009}}

Scantegrity II (Invisible Ink) allows end-to-end cryptographic verification
of optical-scan paper ballots.  Scantegrity ballots may be fully compatible
with conventional optical scan vote tabulation equipment, but are voted
using a marking pen filled with invisible ink.  When the voter marks a
target on the ballot, the ink in the pen reacts with invisible ink on the
paper to disclose a three-letter alphanumeric code in the marked voting
target.  A voter who takes note of this code and the ballot ID number may
check a public bulletin board to check that the ballot was indeed tabulated.

As with Punchscan and Prêt à Voter, there is a cryptographically secure
mapping from the ballot ID number to the code disclosed when the voter marks
a voting target.  The code is displayed on the public bulletin board, and
public verification of the decryption and tally of the contents of the
bulletin board proceeds in a manner similar to that of Punchscan for the
system fielded in Takoma Park, and in a somewhat different manner using
similar principles for the system described in~\cite{chaum2009}.

Scantegrity II is the first end-to-end voting system with ballot privacy to
see use in government elections.  It was used in Takoma Park, Maryland in
2009 and 2011; see~\cite{carback2010}.

\subsection{Helios~\cite{adida2008,adida2009}}

Helios was developed for Web-based Internet voting. Helios provides a
web-browser-based ballot preparation system that can be used to make choices
and encrypt ballots.  After preparing a ballot, the voter has the option of
either casting that ballot or auditing it; this allows the voter to detect
misbehavior on the part of the ballot preparation software. Voter
authentication is not required until after the voter decides to cast the
ballot, so anyone may prepare and audit ballots.  Existing Helios
implementations piggyback on existing university and commercial
authentication mechanisms, and also support an internal login-password
authentication mechanism for which credentials can be distributed by email.

As with the schemes discussed above, all cast ballots are posted in
encrypted form on a public bulletin board so that voters may check that
their ballots have been correctly recorded.  Similarly, after the polls
close, the decryption and vote tally may be checked. Helios uses homomorphic
vote tallying for simplicity, even though a single mixer was used in its
first developments, and mixnets have been used in several of its forks and
descendants; see~\cite{bulens2011} or~\cite{tsoukalas2013} for instance. A recent
variant by~\cite{cortier2014} proposes an enhanced ballot authentication
mechanism. The STAR-Vote system~\cite{star-vote}, designed for public
elections as a result of an invitation from the Travis County election
administration, also borrows various techniques used in Helios and
VoteBox~\cite{sandler2008}.

Helios was used for the election of a Belgian university president in March
2009 and by numerous universities and associations since then, including the
ACM and the IACR. The cryptographic protocols it implements have been
subject to considerable analysis.

\subsection{Norwegian System~\cite{gjosteen2012}}

Between 2011 and 2014, the Norwegian government ran an Internet remote
voting trial.  The system rests on a cryptographic protocol designed by
Scytl, a commercial voting system vendor. Scytl and the Norwegian government
assert that this is an E2E system.  As such, this marks the entry of efforts
(or claims) by commercial voting system vendors to enable E2E elections.
System descriptions have been incomplete, hence it has not been possible to
tell what properties the system possesses. However, the following are clear.
The system's claims to protect voter privacy are weak: ``If the voter's
computer and the return code generator are both honest, the content of the
voter's ballot remains private.''  In addition, the receipt delivered to the
voter proved only that the encrypted ballot was received as cast, not that
it was counted as cast or that the encrypted vote matched the voter's
intent. Thus, by any of the definitions above, it is not an end-to-end
system.

The Norwegian system used a three-channel model, where the voter receives
authorization codes to cast a ballot via the postal system, then uses a
computer to cast an encrypted ballot, and finally obtains a confirmation
code offering a parital end-to-end proof via an SMS message to the voter's
mobile phone.  In addition, as with RIES, voters could cast multiple
ballots; in Norway, only the last ballot cast was counted, and if a voter
votes both on paper at a polling place and by Internet, the paper ballot
overrides the Internet ballot.

The system evolved significantly between its first use in 2011 and 2013,
with added complexity to attempt to assure voters that their ballots were
stored as cast.  In 2013, the Carter Center mounted a serious effort to
observe the Norwegian system in action.  Their report on the operation of
the system, in practice, and the problems they had observing it offers
useful insight into the administration of E2E systems in general as well as
the particulars of the Norwegian system; see~\cite{carter2013}.

\subsection{Remotegrity~\cite{zagorski2013}}

Remotegrity is a remote coded voting system that additionally uses the
notion of a lock-in to provide additional security properties. Remotegrity
voters get a package in the postal mail. The package contains:
\begin{itemize}
  \item a coded voting ballot (a ballot with a code printed against each
    choice). The code may or may not be covered by a scratch-off field (such
    as is used for lottery tickets)
  \item an authentication card that contains: (a) many authentication codes
    under scratch-off (b) a lock-in code under scratch-off and (c) an
    acknowledgement code.
\end{itemize}
Both cards have serial numbers.

To vote, the voter enters (a) both serial numbers, (b) the codes
corresponding to her choices and (c) an authentication code obtained after
scratching-off a surface chosen at random.

She returns to the election website a few hours later to check if her codes
are correctly represented, and to see if the election authority has posted
her acknowledgement code next to the codes. This indicates to her that the
election officials received valid codes for her ballot.

She scratches-off the lock-in code and posts it on the website. This is her
way of communicating to the election custodians, observers and other voters
that her vote is correctly represented on the website. Among all of the E2E
systems (and approximations to E2E systems) discussed here, this is the
first one that asks the voter to take positive action to confirm that the
vote was correctly posted.

As with RIES, if we assume that there is no communication between the
computer used to print the credentials and the computer used to accumulate
the votes, the latter computer does not know the mapping from codes to
candidates, so the vote is not revealed to the computer. Further, because
the computer does not know a valid code corresponding to another candidate
on the ballot, it cannot change the vote.  Finally, and uniquely, because
the computer does not know the acknowledgement code, its presence on the
election website assures the voter that the election officials received a
valid code for her ballot.

The tally is computed from the codes in a verifiable manner that corresponds
to the coded voting system used.

The voter can be sent two ballots. She can choose one to vote with and one
to audit.

If a jurisdiction is nervous about using the Internet for remote voting,
Remotegrity ballots can be mailed in, and voters can check for their codes
on the election website to be assured that their vote correctly reached
election officials.

Remotegrity was used for absentee voting---and Scantegrity for in-person
voting---by the City of Takoma Park in its 2011 municipal election,
see~\cite{zagorski2013}.

\subsection{Wombat~\cite{rosen2011}}

The Wombat voting system is an in-person voting system where the voter votes
on a touch-screen and obtains a printout of his vote with an encryption of
it. The voter can choose to cast or audit the encrypted vote. If she chooses
to audit the vote, she may check if the vote was correctly encrypted. If she
chooses to cast it, the ciphertext is posted online, and she casts the
unencrypted vote in the ballot box (this may be manually counted) and takes
the ciphertext home. The votes are tallied using a verifiable mixnet.

The system has been used for multiple pilot elections in Israel.

\subsection{DEMOS~\cite{kiayias2014}}

DEMOS is a coded vote system where the voter is given a two-part coded
ballot; she audits one part and uses the other to vote. Associated with each
choice on the ballot is (a) a vote code---the encryption of the vote, which
is entered in the voting machine by the voter and (b) a receipt code which
the voter does not enter, but which is posted online next to the vote code.
The voter can check this to ensure her vote reached the election
authorities. The ballot also has a QR code containing all the information on
the ballot. It can be scanned by the voter if she prefers not to manually
enter the vote code. Once the ballot is entirely represented on the
computer, the voter can then make her choices. Note that if the voter scans
the QR code, the computer knows how she voted. The vote codes represent
homomorphic encryptions of the votes and the verifiable tally is obtained in
a standard manner.

A pilot study of DEMOS was carried out in 2014.

\tododmwit{This section should be filled in with material from
  history.tex and comparative\_analysis.tex, and then history.tex and
  comparative\_analysis.tex should be ``retired'' and their resources
  moved into e2e\_viv\_explained\_resources.}
\section{Limitations of Existing Systems}
