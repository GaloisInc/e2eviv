\chapter{E2E VIV Explained\ifdraft{ (Philip/Daniel) (20\%)}{}}
\label{chapter:e2e_viv_explained}

\section{IV, VIV, E2E}
\section{E2E Election Rituals}
\subsection{Pre-Election Phase}
\subsection{Voting}
\subsection{Post-Election Phase}
\section{Shortcomings and Expectations of E2EVIV}
\subsection{Access to Communication/Internet}
\subsection{Accessibility}
\subsection{Usability}
\section{E2E VIV in Practice}
A number of practical voting systems have been developed based on the
principles of E2E VIV. This section describes several systems that
have been used in a real election or in a pilot.

\subsection{RIES~\cite{hubbers2004}}

RIES, the Rijnland Internet Election System, was developed to support
elections to the Rijnland water management board, supplementing the
system of postal voting used by the water board.  It weakens the
receipt-freeness requirement generally accepted for E2E voting systems
while providing universal verifiability and a degree of individual
verifiability.  These compromises are based on the fact that remote
voting generally, including postal voting, has weak coercion
resistance, and adding universal verifiability and any degree of
individual verifiability is a distinct improvement.

Before the election, credentials are mailed to every voter in the form
of a very long number (encoded in Crockford's Base 32)
(see~\cite{crockford2002}).  The same mailing also includes
instructions.  Voters log into an election web site that includes a
client-side voting application written in Javascript.  The algorithms
used by this script are public, and each voter, havign access to all
of the inputs and outputs, may (in principle) check the computations.
This is weaker than the desired individual verifiability, but
nonetheless, far stronger than conventional voting systems.

The client-side application encrypts the vote by passing the voter
authorization code and the public ID of the candidate through a
one-way function to create the encrypted vote.  The encrypted vote is
then placed on the public bulletin board that serves as a ballot box.
At the close of the polls, the election authority releases the
codebook containing the encryptions of all valid credentials with all
candidate IDs. If a voter discloses her credential or her encrypted
vote, the published codebook may be used to violate ballot
secrecy. The developers of RIES judged this violation to be no more
severe than the threats to ballot secrecy inherent in postal voting.

RIES was used for Rijnland Water Board elections starting in 2004, and
in 2006, it was used to allow expatriate voters to participate in the
Dutch parliamentary elections, see~\cite{gonggrijp2009}. The OSCE sent
an election assessment team to observe the use of the system in 2006.
Their report contains observations of critical security features of
the system that could not be observed; see~\cite{osce2007}.

One of the more important practical contributions of RIES involves the
recognition that, when voter authorizations are distributed by post or
more generally, distributed long in advance of the election, a
mechanism must be provided allowing voters to obtain replacement
credentials, for example, by telephone or e-mail, and a mechanism is
needed to invalidate lost credentials.  Adding these mechanisms adds
significant complexity to the system and is a source of some of the
problems reported in the OSCE report.  A second feature of RIES is
that it allows parallel testing during the election, where
pre-invalidated test ballots are deliberately added to the bulletin
board in order to test the network path from selected internet clients
to the server.  This offers the possibility of detecting a variety of
spoofing and denial of service attacks.

\subsection{Introduction}

RIES, the Rijinland Internet Election System was developed by the
Hoogheemraadschap van Riginland, one of Netherland's regional water
management authorities. RIES is patented by Piet Maclaine Point and
Riginland Water Board~\cite{hubbers2004}. The basic design idea is
derived from the master thesis of Maclaine Point's student, Herman
Robers~\cite{robers1998}. It was first introduced as a solution to the
low turnout rate of Water Board Elections in 2004. About 72,000 online
votes were cast, out of 2.2 million eligible voters.

In November 2006, RIES was used by Dutch voters reside outside of
Netherland, to participate in the Lower House Parliamentary
Elections. Around 20,000 voters voted via Internet, which accounts for
91\% of the total eligible voters~\cite{competence2014}. The source
code of RIES was published on June 2008~\cite{gonggrijp2009}.

One of the main distinguishing features of RIES is that it enables
voter to verify---after the election is closed---that their own votes
have been counted correctly, and that the result of the tally
corresponds to the cast votes.

Nedap ES3B, the most widely adopted voting system in Netherland, was
subjected to some major hacks in fall 2006 that resulted in broad news
coverage and discrediting of internet voting. The hacking of Netpad
led to the failure of full adoption of RIES for the 2008 parliament
election. In addition, a study of RIES' published source code in
2008~\cite{gonggrijp2009} reveals serious security holes that make the
system vulnerable to Cross-Site Scripting, SQL injection and
predictable token generation. Arguably, the system accords insider
election administrator too much power and also does not have any
protections from voter coercion in a family situation.

\subsection{Core Architecture \& Operation}

\subsubsection{Basic Architecture \& Design}

The main voting system is written in Java and
Javascript. Cryptographic mechanisms such as DES, 3DES, DESmac, MDC-2,
RSA and SHA-1 are deployed through the voting cycle. The discussion in
this report is based on RIES 2008. Due to the fact that the majority
of the resource on RIES is written in Dutch, the report is only using
the available English documents for analysis.

\subsubsection{Voting Process}

In general, the voting process consists of the following steps:

\begin{itemize}
\item Before the voting, the administrative agency will use
  crypto-hardware to generate a personal key for each voter. These
  keys are printed on ballots and distributed by mail. Furthermore, it
  will generate ballot collections [RN1] for each voter, combine all
  the ballot collections to a pre-election reference table and then
  publish the table on the Internet.
\item During the voting, voter will use the RIES web interface to
  enter its personal key from the ballots, and then select the
  candidate. When successful, the browser will return a technical vote
  on screen which serves as a receipt. Furthermore, voter should
  destroy his ballot with secret key and store the technical vote for
  future verification. All the technical votes are stored on the
  network server SURFnet.
\item After the election is closed, SURFnet will hand over all
  technical votes to the administrative agency. The administrative
  agency computes a hash of every technical vote, validate it using
  the pre-election reference table and then compute the voting
  outcome. Finally, the voting office will publish the total outcome.
\end{itemize}

\subsection{Security \& Trust}

\subsubsection{Security Overview}

As further illustrated in system detail tables, RIES were subject to
several security related testing before deployment and
evaluation. According to the official website, those testing results
were positive. Without access to those reports, however, these
analysts cannot judge the adequacy of the testing.

In terms of trust structure, a significant amount of trust is placed
on election administrators. For voters vote via Internet, although
verifiability is achieved through the voting cycle, they still need to
trust the administrative party/vendor to handle their personal secret
key securely.

For voters who vote via postal ballots, they need to trust the Postal
Votes Processing Bureau to convert their mail votes to technical votes
correctly because they can't validate what has been done to their
votes. This problem, combining with the large amount of hacking
targeting RIES, lead to the failure of full adoption of RIES for the
2008 parliament election.

In addition, a study of RIES' published source code in
2008~\cite{gonggrijp2009} reveals serious security holes that make the
system vulnerable to Cross-Site Scripting, SQL injection and
predictable token generation. Arguably, the system accords insider
election administrators too much power and it also does not have any
protections from voter coercion in a family situation.

\subsubsection{Malware Resiliency}

This research did not find enough information to conclude whether or
not RIES is resistant to DDoS and Malware attacks at Server
side. Since the RIES system assumes that voter's PC is secure (often a
flawed assumption), we can conclude that it is not resistant to
Malware or `Man-in-the-middle-attack' at Client side.

\subsubsection{Malfeasance by Election Official and Dispute Resolution}

When some disputes arise, an umpire can check and recalculate various
steps in the whole process and pass his/her own judgment. But this
only works for a limited type of disputes~\cite{hubbers2008}.

\subsection{Vote Privacy}

Voter privacy has been a significant area of concern for RIES voting
system. First, vote secrecy is highly depending on the way the
personal keys are handled. Although the administrative party/vendor
that generate the personal keys are required to destroy these keys
post-election, threats like insider activities or malware attacks on
the server jeopardize vote secrecy. In addition, the vote server can
link the originating IP address to the vote that is cast. The lack of
anonymous channel creates another risk for voter
privacy~\cite{hubbers2008}.

\subsection{Auditability}

Each individual voter can verify, with his stored technical vote,
whether his actual vote has been correctly casted. This is achieved by
comparing the hash value on his technical vote with the hash value in
the pre- election reference table.

The tally verification can be done by anyone
interested. Theoretically, interested party can download all technical
votes from the network server provider SURFnet, compute the hash value
for each vote, and then compare the results with the pot-election
reference table. However, this does not enable anyone to verify that
the votes are truly as intended.

\subsection{Testing and Deployments}

According to the official website
\textcolor[rgb]{0.078431375,0.3254902,0.6901961}{www.openries.nl, }a
number of independent organizations have evaluated the RIES voting
system before its deployment:

``Various prominent institutions have tested and positively evaluated
RIES: TNO Human Factos from Soesterberg tested usability of the voting
interface; A team of specialists from Peter Landrocks Cryptomathic (in
Aarhus, Denmark) tested the cryptographic principles; Madison Gurka
from Eindhoven tested the server and network setup and security; A
team under supervision of Bart Jacobs (Radboud University Nijmegen)
did external penetration tests.'' (Originally written in Dutch,
translation cited from~\cite{gonggrijp2009}.)

It appears that scientists as well as independent third parties have
looked into various aspects of the design and security of RIES, both
before and after the deployment. However, most of the testing reports
and scientific works are published within Netherland therefore are
written in Dutch. This certainly creates an obstacle for the project
team to access and interpret those documents.

\subsection{Usability}

As mentioned in point 5, there were usability and accessibility tests
conducted on RIES voting system, but the project team could not find
one addressed to the international audience. Based on the available
resources, we know that RIES can accommodate both Internet voting and
the traditional postal ballots voting. It is accessible to the
disabled community in a sense that people can vote at home via
Internet using their own accessibility technologies.

\subsection{Infrastructure}

Available on request in system detail table.

\subsection{Shortcomings}

Most of the testing reports and scientific works are published within
Netherland therefore are written in Dutch. This certainly creates an
obstacle for the project team to access and interpret those findings.

% TODO: what's going on at the end of this paragraph??
Before the 2008 Word Board elections, the ministry hired Fox-IT to
perform the formal approval of RIES-2008. The company had found very
serious problems with the underlying
cryptography~\cite{gedrojc2008}\footnote{The report is in
  Dutch.}. \cite[p3]{gonggrijp2009}

This problem, combining with the large amount of hacking targeting
RIES, lead to the failure of full adoption of RIES for the 2008
parliament election. In addition, a study of RIES' published source
code in 2008~\cite{gonggrijp2009} reveals serious security holes that
make the system vulnerable to Cross-Site Scripting, SQL injection and
predictable token generation.

Another testing report~\cite{hubbers2004} on RIES also reveals
shortcomings in the following areas:
\begin{enumerate*}
  \item the procedure of voter self-check is quite complicated,
  \item the two- channel (mail and internet) voting makes system less
    transparent,
  \item too much power is given to the election administrator and
    SURFnet,
  \item issues with reference table modification due to ballot revoke,
  \item possibilities of collision hashes, and
  \item voter coercion such as family voting.
\end{enumerate*}

\subsection{Prêt à Voter~\cite{chaum2005}}

The Prêt à Voter system uses two-part paper ballots with the candidate
names on one part and the voting targets plus a ballot ID number on
the other.  Typically, the two parts are printed as a single sheet,
with a perforation along which the sheet can be divided after voting.
The order of the candidate names appears to be random, from the
voters' perspective.  After voting, the half of the ballot containing
candidate names is shredded. The voted half forms the cast ballot, and
the voter may take a copy home.

A curious voter may inspect the public record of any cast ballot by
searching for it by the ballot ID number in the public database of
cast ballots.  That database shows the positions that were marked on
that ballot, but crucially, it does not show the identifying letters
or candidate names.  The voter may therefore verify that the positions
marked at the polling place were correctly recorded by the
jurisdiction, but because the voter only has half of the ballot and
there is no public display of the linking of candidate names to ballot
positions, the voter cannot prove to anyone else how the ballot was
voted.

The key to tabulating the votes is that there is a cryptographically
secure mapping from the ballot serial numbers to the apparent random
order of the candidate voting positions.  The decoding of a cast
ballot to a corresponding plain-text ballot, where the voted positions
and candidate names are in a canonical order (alphabetical order, for
example) is performed in multiple steps, by multiple custodians, for
example, using a mixnet. As with the mixnet, no one, even if colluding
with a subset of the custodians, can connect a decoded ballot to the
corresponding cast ballot (and, hence, to the voter).  Unvoted ballots
may be audited before, during and after the election to ensure that
the decoding of cast ballots is being correctly performed (audited
ballots may not be used for voting purposes, however, because the
connection between the voting positions and candidate names is
publicly revealed during the audit). Randomly selected stages in the
decoding can be challenged to prove the integrity of the count, and
decoded ballots are easily counted.  Anyone may therefore check the
count.

A version of Prêt à Voter is planned for use in a governmental
election by the state of Victoria in Australia in November 2014,
see~\cite{burton2012}.  Previously, an attempt was made to use Prêt à
Voter in a student election at the University of Surrey in February
2007, see~\cite{bismark2007} -- this attempt illustrates many of the
perils of working with an election authority.

\subsection{Punchscan~\cite{popoveniuc2006,popoveniuc2010punchscan}}

The Punchscan system used a two-part paper ballot where the top part
had candidate names and candidate numbers (or letters) and the bottom
part had the numbered (or lettered) voting targets.  The top part had
holes punched in it to expose the voting targets below.  The order of
the voting targets for each race appears random to the voter.  Both
halves of the ballot bear an identical serial number.  After marking
the ballot with a bingo dauber, the two halves are separated.  Prior
to separation, the voter can easily see that the correct target has
been marked.  To vote, the voter separates the two sides.  Either side
can be scanned (since the bingo dauber marked both through the hole
and around it) as the cast ballot. The other side is destroyed. A copy
of the cast side may be retained by the voter.

A curious voter may inspect the public record of any cast ballot
exactly as with Prêt à Voter. It does not matter which half of the
ballot the voter retained, because there is no public display of the
numbers that link candidate names to voting positions; only the
position that was marked is displayed.  Again, individual ballots may
be audited, and the key to tabulating the votes is that there is a
cryptographically secure mapping from the ballot serial numbers to the
apparent random order of the candidate voting positions.

Punchscan was used for the graduate student association elections of
the University of Ottawa in 2007, see~\cite{essex2007}. It is likely
the first end-to-end voting system with ballot privacy used in a
binding election.

\subsection{Scantegrity II~\cite{chaum2008,chaum2009}}

Scantegrity II (Invisible Ink) allows end-to-end cryptographic
verification of optical-scan paper ballots.  Scantegrity ballots may
be fully compatible with conventional optical scan vote tabulation
equipment, but are voted using a marking pen filled with invisible
ink.  When the voter marks a target on the ballot, the ink in the pen
reacts with invisible ink on the paper to disclose a three-letter
alphanumeric code in the marked voting target.  A voter who takes note
of this code and the ballot ID number may check a public bulletin
board to check that the ballot was indeed tabulated.

As with Punchscan and Prêt à Voter, there is a cryptographically
secure mapping from the ballot ID number to the code disclosed when
the voter marks a voting target.  The code is displayed on the public
bulletin board, and public verification of the decryption and tally of
the contents of the bulletin board proceeds in a manner similar to
that of Punchscan for the system fielded in Takoma Park, and in a
somewhat different manner using similar principles for the system
described in~\cite{chaum2009}.

Scantegrity II is the first end-to-end voting system with ballot
privacy to see use in government elections.  It was used in Takoma
Park, Maryland in 2009 and 2011; see~\cite{carback2010}.

\subsection{Helios~\cite{adida2008,adida2009}}

Helios was developed for Web-based Internet voting. Helios provides a
web-browser-based ballot preparation system that can be used to make
choices and encrypt ballots.  After preparing a ballot, the voter has
the option of either casting that ballot or auditing it; this allows
the voter to detect misbehavior on the part of the ballot preparation
software. Voter authentication is not required until after the voter
decides to cast the ballot, so anyone may prepare and audit ballots.
Existing Helios implementations piggyback on existing university and
commercial authentication mechanisms, and also support an internal
login-password authentication mechanism for which credentials can be
distributed by email.

As with the schemes discussed above, all cast ballots are posted in
encrypted form on a public bulletin board so that voters may check
that their ballots have been correctly recorded.  Similarly, after the
polls close, the decryption and vote tally may be checked. Helios uses
homomorphic vote tallying for simplicity, even though a single mixer
was used in its first developments, and mixnets have been used in
several of its forks and descendants; see~\cite{bulens2011}
or~\cite{tsoukalas2013} for instance. A recent variant
by~\cite{cortier2014} proposes an enhanced ballot authentication
mechanism. The STAR-Vote system~\cite{star-vote}, designed for public
elections as a result of an invitation from the Travis County election
administration, also borrows various techniques used in Helios and
VoteBox~\cite{sandler2008}.

Helios was used for the election of a Belgian university president in
March 2009 and by numerous universities and associations since then,
including the ACM and the IACR. The cryptographic protocols it
implements have been subject to considerable analysis.


\subsection{Introduction}

Helios was developed by Ben Adida. During his time at Harvard he
decided to create an open audit voting system that would reside on the
Internet. Building upon the ideas of pioneers such as Josh Benaloh,
Ben Adida was able to develop an open source platform for an
End-to-end Verifiable Voting system.

\subsection{Core Architecture \& Operation}

\subsubsection{Basic Architecture \& Design}

The Helios system is setup with a website, a python infrastructure, a
couple servers, and a built-in encryption system for the votes. The
election administrators can from the website, setup an private or
public election, invite other users to join the election, run the
election, and post the results. The system allows for registration
through Google, or Facebook and an alternative login system is in the
works. All the non-sensitive data is housed on a server owned by Ben
Adida and all the private or sensitive data is held on the voter's
computer.

\subsubsection{Voting Process}

In a private election, the administrator inputs the email addresses of
the voters who will be participating and the system emails the voters
their randomly generated login information and the link to the
Internet based election.

Once on the site, the voter follows the on screen prompts and clicks
tabs that correspond with their election choices, followed by clicking
the next button.

At the end of the prompts is an option to check if their ballot has
been encrypted properly and see if their votes have changed. Following
this option allows the voter to verify that their vote was accurate
but also destroys the ballot and prompts the voter to vote again. On
their second try they can skip the encryption check and click
finish. This will send an email to the voter saying that their choices
have been casted and that the results will be shown at the end of the
election. At any time, a voter can follow their old link and vote; the
new vote will replace the old vote.

\begin{figure}
  \centering \capimg[width=2.6665in,height=2.2146in]{page-049-010.jpg}
  \caption{Initial page where the voter can initiate the voting
    process}
  \label{fig:helios-initial}
\end{figure}

\begin{figure}
  \centering \capimg[width=2.852in,height=2.361in]{page-049-011.jpg}
  \caption{This is the actual voting page where the voter makes
    his/her selections}
  \label{fig:helios-voting}
\end{figure}

\begin{figure}
  \centering \capimg[width=2.8146in,height=2.5083in]{page-049-012.jpg}
  \caption{Page that allows the voter to confirm and encrypt the
    ballot}
  \label{fig:helios-confirm}
\end{figure}

\begin{figure}
  \centering \capimg[width=2.9909in,height=2.3244in]{page-050-013.jpg}
  \caption{This is the encryption page where the system encrypts the
    ballot.}
  \label{fig:helios-encrypt}
\end{figure}

\begin{figure}
  \centering \capimg[width=2.9909in,height=2.2583in]{page-050-014.jpg}
  \caption{Confirmation page that the ballot is ready and the voter is
    issued their own unique signature}
  \label{fig:helios-signature}
\end{figure}

\begin{figure}
  \centering \capimg[width=3.0366in,height=2.25in]{page-050-015.jpg}
  \caption{Ballot audit page, which allows the voter to validate the
    encryption on their ballot}
  \label{fig:helios-audit}
\end{figure}

\begin{figure}
  \centering \capimg[width=2.972in,height=2.4075in]{page-051-016.jpg}
  \caption{Final page where the voter puts in his/her credential to
    submit the vote.}
  \label{fig:helios-submit}
\end{figure}

\subsection{Security}

\subsubsection{Security Overview}

When it comes to security Helios does a decent job but is still plague
by various security problems. Orion from the University of Washington
highlighted a number of weaknesses in his security review blog. ``It
is possible, just after the voter casts his/her ballot, for a corrupt
router to intercept the ballot en route to the Helios server and send
the user a fake Helios server success code, causing the `voting booth'
to immediately display a false success message and clear the ballot
from memory.''~\cite{orion2009}

In this scenario if the user fails to realize that his/her vote has
been erased then their vote would end up not counting which would help
the adversary manipulate the election. ``As it currently exists, if
the election administrator allows Helios to administrate the election
(as it seems they suggest doing), it is possible for a corrupt Helios
server to create new, fake voters and cast ballots on their behalf
without easily being discovered.''~\cite{orion2009} This would allow
for a corrupt server to vote for the desired winner and completely
debunk the voting process.

Also, validation is only carried out by the users, so in the off
chance that no one audits the election, the corrupt servers could in
fact manipulate with being detected. ``As currently implemented, the
election administrator (who has the power to add voters and freeze the
election) is authenticated through Google Accounts. Any vulnerability
in the login (weak password, easily guessed security questions, etc.)
could allow an attacker to end the election prematurely or add
additional voters (potentially multiple accounts for the same
voter).''~\cite{orion2009}

Due to the reliance on the Google account to actually administer the
election, if the administrators account was hijacked then an attacker
could do a number of things to hack the system in their favor. Helios
is also susceptible to every possible Internet attack in their various
forms. Helios does not have advanced or cutting edge means to defend
itself against any one Internet attack.

This means that any system crashing or infrastructure hacking attacks
available can be utilized against Helios. Its' only real defense are
the general defenses seen in most if not all up to date
websites. Helios is patched regularly and is mount on a secure
platform in Heroku but outside of this, fulfills the bare requirements
of being considered secure.

\subsubsection{Trust Structure}

``Helios takes an interesting approach: there is only one trustee, the
Helios server itself. Privacy is guaranteed only if you trust
Helios. Integrity, of course, does not depend on trusting Helios: the
election results can be fully audited even if all administrators -- in
this case the single Helios server -- is corrupt.''~\cite{adida2008}

Helios was constructed in such a way that the only thing a user would
have to trust is themselves and the server. Also, because it has an
open audit system, they could always double-check the results to make
sure that the server hasn't been compromised. The assumption that
comes with these elements is that they have not been hacked and they
are in fact safe. In a situation where malware has overtaken either of
these items of trust, then the system would be undermined. This isn't
a problem unique to Helios, but it is a problem none-the-less.

When it comes to the ballots everything is handled and encrypted by
the system itself so at no part of the process does the LEO or
Election Official touch or handle the ballots. The only things that
they have control over is the keys that decrypt the votes which is
only enabled when the election is over. The decrypt keys themselves
allow for the votes to be posted, the officials get to see the results
the same time the participants see them.

\subsubsection{Dispute Resolution}

Dispute resolution is handled by the creator and administrator Ben
Adida. He has up to this point only had one dispute to resolve and
that was handled quickly and everything was straightened out.

\subsection{Auditability}

``A web-based open-audit voting system. Using a modern web browser,
anyone can set up an election, invite voters to cast a secret ballot,
compute a tally, and generate a validity proof for the entire process.

Helios is deliberately simpler than most complete cryptographic voting
protocols in order to focus on the central property of public
audit-ability.''~\cite{adida2008}

The entire basis of Helios is the ability to audit the election after
the polls are closed. If the election is public, then anyone with
Internet access can check the encrypted votes and verify that the
elections were legitimate.

The problem with Helios is that though users can audit the system, the
process by which one would tends to be complicated to those who are
not computer savvy. As mentioned in the usability portion, studies
have shown that it is hard for common users to comprehend and properly
utilize the auditing tools. This means that the auditing capability is
there but the barriers of use are too high for the users.

\subsection{Norwegian System~\cite{gjosteen2012}}

Between 2011 and 2014, the Norwegian government ran an Internet remote
voting trial.  The system rests on a cryptographic protocol designed
by Scytl, a commercial voting system vendor. Scytl and the Norwegian
government assert that this is an E2E system.  As such, this marks the
entry of efforts (or claims) by commercial voting system vendors to
enable E2E elections.  System descriptions have been incomplete, hence
it has not been possible to tell what properties the system
possesses. However, the following are clear.  The system's claims to
protect voter privacy are weak: ``If the voter's computer and the
return code generator are both honest, the content of the voter's
ballot remains private.''  In addition, the receipt delivered to the
voter proved only that the encrypted ballot was received as cast, not
that it was counted as cast or that the encrypted vote matched the
voter's intent. Thus, by any of the definitions above, it is not an
end-to-end system.

The Norwegian system used a three-channel model, where the voter
receives authorization codes to cast a ballot via the postal system,
then uses a computer to cast an encrypted ballot, and finally obtains
a confirmation code offering a parital end-to-end proof via an SMS
message to the voter's mobile phone.  In addition, as with RIES,
voters could cast multiple ballots; in Norway, only the last ballot
cast was counted, and if a voter votes both on paper at a polling
place and by Internet, the paper ballot overrides the Internet ballot.

The system evolved significantly between its first use in 2011 and
2013, with added complexity to attempt to assure voters that their
ballots were stored as cast.  In 2013, the Carter Center mounted a
serious effort to observe the Norwegian system in action.  Their
report on the operation of the system, in practice, and the problems
they had observing it offers useful insight into the administration of
E2E systems in general as well as the particulars of the Norwegian
system; see~\cite{carter2013}.

\subsection{Remotegrity~\cite{zagorski2013}}

Remotegrity is a remote coded voting system that additionally uses the
notion of a lock-in to provide additional security
properties. Remotegrity voters get a package in the postal mail. The
package contains:
\begin{itemize}
  \item a coded voting ballot (a ballot with a code printed against
    each choice). The code may or may not be covered by a scratch-off
    field (such as is used for lottery tickets)
  \item an authentication card that contains: (a) many authentication
    codes under scratch-off (b) a lock-in code under scratch-off and
    (c) an acknowledgement code.
\end{itemize}
Both cards have serial numbers.

To vote, the voter enters (a) both serial numbers, (b) the codes
corresponding to her choices and (c) an authentication code obtained
after scratching-off a surface chosen at random.

She returns to the election website a few hours later to check if her
codes are correctly represented, and to see if the election authority
has posted her acknowledgement code next to the codes. This indicates
to her that the election officials received valid codes for her
ballot.

She scratches-off the lock-in code and posts it on the website. This
is her way of communicating to the election custodians, observers and
other voters that her vote is correctly represented on the
website. Among all of the E2E systems (and approximations to E2E
systems) discussed here, this is the first one that asks the voter to
take positive action to confirm that the vote was correctly posted.

As with RIES, if we assume that there is no communication between the
computer used to print the credentials and the computer used to
accumulate the votes, the latter computer does not know the mapping
from codes to candidates, so the vote is not revealed to the
computer. Further, because the computer does not know a valid code
corresponding to another candidate on the ballot, it cannot change the
vote.  Finally, and uniquely, because the computer does not know the
acknowledgement code, its presence on the election website assures the
voter that the election officials received a valid code for her
ballot.

The tally is computed from the codes in a verifiable manner that
corresponds to the coded voting system used.

The voter can be sent two ballots. She can choose one to vote with and
one to audit.

If a jurisdiction is nervous about using the Internet for remote
voting, Remotegrity ballots can be mailed in, and voters can check for
their codes on the election website to be assured that their vote
correctly reached election officials.

Remotegrity was used for absentee voting---and Scantegrity for
in-person voting---by the City of Takoma Park in its 2011 municipal
election, see~\cite{zagorski2013}.

\subsection{Introduction}

During the mid-2000s, the remote voting system \textbf{Remotegrity
}was proposed and developed by Filip Zagorski, joined by a team of
other distinguished cryptographers that included Poorvi Vora, Richard
T. Carback, David Chaum, Jeremy Clark \& Aleksander
Essex~\cite{zagorski2013}. Remotegrity is an end-to-end verifiable
(E2EV) absentee voting system, which has been architected to work in
concert with an in-person/precinct voting system called
\textbf{\textit{Scantegrity }}(a paper-based system).

Remotegrity adapts the ``code-voting\footnote{Code voting helps in
  achieving privacy by replacing all the elements on a ballot by
  codes, which are cryptographically generated.}'' approach and
features found in the ``mother-system'' Scantegrity. Using this
approach, Remotegrity seeks to protect voter privacy, as well as
provide resiliency against any malware-induced software modifications,
election official or intruder manipulations of vote tallies, and other
potential injuries to election integrity.

Remotegrity was deployed in Takoma Park, Maryland, in 2011, where
election officials used it in a trial of both Scantegrity \&
Remotegrity systems. In addition, the trial included
\textbf{\textit{ranked-choice voting}}\footnote{``Ranked-choice
  voting'' (also called preferential or ``instant run-off voting''
  (IRV) requires voters to rank their choices among the available
  candidates. While IRV election can be structured with different
  rules (for instance, at least two distinct types of IRV structures
  can be created, including e.g., directing voters to ``rank your top
  three choices of candidates by placing a `1' indicating your top
  choice, and `3' your third choice among the field of 8 candidates,''
  or ``Rank each of the 8 candidates in the order of your preference,
  giving a `1' to your top choice candidate and an `8' to your least
  preferred candidate.'' The virtues and evils of IRV are well beyond
  the scope of this Report. Suffice it to say that IRV presents
  substantial voter education hurdles, and that the test-running of
  Remotegrity in concert with IRV eliminates the ability to draw any
  sound conclusions about the usability or deficits in either
  innovation.} methods. In this report we have reviewed the version of
Remotegrity used in the 2011 Takoma Park local election. This analysis
is based on the publicly available research papers and interviews with
the individual architects and developers.

\subsection{Core Architecture \& Operation}

\subsubsection{Basic Architecture \& Design}

Remotegrity uses both the online component as well as the paper
component from \textbf{\textit{Scantegrity. }}Paper ballots are
printed along with an additional feature called the `Authorization
card'. The Authorization card contains codes which are hidden under a
scratch off and which are used in casting the vote (`Auth Code') and
finalizing the casted vote (`Lock Code').

These two printed components are sent to the registered voters via any
postal service and the online component includes voters accessing the
voting site, casting the vote, checking their vote and finally viewing
the results on a publicly accessible Bulletin Board (which is another
website).

Remotegrity is developed in Java. The databases (for example: MySQL)
can be hosted on a cloud infrastructure. The system requires a
separate offline server that is not connected to the Internet to check
the validity of the voters' submitted vote \& authorization
codes. This server is dedicated to maintaining all the cryptographic
keys and validation signatures. Remotegrity developers architected the
system for high security and data integrity and tested it for security
gaps prior to its deployment in Takoma Park.

Remotegrity is designed to ensure that voters will receive unique
codes for the same candidate\footnote{The system utilizes distributed
  key generators \& pseudo-random number generators for generating the
  codes that are printed on the ballots and authorization
  cards.}. Remotegrity is designed so that an election computer
receiving vote codes is able to check whether a code corresponds to a
valid choice on the ballot, without knowing to which vote it
corresponds. This maintains vote privacy while preventing malware on
the voter's computer from changing the vote. The vote codes are
cryptographically calculated using the keys the election officials
generate via automated processes.

\subsubsection{Voting Process}

Printing Ballots: Before ballots are printed the unique codes for
every candidate are calculated along with the Auth and Lock codes for
each voter. The cryptographic values and their relation to candidates
and voters are generated and stored on an offline validation server.

The ballot paper and the authorization card also contain a `Vote
Serial' and `Ack Code' respectively. These are used to validate that
the codes entered came from a particular ballot and an authorization
card. The ballots and the authorization cards are printed and mailed
to all the voters. A sample ballot and the sample authorization card
are shown in \autoref{fig:remotegrity-ballot} and
\autoref{fig:remotegrity-auth-card}.

\begin{figure}
  \centering \capimg[width=3.25in,height=2.4335in]{page-013-004.jpg}
  \caption{Sample Ballot}
  \label{fig:remotegrity-ballot}
\end{figure}

\begin{figure}
  \centering \capimg[width=3.25in,height=2.4154in]{page-014-005.jpg}
  \caption{Sample Authorization Card}
  \label{fig:remotegrity-auth-card}
\end{figure}

Ballot casting: The voters will use the `Vote Code' as shown in
\autoref{fig:remotegrity-ballot} of their preferred candidate and
enter it on the voting portal. Upon entering their choice, the voter
is asked to enter an `Auth Code', which is under a scratch off on the
authorization card. This `Auth Code' is a one-time password
(single-use password) used to validate the voter.

There are 4 Auth codes given to a voter under a scratch off, which are
helpful in case is the voter needs to attempt to vote again (because
their computer performs a denial of service attack or does not respond
correctly to instructions) or in case there is a dispute over the
values entered.

Once the values are entered, the voter is asked to wait for a few
hours and revisit the election web site, allowing the election system
and officials to verify the validity of the codes entered by the
voter. Upon verification the election system signs the code entries
and displays it back to the user. All communication with the offline
computer is performed over an ``air gap''; that is, a human transfers
the data to and from the offline computer at regular intervals (of
about 3-4 hours for the Takoma Park election). This prevents any
malware from online computers from signing false data. Then the voter
is shown his choice along with the Vote Serial and the Ack Code.

The Ack Code is printed on the authorization card (shown in
\autoref{fig:remotegrity-auth-card}) and is never entered by the
voter, and hence, if it is correct, it could not have been generated
by any entity other than the election computer. The presence of a
correct Ack Code on the election website hence communicates to the
voter that the entered vote codes were valid and not manipulated by
the computer used by the voter for voting.

If the voter is satisfied, he/she has to scratch off one of the `Lock
codes' and enter it to use the value to finalize/freeze the vote. Once
the `Lock Code' is used the vote cannot be
changed. \autoref{fig:remotegrity-vote-code} shows Vote codes being
entered. \autoref{fig:remotegrity-auth-code} shows Auth Code being
entered. \autoref{fig:remotegrity-bb} shows the verification back to
the voter along with the Auth and Ack Code-EB3C15.

\begin{figure}
  \centering \capimg[width=2.6665in,height=1.9689in]{page-015-006.jpg}
  \caption{Vote Codes entered -- 6055 \& 2392}
  \label{fig:remotegrity-vote-code}
\end{figure}

\begin{figure}
  \centering \capimg[width=2.6665in,height=2.061in]{page-015-007.jpg}
  \caption{Auth Code entered - 6969-3738-5597-4072}
  \label{fig:remotegrity-auth-code}
\end{figure}

\begin{figure}
  \centering \capimg[width=2.8571in,height=1.8098in]{page-015-008.jpg}
  \caption{User shown his/her choice along with Auth \& Ack Code on a
    public Bulletin Board}
  \label{fig:remotegrity-bb}
\end{figure}

Results: Once the final tally is done the results are displayed on a
publicly accessible bulletin board. For a particular voter, this will
show his candidate's `Vote Code', the `Vote Serial' of the ballot
paper, the `Auth Code' used, the `Lock Code' used and the `Ack code'
of the authorization card. \autoref{fig:remotegrity-lock} shows the
final vote as seen by the user post the completion of the tally. Here
the voter is shown the Lock code as well as the Vote Serial along with
other codes

\subsection{Security}

\subsubsection{Security Overview}

Remotegrity utilizes the ``code-voting'' approach and features found
in the Scantegrity voting system. It thereby seeks to protect voter
privacy, provides resiliency against any malware-induced software
modifications, election official or intruder manipulations of vote
tallies, and other potential injuries to election integrity. It uses
distributed key generators and pseudo random generators are used to
build a (threshold) shared secret amongst the election officials,
which is then used to generate the codes.

\subsubsection{Malware Resiliency}

In the voting process the values entered by the voter are the Vote,
`Auth \& Lock codes'. The values, which are never entered by the
voter, are Vote Serial and the `Ack Code' (which are present on the
ballot and authorization card respectively, but only displayed
electronically once the vote is casted).

Considering a case where the malware makes changes to the entered Vote
code during ballot casting, this will be rejected by the election
official while the initial verification is performed by the system,
since this vote code will not generate a relation with the
corresponding relation to the Vote serial for this particular
voter. Also, since the vote codes are generated randomly there is low
probability that the malware will be able to guess a correct
alternative code for a particular candidate.

\begin{figure}
  \centering \capimg[width=2.25in,height=1.5764in]{page-016-009.jpg}
  \caption{Final vote confirming the Lock code \& Vote Serial
    (2-456922)}
  \label{fig:remotegrity-lock}
\end{figure}

Considering the malware is able to guess a valid code for a candidate,
it should also know a valid Auth \& Lock Code for confirming the
choice. Since the Auth \& Lock Codes are generated randomly and
sufficient in length, a malware with high probability will not be able
to guess the correct alternative Auth \& Lock Code (which would be
still under a scratch off). Any chances of a malware changing the Auth
\& Lock Code will also be rejected since this code will not generate
the relation with the Ack Code for that particular authorization card
for the voter.

\subsubsection{Malfeasance by Election Official and Dispute Resolution}

The use of Auth codes and the Lock codes under a scratch off makes
sure that there is no malicious activity by an election authority. As
we mentioned earlier an election system/computer has to sign every
verified entry before posting it on bulletin board, this makes sure
that the election official is responsible for the results shown to the
voter. Another feature of Remotegrity requires the use of a new Auth
code every time a signed entry is displayed, thus if an election
official alters the values on the bulletin board after the voter
submits the vote means that they (election authority or computer)
would require a new Auth code. Similarly, if they attempted to lock-in
a vote that the voter did not approve, they would need a Lock code
which was not scratched off by the voter. This makes it easier to
detect any alterations made by an election official and point out the
discrepancy. These properties of the system also allow in maintaining
vote integrity and helps in resolving any disputes.

\subsubsection{Voter Privacy}

Remotegrity utilizes code voting. This is a feature in which the
identities of the voter and the candidates are replaced by
cryptographically generated codes and the relation/binding between
them is retained for verification purpose. In Remotegrity, the
candidates are replaced by `Vote Code' present on the printed ballot
and the validity of the voter is verified by the use of the one-time
password known as the `Auth Code'.

Every voter gets a generated `Vote Serial' and `Ack code,' which
allows the voter to verify that the choice of the candidate entered by
him on the voting system was recorded correctly. Since the final
results are displayed publicly as codes entered by the voter, only the
voter knows his vote/choice and does not make any sense to anyone
else. This feature of having a publicly verifiable election of
Remotegrity and the back end of Scantegrity makes this system
universally verifiable.

Details on \textbf{\textit{Auditability, Trust structure }}and other
aspects can be found in the System detail table.

\subsection{Infrastructure required}

By the election official --
\begin{enumerate}
\item Officials who will generate the master secrets for generating
  the codes
\item Hardware for printing with the capability of having the Auth and
  Lock codes under a scratch off
\item A web server infrastructure in case standalone hosting is done
  or a cloud infrastructure can be used
\end{enumerate}
By the voter --
\begin{enumerate}
  \item A computer or a smart phone with an Internet connection.
\end{enumerate}

\subsection{Shortcomings}

We have seen so far that the Remotegrity voting system which was used
in Takoma Park is an end to end verifiable absentee voting system
which is capable of ensuring the integrity of the votes, results,
privacy of the voters, resiliency to changes made by malware and an
election official, dispute resolution. However, this system is not
designed to be resistant towards voter-coercion. Moreover, like any
other web facing infrastructure this system by itself is not capable
of coping up with any kind of denial of service attack. This risk can
however be reduced if the system is deployed on a cloud infrastructure
which provides very high percentage of up time. Also, in case of a
denial of service attack, if voters are not able to access the
internet; this system can fall back to a regular mail-absentee ballot
system. We also did not have any data on the usability aspects of this
voting system.

\subsection{Wombat~\cite{rosen2011}}

The Wombat voting system is an in-person voting system where the voter
votes on a touch-screen and obtains a printout of his vote with an
encryption of it. The voter can choose to cast or audit the encrypted
vote. If she chooses to audit the vote, she may check if the vote was
correctly encrypted. If she chooses to cast it, the ciphertext is
posted online, and she casts the unencrypted vote in the ballot box
(this may be manually counted) and takes the ciphertext home. The
votes are tallied using a verifiable mixnet.

The system has been used for multiple pilot elections in Israel.

\subsection{DEMOS~\cite{kiayias2014}}

DEMOS is a coded vote system where the voter is given a two-part coded
ballot; she audits one part and uses the other to vote. Associated
with each choice on the ballot is (a) a vote code---the encryption of
the vote, which is entered in the voting machine by the voter and (b)
a receipt code which the voter does not enter, but which is posted
online next to the vote code.  The voter can check this to ensure her
vote reached the election authorities. The ballot also has a QR code
containing all the information on the ballot. It can be scanned by the
voter if she prefers not to manually enter the vote code. Once the
ballot is entirely represented on the computer, the voter can then
make her choices. Note that if the voter scans the QR code, the
computer knows how she voted. The vote codes represent homomorphic
encryptions of the votes and the verifiable tally is obtained in a
standard manner.

A pilot study of DEMOS was carried out in 2014.

\todoacf{This section should be filled in with material from
  history.tex and comparative\_analysis.tex, and then history.tex and
  comparative\_analysis.tex should be ``retired'' and their resources
  moved into e2e\_viv\_explained\_resources.}
\section{Limitations of Existing Systems}
