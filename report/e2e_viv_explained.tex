\chapter{E2E-VIV Explained\ifdraft{ (Philip/Daniel/Adam) (100\%)}{}}
\label{chapter:e2e_viv_explained}

\tododmwit{signposting; some content is here, but we need an intro and
transitions explaining what content is about to happen}

\section{Goals}
Typical Internet voting election processes have six phases:

\begin{description}
  \item[Setup] During the setup phase, the election officials gather the
    information needed to run an election. This includes gathering
    registration information for all voters, identifying the issues and
    races that will be voted on, designing and specializing ballots,
    sending instructions and other information about the election to voters,
    and so on.
  \item[Distribution] Once the election has been set up, election officials
    must distribute ballots to the voters. Different voting system
    architectures use different mechanisms, including postal mail, email, or
    by having voters interact with a website\footnote{We distinguish between
    sending instructions to voters and distributing ballots; there is no
    hard and fast rule for the distinction, but a rule of thumb is that
    instructions are applicable to many voters, whereas anything that has
    been specialized for a single voter is part of the ballot and falls
    under the distribution phase.}.
  \item[Voting] Voters then fill out their ballots, often with the help of
    software installed on their own computers.
  \item[Casting] Completed ballots are then returned to the election
    officials; as with distribution, different architectures use different
    mechanisms.
  \item[Tallying] The tallying phase includes the remainder of the election
    finalization tasks: counting votes and announcing the election outcome
    are common to almost every process, though some include other
    miscellaneous tasks like publishing certain information needed for
    audits.
  \item[Auditing] Some elections will inevitably be disputed; in such cases,
    there is a final phase in which interested parties look for evidence
    that the election outcome is correct (or not!).
\end{description}

One major concern for Internet voting involves ballot integrity during the
distribution, voting, and casting phases. For the election outcome to be
correct, it is important that the ballot that is received by and displayed
to the voter match the ballot that was created and sent by the election
officials; that the computer used to fill out the ballot faithfully reports
the intention of the voter; and that the filled out ballot be received by
the election officials exactly as it was when it was sent by the voter.
Typical Internet communications involve not just the computers owned by the
two parties communicating, but also many intermediary computers controlled
by neither party. A good election system needs to account for this, making
it impossible for these intermediates to intercept ballots for viewing or
modification during transit. 

Another concern is that voters computers are
rarely administered by system administration professionals, and as a result many of them are compromised
by outside forces. One consequence of this is that the voting phase itself
may become corrupted: even if the ballot arrives unchanged to the voter,
malware on the voter's computer may change the way the ballot is displayed
or the way the vote is recorded before casting the ballot. It can be difficult
to design a system that is resistant to this kind of attack without
seriously sacrificing the usability of the system. Some systems use
alternative distribution mechanisms as cross-checks; for example, sending
a code to the voter by postal mail which can be used to check that the
ballot displayed by their computer is correct.

To the extent that it is possible, it is desirable for Internet voting to be
private and anonymous. Voters should feel comfortable voting they way they
like (and not feeling pressured to vote for a particular candidate or to
vote a particular way on some issue); the fewer people who know or can find
out the way a given voter voted, the more comfortable they can feel. On the
other hand, election officials only want to record votes from people who are
registered to vote, and even then want to record only one vote from each
voter. Thus there is a tension during the vote casting phase between
retaining the anonymity of votes and ensuring that a vote is coming from
somebody who ought to be able to vote.
% I keep struggling with the fact that I think vote and ballot are two different things and that you are using them 
% interchangeably. A ballot can have many votes on it... I think people cast "ballots". 
% Is it "vote casting" or as I would say it ballot casting - they vote by casting a ballot,
% election officials record one ballot to a voter, but that ballot could contain
% many votes on it... isn't the ballot tied to the voter, and then the votes to the ballot,
% or maybe, I am using this language wrong - just thought I would note it. 
\tododmwit{ask Joe for a reference to the audit that showed that voting
terminal logs retained exactly who voted for what}
One popular approach to this problem in existing systems is to initially
require each vote to be tied to the voter who cast it long enough to decide
whether to include the vote in the later tally or not; then to keep the vote
but delete the information about who cast the vote. This approach can work;
however, audits of systems that take this approach show that it is all too
easy to accidentally retain the connection between votes and voters longer
than intended, and make this information much more widely visible than
intended. From the privacy side of the tradeoff, it would be better if the
voter could be confident that there was no connection stored because the
information they send to the election officials during the casting phase
does not include any personally identifying material.

There is a subtle distinction being made here. We certainly want our
Internet voting systems to be correct, private, secure, and so forth. It is
important for the people developing these systems to verify that they are
correct and take an active role in seeking out and eliminating defects in
the system. But the goal of verifiable Internet voting is to go even
farther: not just correct, but \emph{visibly} correct. That is, it must be
possible for the parties using the system to be able to \emph{check} that
the system is behaving correctly, without trusting in the abilities of the
people who created the system to avoid bugs or trusting in the inability of
third parties to influence the behavior of the system. As applied to
anonymity: since it is not easy to prove to somebody else that you have
deleted some information, one must simply avoid giving them that information
in the first place. 

This theme---of being not just correct, but
verifiable---is one of the central ones of verifiable Internet voting, and
is a critical part of the defense against the software bugs, security
vulnerabilities, and sophisticated cybercrimes that history tells us are
sure to crop up.\tododmwit{do we need some evidence of security
vulnerabilities and cybercrime\ldots?}

The tallying process provides a particularly good example of the difference
between correctness and verifiability. We certainly want the election system
to count the votes correctly; but the goal of verifiability is to provide
some evidence to voters that the election outcome is correct. For example,
some systems allow voters to check that their vote was included in the
election outcome; some allow voters to check that the system is recording
the content of their votes correctly; some even allow voters to check that
the number of people that voted for a given candidate is accurately
calculated without revealing any of the individual votes. Meeting these
verification goals without violating the anonymity and privacy goals can be
a balancing act.

Each of these individual goals contribute to a single top-level goal:
end-to-end verifiability. The ``end-to-end'' property is that the whole
election process produces a result that matches the intentions of the
voters. The subgoals of this are summarized with the catchphrase, ``Cast as
intended; recorded as cast; and counted as recorded.'' This recapitulates
the concerns discussed above; ``cast as intended'' is the demand that
casting use secure communications and other mechanisms to ensure that
malware and outsiders cannot change the vote, ``recorded as cast'' is the
demand that the election system itself correctly interprets a vote, and
``counted as recorded'' is the demand that the tallying process be faithful;
and all of these demands are subject to not just correctness but
verifiability, so that a voter can convince themselves that these properties
hold even if they suspect that the system or election officials have been
corrupted.

\section{Shortcomings and Expectations of E2E-VIV}

As discussed in \autoref{chapter:remote_voting}, there are several
difficulties with current voting processes: voters with disabilities cannot
vote unassisted, communication channels with remote voters are slow and
unreliable, vote tallying is labor-intensive and error-prone, and
election audits are costly. Additionally, there is little visibility into
the election process, meaning that individual voters and, in some cases,
even auditors, must trust the reports of election officials and voting
hardware vendors on election outcomes and processes. Internet voting may be
able to alleviate some of these concerns. Voters with disabilities could
potentially use their own familiar hardware, such as Braille displays,
screen readers, sip-and-puff input devices, and so on, to participate in the
election. Internet communications are traditionally speedy (seconds per
message rather than weeks) and relatively robust compared to overseas postal
mail. In most systems, tallying is automated and fast. Auditing can still be
a challenge, though there is some hope that verifiable systems can make
elections more transparent for this purpose, too.

There are also some serious challenges in rolling out an Internet voting
system. \autoref{chapter:architecture} discusses the feasibility of
producing a system that meets the security and verifiability goals we have
touched on above. In addition to those concerns, the ability of normal
voters to use the system to cast their vote in the way they intend to vote
is a major goal; as we discuss below, current systems do not meet this goal
very well. One component of this is the system itself; though another that
is common to all Internet voting systems is the need for voters to have
Internet access. This is not necessarily possible for all overseas and
military voters.\tododmwit{would be nice to have some hard data about this}

\section{E2E-VIV in Practice}

A number of practical voting systems have been developed based on the
principles of E2E-VIV. This section describes several systems that
have been used in a real election or in a pilot.

\subsection{RIES}
\label{sec:ries} 

RIES, the Rijnland Internet Election System~\cite{hubbers2004}, was
first used in 2004 to support elections to the Rijnland water
management board, supplementing the system of postal voting used by
the water board. A subsequent version was used to allow expatriate
voters to participate in the Dutch parliamentary elections
\cite{gonggrijp2009}.

Before a RIES election, credentials are mailed to every voter in the
form of a very long number. The same mailing also includes
instructions for the voter.

During the election, voters log into an election web site that
includes a client-side voting application written in JavaScript. The
client-side application encrypts the vote by passing the voter
authorization code and the public ID of the candidate through a
one-way function to create the encrypted vote. The encrypted vote is
then placed on a public bulletin board that serves as a ballot box.

At the close of the polls, the election authority releases the final
vote tallies along with a codebook containing the encryptions of all
valid credentials with all candidate IDs.

The algorithms and protocols used by RIES are public, and each voter,
having access to all of the inputs and outputs, may (in principle)
check the computations. This is weaker than the desired individual
verifiability, but nonetheless, far stronger than conventional voting
systems.

The Organization for Security and Co-operation in Europe (OSCE) sent
an election assessment team to observe the use of RIES in 2006. Their
report contains observations of critical security features of the
system that could not be observed \cite{osce2007}. Further
weaknesses were revealed by the Eindhoven Institute for the Protection
of Systems and Information (EiPSI) in 2008 \cite{hubbers2008},
notably that:

\begin{itemize}
  \item the procedure of voter self-check is quite complicated,
  \item the two-channel (mail and Internet) voting makes system less
    transparent,
  \item too much power is given to the election administrator and the
    system's Internet host,
  \item issues arise when modifying the codebook due to a revoked
    ballot, and
  \item there are realistic ways to forge votes via cryptographic hash
    collisions.
\end{itemize}

One of the more important lessons learned through RIES is that when
voter authorizations are distributed long in advance of the election,
a mechanism must be provided allowing voters to obtain replacement
credentials and invalidate lost credentials. The mechanism adds
significant complexity to the system, and is a source of some of the
problems reported in the OSCE and EiPSI reports.

Another feature of RIES rife with tradeoffs is the ability to perform
testing during the election: pre-invalidated test ballots are
deliberately added to the bulletin board in order to test the network
path from selected Internet clients to the server. While such testing
in principle can increase confidence in the election integrity, in
practice it opens the system to spoofing and denial of service
attacks. Furthermore in the RIES implementation the system is aware of
the fact that it is processing a testing ballot, and all of the test
ballots were voted identically from the same computer, limiting the
confidence added at the expense of these vulnerabilities.

In the wake of these critical reports, plans to use RIES in the 2008
Dutch parliamentary elections were scrapped, and Internet voting as a
whole was banned in the Netherlands.

\subsection{Prêt à Voter}
\label{sec:pret-voter}

The state of Victoria in Australia held a governmental election in
November 2014, using a version of the Prêt à Voter
system~\cite{chaum2005, burton2012}. An attempt was also made to use
Prêt à Voter in a student election at the University of Surrey in
February 2007~\cite{bismark2007}. The failure of this attempt
illustrates many of the pitfalls of adapting a research system to an
actual election, such as a short timetable, a lack of clear
requirements, and the need for rigorous implementation practices.

Prêt à Voter uses two-part paper ballots with the candidate names on
one part and the voting targets plus a ballot ID number or barcode on
the other part. Typically, the two parts are printed as a single sheet
with a perforation to divide the sheet after voting.

From the voter's perspective, the order of the candidate names on the
ballot appears to be random. The voter marks her choice next to the
candidate name of her choice, separates the two parts of the ballot,
and destroys the candidate names. She may take a copy of the voted
part home for later verification.

For tabulation, there is a cryptographically secure mapping from the
ballot ID numbers to the apparent random order of the candidate voting
positions. Multiple custodians using a mixnet or similar technique use
this mapping to decode cast ballots into anonymized plain-text ballots
which are then posted to a bulletin board.

Unvoted ballots may be audited before, during and after the election
to ensure that the decoding of cast ballots is being correctly
performed. Randomly selected stages in the decoding can be challenged
to prove the integrity of the count, and the plain-text decoded
ballots are easily counted for verification by any interested party.

An individual voter may also search for their voted ballot ID on the
bulletin board. This reveals the positions that were marked on that
ballot, but crucially, it does not show the corresponding candidate
names. The voter may therefore verify that the positions marked at the
polling place were correctly recorded by the election officials, but
because the voter no longer has the part of the ballot linking
candidate names to ballot positions, the voter cannot prove to anyone
else how the ballot was voted.

\subsection{Punchscan}
\label{sec:punchscan}

Punchscan~\cite{popoveniuc2006,popoveniuc2010punchscan} was used for
the graduate student association elections of the University of Ottawa
in 2007~\cite{essex2007}. It is likely the first E2E voting system
with ballot privacy used in a binding election.

The election experience for a Punchscan voter is very similar to that
of Prêt à Voter. The system uses a two-part paper ballot where the top
part has candidate names and candidate numbers (or letters) and the
bottom part has numbered (or lettered) voting targets. Holes punched
in the top part expose the voting targets below. The order of the
voting targets for each race appears random to the voter. Both halves
of the ballot bear an identical serial number.

The voter casts their vote by marking her choice with a bingo dauber,
and the two halves are separated. Either side can be scanned (since
the bingo dauber marked both through the hole and around it) as the
cast ballot. The other side is destroyed, and a copy of the cast side
may be retained by the voter.

A curious voter may inspect the public record of any cast ballot
exactly as with Prêt à Voter. It does not matter which half of the
ballot the voter retained; neither half contains the information necessary to determine the vote. Again, individual ballots may
be audited, and the key to tabulating the votes is that there is a
cryptographically secure mapping from the ballot serial numbers to the
apparent random order of the candidate voting positions.

\subsection{Scantegrity II}
\label{sec:scantegrity-ii}

Scantegrity II (Invisible Ink)~\cite{chaum2008,chaum2009} was used in
the Takoma Park, Maryland municipal elections in
2009~\cite{carback2010}. In 2011, it was used for in-person voting
with Remotegrity (\autoref{sec:remotegrity}) used for absentee voting. The
2009 Takoma Park election was the first use of an E2E system with
ballot privacy in binding governmental elections.

Before the election, officials generate the seed to a pseudorandom
number generator using a secret sharing scheme. Three-letter
alphanumeric codes are created for each choice on each printed ballot
using this seed, and additional tables are created so that interested
parties can later confirm that the tally was computed correctly.

During the election, the voter experience is nearly identical to that
of conventional optical-scan paper ballots. When the voter marks their
choice, the ink in the pen reacts with invisible ink on the paper to
disclose the three-letter code in the marked voting target. The ballot
ID number and the displayed code are posted to a public bulletin
board.

After the election, public verification of the final tally proceeds
with the public bulletin board in a manner similar to that of
Punchscan and Prêt à Voter.

In addition to the public verification, an individual voter who takes
note of their ballot ID number and the code revealed from invisible
ink may use the public bulletin board to check that their ballot was
indeed tabulated, though this information is not sufficient to prove
that they voted a particular way.

\subsection{Remotegrity}
\label{sec:remotegrity}
Remotegrity~\cite{zagorski2013} is a remote coded voting system that
was used for absentee voting alongside Scantegrity
(\autoref{sec:scantegrity-ii}) for in-person voting for the 2011
Takoma Park, Maryland municipal elections.

Remotegrity voters receive a coded voting ballot and an authentication
card in the mail. The codes on the ballot can be covered by a
lottery-style scratch-off field. The authentication card contains
several authentication codes under scratch-off, a lock-in code under
scratch-off, and an acknowledgment code. Both cards have serial
numbers. The voter can be sent two ballots so that she can use one for
auditing purposes.

To vote, the voter enters both serial numbers, the codes corresponding
to her choices, and an authentication code obtained after
scratching-off a surface chosen at random.

She returns to the election website a few hours later to check if her
codes are correctly represented, and to see if the election authority
has posted her acknowledgment code next to the codes. This indicates
to her that the election officials received valid codes for her
ballot.

She scratches off the lock-in code and posts it on the website. This
affirms to the election officials, observers and other voters that her
vote is correctly represented on the website.

Among all of the systems discussed here, this is the first one that
asks the voter to take positive action to confirm that the vote was
correctly posted.

As with RIES, if we assume that there is no communication between the
computer used to print the credentials and the computer used to
collect the votes, the latter computer does not know the mapping
from codes to candidates, so the vote is not revealed to the
computer. Further, because the computer does not know a valid code
corresponding to another candidate on the ballot, it cannot change the
vote. Finally, and uniquely, because the computer does not know the
acknowledgment code, its presence on the election website assures the
voter that the election officials received a valid code for her
ballot.

The tally is computed from the codes in a verifiable manner that
corresponds to the coded voting system used.

If a jurisdiction is nervous about using the Internet for remote
voting, Remotegrity ballots can be mailed in, and voters can check for
their codes on the election website to be assured that their vote
correctly reached election officials.

\subsection{Helios}

Helios~\cite{adida2008,adida2009} is a system developed for web-based
Internet voting. It was used for the election of a Belgian university
president in March 2009 and by numerous universities and associations
since then, including the Association for Computing Machinery and the
International Association for Cryptologic Research.

Before a Helios election, officials input the email addresses of the
voters who will be participating. The system emails the voters their
randomly-generated login information and the link to the election
website.

During the election, the voter enters their choices on the
website. After entering her choices, the voter has an option to spoil
their ballot in order to verify that it was recorded correctly. Upon
completing a non-spoiled ballot, the system sends an email confirming
the receipt of their vote, though not their choices. At any time
before the close of the election, the voter can repeat these steps and
the new vote will replace the old vote.

After the election, Helios uses homomorphic vote tallying with the
optional addition of mixers and mixnets in some
derivatives~\cite{bulens2011,tsoukalas2013}.

Voter authentication is not required until after the voter decides to
cast the ballot, so any interested party may prepare and audit
ballots. All cast ballots are posted in encrypted form on a public
bulletin board so that voters may check that their ballots have been
correctly recorded. Similarly, after the polls close, the decryption
and vote tally may be checked.

\subsection{Norwegian System}

Between 2011 and 2014, the Norwegian government ran an Internet remote
voting trial~\cite{gjosteen2012} using a cryptographic protocol
designed by Scytl, a commercial voting system vendor. 
% If the claim is
% accurate, this is among the first efforts by commercial voting system vendors to
% enable E2E elections. (Note that Remotegrity has also been commercialized and it was used in 2011). 

The Norwegian system uses a three-channel model involving postal mail,
the Internet, and SMS text messaging. Before the election, the voter
receives authorization codes to cast a ballot via postal mail.

During the election, the voter uses a computer to cast an encrypted
ballot. The voter can cast multiple ballots; only the last ballot cast
is counted, and if a voter votes both on paper at a polling place and
by Internet, the paper ballot overrides the Internet ballot. After
casting a ballot, the voter receives a confirmation code offering a
partial end-to-end proof via an SMS message.

Available descriptions of the Norwegian system are incomplete, so it
is not possible to analyze the system in depth. However the system's
claims to protect voter privacy are weak: ``If the voter's computer
and the return code generator are both honest, the content of the
voter's ballot remains private.'' In addition, the receipt delivered
to the voter proves only that the encrypted ballot was received as
cast, not that it was counted as cast or that the encrypted vote
matches the voter's intent.

The system evolved significantly between its first use in 2011 and
2013, with added complexity to attempt to assure voters that their
ballots were stored as cast. In 2013, the Carter Center mounted a
serious effort to observe the Norwegian system in action. Their
report on the operation of the system and the problems
they had observing it offers useful insight into the administration of
E2E systems in general as well as the particulars of the Norwegian
system~\cite{carter2013}.

Scytl and the
Norwegian government assert that this is an E2E system. However, the paper says that, if the voter's encryption software colludes with the return code generator (both presumably run on software written by Scytl), it can lead the voter into believing her vote was cast accurately even when it is not. Thus its property of voter-verifiability relies on voting system software behaving honestly. Additionally, the system does not provide a proof of the tally and hence the tally is not universally verifiable. Finally, there is no information that enables a voter to know which of her votes was counted (if she casts multiple vote to ward off coercion). Because the correctness of the tally is not possible without the correct votes being included in the count, the voting public also does not have the information to determine that only one vote was counted for each voter. For all these reasons, Scytl as used in the Norwegian elections is not considered an E2E system. 

\subsection{Wombat}

The Wombat voting system~\cite{rosen2011} has been used for multiple
pilot elections in Israel. It is an in-person voting system where the
voter votes on a touch-screen and obtains a printout of her vote with
an encryption of it. The voter can choose to cast or audit the
encrypted vote. If she chooses to audit the vote, she may check if the
vote was correctly encrypted. If she chooses to cast it, the
ciphertext is posted online, and she casts the unencrypted vote in the
ballot box (this may be manually counted) and takes the ciphertext
home. The votes are tallied using a verifiable mixnet.

\subsection{DEMOS}

DEMOS~\cite{kiayias2014} is a coded vote system where the voter is
given a two-part coded ballot; she audits one part and uses the other
to vote. Associated with each choice on the ballot is a vote
code---the encryption of the vote, which is entered in the voting
machine by the voter, and a receipt code which the voter does not
enter, but which is posted online next to the vote code.

The voter can check the receipt to ensure her vote reached the
election authorities. The ballot also has a QR code containing all the
information on the ballot which can be scanned by the voter if she
prefers not to manually enter the vote code. Once the ballot is
entirely represented on the computer, the voter can then make her
choices. Note that if the voter scans the QR code, the scanning
computer knows how she voted. The vote codes represent homomorphic
encryptions of the votes and the verifiable tally is obtained in a
standard manner.

A pilot study of DEMOS was carried out during the 2014 European
Elections in Greece.

\section{Limitations of Existing Systems}
\label{sec:limit-exist-syst}

E2E systems inherit many of the limitations of traditional voting
systems. Reliability of equipment, reliance on procedure, trust in
insiders, and accessibility are all problems with traditional
in-person voting systems. For remote systems, the integrity of postal
systems, turnaround time for mailed materials, access to Internet or
fax technology, and reliability of Internet servers are all
well-documented obstacles to voting. \todoacf{reference other sections
  about this}

Existing E2E systems mitigate some of these limitations. For example,
code voting limits the ability for attacks against postal mail systems
to change the candidates marked on voted ballots. However if an
attacker simply intercepts and destroys the voted ballot, a
replacement might not arrive in time for that voter to participate in
the election. To mitigate this, election officials might choose to
instead accept voted ballots via fax, email, or website, but such
expedient measures often trade off the verifiability that makes an E2E
system desirable in the first place.

In this section, we examine the limitations of E2E systems with a
particular focus on the limitations that are unique to or exacerbated
by E2E characteristics.

\subsection{Voter Secrecy}

Systems like Prêt à Voter (\autoref{sec:pret-voter}) and Punchscan
(\autoref{sec:punchscan}) rely on a randomized candidate order or a code
on printed ballots to ensure voter secrecy. Voted ballots must appear
on a public bulletin board in order to verify the election results,
and so to protect secrecy only the selected position or code is
visible on the final ballot along with a ballot ID.

If an insider is able to review the printed ballots before the
election, they can record how the candidate positions are arranged for
each ballot ID and therefore identify which candidate is marked on the
voted ballots, thus violating secrecy~\cite{burton2012}.

Recent writing on Prêt à Voter recommends printing ballots on demand
at polling places in order to limit this
possibility~\cite{ryan2009}. Printing on demand introduces additional
problems and expense compared to centralized printing. More printing
equipment is required at each polling place, that equipment can break
or be difficult to operate, and the printing equipment must have some
way of communicating with the rest of the election infrastructure to
ensure it has, for example, the correct cryptographic seeds for
generating new ballots.

Scantegrity II (\autoref{sec:scantegrity-ii}) uses invisible ink to hide
the vote codes on unvoted ballots, and Remotegrity
(\autoref{sec:remotegrity}) can use scratch-off fields to hide vote codes
and other information required to cast a ballot. These techniques
limit the opportunity for insiders to learn secrecy-compromising
information without being detected through the presence of a marked or
damaged ballot.

Even with techniques to mitigate insider foreknowledge of the ballots,
secrecy still can depend on voters and poll workers correctly
following procedures. A voter can leave the polling place with a
complete Prêt à Voter ballot, for example, failing to shred the half
with the candidate order. With both halves of their ballot, they can
prove how they voted, losing receipt-freedom.

RIES (\autoref{sec:ries}) makes a deliberate secrecy tradeoff by
weakening the receipt-freeness requirement in exchange for providing
universal verifiability and a degree of individual verifiability. The
results of an entire election can be independently audited with only
the information publicly available after the election. However if a
voter discloses her credential or her encrypted vote, the same public
information may be used to violate ballot secrecy. The developers of
RIES judged this violation to be no more severe than the threats to
ballot secrecy inherent in postal voting, and therefore worth
accepting for the benefit to verifiability.

\subsection{Ballot Stuffing}

As when ensuring voter secrecy, many E2E systems depend on correct
procedures to defend against ballot stuffing. For example, during the
University of Ottawa elections using Punchscan, more ballots were cast
than voters recorded in the pollbook. In this case, ballot stuffing
can be caught after the fact by poll workers, but is not an inherently
verifiable property of the system, and requires trust in the accuracy
of the poll workers.

In the Helios system, officials can enter voters by email address, and
so there is limited protection against insider ballot stuffing. Helios
relies on individual voters verifying their votes. While an interested party may verify the tally for the entire election by checking that the collection of encrypted votes is counted correctly,
there is no provision for the interested party to determine if votes were fraudulently cast on behalf of voters who had not voted~\cite{orion2009}.

A pre-election step that publicly publishes tables of valid ballot IDs
can help mitigate this problem, but also creates others. All votes in
the final tally have an (anonymized) provenance that can be traced
back to before the election began and presumably cross-checked with
voter registration rolls. However having a fixed set of ballot IDs can
make it harder to replace lost, stolen, or spoiled ballots, or to
providing for late or same-day voter registration.

\subsection{Dispute Resolution}

Note that E2E systems provide voter verifiability: they enable a voter to determine if her vote was accurately recorded. However, in the instance when the vote is not accurately recorded, not all E2E systems provide the voter with evidence that she may use to convince an independent party of the problem. This presents a loophole that may be exploited by dishonest voters, who may claim that their vote is not accurately recorded when it is, calling into question an honest election. 

An E2E system with the dispute resolution property enables a voter to present evidence to support claims of election fraud. As a result, it enables a third party to resolve a dispute between a voter (claiming her vote is inaccurately recorded) and the voting system (claiming it is accurate). 

All cryptographic protocols in the literature that provide dispute resolution require either paper or a second channel (such as a smartphone) in addition to the machine the voter votes from, whether in a polling booth or remotely. (We consider only protocols intended for use by voters voting from machines that are not trusted). 

This implies that any existing voting system with dispute resolution needs the use of a second electronic channel or paper.  

\subsection{Infrastructure \& Equipment}

Election equipment fails in practice. An E2E system must be resilient
to failures while not giving up E2E properties. A system that lacks
robust fallback mechanisms is not itself robust, but is only as strong
as its weakest fallback. For example, if a remote voting website fails
and election officials resort to accepting voted ballots by email, E2E
guarantees are lost for all of the emailed ballots.

In addition to being more sensitive to failures, verifiable election
systems often require more sophisticated equipment than traditional
systems. For in-person voting, a verifiable system might require
ballots to be printed on demand, a high-quality shredder for two-part
ballots, and more sophisticated assistive devices. This complexity
incurs additional cost and poll worker training requirements.

Many E2E systems post encrypted ballots during an election to a public
bulletin board. In order to update the bulletin board in real time,
these election systems are distributed systems, networked via
traditional means or via a manual air gap. Depending on the networking
scheme this can open equipment to distributed denial of service (DDoS)
attacks, network partitions, inconsistency, and other problems
inherent to distributed systems.

Internet systems compound the difficulties of distributed systems by
requiring the systems to be accessible via the public Internet,
increasing the possibilities for DDoS and other malicious
attacks. Furthermore, many systems allow voters to use their own
computers to vote, leading to pitfalls inherent when election
officials lack control over the voting environment. Malware on the
voter's computer might undermine security, incompatibilites might
arise due to operating systems or web browser versions, and the
network infrastructure between the voter and the central election
system might be compromised with a man-in-the-middle attack.

\subsection{Usability}

Traditional election systems struggle with usability. Most often, however, the problems can be addressed with special attention to user interface design. Verifiable
systems add more steps and complexity, presenting usability complications that have not been fully studied. The mechanics of marking a ballot become more complex with
code voting as in Remotegrity, and position or shape matching as in
Prêt à Voter and Punchscan. Individual verification, not even possible
in traditional systems, is an entirely new process that voters must
master to take full advantage of E2E guarantees. Many E2E systems have attempted to reduce the additional effort for voters who are not interested in participating in election verifiability. 

In 2014, a team of researchers from Rice University undertook a
quantitative, experimental study of the usability of Helios, Prêt à
Voter, and Scantegrity II~\cite{acemyan2014usability}. They aimed to
quantify usability using the ISO 9241-11 standard axes of
effectiveness, efficiency, and satisfaction. Their results show that
these systems broadly fail on these axes even for typical voters who
are uninterested in performing additional verification steps.

The Rice study found the systems were not effective as significant
number of voters failed to cast a ballot with each
system. Troublingly, many of those voters thought they had in fact
successfully cast a ballot; in a real election they would have left
the voting process unfinished without even knowing to ask a poll
worker for assistance. By contrast, traditional systems have
near-100\% success rates~\cite{byrne2007usability}.

The systems also lacked efficiency, as they all required significantly
more time -- almost twice as long -- to complete as a traditional
system.

The usability of an election system is crucial for that system to not
disenfranchise voters, and for voters to generally have confidence in
the election results. The Rice study shows that adding E2E guarantees
can be a Pyhrric victory when the resulting system is unusable for
non-expert voters.

The results of the Rice study have been challenged by McBurnett et al~\cite{mcburnett2014}. 
\subsection{Accessibility}

There are ability requirements for many E2E systems in various stages
of the voting process. For example, a sighted voter is able to see the
correspondence between candidate position and marking position on a
Punchscan ballot, but a non-sighted voter cannot without
assistance. In addition to obstacles to marking a ballot, some schemes with
individual verification lack provisions for disabled voters to
participate in individual verification without assistance. Information
required for verification is frequently delivered through a paper
receipt, an invisible ink code, or requires writing down receipt
data.

Accessible verification protocols have been proposed that take care to
protect voter secrecy and allow participation in individual
verification~\cite{chaum2009accessible}. However, these protocols
require using accessibility equipment with an audio, sip-puff, or
switch interface to read and mark the unencrypted ballot. The device
must therefore be trusted not to record the votes, which would violate
voter secrecy. The device must also represent the ballot faithfully to
the voter so that votes are recorded as intended.

Requiring trust in assistive devices is not unique to E2E
systems~\cite{runyan2007improving}. In non-E2E systems, though, voters are required to trust many aspects of the election. In the context of having to trust the
chain of custody of ballots, the integrity of poll workers, and the
outcomes of any audits, having to trust an assistive device is a
relatively small concession to make in an already-flawed system.

On the other hand, a well-designed E2E system requires a much smaller
base of trust for voters to have confidence in the results of an
election. The additional requirement of trusting an assistive device is not a flaw of the E2E system, but, instead, a limitation inherent to the use of assistive devices. It may be mitigated by enabling the voter with the need for assistance to use multiple assistive devices, and by requiring that all voters, whether in need of assistance or not, use similar assistive devices or a universal interface. This ensures that the assistive device and/or interface is tested by multiple users. 
% By requiring an expanded base of trust in order to be
% accessible, the existing E2E systems undermine their E2E properties.

\subsection{Social \& Political}

Novel election systems face a difficult bootstrapping problem: in
order to be adopted in large-scale elections, they must have a
successful track record. However in order to build up that track
record, systems must be successful despite the limited resources
available during small-scale pilot programs. With limited resources,
corners are cut in the implementation of the election system leading
to a greater chance that problems with equipment, software, and
support will undermine confidence in the system.

This confidence in election systems generally, and E2E systems in
particular, is fragile in the eyes of the public. When election
systems fail during an election or are revealed to have substantial
integrity issues, the perception of all similar systems is tainted, no
matter the differences between specific systems or the reassurance of
E2E guarantees. Failure of a legacy computerized system can poison the
well and make the public reject a novel system by association.

For example, The Federal Constitutional Court of Germany issued a
decision in 2009 in the wake of a hacking demonstration on electronic
voting machines used in previous elections~\cite{germany2009decision}.
They decided that electronic systems may only be used in elections if
``the result can be examined reliably and without any specialist
knowledge of the subject'', a standard which E2E systems have not been
able to meet in practice~\cite{byrne2007usability}. Similarly after
reports critical of RIES, a popular movement successfully advocated
for a ban on Internet voting in the Netherlands.

Broader computer security concerns are becoming topics of household
conversation with vulnerabilities like Heartbleed and droves of
personal data compromises making the headlines. These concerns rightly
make the public wary of any system with a computerized component, even
if the Internet is not involved. The challenge for E2E systems is to
overcome this broader skepticism by demonstrating integrity in a way
accessible to non-experts without making it more difficult to vote.

%%% Local Variables:
%%% mode: latex
%%% TeX-master: "report"
%%% End:
