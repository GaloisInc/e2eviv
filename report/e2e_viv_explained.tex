\chapter{E2E VIV Explained\ifdraft{ (Philip/Daniel/Adam) (25\%)}{}}
\label{chapter:e2e_viv_explained}

As discussed in \autoref{chapter:remote_voting}, there are several
difficulties with current voting processes: voters with disabilities cannot
vote unassisted, communication channels with remote voters are slow and
unreliable, vote tallying is labor-intensive and error-prone, and
election audits are costly. Additionally, there is little visibility into
the election process, meaning that individual voters and, in some cases,
even auditors, must trust the reports of election officials and voting
hardware vendors on election outcomes and processes. Election officials are
naturally seeking technology that can mitigate these problems, such as
automated, computer-based vote tallying to reduce tallying and auditing
costs, computerized ballot completion with accessibility technology to
assist disabled voters, Internet-based vote submission to increase the speed
and reliability of communicating with remote voters, and cryptographic
techniques for providing visibility without violating voter privacy. Below,
we give an overview of the goals of these technologies, discuss their
success at these goals, and highlight some common pitfalls.

\section{IV, VIV, E2E}

The core idea of Internet voting is deceptively simple: take a system that
allows remote voting via postal mail---which is already done in many
places---and replace some or all uses of postal mail in that process with
Internet-based delivery mechanisms. The draw of this idea is clear: messages
can be sent between election officials and voters in seconds, no matter how
distant, instead of weeks, and messages in typical Internet communication
methods are rarely lost, while up to half of overseas postal messages get
lost.

The general process is something like this:

\newcommand\stepref[1]{\hyperref[#1]{step~\ref*{#1}}}
\begin{enumerate}
  \item\label{enum:setup} Well before the election, officials compile a list
    of registered voters. They use this to populate a central database with
    contact information, details about what each voter's ballot should
    contain, and so forth. General instructions and other information about
    the election may be broadcast at or near this time---for example, via a
    central website.
  \item\label{enum:share secrets} Election officials contact each voter
    individually to set up some shared, but secret, information specific to
    that voter. For example, this might include an empty ballot with secret
    numbers associated with each candidate; or a cryptographic key-pair that
    the voter can use during communications with the officials. This is
    sometimes done by traditional means (like postal mail) and sometimes not
    done at all.
  \item\label{enum:send blank ballot} Once a means of communication is
    established in \stepref{enum:share secrets}, the election officials send
    the voter a blank ballot.
  \item\label{enum:complete ballot} The voter fills out the blank ballot on
    their computer, perhaps using custom software.
  \item\label{enum:send completed ballot} Again using the communication
    method established in \stepref{enum:share secrets}, the voter sends the
    completed ballot back to the election officials. Many systems have an
    additional step in which the election officials confirm receipt of the
    ballot.
  \item\label{enum:tally} After the election is over, all ballots sent in
    are tallied, and the outcome of the election is announced. In some
    systems, other supplemental information is often announced, such as
    whose votes were counted.
\end{enumerate}

Unfortunately, simply switching to Internet-based communication does not
solve all the problems discussed above, and introduces new problems of its
own.

\tododmwit{expand each of these}
\begin{description}
  \item[Secure transport] For the integrity of the vote, it is important
    that the empty ballot sent in \stepref{enum:send blank ballot} and the
    completed ballot sent in \stepref{enum:send completed ballot} arrive at
    their destination unmodified. The Internet includes a lot of hardware
    that is controlled by neither the election officials nor the voter, so
    ensuring this property can be quite a challenge. Some systems use a
    different communication channel in \stepref{enum:share secrets} to give
    greater confidence in this property, using information communicated by
    mail to cross-check information communicated over the Internet.
  \item[Private transport] Normally, votes are anonymous---there is no
    connection between a cast vote and the person who cast it. However,
    extant Internet communication protocols cannot support this privacy
    guarantee (and vote-by-mail systems typically sacrifice this property as
    well).
  \item[Attack surfaces] The voter is presumably using his own computer.
    It is possible that his computer has been taken over by somebody else.
    This kind of problem is completely fresh compared to paper voting.
  \item[Verifiability] Voters should be able to independently verify that
    their votes were cast and counted correctly, without needing to place
    trust in the communication medium or the election officials. Auditors
    should be able to check that the election was executed correctly without
    trusting the voting hardware and software vendors. Paper ballots support
    auditing well, but make individual voter confidence difficult.
\end{description}

\section{E2E Election Rituals}
\subsection{Pre-Election Phase}
\subsection{Voting}
\subsection{Post-Election Phase}
\section{Shortcomings and Expectations of E2EVIV}
\subsection{Access to Communication/Internet}
\subsection{Accessibility}
\subsection{Usability}
\section{E2E VIV in Practice}
\tododmwit{This section should be filled in with material from
  history.tex and comparative\_analysis.tex, and then history.tex and
  comparative\_analysis.tex should be ``retired'' and their resources
  moved into e2e\_viv\_explained\_resources.}
\section{Limitations of Existing Systems}
