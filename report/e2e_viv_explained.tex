\chapter{E2E VIV Explained\ifdraft{ (Philip/Daniel/Adam) (45\%)}{}}
\label{chapter:e2e_viv_explained}

As discussed in \autoref{chapter:remote_voting}, there are several
difficulties with current voting processes: voters with disabilities cannot
vote unassisted, communication channels with remote voters are slow and
unreliable, vote tallying is labor-intensive and error-prone, and
election audits are costly. Additionally, there is little visibility into
the election process, meaning that individual voters and, in some cases,
even auditors, must trust the reports of election officials and voting
hardware vendors on election outcomes and processes. Election officials are
naturally seeking technology that can mitigate these problems, such as
automated, computer-based vote tallying to reduce tallying and auditing
costs, computerized ballot completion with accessibility technology to
assist disabled voters, Internet-based vote submission to increase the speed
and reliability of communicating with remote voters, and cryptographic
techniques for providing visibility without violating voter privacy. Below,
we give an overview of the goals of these technologies, discuss their
success at these goals, and highlight some common pitfalls.

\section{IV, VIV, E2E}

The core idea of Internet voting is deceptively simple: take a system that
allows remote voting via postal mail---which is already done in many
places---and replace some or all uses of postal mail in that process with
Internet-based delivery mechanisms. The draw of this idea is clear: messages
can be sent between election officials and voters in seconds, no matter how
distant, instead of weeks, and messages in typical Internet communication
methods are rarely lost, while up to half of overseas postal messages get
lost.

The general process is something like this:

\newcommand\stepref[1]{\hyperref[#1]{step~\ref*{#1}}}
\begin{enumerate}
  \item\label{enum:setup} Well before the election, officials compile a list
    of registered voters. They use this to populate a central database with
    contact information, details about what each voter's ballot should
    contain, and so forth. General instructions and other information about
    the election may be broadcast at or near this time---for example, via a
    central website.
  \item\label{enum:share secrets} Election officials contact each voter
    individually to set up some shared, but secret, information specific to
    that voter. For example, this might include an empty ballot with secret
    numbers associated with each candidate; or a cryptographic key-pair that
    the voter can use during communications with the officials. This is
    sometimes done by traditional means (like postal mail) and sometimes not
    done at all.
  \item\label{enum:send blank ballot} Once a means of communication is
    established in \stepref{enum:share secrets}, the election officials send
    the voter a blank ballot.
  \item\label{enum:complete ballot} The voter fills out the blank ballot on
    their computer, perhaps using custom software.
  \item\label{enum:send completed ballot} Again using the communication
    method established in \stepref{enum:share secrets}, the voter sends the
    completed ballot back to the election officials. Many systems have an
    additional step in which the election officials confirm receipt of the
    ballot.
  \item\label{enum:tally} After the election is over, all ballots sent in
    are tallied, and the outcome of the election is announced. In some
    systems, other supplemental information is often announced, such as
    whose votes were counted.
\end{enumerate}

Unfortunately, simply switching to Internet-based communication does not
solve all the problems discussed above, and introduces new problems of its
own.

\tododmwit{expand each of these}
\begin{description}
  \item[Secure transport] For the integrity of the vote, it is important
    that the empty ballot sent in \stepref{enum:send blank ballot} and the
    completed ballot sent in \stepref{enum:send completed ballot} arrive at
    their destination unmodified. The Internet includes a lot of hardware
    that is controlled by neither the election officials nor the voter, so
    ensuring this property can be quite a challenge. Some systems use a
    different communication channel in \stepref{enum:share secrets} to give
    greater confidence in this property, using information communicated by
    mail to cross-check information communicated over the Internet.
  \item[Private transport] Normally, votes are anonymous---there is no
    connection between a cast vote and the person who cast it. However,
    extant Internet communication protocols cannot support this privacy
    guarantee (and vote-by-mail systems typically sacrifice this property as
    well).
  \item[Attack surfaces] The voter is presumably using his own computer.
    It is possible that his computer has been taken over by somebody else.
    This kind of problem is completely fresh compared to paper voting.
  \item[Verifiability] Voters should be able to independently verify that
    their votes were cast and counted correctly, without needing to place
    trust in the communication medium or the election officials. Auditors
    should be able to check that the election was executed correctly without
    trusting the voting hardware and software vendors. Paper ballots support
    auditing well, but make individual voter confidence difficult.
\end{description}

\section{E2E Election Rituals}
\subsection{Pre-Election Phase}
\subsection{Voting}
\subsection{Post-Election Phase}
\section{Shortcomings and Expectations of E2EVIV}
\subsection{Access to Communication/Internet}
\subsection{Accessibility}
\subsection{Usability}
\section{E2E VIV in Practice}

A number of practical voting systems have been developed based on the
principles of E2E VIV. This section describes several systems that
have been used in a real election or in a pilot.

\todoacf{Sort these somehow; make a timeline?}

\subsection{RIES~\cite{hubbers2004}}

RIES, the Rijnland Internet Election System, was first used in 2004 to
support elections to the Rijnland water management board,
supplementing the system of postal voting used by the water board. A
subsequent version was used to allow expatriate voters to participate
in the Dutch parliamentary elections \cite{gonggrijp2009}.

\subsubsection{Election Procedure}

\begin{itemize}

\item Before the election, credentials are mailed to every voter in
  the form of a very long number. The same mailing also includes
  instructions for the voter.

\item During the election, voters log into an election web site that
  includes a client-side voting application written in JavaScript. The
  client-side application encrypts the vote by passing the voter
  authorization code and the public ID of the candidate through a
  one-way function to create the encrypted vote. The encrypted vote
  is then placed on a public bulletin board that serves as a ballot
  box.

\item At the close of the polls, the election authority releases the
  final vote tallies along with a codebook containing the encryptions
  of all valid credentials with all candidate IDs.

\end{itemize}

\subsubsection{Verifiability}

The algorithms and protocols used RIES are public, and each voter,
having access to all of the inputs and outputs, may (in principle)
check the computations. This is weaker than the desired individual
verifiability\todoacf{how/why?}, but nonetheless, far stronger than
conventional voting systems.

\subsubsection{Analysis}

RIES makes a tradeoff by weakening the receipt-freeness requirement in
exchange for providing universal verifiability and a degree of
individual verifiability. The results of an entire election can be
independently audited with only the information publicly available
after the election. However if a voter discloses her credential or her
encrypted vote, the same public information may be used to violate
ballot secrecy. The developers of RIES judged this violation to be no
more severe than the threats to ballot secrecy inherent in postal
voting, and therefore worth accepting for the benefit to
verifiability.

The Organization for Security and Co-operation in Europe (OSCE) sent
an election assessment team to observe the use of RIES in 2006. Their
report contains observations of critical security features of the
system that could not be observed \cite{osce2007}. Further
weaknesses were revealed by the Eindhoven Institute for the Protection
of Systems and Information (EiPSI) in 2008 \cite{hubbers2008},
notably that:

\begin{itemize}
  \item the procedure of voter self-check is quite complicated,
  \item the two-channel (mail and Internet) voting makes system less
    transparent,
  \item too much power is given to the election administrator and the
    system's Internet host,
  \item issues arise when modifying the codebook due to a revoked
    ballot, and
  \item there are realistic ways to forge votes via cryptographic hash
    collisions.
\end{itemize}

One of the more important lessons learned through RIES is that when
voter authorizations are distributed long in advance of the election,
a mechanism must be provided allowing voters to obtain replacement
credentials and invalidate lost credentials. These mechanisms add
significant complexity to system, and is a source of some of the
problems reported in the OSCE and EiPSI reports.

Another feature of RIES rife with tradeoffs is the ability to perform
testing during the election: pre-invalidated test ballots are
deliberately added to the bulletin board in order to test the network
path from selected Internet clients to the server. While such testing
in principle can increase confidence in the election integrity, in
practice it opens the system to spoofing and denial of service
attacks. Furthermore in the RIES implementation the system is aware of
the fact that it is processing a testing ballot, and all of the test
ballots were voted identically from the same computer, limiting the
confidence added at the expense of these vulnerabilities.

In the wake of these critical reports, plans to use RIES in the 2008
Dutch parliamentary elections were scrapped, and Internet voting as a
whole was banned in the Netherlands.

\subsection{Prêt à Voter~\cite{chaum2005}}
\label{sec:pret-voter}

The state of Victoria in Australia held a governmental election in
November 2014, using a version of the Prêt à Voter
system~\cite{burton2012}. An attempt was also made to use Prêt à Voter
in a student election at the University of Surrey in February
2007~\cite{bismark2007}. The failure of this attempt illustrates many
of the pitfalls of adapting a research system to an actual election,
such as a short timetable, a lack of clear requirements, and the need
for rigorous implementation practices.

\subsubsection{Election Procedure}

Prêt à Voter uses two-part paper ballots with the candidate names on
one part and the voting targets plus a ballot ID number or barcode on
the other part.

\begin{itemize}

\item Before the election, the paper ballots are printed. Typically,
  the two parts are printed as a single sheet with a perforation to
  divide the sheet after voting.

\item From the voter's perspective, the order of the candidate names
  on the ballot appears to be random. The voter marks her choice next
  to the candidate name of her choice, separates the two parts of the
  ballot, and destroys the candidate names. She may take a copy of the
  voted part home for later verification.

\item For tabulation, there is a cryptographically secure mapping from
  the ballot ID numbers to the apparent random order of the candidate
  voting positions. Multiple custodians using a mixnet or similar
  technique use this mapping to decode cast ballots into anonymized
  plain-text ballots which are then posted to a bulletin board.

\end{itemize}

\subsubsection{Verifiability}

Unvoted ballots may be audited before, during and after the election
to ensure that the decoding of cast ballots is being correctly
performed. Randomly selected stages in the decoding can be challenged
to prove the integrity of the count, and the plain-text decoded
ballots are easily counted for verification by any interested party.

An individual voter may also search for their voted ballot ID on the
bulletin board. This reveals the positions that were marked on that
ballot, but crucially, it does not show the corresponding candidate
names. The voter may therefore verify that the positions marked at the
polling place were correctly recorded by the election officials, but
because the voter no longer has the part of the ballot linking
candidate names to ballot positions, the voter cannot prove to anyone
else how the ballot was voted.

\subsubsection{Analysis}

Although Prêt à Voter in theory offers strong protection from
coercion, in practice much of that protection depends on the rigor
with which election procedures are followed~\cite{burton2012}. If the
part of the ballot with candidate names is not destroyed after voting,
it can be used to violate secrecy. An insider with access to printed
paper ballots before the election can record both parts of the ballot
together and later use that information to learn a voter's choice from
the bulletin board.

Since there is no pre-election bulletin board posting of the valid
ballot IDs, there is also potential for ballot stuffing by
insiders. This can be detected by cross-referencing polling place
data with the bulletin board after the fact, but that requires
additional trust in the poll workers.

\subsection{Punchscan~\cite{popoveniuc2006,popoveniuc2010punchscan}}
\label{sec:punchscan}

Punchscan was used for the graduate student association elections of
the University of Ottawa in 2007~\cite{essex2007}. It is likely
the first E2E voting system with ballot privacy used in a binding
election.

\subsubsection{Election Procedure}

The election experience for a Punchscan voter is very similar to that
of Prêt à Voter. The system uses a two-part paper ballot where the top
part has candidate names and candidate numbers (or letters) and the
bottom part has numbered (or lettered) voting targets. Holes punched
in the top part expose the voting targets below. The order of the
voting targets for each race appears random to the voter. Both halves
of the ballot bear an identical serial number.

The voter casts their vote by marking her choice with a bingo dauber,
and the two halves are separated. Either side can be scanned (since
the bingo dauber marked both through the hole and around it) as the
cast ballot. The other side is destroyed, and a copy of the cast side
may be retained by the voter.

\subsubsection{Verifiability}

A curious voter may inspect the public record of any cast ballot
exactly as with Prêt à Voter. It does not matter which half of the
ballot the voter retained, because there is no public display of the
numbers that link candidate names to voting positions; only the
position that was marked is displayed. Again, individual ballots may
be audited, and the key to tabulating the votes is that there is a
cryptographically secure mapping from the ballot serial numbers to the
apparent random order of the candidate voting positions.

\subsubsection{Analysis}

Punchscan elections rely on procedure to maintain many of its
desirable properties, much like Prêt à Voter. For example, during the
University of Ottawa elections, more ballots were cast than voters
recorded in the pollbook, showing that ballot stuffing can be caught
after the fact by poll workers, but is not an inherently verifiable
property of the system.


\subsection{Scantegrity II~\cite{chaum2008,chaum2009}}
\label{sec:scantegrity-ii}

Scantegrity II (Invisible Ink) was used in the Takoma Park, Maryland
municipal elections in 2009~\cite{carback2010}. In 2011, it was used
for in-person voting with Remotegrity (\ref{sec:remotegrity}) used for
absentee voting. The 2009 Takoma Park election was the first use of an
E2E system with ballot privacy in binding governmental elections.

\subsubsection{Election Procedure}

\begin{itemize}

\item Before the election, officials generate the seed to a
  pseudorandom number generator using a secret sharing
  scheme. Three-letter alphanumeric codes are created for each choice
  on each printed ballot using this seed, and additional tables are
  created so that interested parties can later confirm that the tally
  was computed correctly.

\item During the election, the voter experience is nearly identical to
  that of conventional optical-scan paper ballots. When the voter
  marks their choice, the ink in the pen reacts with invisible ink on
  the paper to disclose the three-letter code in the marked voting
  target. The ballot ID number and the displayed code are posted to a
  public bulletin board.

\item After the election, public verification of the final tally
  proceeds with the public bulletin board in a manner similar to that
  of Punchscan and Prêt à Voter.

\end{itemize}
\subsubsection{Verifiability}

In addition to the public verification, an individual voter who takes
note of their ballot ID number and the code revealed from invisible
ink may use the public bulletin board to check that their ballot was
indeed tabulated, though this information is not sufficient to prove
that they voted a particular way.

\subsubsection{Analysis}

\todoacf{Roll this in with other analysis in \ref{sec:limit-exist-syst}}

\subsection{Remotegrity~\cite{zagorski2013}}
\label{sec:remotegrity}
Remotegrity is a remote coded voting system that was used for absentee
voting alongside Scantegrity (\ref{sec:scantegrity-ii}) for in-person
voting for the 2011 Takoma Park, Maryland municipal elections.

Remotegrity voters receive a coded voting ballot and an authentication
card in the mail. The codes on the ballot can be covered by a
lottery-style scratch-off field. The authentication card contains
several authentication codes under scratch-off, a lock-in code under
scratch-off, and an acknowledgment code. Both cards have serial
numbers. The voter can be sent two ballots so that she can use one for
auditing purposes.

To vote, the voter enters both serial numbers, the codes corresponding
to her choices, and an authentication code obtained after
scratching-off a surface chosen at random.

She returns to the election website a few hours later to check if her
codes are correctly represented, and to see if the election authority
has posted her acknowledgment code next to the codes. This indicates
to her that the election officials received valid codes for her
ballot.

She scratches off the lock-in code and posts it on the website. This
affirms to the election officials, observers and other voters that her
vote is correctly represented on the website.

Among all of the systems discussed here, this is the first one that
asks the voter to take positive action to confirm that the vote was
correctly posted.

As with RIES, if we assume that there is no communication between the
computer used to print the credentials and the computer used to
collect the votes, the latter computer does not know the mapping
from codes to candidates, so the vote is not revealed to the
computer. Further, because the computer does not know a valid code
corresponding to another candidate on the ballot, it cannot change the
vote. Finally, and uniquely, because the computer does not know the
acknowledgment code, its presence on the election website assures the
voter that the election officials received a valid code for her
ballot.

The tally is computed from the codes in a verifiable manner that
corresponds to the coded voting system used.

If a jurisdiction is nervous about using the Internet for remote
voting, Remotegrity ballots can be mailed in, and voters can check for
their codes on the election website to be assured that their vote
correctly reached election officials.

\subsection{Helios~\cite{adida2008,adida2009}}

Helios is a system developed for web-based Internet voting. It was
used for the election of a Belgian university president in March 2009
and by numerous universities and associations since then, including
the Association for Computing Machinery and the International
Association for Cryptologic Research.

\subsubsection{Election Procedure}

\begin{itemize}

\item Before the election, officials input the email addresses of the
  voters who will be participating. The system emails the voters their
  randomly-generated login information and the link to the election
  website.

\item During the election, the voter enters their choices on the
  website. After entering her choices, the voter has an option to
  spoil their ballot in order to verify that it was recorded
  correctly. Upon completing a non-spoiled ballot, the system sends an
  email confirming the receipt of their vote, though not their
  choices. At any time before the close of the election, the voter can
  repeat these steps and the new vote will replace the old vote.

\item After the election, Helios uses homomorphic vote tallying with
  the optional addition of mixers and mixnets in some
  derivatives~\cite{bulens2011,tsoukalas2013}.

\end{itemize}

\subsubsection{Verifiability}

Voter authentication is not required until after the voter decides to
cast the ballot, so any interested party may prepare and audit
ballots. All cast ballots are posted in encrypted form on a public
bulletin board so that voters may check that their ballots have been
correctly recorded. Similarly, after the polls close, the decryption
and vote tally may be checked.

\subsubsection{Analysis}

Because officials can enter voters by email address, Helios provides
limited protection against insider ballot stuffing. Due to the
complexity of publicly auditing the election results, this would be
difficult to detect~\cite{orion2009}.

\subsection{Norwegian System~\cite{gjosteen2012}}

Between 2011 and 2014, the Norwegian government ran an Internet remote
voting trial using a cryptographic protocol designed by Scytl, a
commercial voting system vendor. Scytl and the Norwegian government
assert that this is an E2E system, which if accurate is the first
effort by commercial voting system vendors to enable E2E
elections.

\subsubsection{Election Procedure}

The Norwegian system uses a three-channel model involving postal mail,
the Internet, and SMS text messaging.

\begin{itemize}

\item Before the election, the voter receives authorization codes to
  cast a ballot via postal mail.

\item During the election, the voter uses a computer to cast an
  encrypted ballot. The voter can cast multiple ballots; only the
  last ballot cast is counted, and if a voter votes both on paper at a
  polling place and by Internet, the paper ballot overrides the
  Internet ballot. After casting a ballot, the voter receives a
  confirmation code offering a partial end-to-end proof via an SMS
  message.

\end{itemize}

\subsubsection{Verifiability \& Analysis}

Available descriptions of the Norwegian system are incomplete, so it
is not possible to analyze the system in depth. However the system's
claims to protect voter privacy are weak: ``If the voter's computer
and the return code generator are both honest, the content of the
voter's ballot remains private.'' In addition, the receipt delivered
to the voter proves only that the encrypted ballot was received as
cast, not that it was counted as cast or that the encrypted vote
matches the voter's intent.

The system evolved significantly between its first use in 2011 and
2013, with added complexity to attempt to assure voters that their
ballots were stored as cast. In 2013, the Carter Center mounted a
serious effort to observe the Norwegian system in action. Their
report on the operation of the system and the problems
they had observing it offers useful insight into the administration of
E2E systems in general as well as the particulars of the Norwegian
system~\cite{carter2013}.

\subsection{Wombat~\cite{rosen2011}}

The Wombat voting system has been used for multiple pilot elections in
Israel. It is an in-person voting system where the voter votes on a
touch-screen and obtains a printout of her vote with an encryption of
it. The voter can choose to cast or audit the encrypted vote. If she
chooses to audit the vote, she may check if the vote was correctly
encrypted. If she chooses to cast it, the ciphertext is posted online,
and she casts the unencrypted vote in the ballot box (this may be
manually counted) and takes the ciphertext home. The votes are tallied
using a verifiable mixnet.

\subsection{DEMOS~\cite{kiayias2014}}

DEMOS is a coded vote system where the voter is given a two-part coded
ballot; she audits one part and uses the other to vote. Associated
with each choice on the ballot is

\begin{itemize}
\item a vote code---the encryption of the vote, which is entered in
  the voting machine by the voter, and
\item a receipt code which the voter does not enter, but which is
  posted online next to the vote code.
\end{itemize}

The voter can check the receipt to ensure her vote reached the
election authorities. The ballot also has a QR code containing all the
information on the ballot which can be scanned by the voter if she
prefers not to manually enter the vote code. Once the ballot is
entirely represented on the computer, the voter can then make her
choices. Note that if the voter scans the QR code, the scanning
computer knows how she voted. The vote codes represent homomorphic
encryptions of the votes and the verifiable tally is obtained in a
standard manner.

A pilot study of DEMOS was carried out in 2014.

\todoacf{This section should be filled in with material from
  history.tex and comparative\_analysis.tex, and then history.tex and
  comparative\_analysis.tex should be ``retired'' and their resources
  moved into e2e\_viv\_explained\_resources.}
\section{Limitations of Existing Systems}
\label{sec:limit-exist-syst}
%%% Local Variables:
%%% mode: latex
%%% TeX-master: "report"
%%% End:
