\chapter{History/Summary of E2E Voting System Development}

\section{Origin}

The phrase \emph{end to end} is generally attributed to~\cite{saltzer1981},
where it was applied to a class of arguments about computer networks: good
system design ensures that---regardless of the application and what
assumptions can be made of its properties---certain high-level properties
are satisfied at the endpoints of a communication channel.  As applied to
voting, \cite{jones2002}~translated this to the question "How close does the
outcome of the election, as reflected in the official canvass, come to the
actual intent of the voters who participated in the election?" The term is
also used in Ben Adida's dissertation~\cite{adida2006}, which states that
the verifiability of a voting system is end-to-end if it is "preserved from
start to finish, regardless of what happens in between". More specifically,
the ability to verify correctness of the output of an end-to-end voting
system does not depend on securing chain of custody, nor on checking the
software or hardware used to carry out the election. That is, it does not
depend on monitoring the path the votes take from the voter to the tally, or
on assumptions regarding the security of that path. Rather, voting system
design ensures that one may examine only the output (the tally, other data
and a mathematical proof of tally correctness) to verify the election.

During the recent years, various authors provided various definitions of
end-to-end verifiable voting systems. These definitions differ in their
level and precision of formalism, and in their relation to  specific voting
systems. For instance:
\begin{itemize}
  \item \cite{lowry2009}~define end-to-end independently verifiable voting
    systems as ones where an honest observer may determine whether the
    outcome "correctly represents the votes cast by voters".
  \item \cite{kremer2010}~define end-to-end verifiability, using the
    applied-pi calculus as modelling framework, as the conjunction of three
    properties: (i) individual verifiability (voters can verify that their
    vote was captured properly); (ii) eligibility verifiability (anyone can
    verify that all the ballots included in the tally come from eligible
    voters) and (iii) universal verifiability (anyone can verify that the
    tally is correctly computed from those votes).
  \item \cite{popoveniuc2010}~provide a somewhat technical definition of
    end-to-end verifiable elections, as ones that pass a set of six checks
    on whether: (i) presented ballots are well-formed; (ii) cast ballots are
    well-formed; (iii) ballots are recorded as cast; (iv) ballots are
    tallied as recorded; (v) consistency: whether the collection of ballots
    subjected to the check for (iii) is the same as that checked for (iv);
    (vi) each recorded ballot is subjected (by at least one voter) to the
    check of (iii).
  \item \cite{kusters2010}~define verifiability as the guarantee that
    auditors will only accept election results that are consistent with the
    votes cast by eligible voters. They also advocate the requirement of a
    stronger notion of verifiability, that is, accountability, which
    requires that, when the audit process of a verifiable system results in
    a failure, it should also point to the election participant (voter,
    authority, ...) that misbehaved.  Their definition is instantiated both
    in a symbolic and in a computational model.
\end{itemize}

\section{Background}

Research in the use of cryptography to design voting systems with end to end
properties dates back to the early eighties. Much literature exists on
desirable properties of voting systems. Inspired from~\cite{kremer2005}, we
list the following desirable voting system properties as an example:

\begin{itemize}
  \item Fairness: no early results can be obtained which could influence the
    remaining voters.
  \item Eligibility: only legitimate voters can vote, and only once.
  \item Privacy: the fact that a particular voter voted in a particular way
    is not revealed to anyone.
  \item Individual verifiability: a voter can verify that her vote was
    correctly included in the collection of votes.
  \item Universal verifiability: anyone can verify that the published
    outcome correctly represents that same collection of votes.
  \item Receipt-freeness: a voter cannot prove that she voted in a certain
    way (this is important to protect voters from coercion).
\end{itemize}

Of these, properties 4 and 5, taken together, impose an end-to-end
constraint on the process of vote collection and tabulation, while property
2 protects the vote from dilution, and properties 1, 3 and 6 work to prevent
coercion of voters. (We observe that these six properties do not allow
anyone to verify that \emph{all} votes were [properly] included in the
tally.)

The end-to-end constraints can be satisfied trivially in small elections
where preventing coercion is not an issue.  Everyone involved in a show of
hands can verify that their hand is up or down, and everyone involved can
count all of the votes to verify the result.  Similarly, if voters sign
their paper ballots and post them on a bulletin board, each voter can verify
that their ballot is on the board, which represents the collection of votes,
and anyone can verify the result.

Paper ballots, whether hand counted or machine counted, come close enough to
meeting these constraints that their failures are worth analyzing.
\cite{jones2002}~observed that, while voters can check that their ballots
were marked as intended, they cannot check that hand-made markings on
ballots meet the criteria that will be used in the count.  Similarly, while
voters can observe that their ballots went into the ballot box, everyone
cannot observe the chain of custody of the ballots between the time they go
into the box and the time the box is opened for counting.  Finally, the
number of people who can directly observe the count is strictly limited, so
we do not have universal verifiability.

Numerous voting systems have been proposed that attempt to meet these
constraints.  The following papers describe key technologies on which
practical systems have been based.  Note that these predate Kremer and
Ryan's enumeration of voting system properties; that enumeration is, in
large part, one attempt to formalize the problem that these and other new
voting systems are attempting to solve:

\begin{itemize}
  \item \cite{chaum1981}~proposed the idea of a cryptographic \emph{mix net}
    with applications to anonymous e-mail and electronic voting. In this
    approach, votes are encrypted and then shuffled by multiple
    \emph{mixes}, each mix partially decrypts votes as well. In the final
    set of fully-decrypted and shuffled votes the link between a voter and
    her vote is completely destroyed. This set of votes may be counted. This
    approach established one of the two main paradigms for later proposals
    for end-to-end verifiable voting.
  \item \cite{benaloh1985}~introduced \emph{homomorphic encryption} as an
    alternative to mix nets, and this approach is the basis of the other
    main paradigm.  In this approach, encrypted votes are counted without
    decryption, and only after counting are the totals decrypted.
  \item \cite{chaum2001}~introduced \emph{code voting}, embodied in the
    SureVote system. This approach forms the basis of many remote voting
    systems where the voting machine is not trusted. Voters obtain
    codesheets with a code corresponding to each candidate, and enter the
    code, instead of the candidate name, into the voting machine.
\end{itemize}

Building on these ideas, a number of practical voting systems have been
developed.  Each of the following systems has actually been used in a real
election or in a pilot:

\subsection{RIES~\cite{hubbers2004}}

RIES, the Rijnland Internet Election System, was developed to support
elections to the Rijnland water management board, supplementing the system
of postal voting used by the water board.  It weakens the receipt-freeness
requirement generally accepted for E2E voting systems while providing
universal verifiability and a degree of individual verifiability.  These
compromises are based on the fact that remote voting generally, including
postal voting, has weak coercion resistance, and adding universal
verifiability and any degree of individual verifiability is a distinct
improvement.

Before the election, credentials are mailed to every voter in the form of a
very long number (encoded in Crockford's Base 32)
(see~\cite{crockford2002}).  The same mailing also includes instructions.
Voters log into an election web site that includes a client-side voting
application written in Javascript.  The algorithms used by this script are
public, and each voter, havign access to all of the inputs and outputs, may
(in principle) check the computations.  This is weaker than the desired
individual verifiability, but nonetheless, far stronger than conventional
voting systems.

The client-side application encrypts the vote by passing the voter
authorization code and the public ID of the candidate through a one-way
function to create the encrypted vote.  The encrypted vote is then placed on
the public bulletin board that serves as a ballot box.  At the close of the
polls, the election authority releases the codebook containing the
encryptions of all valid credentials with all candidate IDs. If a voter
discloses her credential or her encrypted vote, the published codebook may
be used to violate ballot secrecy. The developers of RIES judged this
violation to be no more severe than the threats to ballot secrecy inherent
in postal voting.

RIES was used for Rijnland Water Board elections starting in 2004, and in
2006, it was used to allow expatriate voters to participate in the Dutch
parliamentary elections, see~\cite{gonggrijp2009}. The OSCE sent an election
assessment team to observe the use of the system in 2006.  Their report
contains observations of critical security features of the system that could
not be observed; see~\cite{osce2007}.

One of the more important practical contributions of RIES involves the
recognition that, when voter authorizations are distributed by post or more
generally, distributed long in advance of the election, a mechanism must be
provided allowing voters to obtain replacement credentials, for example, by
telephone or e-mail, and a mechanism is needed to invalidate lost
credentials.  Adding these mechanisms adds significant complexity to the
system and is a source of some of the problems reported in the OSCE report.
A second feature of RIES is that it allows parallel testing during the
election, where pre-invalidated test ballots are deliberately added to the
bulletin board in order to test the network path from selected internet
clients to the server.  This offers the possibility of detecting a variety
of spoofing and denial of service attacks.

\subsection{Prêt à Voter~\cite{chaum2005}}

The Prêt à Voter system uses two-part paper ballots with the candidate names
on one part and the voting targets plus a ballot ID number on the other.
Typically, the two parts are printed as a single sheet, with a perforation
along which the sheet can be divided after voting.  The order of the
candidate names appears to be random, from the voters' perspective.  After
voting, the half of the ballot containing candidate names is shredded. The
voted half forms the cast ballot, and the voter may take a copy home.

A curious voter may inspect the public record of any cast ballot by
searching for it by the ballot ID number in the public database of cast
ballots.  That database shows the positions that were marked on that ballot,
but crucially, it does not show the identifying letters or candidate names.
The voter may therefore verify that the positions marked at the polling
place were correctly recorded by the jurisdiction, but because the voter
only has half of the ballot and there is no public display of the linking of
candidate names to ballot positions, the voter cannot prove to anyone else
how the ballot was voted.

The key to tabulating the votes is that there is a cryptographically secure
mapping from the ballot serial numbers to the apparent random order of the
candidate voting positions.  The decoding of a cast ballot to a
corresponding plain-text ballot, where the voted positions and candidate
names are in a canonical order (alphabetical order, for example) is
performed in multiple steps, by multiple custodians, for example, using a
mixnet. As with the mixnet, no one, even if colluding with a subset of the
custodians, can connect a decoded ballot to the corresponding cast ballot
(and, hence, to the voter).  Unvoted ballots may be audited before, during
and after the election to ensure that the decoding of cast ballots is being
correctly performed (audited ballots may not be used for voting purposes,
however, because the connection between the voting positions and candidate
names is publicly revealed during the audit). Randomly selected stages in
the decoding can be challenged to prove the integrity of the count, and
decoded ballots are easily counted.  Anyone may therefore check the count.

A version of Prêt à Voter is planned for use in a governmental election by
the state of Victoria in Australia in November 2014, see~\cite{burton2012}.
Previously, an attempt was made to use Prêt à Voter in a student election at
the University of Surrey in February 2007, see~\cite{bismark2007} -- this
attempt illustrates many of the perils of working with an election
authority.

\subsection{Punchscan~\cite{popoveniuc2006,popoveniuc2010}}

The Punchscan system used a two-part paper ballot where the top part had
candidate names and candidate numbers (or letters) and the bottom part had
the numbered (or lettered) voting targets.  The top part had holes punched
in it to expose the voting targets below.  The order of the voting targets
for each race appears random to the voter.  Both halves of the ballot bear
an identical serial number.  After marking the ballot with a bingo dauber,
the two halves are separated.  Prior to separation, the voter can easily see
that the correct target has been marked.  To vote, the voter separates the
two sides.  Either side can be scanned (since the bingo dauber marked both
through the hole and around it) as the cast ballot. The other side is
destroyed. A copy of the cast side may be retained by the voter.

A curious voter may inspect the public record of any cast ballot exactly as
with Prêt à Voter. It does not matter which half of the ballot the voter
retained, because there is no public display of the numbers that link
candidate names to voting positions; only the position that was marked is
displayed.  Again, individual ballots may be audited, and the key to
tabulating the votes is that there is a cryptographically secure mapping
from the ballot serial numbers to the apparent random order of the candidate
voting positions.

Punchscan was used for the graduate student association elections of the
University of Ottawa in 2007, see~\cite{essex2007}. It is likely the first
end-to-end voting system with ballot privacy used in a binding election.

\subsection{Scantegrity II~\cite{chaum2008,chaum2009}}

Scantegrity II (Invisible Ink) allows end-to-end cryptographic verification
of optical-scan paper ballots.  Scantegrity ballots may be fully compatible
with conventional optical scan vote tabulation equipment, but are voted
using a marking pen filled with invisible ink.  When the voter marks a
target on the ballot, the ink in the pen reacts with invisible ink on the
paper to disclose a three-letter alphanumeric code in the marked voting
target.  A voter who takes note of this code and the ballot ID number may
check a public bulletin board to check that the ballot was indeed tabulated.

As with Punchscan and Prêt à Voter, there is a cryptographically secure
mapping from the ballot ID number to the code disclosed when the voter marks
a voting target.  The code is displayed on the public bulletin board, and
public verification of the decryption and tally of the contents of the
bulletin board proceeds in a manner similar to that of Punchscan for the
system fielded in Takoma Park, and in a somewhat different manner using
similar principles for the system described in~\cite{chaum2009}.

Scantegrity II is the first end-to-end voting system with ballot privacy to
see use in government elections.  It was used in Takoma Park, Maryland in
2009 and 2011; see~\cite{carback2010}.

\subsection{Helios~\cite{adida2009,adida2009}}

Helios was developed for Web-based Internet voting. Helios provides a
web-browser-based ballot preparation system that can be used to make choices
and encrypt ballots.  After preparing a ballot, the voter has the option of
either casting that ballot or auditing it; this allows the voter to detect
misbehavior on the part of the ballot preparation software. Voter
authentication is not required until after the voter decides to cast the
ballot, so anyone may prepare and audit ballots.  Existing Helios
implementations piggyback on existing university and commercial
authentication mechanisms, and also support an internal login-password
authentication mechanism for which credentials can be distributed by email.

As with the schemes discussed above, all cast ballots are posted in
encrypted form on a public bulletin board so that voters may check that
their ballots have been correctly recorded.  Similarly, after the polls
close, the decryption and vote tally may be checked. Helios uses homomorphic
vote tallying for simplicity, even though a single mixer was used in its
first developments, and mixnets have been used in several of its forks and
descendants; see~\cite{bulens} or~\cite{tsoukalas} for instance. A recent
variant by~\cite{cortier} proposes an enhanced ballot authentication
mechanism. The STAR-Vote system~\cite{star-vote}, designed for public
elections as a result of an invitation from the Travis County election
administration, also borrows various techniques used in Helios and
VoteBox~\cite{votebox}.

Helios was used for the election of a Belgian university president in March
2009 and by numerous universities and associations since then, including the
ACM and the IACR. The cryptographic protocols it implements have been
subject to considerable analysis.

\subsection{Norwegian System~\cite{gjosteen2012}}

Between 2011 and 2014, the Norwegian government ran an Internet remote
voting trial.  The system rests on a cryptographic protocol designed by
Scytl, a commercial voting system vendor. Scytl and the Norwegian government
assert that this is an E2E system.  As such, this marks the entry of efforts
(or claims) by commercial voting system vendors to enable E2E elections.
System descriptions have been incomplete, hence it has not been possible to
tell what properties the system possesses. However, the following are clear.
The system's claims to protect voter privacy are weak: "If the voter's
computer and the return code generator are both honest, the content of the
voter's ballot remains private."  In addition, the receipt delivered to the
voter proved only that the encrypted ballot was received as cast, not that
it was counted as cast or that the encrypted vote matched the voter's
intent. Thus, by any of the definitions above, it is not an end-to-end
system.

The Norwegian system used a three-channel model, where the voter receives
authorization codes to cast a ballot via the postal system, then uses a
computer to cast an encrypted ballot, and finally obtains a confirmation
code offering a parital end-to-end proof via an SMS message to the voter's
mobile phone.  In addition, as with RIES, voters could cast multiple
ballots; in Norway, only the last ballot cast was counted, and if a voter
votes both on paper at a polling place and by Internet, the paper ballot
overrides the Internet ballot.

The system evolved significantly between its first use in 2011 and 2013,
with added complexity to attempt to assure voters that their ballots were
stored as cast.  In 2013, the Carter Center mounted a serious effort to
observe the Norwegian system in action.  Their report on the operation of
the system, in practice, and the problems they had observing it offers
useful insight into the administration of E2E systems in general as well as
the particulars of the Norwegian system; see~\cite{carter2013}.

\subsection{Remotegrity~\cite{zagorski2013}}

Remotegrity is a remote coded voting system that additionally uses the
notion of a lock-in to provide additional security properties. Remotegrity
voters get a package in the postal mail. The package contains:
\begin{itemize}
  \item a coded voting ballot (a ballot with a code printed against each
    choice). The code may or may not be covered by a scratch-off field (such
    as is used for lottery tickets)
  \item an authentication card that contains: (a) many authentication codes
    under scratch-off (b) a lock-in code under scratch-off and (c) an
    acknowledgement code.
\end{itemize}
Both cards have serial numbers.

To vote, the voter enters (a) both serial numbers, (b) the codes
corresponding to her choices and (c) an authentication code obtained after
scratching-off a surface chosen at random.

She returns to the election website a few hours later to check if her codes
are correctly represented, and to see if the election authority has posted
her acknowledgement code next to the codes. This indicates to her that the
election officials received valid codes for her ballot.

She scratches-off the lock-in code and posts it on the website. This is her
way of communicating to the election custodians, observers and other voters
that her vote is correctly represented on the website. Among all of the E2E
systems (and approximations to E2E systems) discussed here, this is the
first one that asks the voter to take positive action to confirm that the
vote was correctly posted.

As with RIES, if we assume that there is no communication between the
computer used to print the credentials and the computer used to accumulate
the votes, the latter computer does not know the mapping from codes to
candidates, so the vote is not revealed to the computer. Further, because
the computer does not know a valid code corresponding to another candidate
on the ballot, it cannot change the vote.  Finally, and uniquely, because
the computer does not know the acknowledgement code, its presence on the
election website assures the voter that the election officials received a
valid code for her ballot.

The tally is computed from the codes in a verifiable manner that corresponds
to the coded voting system used.

The voter can be sent two ballots. She can choose one to vote with and one
to audit.

If a jurisdiction is nervous about using the Internet for remote voting,
Remotegrity ballots can be mailed in, and voters can check for their codes
on the election website to be assured that their vote correctly reached
election officials.

Remotegrity was used for absentee voting---and Scantegrity for in-person
voting---by the City of Takoma Park in its 2011 municipal election,
see~\cite{zagorski2013}.

\subsection{Wombat~\cite{rosen2011}}

The Wombat voting system is an in-person voting system where the voter votes
on a touch-screen and obtains a printout of his vote with an encryption of
it. The voter can choose to cast or audit the encrypted vote. If she chooses
to audit the vote, she may check if the vote was correctly encrypted. If she
chooses to cast it, the ciphertext is posted online, and she casts the
unencrypted vote in the ballot box (this may be manually counted) and takes
the ciphertext home. The votes are tallied using a verifiable mixnet.

The system has been used for multiple pilot elections in Israel.

\subsection{DEMOS~\cite{kiayias2014}}

DEMOS is a coded vote system where the voter is given a two-part coded
ballot; she audits one part and uses the other to vote. Associated with each
choice on the ballot is (a) a vote code---the encryption of the vote, which
is entered in the voting machine by the voter and (b) a receipt code which
the voter does not enter, but which is posted online next to the vote code.
The voter can check this to ensure her vote reached the election
authorities. The ballot also has a QR code containing all the information on
the ballot. It can be scanned by the voter if she prefers not to manually
enter the vote code. Once the ballot is entirely represented on the
computer, the voter can then make her choices. Note that if the voter scans
the QR code, the computer knows how she voted. The vote codes represent
homomorphic encryptions of the votes and the verifiable tally is obtained in
a standard manner.

A pilot study of DEMOS was carried out in 2014.


\section{Typical Use}

All of these E2E voting systems rest on a similar information flow, so we
organize our discussion of the typical use in terms of this flow.

(Perhaps this section is best done after a similar, but more technical,
effort by the cryptographic protocol team.)

\subsection{Pre-election phase}

Before the polls open---and in cases where the postal system is used, long
before---there are a number of operations that must be carried out and data
provided to the public. All the data is generally posted on an election
website. The custodians may use public (asymmetric) keys and/or symmetric
keys to carry out the cryptographic operations performed before, during and
after the election. Public keys are declared before the election. To
construct secret keys, election custodians may construct a single master
secret from individual passwords. The master secret is known as a shared
secret, and is used to construct all other secret keys used in the election.

In addition to cryptographic keys, an important work product of the
pre-election phase is a set of voting credentials that must be distributed
to voters to allow them to vote.  With paper-ballot elections (whether E2E
verifiable or conventional), the blank paper ballots are the credentials;
these are distributed either by post or by election clerks who directly hand
them to voters.  Where Internet voting is involved, the credentials may take
the form of passwords, physical tokens such as smart-cards, or hybrids such
as a smart-card plus a PIN.

The custodians of the election must, in come cases, make a public commitment
of certain results of their pre-election computations.  They may do this,
for example, by publishing a secure hash of the results. In addition, the
custodians may also post commitments to how they will process votes. Both
sets of commitments will enable the post-election audits.

Voter privacy requires that the secrets used in setting up the election not
be leaked to insiders.  In all of the systems (whether voters use paper
ballots or paper credentials, or directly enter their vote on a voting
machine), information leakage---from the system that prints the ballots or
credentials or encrypts the vote---to the system that accumulates the votes
would permit privacy violation and in some cases, selective
disenfranchisement. While procedures may be used to reduce the probability
of such leakage, it is very difficult to guarantee that such leakage will
not/did not occur.

\subsection{Voting}

In an E2E voting system, as in any voting system, a voter is presented
candidate choices. Once the voter makes a choice, she receives a
cryptographic receipt.
\begin{itemize}
  \item For the (electronic or paper) ballot serial number used by her, if
    the voting system does not cheat, this receipt represents one candidate
    only, the one chosen by her.
  \item However, in a secret-ballot E2E system, the receipt cannot be used
    to determine the candidate. E2E systems that rely on the security of
    cryptographic primitives to enable privacy are said to provide
    \emph{computational privacy}. Systems that do not rely on such an
    assumption are said to provide \emph{unconditional privacy}.
\end{itemize}

It is important that the voter be able to check that the receipt correctly
represents her vote; that is, that the voting system did not cheat. In the
E2E voting systems that have been used, this end is generally achieved using
the cast-or-audit approach due to~\cite{benaloh2006}, as follows. The voter
needs her credentials only to cast a receipt (and not to obtain it). She
hence obtains receipts for many different votes. She casts only one of
these, corresponding to her chosen candidate (the other uncast receipts may
correspond to any candidates, including her own). She challenges all the
uncast receipts. The voting system provides a public proof of the votes
encrypted by each challenged receipt. Note that a voter cannot cast a
challenged receipt, because the corresponding vote is public. However,
because the voting system does not know which receipts will be challenged,
it is caught, with high probability (assuming challenges are suitably
random), if it tries to encrypt incorrect votes. Note also that all voters
need not challenge receipts. Also note that observers and auditors who are
not voters may also obtain receipts and challenge them. As long as the
machine cannot tell which receipt will be challenged, it will be caught with
high probability (again, if challenges are suitably random) if it cheats.

The voter may take the voted receipt and challenged receipts and proofs home
with her if she votes from a polling booth. Alternately, she is presented
the receipts and proofs in electronic form if she votes remotely using an
electronic channel. In addition to the proofs for challenged receipts, the
custodians post all voted receipts on the election website and compute the
tally from the voted receipts. The custodians may use secret keys to compute
the tally, however the computation must be performed in an auditable
fashion. Additionally, for ballot privacy, the audits must not reveal
information on the secrets or the individual votes beyond that already
contained in the correct tally.

\subsection{Post-election phase}

Voters who wish to can check that their receipts are represented correctly
on the election website. If a large enough number of voters independently
perform this check, and if the voting system cannot predict who will check,
a voting system attempting to use different receipts for the tally will be
caught with high probability (if voters who check amount to a random sample
of all voters).

A post-election audit that checks that the tally is correctly computed from
the posted receipts is performed. This audit uses data that can be made
public without revealing information on secrets or any individual votes
(beyond the information contained in the correct tally). The audit will
typically require reliable confirmation that a certain minimum number of
independent voters had correctly checked their receipts, and that no one had
found an error.

Additionally, the post-election audit also checks the correctness of the
proofs for challenged receipts.

\section{What is not covered}

E2E techniques make it possible to detect manipulations of an election that
can lead to incorrect results. They do not, however, prevent such
manipulation. Additionally, by themselves, they do not guarantee anything
regarding the privacy of the votes (voting by a public show of hands is E2E
verifiable) or the availability of the election results (E2E does not
prevent destroying all paper and/or electronic ballots before tally, though
such destruction would be detected).

Of course, E2E verifiability does not contradict these properties: all the
E2E systems described above provide these properties at various levels, and
often in ways that are more robust than in the currently deployed non-E2E
systems. Depending on the intended uses of the systems, various solutions
are proposed, balancing security, usability and versatility.

In practice, the privacy of the votes will typically rely on a set of
trustees to not collude and, in some cases, on the device used to prepare
the ballot to not reveal the ballot content (when there is such a device).

Getting election results will depend on the availability of paper ballots
and/or voting devices, or of websites serving the ballot preparation system.
It will also depend on the protection of the urns (paper, or electronic):
ballot stuffing or hacking of an election server will be detected thanks to
E2E verifiability, but is still likely to make it impossible to obtain
election results.

E2E techniques also often rely, at various levels, on the availability of
some specific components. It is typically assumed, for instance, that
auditors and voters can run their verification procedure on at least one
device that is not corrupted. It is also often assumed that a broadcast
channel, providing consistent data to everyone, is available for getting
information such as election description, audit data, or public randomness.
Such channels are sometimes simply assumed to be provided by a public
website, but various other techniques have also been implemented, including
the use of newspapers, digital signatures, or more sophisticated distributed
cryptographic protocols.

Some (though not all) E2E designs are not able to resolve disputes between
the voting system (which may, for example, insist that it has honestly
represented voter receipts) and a group of voters (who might be insisting
otherwise). With these voting systems, hence, the public would not be able
to determine if the complaints of a group of voters were genuine.

\section{Strengths}

Paper ballot systems using statistical audits require a secure chain of
custody. This is not always straightforward to ensure. It is also not easy
to detect any violations of this assumption. Finally, only a certain elite
few are in a position to detect these violations.

E2E systems, on the other hand, democratize access to the data that reveals
attempts at election fraud. They also do not require the voter or observer
desirous of checking the correctness of election outcome to trust a certain
set of officials, software or hardware. To the extent that software or
hardware needs to be used to check the election outcome, an E2E system
allows the individual to choose which software or hardware she would use. To
the extent that it requires trust in election procedures, these are
performed during the election in a publicly-viewable manner when they deal
with the collection of votes or ballots in general. When they deal with an
individual vote, they are viewable during the voting process by the
corresponding individual voter.

\section{Weaknesses}

E2E systems require the participation of voters and observers to detect
problems.

E2E systems providing computational privacy open up the possibility of vote
exposure if the cryptography used is insecure, or if a set of custodians
collude. Note that this vote exposure relates the receipt to the vote. If
the voter reveals her receipt, it then relates the voter to the vote through
the receipt.

E2E systems increase the number of steps in the voting process and hence
decrease its usability. The most secure E2E voting systems are
paper-ballot-based, and hence very inaccessible.

\section{Potential points of failure}

The most vulnerable point in an E2E voting system is the election website.
Additionally, an E2E system can fail if the voting process is not adequately
communicated to voters, or if a sufficient number of voters does not check
encryption-correctness or the presence of receipts on the website. Both
aspects of an E2E election need special attention.

% citation information, such as it is, included below:
% "Saltzer, Reed, and Clark [1981]":https://dl.acm.org/citation.cfm?id=357402
% "Jones [2002]":http://homepage.cs.uiowa.edu/~jones/voting/west02/
% "Adida [2006]":http://hdl.handle.net/1721.1/38302
% "Lowry and Vora [2009]":http://csrc.nist.gov/groups/ST/e2evoting/documents/VORA_DesirableProperties_101309.pdf
% "Kremer et al [2010]":http://dx.doi.org/10.1007/978-3-642-15497-3_24
% "Popoveniuc et al [2010]":https://www.usenix.org/legacy/event/evtwote10/tech/full_papers/Popoveniuc.pdf
% "Küsters et al [2010]":http://doi.acm.org/10.1145/1866307.1866366
% "Kremer and Ryan [2005]":http://link.springer.com/chapter/10.1007/978-3-540-31987-0_14
% "Chaum [1981]":http://dl.acm.org/citation.cfm?doid=358549.358563
% "Zagórski et al [2013]":http://link.springer.com/chapter/10.1007%2F978-3-642-38980-1_28#page-1Zagórski
% "Rosen et al [2011]":http://www.wombat-voting.com/how-to-vote
% "Benaloh [2006]":https://www.usenix.org/legacy/events/evt06/tech/full_papers/benaloh/benaloh.pdf
% "Benaloh [1985]":http://www.computer.org/csdl/proceedings/focs/1985/5428/00/542800372-abs.html
% "Chaum [2001]":http://web.archive.org/web/20041214134006/http://www.vote.caltech.edu/wote01/pdfs/surevote.pdf
% "Hubbers et al [2004]":https://pms.cs.ru.nl/iris-diglib/src/getContent.php?id=2004-Hubbers-VotingAction
% "Chaum, Ryan and Steve Schneider [2005]":http://link.springer.com/chapter/10.1007/11555827_8
% "Kiayias et al [2014]":http://www.demos-voting.com/demos_-_elektronike_psephophoria/Europaikes_Ekloges_2014_files/euro2014.pdf
% "Popoveniuc and Hosp [2006]":http://popoveniuc.com/papers/popoveniuc_hosp_punchscan_introduction.pdf
% "Popoveniuc [2010]":http://link.springer.com/chapter/10.1007/978-3-642-12980-3_15
% "Crockford 2002":http://www.crockford.com/wrmg/base32.html
% "Gonggrijp et al, 2009":http://link.springer.com/chapter/10.1007%2F978-3-642-04135-8_10
% "OSCE 2007":http://www.osce.org/odihr/elections/netherlands/24322?download=true
% "Bismark et al, 2007":http://www.researchgate.net/publication/224136367_Experiences_Gained_from_the_first_Prt__Voter_Implementation
% "Burton et al, 2012":https://www.usenix.org/system/files/conference/evtwote12/evtwote12-final9_0.pdf
% "Essex, Clark, Carback and Popoveniuc, 2007":http://www.punchscan.org/papers/pip_essex.pdf
% "Chaum et al 2008":https://www.usenix.org/legacy/event/evt08/tech/full_papers/chaum/chaum.pdf
% "Chaum et al 2009":http://ieeexplore.ieee.org/xpl/login.jsp?tp=&arnumber=5290135
% "Carback et al 2010":https://www.usenix.org/legacy/events/sec10/tech/full_papers/Carback.pdf
% "Adida 2008":https://www.usenix.org/legacy/events/sec08/tech/full_papers/adida/adida.pdf
% "Adida et al, 2009":https://www.usenix.org/legacy/event/evtwote09/tech/full_papers/adida-helios.pdf
% "Carter Center, 2013":http://www.cartercenter.org/resources/pdfs/peace/democracy/Carter-Center-Norway-2013-study-mission-report2.pdf
% "VoteBox":http://www.usenix.org/events/sec08/tech/sandler.html
% "Bulens et al":http://www.usenix.org/events/evtwote11/tech/final_files/Bulens.pdf
% "Tsoukalas et al":https://www.usenix.org/conference/evtwote13/workshop-program/presentation/Tsoukalas
% "Cortier et al":http://hal.inria.fr/docs/01/01/84/47/PDF/rapport.pdf
% "STAR-Vote":https://www.usenix.org/conference/evtwote13/workshop-program/presentation/bell
% "Gjøsteen 2012":http://link.springer.com/chapter/10.1007/978-3-642-32747-6_1
