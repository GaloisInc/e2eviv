\chapter{Crypto Specification\ifdraft{(Aggelos, Joe) (100\%)}{}}
\label{chapter:crypto_spec}

As no existing E2E-VIV system, nor E2E protocol, fulfills the
requirements set forth in this report, we cannot provide a full
cryptographic system or protocol specification.  The development and
verification of such a specification will be one of the primary
deliverables of a phase two of this project.

In order to frame that speculative future research, a formalized ideal
functionality for an E2E system is a useful foundation for examining
and comparing E2E protocols. Consequently, this chapter provides such
a foundation.  

This chapter also discusses the appropriate cutting-edge technologies
that should be used to mechanize and verify E2E protocols and their
cryptographic algorithms. Mechanization and formal verification is
mandatory for this work. E2E protocols are of critical import for
public elections, as national digitally supported elections are a
critical system that directly impacts national security.

\section{Crypto Specification}

Cryptographic specifications are typically written ``on paper'' in
peer-reviewed articles. With increasing frequency, algorithms and
protocols are mechanized, either within general-purpose logic
frameworks such as Coq, or in specialized environments such as
EasyCrypt. The foundations discussed in the following sections has not
yet been mechanized, but must be as a first step toward any future E2E
protocol work. A discussion of the modern best-practices for such
mechanization and verification is included in the conclusion of this
chapter.

\subsection{Ideal Functionality of an E2E System}

In this section we introduce a template for expressing the core of an
E2E verifiable system as an ideal functionality. An ideal
functionality is an abstraction that expresses the I/O interfaces of
the system with the parties that are involved in the e-voting process
as well as the way the system is supposed to react on such
inputs. Furthermore, the ideal functionality specification includes an
I/O interface with the adversary that expresses precisely the type of
information that is leaked to the adversary (the output channel of the
interface) and the level of influence the adversary may have on the
actions taken by the functionality (the input channel of the
interface).

The ideal functionality is supposed to operate in an ``ideal world''
where parties have direct access to its interfaces. This means that
the adversary is not able to block or high-jack the communication of
parties to and from the ideal functionality. The only way for the
adversary to interfere is through its own interface. This emphasizes
that the ideal functionality has precedence over the adversary in the
ideal world. Contrary to that, in the “real world” such an ideal
functionality does not exist and hence parties have to resort in the
execution of a protocol that intends to implement the ideal
functionality in reality. The conditions under which a protocol can be
said to realize an ideal functionality are explained below.

The intent of the ideal functionality is to express succinctly and in
tandem all the required properties of a system. A protocol is said to
realize an ideal functionality if it is possible to translate any
attack in the real world to an attack in the ideal world in a way that
no matter how the system is operated by the parties, it is impossible
to achieve any distinguishing effect between the two worlds. The
hallmark of a safe implementation of the ideal functionality is
exactly this indistinguishability property. Establishing it requires a
“security proof” that is constructive and algorithmic in nature. Given
any real world adversary, an ideal world adversary (usually referred
to as the simulator) is constructed that achieves the above
indistinguishability property.

An ideal functionality operates in the context of an ideal world
execution, a simulation that involves the following parties: the
functionality itself, the environment and the adversary. The
environment is the main driver of the execution that describes the
sequence of actions that take place in the interfaces of the ideal
functionality. It is helpful to think of these actions as serialized
however formulations of concurrent such executions are also
possible. The environment is not concerned with how the ideal
functionality operates as it only provides input and receives output
from the interfaces of the functionality. At the same time the
environment communicates with the adversary in some arbitrary
unrestricted fashion (no assumptions are made about the interface of
the environment with the ideal functionality). The latter property is
essential to ensure the arbitrary composability of the election system
within larger systems (that may involve other components such as
deliberation platforms and so on). If the interface between
environment and adversary is specified and restricted in some sense
then it is typically the case that realizing the functionality becomes
simpler however this may be done at the expense of sacrificing the
composability of the protocol.

For our objective, we will provide only a template for an ideal
functionality for E2E voting. This emphasizes the fact that further
research will be required to establish a precise formulation of this
ideal functionality and furthermore there could be many different
versions of the ideal functionality capturing similar but potentially
different facets of the e-voting design problem.

\subsubsection{Interfaces of the Ideal Functionality}

The E2E ideal functionality has a number of interfaces that are
described below. The interfaces are given to parties that the ideal
functionality is able to identify. Some interfaces may be given to any
party (without the functionality being concerned about the identity of
this party).

The administrator interface is the interface that is given to the
operator of the election system. This interface enables the
administrator to setup an election, an action that involves the
description of the ballot together with any restrictions and
constraints that need to be applied to the ballot casting phase (e.g.,
how many and what choices are valid per question and so on). The
initialization message should specify the type of election function
(e.g., a plurality vote) that should be applied to the inputs
collected from the voters as well as the list of eligible voters which
should be a subset of the parties that the functionality is able to
identify. The administrator is also responsible for opening and
closing the polls.

The voter interface is the interface that is provided to the
voters. It allows a voter to cast a vote for the election that is
controlled by the ideal functionality as long as the election is
open. The interface is also able to return to the voter some feedback
information that will enable the voter to verify that her vote has
been correctly recorded and included in the final tally.

The auditor interface is the interface that is provided to any
interested party and it allows the auditing of the election result and
of the election process in general. The auditor interface makes public
all information about the election including the list of eligible
voters. Furthermore after the closing of the polls, this interface,
given the feedback of a voter, it will enable a response on whether
the vote that has been recorded by the ideal functionality matches the
original intent of the voter.

The adversary interface enables the adversary to learn and influence
the way the ideal functionality operates in a number of ways. First it
enables the adversary to corrupt any party that is involved in the
process. Corrupted parties are assumed to be under the control of the
adversary. The identities of corrupted parties are kept by the ideal
functionality, which may modify its operation depending on which
parties are corrupted. Most importantly, in case of a corrupted
administrator, the ideal functionality will allow the modification of
the voters’ submitted votes.

\subsubsection{Ideal Functionality}

We now describe a general template describing how the functionality
reacts when given input in any of its available interfaces. As
discussed above, the description we give here should be interpreted as
a general guide for expressing the syntax and properties of an ideal
functionality for an E2E verifiable e-voting system rather than the
definitive final formulation of such functionality. Using the below as
a basis a number of different functionalities may be derived that
share the same interfaces but differ slightly in the way they operate
when given inputs to one or more of their interfaces.

The functionality recognizes a number of parties some of which are
given the special role of administrator. Furthermore, it is
parameterized by a predicate $P(.,.)$ that determines the precision of
the functionality i.e., how sensitive it is to adversarial
modifications in the final tally compared to the tally that is
calculated based on the recorded votes. Intuitively the $P(.,.)$
predicate captures the fact that we may implement an E2E voting
procedure via a protocol that cannot prevent with overwhelming
probability an adversary from switching a handful of votes. Note that
absolute precision can be achieved by setting $P(.,.)$ to be the
equality predicate; in any case we require that for any $x,y$ if $x=y$
then it holds that $P(x,y)$. We proceed to the description of the
actions taken by the functionality.

\begin{itemize}
\item Given an initialization message in the administrator interface
  it will parse it to extract the list of eligible voters and the
  description of the ballot. It will forward the initialization
  message to the adversary. Assuming the adversary enables it, will
  check that the list of eligible voters is a subset of the list of
  parties it can identify and it will respond with success to the
  calling party. Otherwise it will respond with failure.
\item Given an open polls message in the administrator interface it
  will switch its internal state to accepting votes after taking
  permission from the adversary.
\item Given a cast vote message in the voter interface it will parse
  it to extract a choice for the election's questions. Then, assuming
  that the administrator is not corrupted, it will notify the
  adversary about it without communicating the choices of the voter;
  In case the administrator is corrupted the choices of the voter will
  be provided to the adversary. When activated, the adversary will
  decide whether a response back to the voter is to be provided; Given
  permission, the ideal functionality will check that the voter is
  among the eligible voters and it will check that the vote is valid
  given the election's definition. If this is the case it will store a
  record with the voter's identity and the choices she selected. Then,
  it will request the feedback for the voter to be specified by the
  adversary. This feedback will be returned to the voter in order to
  signify that the ballot-casting submission has been accepted. The
  adversary is free to provide the feedback string however the
  functionality will restrict it to be unique: the adversary will not
  be allowed to use the same feedback string twice. In any other
  respect, the precise structure of this string is entirely left to
  the discretion of the ideal world adversary. The feedback string is
  appended to the voter record and is kept in the local state of the
  ideal functionality.
\item Given a close polls message in the administrator interface the
  functionality will switch its internal state not to accept votes
  anymore after taking permission from the adversary.  Then, it will
  calculate the final result based on the records that are kept for
  the eligible voters that have voted. It will forward the tally to
  the adversary. This will be called the {\em calculated tally}. The
  adversary will respond with a possibly modified tally that will be
  referred to as the {\em final tally}. If the administrator is honest
  the final tally will be forced by the functionality to be equal to
  the calculated tally (independently of what the adversary
  suggested). Both calculated tally and final tally will be recorded
  in the local state of the ideal functionality.
\item Given a read tally message in any interface it will forward the
  request to the adversary. Assuming the adversary enables it, it will
  return the final tally (note that this may be different from the
  calculated tally in case the administrator is corrupted). The
  functionality may also reveal together with the final tally the list
  of feedback information given to each eligible voter in association
  with the voter identities.
\item Given a modify voter record message in the adversary interface,
  the functionality will parse it to extract the voter's choice. Then,
  assuming the functionality has been notified earlier that the
  administrator is corrupted, that the polls are still open and that
  the alternate voter's choice is valid given the election's
  definition, it will change the voter's choice in the table where it
  keeps the voter records making a note that the voter's choice has
  been switched maliciously.
\item Given a verify election record in the auditor interface, the
  functionality will parse it to extract a voter feedback string. It
  will forward this message to the adversary. If the adversary enables
  it, it will attempt to identify the record of a voter that was given
  such string and then return to the auditor a single bit signifying
  whether the original voter intent has been tampered with or
  not. Furthermore, it will utilize the predicate $P(.,.)$ and it will
  pass to the predicate the calculated tally and the final tally. It
  will return to the callee the output of the predicate.
\end{itemize}

\subsubsection{Security Characteristics}

Using the template ideal functionality of the above section we next
argue how a number of required properties are captured. We examine the
properties individually.

\paragraph{Voter Privacy.}  The ideal functionality ensures voter
privacy up to a certain level by not disclosing the voter's choice
when this is given to the functionality in the voter interface
provided that the administrator is honest. We stress that voter
privacy is not absolute since the revelation of the calculated tally
to the adversarial interface when the polls close reveals some
information about the choices of the voters. In extreme scenarios
(e.g., only a single voter casts a ballot, a single honest voter
exists among adversarial voters or everyone votes in the same way) no
voter privacy remains after the tally is revealed to the adversary and
naturally no e-voting implementation can be expected to protect
privacy in these scenarios.

\paragraph{E2E Verifiability.} The verifiability that is provided by
the system is captured by the actions taken in the auditor
interface. The auditor can use the feedback provided by the system
after ballot casting to check whether the voter record has been
modified by the adversary. Observe that such a modification can happen
only in the case that the administrator is corrupted. It is easy to
see that if all voters audit their ballot in the auditor interface (or
delegate this task to a third party that performs it) it is guaranteed
that the $P(.,.)$ predicate holds between the calculated tally and the
final tally. In the other extreme, if nobody audits it is easy to see
that no guarantee whatsoever is given for the final tally, which may
deviate arbitrarily from the calculated tally. The intermediate
setting where voters perform auditing with some probability will
ensure the correctness of the tally in a statistical sense. We note
that this is one of the distinctions of this ideal functionality
compared to ideal functionality for ``secure multiparty computation''
which is a standard cryptographic notion and the output, if produced,
is guaranteed to be correct independently of the actions of the
adversary.

\paragraph{Receipt-freeness.} Receipt-freeness ensures that the voter
does not obtain any feedback from the system that can be used to
identify how she voted. Specifically, observe that, provided the
administrator is honest, the feedback string is generated by the
adversary and is independent of the voters' choices. It follows that
this feedback cannot carry any voter choice information. On the other
hand, if the administrator is corrupted observe that this property is
not preserved: the feedback may be chosen as a function of the voter's
choices and hence the receipt-freeness property cannot hold. Note that
if a voter is corrupted after her vote is cast, the template ideal
functionality will not divulge the voter record that it has kept in
its local state. This points to the fact that any implementation of
the ballot casting protocol should not leave traces from the ballot
casting stage that unequivocally bind the procedure to a specific
voter choice.

\subsection{Corruption Robustness}

The ideal functionality enables the corruption of any party in the
election process. However, it keeps track of the parties that are
corrupted and it still provides some security guarantees even in the
case they are corrupted. We examine the effects of corruption from
different perspectives.

\subsubsection{Corrupt Election Administrator}

A corrupt election administrator has a drastic effect in the way the
ideal functionality operates. Note that in the template ideal
functionality we express by a single entity the administrator of the
election. In a protocol implementation that realizes the ideal
functionality, the administrator may be implemented by a quorum of
parties/trustees that collectively share the responsibility of the
election administration. In such case, corrupting the administrator
would amount to corrupting a sufficiently large number of trustees
(the exact number is determined by the specifics of the
implementation).  When operating the ideal functionality in the
presence of a malicious administrator we observe that privacy and
receipt-freeness is lost and moreover the voter intent as captured by
the functionality can be modified by the adversary. Despite this, the
functionality still offers a faithful auditing step and informs the
auditor regarding the state of the voter intent as long as the auditor
provides the feedback that was obtained by the voter at the completion
of ballot casting. Note that the functionality enforces that the
feedback provided, uniquely identifies each voter and thus any auditor
that has at her possession proper feedback from a certain number of
honest voters is guaranteed to be able to check the status of an equal
number of voter records as preserved in the local state of the ideal
functionality.

\subsubsection{Corrupt Voters} 

A corrupt voter is a voter that is controlled by the adversary. Voters
may be corrupted at any moment either at the onset of the system
operation or even during the ballot casting process. It is expected
that an adversary controlling a certain number of voters is also able
to shift an equal amount of votes in the final tally. However, the
power of such an adversary is restricted to this level of capability
and corrupt voters by themselves are incapable of disrupting the
auditing process or jeopardizing the privacy of the other voters.

\subsubsection{Corrupt Implementations} 

A corrupt implementation in the context of an ideal functionality is
any implementation that fails to realize the ideal functionality. In
more detail, if an implementation is corrupt, this can be demonstrated
via the existence of a real world adversary such that no matter how it
is transformed into the ideal world there is an environment that
produces a sequence of actions that lead to a distinguishing event
between the real and the ideal world.

\subsection{Absent Security Properties}

In the template ideal functionality a number of security aspects are
not dealt with explicitly. The fact that are not directly addressed is
intentional as their inclusion would make the security specification
substantially more complex. In this way, the template functionality
provides a baseline for security that is a minimum threshold for
end-to-end verifiability and privacy.  Given that a scheme attains at
least the level of security suggested by the ideal functionality it
should then be analyzed with respect to these additional properties as
they can be desirable at least to a certain extent; whether a scheme
is suitable for deployment in a certain context may depend also on its
performance on these additional security characteristics.  We
elaborate on them below.

\paragraph{Denial of service attacks.} The template ideal
functionality enables the adversary to prevent voters completing the
ballot-casting protocol and may also prevent the tally from becoming
available. From a definitional point of view, expressing such level of
security is feasible by assuming certain qualities of the underlying
communication and message passing mechanisms that are employed in the
implementation. One way to extend the functionality to capture such a
setting is to oblige the adversary to deliver its messages by certain
deadlines; in such case the functionality may go ahead and deliver
messages in its output interfaces without requiring the adversary to
enable suc messages. In order to do this formally, a notion of time
will have to be introduced in the model. This may be achieved by
introducing a global clock functionality based on which such deadlines
can be expressed.

\paragraph{Coercion.} Even though the functionality does not permit
coercion via the voter feedback it provides, the adversary may still
achieve coercion by corrupting a voter (e.g., hacking into the voter's
PC) and assuming her role. Extending the model to handle such coercion
is feasible by mediating in a more strict way the way voter corruption
takes place.

\paragraph{Sybil attacks.} The set of voters is predetermined and
integrated into the functionality. Hence, the adversary cannot
manipulate the list of voters. It follows that the template
functionality is applicable to the setting where the list of voters is
predetermined, assumed to be public and the adversary may not tamper
with it. The setting where the adversary can manipulate the system via
the introduction of fake identities in the eligible voter list is not
explicitly addressed. Nevertheless, note that the functionality can
produce the list of identities that have participated in voting and in
such case it can be possible for auditors to perform a check in order
to verify whether the active voters correspond to real persons (e.g.,
by selecting a small random sample of them and performing an in-person
interview).

\paragraph{Privacy beyond the voters' choices.} The template
functionality as written leaks to the adversary a number of voter
behavior specific aspects including the time that a certain voter
casts a ballot as well as the final list of voters that have
participated in the election. As mentioned above, this is also a
security feature as it enables the auditors to identify the list of
active voters and validate their participation. Increasing privacy and
reducing auditability by hiding the list of active voters is possible
in the model.

\paragraph{Accountability.} As written, the template functionality
does not provide any mechanism to resolve disputes between sets of
voters that claim their voter records have been compromised and an
election administrator that claims the particular set of voters
intends to disrupt the election by making false claims. Such dispute
resolution mechanisms may be implemented judicially outside the
security model taking into account threshold conditions (e.g., if the
number of complaints is below a threshold then disputes are resolved
in favor of the election administrator; note that this will affect the
verifiability of the election in a statistical sense). However, it is
also possible to enhance the model with accountability by having the
ideal functionality present some information about the corruption
state of parties (currently no such information is divulged to the
auditor).

\section{Contextual Analyses of Primary E2E Protocols}

In this section we overview a list of candidate e-voting systems from
the perspective of privacy and verifiability in the context of the
template ideal functionality. Further investigations will be able
determine whether the schemes below (or close variants) are capable of
realizing some instantiation of the template ideal functionality.  A
common characteristic in all the systems we list below is the
existence and general availability of a public-bulletin board that
maintains unambiguously the transcript of the election. This is the
transcript that will be used by the auditors to validate the election
result together with feedback information collected from the voters.
We remark that the above are in some sense minimal assumptions. If the
voters are not able to have a unique view of the election transcript
then it is impossible to have verifiability overall and the the
template ideal functionality would immediately disqualify a system
where an adversary can disrupt the voters' unique view of the election
transcript.

\subsection{Demos}

Demos is a system that was recently put forth in \cite{kiayias2014}. The
central construction idea views the administrator as operating in a
three move protocol known as a Sigma protocol. At the initial stage,
the administrator precomputes a sequence of encodings that contain all
possible ways that a voter can vote in an election and posts them in a
public bulletin board.  The administrator then distributes ballots
that enable voters to cast their vote by essentially pointing to
specific encoded information in the public bulletin board. The
election terminates by having the administrator produce a proof in the
bulletin board that the tally computation is correct.

A key feature of the system is that the administrator precomputes for
each voter two equivalent ways to cast a ballot and the voter is free
to choose at random one of the two (say ``A'' or ``B''). While the A-B
decision does not affect the way the voter interacts with the system,
it introduces voter-side entropy that is independent of the voter's PC
or device and can be collected and utilized to ensure the calculation
of the tally is correct. Taking advantage of this feature DEMOS is
capable of producing a proof that unequivocally ensures the result is
correct (without having to rely on any additional assumption beyond
the availability of the election transcript and the feedback from the
voters).

From the implementation point of view, the voter interface requires
the ballot encoding information. Given this information, ballot
casting is straightforward as it is only needed to submit the
corresponding pointer to the bulletin board. Note that the pointer
itself does not violate the privacy of the voter since it merely
points to an ciphertext that hides the actual choice under a
cryptographic computational assumption (specifically, a variant of the
Decisional Diffie Hellman assumption).

The verifiability of the system hinges on the fact that the election
administrator will have to prove that the tally is correct by opening
any of the unused ballot encodings in the public bulletin board. In
order to attack verifiability effectively the administrator will have
to guess the choice made by the voters in the A-B decision. While this
may be feasible for just a handful of voters, it is shown in
\cite{kiayias2014} that the probability of not getting caught drops
exponentially in the distance of the deviation of the final tally from
the calculated tally (in our present terminology).

Demos (or a Demos variant) has potential has the potential to realize
some useful instantiation of the template ideal functionality given
the proof arguments presented in \cite{kiayias2014} assuming there is
sufficient entropy in the A-B decision that is performed by the
voters.

\subsection{Helios}

Helios~\cite{adida2008} culminates a long sequence of previous works
that used client side encryption combined with a ciphertext processing
stage that breaks the connection between the identity of the voter and
her choices during ballot casting. The voter casts a ballot by
utilizing a public-key encryption operation under a key that is
provided by the election administrator. This key is typically
generated by a quorum of trustees who collectively control the
decryption operation under a certain threshold condition.

The verifiability features of such a system are based on two
mechanisms. First, the voter is allowed to challenge her client side
encryption device (be it a PC, smartphone or other system). This is
performed at a moment where the device has produced a ciphertext; the
voter is asked whether to cast it or audit it. In case of an audit the
coins used for producing the ciphertext are presented and the voter
obtains a transcript of information that can be used to verify that
the ciphertext is properly constructed. Note that such information
cannot be verified visually or by hand and the voter is required to
keep this information and verify it using a second device (in order
for the check to be meaningful this auditing device should run code
that is independent from the voter's main device). When the voter has
performed a number of audits and kept the results she can cast her
encrypted ballot. With each ciphertext that is produced a ``smart
ballot tracker'' information is provided as well, that is the hash of
the ciphertext. This can be used by the voter to verify that the
ciphertext that was generated is the same one that appears in the
bulletin board.

Helios utilizes non-interactive zero-knowledge (NIZK) proofs for (i)
ensuring that the encrypted votes are of the proper form, and (ii)
ensuring that the decryption is performed properly by the
trustees. The presence and verification of those NIZKs is essential
for the verifiability and privacy aspects of the system.

Overall, auditing in Helios requires (i) the verification of the
artifacts of the cast-or-audit process, (ii) checking that the smart
ballot tracker corresponds to the hash of the ciphertext that is
posted on the bulletin board and finally (iii) verifying the
zero-knowledge proofs that ensure that the encrypted ballots are
properly encoded and that the ciphertext processing (be it homomorphic
or based on mix-nets) is properly executed.

The current implementation of Helios utilizes encryption with an
additive homomorphic property so that vote tallying can be executed
over the submitted ciphertexts. An alternative implementation that
utilizes mix-nets for ciphertext processing has been proposed
\cite{tsoukalas2013}.

Helios (or a Helios variant) has the potential to realize some useful
instantiation of the template ideal functionality given that it can
satisfy end-to-end verifiability assuming the voters perform the
audit-or-check procedure following a probabilistic strategy. It should
be noted that such proof will require the validity of the NIZK
components that in the present system formulation relies on an
abstraction called the random oracle model.

\subsection{Norwegian System}

The Norwegian system \cite{gjosteen2012} is based on a protocol developed
by Scytl. The administrator will generate for each voter a list of
receipt codes, one for each possible choice. These are handed to the
voters ahead of time. The voters submit their encrypted vote to the
administrator. Using a special ciphertext processing operation it is
possible for the administrator to extract from each ciphertext the
code that corresponds to the choice of the voter and submit it back to
the voter via an independent channel (e.g., if voting takes place
using a PC the code can be transmitted back to a smartphone). This is
argued to ensure the voter that the system has correctly recorded her
intent without violating the privacy of the voter. The reason for that
is that the code itself is pseudorandomly dependent on the choice of
the voter and it is independently selected for each voter. Thus, the
code by itself, reveals no useful information about the choice of the
voter. At the same time, the ciphertext processing step ensures that
the server calculating the code has to apply the proper function and
extract the proper value while it is hard to guess an alternative
value (given the pseudorandomness of the codes).

When the voter receives the code she can compare it with the
information that is available to her and thus verify that the system
correctly recoded her intention.

Contrary to Helios or Demos, verifiability in this system can only
hold provided that the adversary does not control the servers and the
device used by the voter to cast the ballot. Indeed, in such case it
is possible that the system deceives the voter that the proper choice
was recorded while something else has been recorded instead. It
follows that the verifiability guarantee for the Norwegian system is
weaker (note that the template functionality can accommodate such
weaker guarantees by suitably restricting the number and type of
corrupt parties in the real execution; nevertheless, such restrictions
would be preferable to be avoided if possible).

\subsection{Remotegrity}

Remotegrity \cite{zagorski2013} is a front-end system for Internet
voting that can be combined with a back-end (such as Scantegrity-II
\cite{chaum2008scantegrity} ) to offer a complete e-voting system with
end-to-end verifiability guarantees.

During initialization the system administrator precomputes a number of
encoded ballots as well as commits to a properly formed set of tables
of commitments in a public bulletin board. For each candidate choice
there is a unique vote-code that is assigned to each voter and the
tables commit to a correspondence between vote-codes and
candidates. During the cast protocol the voter can select to audit or
cast a vote. Audited ballots are spoiled and their correspondence in
the tables will be revealed at the end of the process. Cast ballots on
the other hand record a certain vote-code and the corresponding
election choice will be revealed in a way that the correspondence to
the voter will remain hidden.

This is achieved via the interaction of two tables of commitments that
become partially open depending on a source of public and verifiable
randomness.

Assuming that the adversary is incapable of biasing the independent
random source that provides the challenge for the partial opening of
the commitment tables it is possible to argue that the system
satisfies a level of verifiability without sacrificing privacy.

Based on this, the system has the potential for privacy and
verifiability only in case of an adversarial environment where the
randomness used to challenge the administrator remains unbiased and
outside the control of the adversary. This is tight, as it is easy to
see that in case the adversary completely controls the randomness used
in the challenge stage she can produce any tally that is independent
of the voter's intent.

Regarding the randomness used there are several proposals for natural
or human public phenomena that have been proposed for use in this
context (for instance, stockmarket end of day quotes, weather
measurements are potential candidates). We stress that using such
sources of randomness is not straightforward as the randomness quality
is limited and a proper extraction of uniform randomness needs to be
applied.

\subsection{RIES}

The Rijnland Internet Election System (RIES) was used between
2004-2006 for elections in Dutch District Water Boards and in 2006 for
the Dutch expats in the parliamentary elections. The system is simpler
than the systems examined above.

The administrator produces a public-key encryption pair and one
symmetric encryption key for each voter. Subsequently voters can
submit vote-codes that are calculated as functions of their symmetric
encryption key. The administrator who knows all symmetric encryption
keys publishes a table of hashes on the vote-codes used by the
voters. A voter interacts with her device providing the symmetric
encryption key. The device prepares the vote-code (which is computable
using the vote-code) and maintains that vote-code as a receipt for the
voter. Meanwhile the administrator decrypts all votes and posts them
to the bulletin board in some predetermined order (that also depends
on the symmetric encryption key).

Note that it is possible to calculate the final tally using the
information that is put in the bulletin board by matching them with
the hashed vote-codes. However, the system is not a candidate for
realizing a reasonable variant of the template ideal functionality
given that it produces receipts that unequivocally identify the way
that voters choose their selection during ballot-casting. Indeed, the
system is not receipt-free and hence cannot be used as a reasonable
implementation of the ideal functionality.

\subsection{Wombat}

The wombat system \cite{ben2012new} falls in the same category as Helios
and Zeus with the exception that it uses an onsite ballot-casting
setup that also preserves the original voter intent in paper. The
voters generate ciphertexts with the aid of a device with a printer
that presents the voter choice both in ciphertext form as well as in
plaintext form. The voter may choose to audit the combination using
the same exact logic as the audit or cast mechanism described above in
the Helios section.  When the voter decides to cast the ballot, she
tears the plaintext part and places it in a ballot box. The ciphertext
part is scanned and published in a bulletin board.

The voter retains the encoding of the ciphertext that was cast as a
fingerprint to match it with the information in the bulletin board
(this plays the same role as the smart ballot tracker in the case of
Helios).

In order to preserve the privacy of the voters a mix-net final step is
employed to ensure that ciphertexts become disassociated from the
voters themselves. Overall the same verifiability and privacy
arguments made in the case of Helios would apply here (while taking
into account the fact that wombat is an on-site system).

\newcommand{\PaV}{Pr{\^{e}}t {\`{a}} Voter}

\subsection{vVote/\PaV}

The vVote system is an adaptation of the \PaV system for the
Australian state of Victoria elections \cite{culnane2014vvote}. It retains the
general structure of \PaV \cite{ryan2009}
while it modifies the way the encryption of voter choices is performed
so that it suits the style of elections of Victoria.

Recall that \PaV paper ballots are divided into a left and
right hand side. Each ballot contains a permutation of the choices
that are available to the voter. The voter is invited to enter her
choices in the right-hand-side while the names of the candidates
appear (randomly permuted) in the left-hand-side. After marking, the
left-hand-side is destroyed while the right-hand-side is scanned so
that it is posted in the bulletin board as well as it is kept as a
receipt. The right-hand-side contains also the encoding of a
ciphertext that enciphers the permutation that was used in the ballot.

Verifiability can be based on an audit-or-cast process on the printed
ballot and the verification of the process in the bulletin
board. Specifically, the voter can choose to spoil a ballot and reveal
the encryption of the permutation which will enable the comparison of
the encrypted permutation to the one physically shown in the the
ballot. The voter can also track her ballot in the bulletin board
(recall that the right hand side contains an encoding of the
ciphertext). Finally the result is revealed after a mix-net operation
which includes zero-knowledge proofs of knowledge to ensure that the
mixing process is done correctly (namely, no ballots are substituted
or removed at each mixing step).

In principle the logic under which verifiability can be argued for
vVote/\PaV is similar to that of Helios and Wombat.

\section{Realizing Ideal Functionality}

In this section we describe some general directions and tools
regarding the realization of the ideal functionality. We start with
the description of relevant cryptographic tools and then we move on to
describe how it is possible to mechanize the security proof of
realizing the ideal functionality.  A case study for one system is
presented. Finally the case for open protocols and software
independence is made.

\subsection{Commonly Used Cryptographic Tools}

A number of cryptographic tools play an important role in the design
of e-voting schemes. We provide a list of some of these tools that
have found use in the e-voting design of the systems we overviewed in
the previous section.

\paragraph{Commitment schemes.} A (non-interactive) commitment scheme
is comprised of two algorithms $(\mathrm{Commit}, \mathrm{Verify})$
and enables a party to produce a pair
$(\psi, \rho) \leftarrow \mathrm{Commit}(m)$ so that $\psi$
``commits'' to the value $m$ without revealing much information about
$m$. Specifically, a party can commit to $m$ by sending $\psi$ to a
public-bulletin board. At a later time, the party can reveal $m,\rho$
and any interested party can verify that indeed the process was done
correctly by running the predicate $\mathrm{Verify}(\rho, m, \psi)$
and accepting the opening provided that the predicate returns true.
It is possible that the two algorithms are parameterized by a public
parameter $p$ that is available to all parties. In this case an
algorithm may be added in the description that generates the parameter
$p$. Depending on the type of commitment this algorithm may be
required to be executed honestly. The security model requires that
commitments are hiding, i.e., $\psi$ does not reveal any information
about $m$, and binding, i.e., it is infeasible for the party that
makes a commitment to $m$ to equivocate it and open it to a different
value $m'$.

\paragraph{Public-key encryption.} A public-key encryption is
comprised of three algorithms
$(\mathrm{Gen},\mathrm{Enc}, \mathrm{Dec})$.  The algorithm
$\mathrm{Gen}$ produces a pair of public and secret-key denoted by
$(pk,sk)$ while $\mathrm{Enc}(pk, \cdot )$ on input plaintext $m$ it
samples a ciphertext $\psi$ that corresponds to $m$. Finally, the
algorithm $\mathrm{Dec}(sk, \cdot )$ returns the plaintext that
corresponds to the input $\psi$. The security model requires at
minimum that an adversary that is given the public-key $pk$, is
incapable of distinguishing between two ciphertexts that are sampled
corresponding to different plaintexts, even if such plaintexts are
adversarially chosen.

\paragraph{Additively homomorphic encryption.} A homomorphic
(public-key) encryption scheme adds the following property to the
$\mathrm{Enc}(pk, \cdot)$ algorithm.  First the space of plaintexts is
endowed with a group structure over a binary addition operation
$+$. Then, given $\psi_1 = \mathrm{Enc}(pk, m_1)$ and
$\psi_2 = \mathrm{Enc}(pk, m_2)$ it is possible to compute a
ciphertext $\psi$ that belongs to the encryptions
$\mathrm{Enc}(pk, m_1 + m_2)$. Furthermore, we also require that if
at least one of $\psi_1, \psi_2$ is sampled uniformly over all
ciphertexts then the output $\psi$ is following the uniform
distribution over all ciphertexts of $m_1 + m_2$.

Additive homomorphic encryption enables processing of ciphertexts in
various ways that are useful in e-voting systems.  Two important uses
of such schemes are as follows: (i) Given $k$ ciphertexts
$\psi_1, \ldots, \psi_k$ encoding $v_1, \ldots, v_k \in \{0,1\}$ it is
possible to derive a single ciphertext $\psi$ that encodes
$T= \sum^k_{i=1} v_i$. Such $T$ can be used to derive the tally of an
election in case each plaintext $v_i$ corresponds to an election
choice made by the $i$-th voter.  (ii) Given
$\psi = \mathrm{Enc}(pk, m)$ it is possible to ``refresh'' the
randomness of $\psi$ by processing $\psi$ together with
$\mathrm{Enc}(pk, 0)$. The resulting ciphertext is uniformly
distributed over the ciphertexts encoding $m$.

\paragraph{Secret-sharing.} A secret-sharing scheme is comprised of
two algorithms $(\mathrm{Gen}, \mathrm{Rec})$ that operate as
follows. Given parameters $(n,t)$ and a value $s$ the algorithm
$\mathrm{Gen}$ produces $n$ ``shares'' $s_1, \ldots, s_n$. The
algorithm $\mathrm{Rec}$ given any subset of size at least $t$ from
the set $\{s_1, \ldots, s_n\}$ reconstructs the value $s$.  The
security guarantee that we require for a secret-sharing scheme is that
if the adversary has any set of size less than $t$ values from
$s_1,\ldots, s_n$, it should be incapable of finding any non-trivial
information about $s$.  The value $t$ is called the threshold of the
secret-sharing scheme.

An additional property for secret-sharing schemes is verifiability,
where the reconstruction algorithm is capable of detecting shares that
are incorrect based on some public information that is provided by the
$\mathrm{Gen}$ algorithm.

\paragraph{Threshold encryption.} A threshold (public-key) encryption
is equipped with a multiparty protocol that implements the
$\mathrm{Gen}()$ public/secret-key generation algorithm between a set
of parties. Specifically, this protocol is parameterized by two
integers $(t,n)$ and enables a set of $n$ parties (sometimes called
trustees) to produce the value $pk$ as well as a secret-sharing of the
value $sk$ with threshold $t$. The shares of the secret-key $sk$ are
kept by the trustees and can be revealed when it is time to decrypt a
ciphertext.

A threshold encryption scheme is also capable of achieving threshold
decryption if there is a protocol that implements the algorithm
$\mathrm{Dec}()$ by the trustees so that each one of them uses her own
share of the secret-key $sk$ but is not required to reveal it.
Threshold decryption is particularly important in practice since it is
typically desirable to apply the decryption function $\mathrm{Dec}$
selectively only to specific ciphertexts that may be identified by the
system that utilizes threshold encryption.

The property required by a threshold encryption is similar to that of
secret-sharing.  The adversary should be incapable of finding any
information about a target ciphertext if it controls less than $t$
trustees. Importantly, the adversary should be incapable of doing so,
despite the fact it may be participating on various threshold
decryption operations running concurrently on ciphertexts other than
the target ciphertext.

\paragraph{Zero-knowledge proofs.} A zero-knowledge (ZK) proof is a
protocol between two parties, called the prover and the verifier, that
is associated with a language $L = \{x \mid \exists w : R(x,w)\}$
where $R$ is a polynomial-time predicate in a parameter $k$ and $x,w$
are strings of length $k$.  The protocol enables the prover to
convince the verifier that she is in possession of a ``witness'' $w$
regarding the fact that $x\in L$.

The properties required by a ZK proof is completeness, soundness and
zero-knowledge.  Informally, completeness requires that the verifier
accepts an honest prover, soundness expresses that the prover cannot
cheat (e.g., that she cannot convince the verifier if she does not
know the witness) and zero-knowledge that the verifier cannot learn
anything significant about $w$ from interacting with the prover beyond
the fact that $x\in L$. Sometimes it may be the case that $x\in L$ is
a given fact and the protocol has the objective to demonstrate simply
that the prover is in possession of the witness $w$.  Protocols with
this property are called ZK proofs of knowledge.

An important variation of ZK proofs are non-interactive ZK (NIZK)
proofs, where the prover is capable of producing a string $\pi$ in one
move that by itself can convince the verifier regarding the status of
the statement $x\in L$ (or that the prover is in possession of the
witness $w$). A NIZK requires a public parameter $p$ that should be
honestly produced (independently of the prover and the verifier).

\paragraph{Mix-nets.} A mix-net is a protocol executed sequentially by
a set of servers on a set of ciphertexts that originate from a
public-key encryption scheme.  Specifically, a mix-net can be built
upon a single algorithm $\mathrm{Shuffle}()$ that is given the
public-key $pk$ as well as a sequence of ciphertexts
$\vec\psi = (\psi_1,\ldots, \psi_n)$.  The algorithm produces a
sequence of ciphertexts $\vec\psi' = (\psi_1', \ldots, \psi'_n)$ as
well as a ``proof'' $\pi$. The proof $\pi$ can be in the form of a
NIZK that ensures the following fact about the values
$(pk, \vec \psi, \vec \psi')$. Let $M$ be the sequence of plaintexts
that in the $i$-th position is equal to
$ \mathrm{Dec}(\psi_i) \rangle$ and similarly $M'$ the sequence of
plaintexts that in the $i$-position is equal to
$\mathrm{Dec}(\psi_i') \rangle$.  It should hold that there is a
permutation $\mu$ over $n$ elements such that the $j$-th element of
$M$ equals to the $\mu(j)$ element of $M'$ for all $j=1,\ldots, n$.

Given the $\mathrm{Shuffle}()$ algorithm observe that we can apply it
sequentially on the same sequence of ciphertexts. If a sequence of $m$
servers apply $\mathrm{Shuffle}()$ one after the other, it is easy to
see that the correspondence between the original sequence of
ciphertexts and the final sequence of ciphertexts will be hidden
provided that at least one server remains honest.

\subsection{Other Potentially Useful Cryptographic Tools}

There are several other cryptographic tools that have not yet seen
much application in the context of E2E protocol design. Several are
briefly summarized here, in part because they represent opportunities
for novel research directions in E2E protocol design. They also
described because some naive proposals for election systems include
the use of these tools, and should be identified and rejected.

\paragraph{Hashchains and blockchains.} A hashchain is a data store
that holds a ledger of sequential records in a cryptographically
secure fashion~\cite{HC}. Hashchains are commonly used to record a
sequence of causally dependent events with cryptographic integrity and
non-repudiation. Within the context of election systems, hashchains
are a common means by which to record privacy-preserving election logs
for post-election audit, construct digital ballot boxes, craft public
bulletin boards that contain evidence of an election's correctness and
security properties, and more~\cite{HC-for-elections}.

A hashchain $H$ is constructed by the repeated application of a
cryptographic hash function $h$. Linear hashchains are of the form
$H^{k}=h(x_{k} \otimes h(...h(x_{2} \otimes h(x_{1} \otimes h(\perp
\otimes \perp)))))$
where each (identical) hash $h$ is applied to a fusion $\otimes$
(often defined as xor) of the previous hashchain and a new ledger
element $x$. The bottom element $\perp$ can be one of a number of
potential root values (e.g., a prefix of the public key of a given
election), depending upon the context of the application of the
hashchain. 

Non-linear hashchains are trees of linear hashchains~\cite{NLHC}. They
are useful in the context of disconnected devices with causally or
temporally connected ledgers (e.g., disconnected DRMs used in a given
election)~\cite{NLHC-for-elections}.

A blockchain is a distributed data store that holds a ledger of
transactions in a cryptographically secure
fashion~\cite{blockchains}. It is a kind of distributed hashchain,
wherein multiple computers compute and communicate to determine what
the next data element of the hashchain/tree using a consensus
protocol. Blockchains often rely upon a cryptographic work factor to
determine consensus as well as to prevent manipulation of the ledger
by powerful adversaries.

Proposals to use blockchains for elections are
plentiful~\cite{blockchains-for-elections}, but have been shown to be
naive in most instances and inappropriate as a foundation for a public
election E2E protocol. The only reasonable proposal for the use of
blockchains for parts of an election protocol is from Clark and
Essex~\cite{clark2012commitcoin}.

\paragraph{Multi-party computation and linear secret sharing.} Secure
multi-party computation (MPC) is a collaborative privacy-preserving
computation technology.  MPC permits a (typically small) collection of
parties to compute a collaborative result without any parties gaining
any knowledge about the inputs provided by the parties, expect what
can be determined from the output of the
computation~\cite{chaum1988multiparty, goldwasser1987play}. 

Since systems compute on encrypted data, computation can take place on
public systems, such as in a public cloud. Moreover, since systems can
communicate using a published public protocol to collaboratively
compute, parties can be implemented by multiple, cooperating or
competing, organizations.

In the kind of MPC known as linear (or additive) sharing, computation
proceeds on data that appears entirely
random~\cite{Cramer00}. Certain operations, such as addition
or logical-XOR can be performed locally, but operations such as
multiplication or logical-AND require a network communication between
the parties. Consequently, the computational overhead of MPC is large,
and the cost is measured in orders of magnitude slowdown with respect
to computing in the clear.

However, efficiency improvements over the last few years have shifted
the potential applicability of MPC from just micro-benchmarks to
user-level applications, including some that have a data volume
comparable to elections. Consequently, there may be real opportunities
in the use of MPC for E2E elections.

\paragraph{xwunctional encryption.} Functional encryption (FE) is a
kind of public-key cryptography in which possessing a secret key
permits one to learn a function of what the ciphertext is encrypting,
and nothing more~\cite{boneh2011functional}.

More precisely, a FE scheme for a given functionality $F$ consists of
the following four algorithms:
\begin{enumerate}
\item $(pk, msk) \leftarrow Setup(1^\lambda)$: creates a public key
  $pk$ and a master secret key $msk$,
\item $sk \leftarrow Keygen(msk, k)$: uses the master secret key to
  generate a new secret key $sk$ for value $k$,
\item $c \leftarrow Enc(pk, x)$: uses the public key to encrypt a
  message $x$, and
\item $F(k, x) \leftarrow Dec(sk, c)$: uses secret key to calculate a
  function of the value $c$ encrypts.
\end{enumerate}

FE's primary use is in encrypted databases whose security properties
are determined, in part, by the computations permitted on its
encrypted data~\cite{Popa11}. Consequently, FE may prove useful in
the storing of, or computing on, election data like ballots, audit
logs, and more.

FE encryption generalizes several existing primitives including
identity-based encryption (IBE) and attribute-based encryption (ABE),
both of which may be useful in the context of authentication in E2E
protocols~\cite{shamir1985identity,goyal2006attribute}.

\paragraph{Fully homomorphic encryption.} Fully homomorphic encryption
(FHE) is a more powerful version of the aforementioned partial
homomorphic encryption (in the form of additive or multiplicative
homomorphic encryption) cryptosystem.  FHE permits arbitrary
computation on encrypted data, but is prohibitively slow (several
orders of magnitude slower than plaintext computation) and ciphertexts
are enormous (dozens to hundreds of megabytes in size).  

It is unclear if FHE has potential application in the context of E2E
protocols, but given the swelling interest in FHE and the availability
of open source implementations of FHE libraries, surly we will see its
applications, particularly in the context of cloud deployments of
election systems.

\paragraph{Verifiable computing.} Verifiable computing is a new area
of R\&D which focuses on the offloading of computation to an untrusted
system (say, an untrusted DRM evoting machine or a cloud hosting
service), and yet having a means by which to check that the
computation was performed correctly.

Several technologies exist to help verify that a computation performed
by untrusted workers is correct, including the use of secure
coprocessors, Trusted Platform Modules (TPMs), interactive proofs, and
more~\cite{king2013trusted,perez2006vtpm,sparks2007security}. These verifications are either interactive,
which require the client to interact with the worker to verify the
correctness proof, or are non-interactive protocols which can be
proven in the random oracle model.

The utility of verifiable computing in the context of E2E-VIV is
obvious, but the state-of-the-art is still far from being able to
handle the kind of data volumes mandated by public elections. It is to
be expected that new results in both verifiable computing, and E2E
protocols that presume the existence of efficient verifiable
computing, will be forthcoming.

\section{Formal Mechanization of Ideal Functionality}

As mentioned above, in order to contextualize E2E protocol designs,
especially to compare-and-contrast them from an objective framing, the
ideal functionality must be mechanized.

The premiere environments in which to perform such mechanization are
Coq, EasyCrypt, CryptoVerif, and Cryptol. Each choice has pros and
cons, and it is certainly a reasonable to consider mechanizing
protocols, and their dependent algorithms, in multiple environments.
In general, any framework chosen must be trusted by cryptographers,
open source, and have liberal licensing.

There are several other excellent useful environments that should be
considered for future mechanization work, such as higher-order
frameworks such as PVS, Isabelle, and HOL, and protocol-centric tools
such as F\*, CVK, and ProVerif, which we will not describe in detail
here.

\paragraph{Coq.} Coq is a general purpose logical framework based upon
the Calculus of Inductive Constructions. It has a small trusted core,
proofs are first-class constructs, and there are several libraries
available for reasoning about not only cryptographic algorithms and
protocols, but also the correctness of programs written in a variety
of languages. A significant amount of recent research focusing on the
verification of cryptographic algorithms and protocols has used Coq,
including Princeton's work on formally verifying SHA and HMAC,
Harvard's work on reasoning with the Foundational Cryptography
Framework, and IMDEA's early work on their precursor to EasyCrypt,
CertiCrypt.

\paragraph{EasyCrypt.} EasyCrypt is a toolset for reasoning about
relational properties of probabilistic computations with adversarial
code. Its main application is the construction and verification of
game-based cryptographic proofs. Initial applications of EasyCrypt
focused on encryption and signature schemes, but recent extensions
reason about the security of cryptographic systems, which achieve
specific functionalities through intricate combinations of several
primitives. These developments have expanded significantly the scope
of potential applications of EasyCrypt, as reflected in the recent
formalization of secure function evaluation and verifiable
computation. Moreover, they have enabled the formalization of examples
that were previously out of scope, for instance modular proofs of
security for key-exchange protocols and E2E protocols like
Helios. EasyCrypt was custom-designed by Barthe and his colleagues at
IMDEA and INRIA.

\paragraph{CryptoVerif.} CryptoVerif is a computationally sound
mechanized prover for cryptographic protocols created by Blanchet and
his collaborators. It has been used to reason about the secrecy and
correspondence properties of numerous protocols and gives a bound on
the probability of any attack.

\paragraph{Cryptol.} Cryptol is a domain-specific language for
specifying cryptographic algorithms. A Cryptol implementation of an
algorithm resembles its mathematical specification more closely than
an implementation in a general purpose language. Using a specification
in Cryptol, programmers can generate their own test vectors, prove
theorems, and (using other tools) verify equivalence to their own
programs, or even generate code or hardware from the specification.
Cryptol version 2 is open source and available under a 3-clause BSD
license. Galois's goal is that it becomes a widely adopted standard
for expressing cryptographic algorithms.

Our recommendation is to experiment with each of these alternatives,
formalizing ideal functionality and at least one E2E protocol. Based
upon that case study, an objective decision can be made about which
toolset to use in the long run. 

\subsection{Recommendations}

Based upon our experience, the most promising approach is to use Coq
to formalize systems and reason about implementations written in C
(using CompCert and VST), Cryptol to specify cryptographic algorithms
and simple protocols and reason about their, and their
implementations' (in LLVM or JVM), correctness, use EasyCrypt to
reason about algorithms' and protocols' security properties, and use
F\* to specify and reason about protocols that need a Javascript or
.Net implementations.

\section{Specification of Open Protocols} 

E2E protocols, while cryptographic protocols, are at their core, just
protocols. They specify the means by which different subsystems
communicate with each other. If we presume that different subsystems
may be implemented by different parties, and that multiple independent
implementations of critical components (such as tabulation and
verification subsystems) is mandatory, then the precise specification
of open protocols is critical.

The best practice for protocol specification is to formalize the
protocol using a precise mechanization, provide a reference
implementation, and provide a means by which any implementation can be
validated against the specification. Given the aforementioned
toolsets---ProVerif, EasyCrypt, CryptoVerif, and F\*---specifying the
protocols is a straightforward proposition. Verifing the protocol's
correctness and security properties is a much more challenging, though
not insurmountable, proposition. 

The real challenge is in verifying an implementation's
correctness. Full blown formal verification of implementations, while
possible, takes a significant amount of effort, typically several man
months of work. Moreover, if a specification is not written with the
intention of supporting a protocol under evolution, re-verifying
changes in an implementation of a protocol can require as much effort
as the original verification.

Rigorous validation of a protocol, on the other hand, takes
significantly less effort and is more flexible in the face of protocol
evolution. Recent work from Cambridge focusing on the rigorous
specification and implementation of TLS~\cite{nqsb-TLS} in a pure
functional style exemplifies the kind of reference implementation that
should be written for an E2E protocol.

\section{The Case for Software Independence}

A voting system is software independent if an undetected change or
error in its software cannot cause an undetected change or error in an
election outcome.  A way to achieve software independence is to endow
the voting system with the capability to record voter intent
physically in a way that is immediately verifiable by the voter at the
time of ballot-casting. Given such record, one would subsequently
perform a random audit to ensure that the published election outcome
(generated by software) does not deviate from the result that would
have been calculated if the physical record was used in the tallying
process.

A notable example of such procedure currently in place is in the state
of Connecticut where optical scan voting machines are used. The marked
paper ballots are retained after the election in each precinct. After
the end of the election when the tally is announced, a sample of
precincts is randomly elected at the state level and a manual hand
count is performed. The results of the hand count are audited and a
statistical analysis is conducted to confirm the e-voting tally at a
certain confidence level.  See \cite{antonyan2009state} for more
details regarding the procedure that is followed.

%connecticut-audit
%State-wide Elections, Optical Scan Voting Systems, and the Pursuit of Integrity – ieee.pdf
%IEEE TRANSACTIONS ON INFORMATION FORENSICS AND SECURITY, Volume: 4 Issue: 4 Part: 1, December 2009


%%% Local Variables:
%%% mode: latex
%%% TeX-master: "report"
%%% End:
