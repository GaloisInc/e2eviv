% Overview of the Hack-a-PDF Attack
% Galois, Inc.

\documentclass{article}
\usepackage{lucidbry}

\setlength{\textwidth}{6.5in}
\setlength{\oddsidemargin}{0pt}
\setlength{\evensidemargin}{\oddsidemargin}
\setlength{\textheight}{9in}
\setlength{\topmargin}{0pt}
\advance \topmargin by -\headheight
\advance \topmargin by -\headsep

\usepackage[pdftex,bookmarks=false,a4paper=false,
            plainpages=false,naturalnames=true,
            colorlinks=true,pdfstartview=FitV,
            linkcolor=blue,citecolor=blue,urlcolor=blue,
            pdfauthor="Daniel M. Zimmerman"]{hyperref}

\usepackage{color}
\usepackage{listings}
\usepackage{cite}
\usepackage{float}
\usepackage{xspace}

\definecolor{keywordcolor}{rgb}{0.5,0,0.33}
\definecolor{identifiercolor}{rgb}{0,0,0.75}
\definecolor{commentcolor}{rgb}{0.3,0.3,0.3} 

\def\sectionautorefname{Section}
\def\subsectionautorefname{Section}
\def\subsubsectionautorefname{Section}

\newcommand{\eg}{e.g.,\xspace}
\newcommand{\ie}{i.e.,\xspace}
\newcommand{\etc}{etc.\xspace}

\begin{document}

\title{Modifying an Off-the-Shelf Wireless Router for PDF Ballot Tampering}
\author{Daniel M.~Zimmerman and Joseph R.~Kiniry \\ Galois, Inc. \\
  421 SW 6th Ave., Suite 300, Portland, OR 97204}

\maketitle

\abstract{In order to highlight the dangers associated with Internet
  voting carried out over electronic mail with PDF forms, we show that
  an off-the-shelf home Internet can be easily modified to silently
  alter election ballots. The modification is nearly undetectable and
  can be carried out in a way that leaves no evidence to be found in a
  post-election investigation.}

\section{Introduction}

A number of governments, at various levels, have expressed interest in
the establishment of Internet voting systems. Examples include the
state of Alaska, which is carrying out an Internet voting trial for
the 2014 midterm elections; Washington, D.C., which in 2010 developed
an Internet voting pilot project for absentee voters that was
successfully attacked by an academic research group;~\cite{DCVoting}
and the nation of Estonia, which has had Internet voting since 2005
that exhibits significant security flaws.~\cite{EstoniaEVoting}

One mechanism proposed for Internet voting, which was used as a
fallback mechanism after the Washington, D.C. Internet voting system
was successfully hacked and removed from service and is also available
to current Alaska voters, involves ballots rendered as standard Adobe
Portable Document Format (PDF) forms. These forms are made available
to voters on a web site; the voters then use standard software (\eg
Acrobat Reader, Preview, \etc) to fill out the forms and submit the
completed forms via electronic mail to the appropriate election
authority.

Unfortunately, this mechanism is vulnerable to attack at several
levels: malicious software on the user's computer could modify or
invalidate a vote; a malicious election authority could intentionally
miscount (or simply ``lose'') received votes; malicious third parties
could masquerade as the election authority, or perform denial of
service attacks against the actual election authority to prevent votes
from being cast or to flood them with invalid ballots; and more. Here
we describe a more subtle attack at the transport level, which changes
the raw data traveling through the electronic mail system between the
voter's computer and the election authority.

\section{The Attack}

Our attack transparently and untraceably alters a ballot to change a
voter's choices once the ballot has been sent as an email attachment
to the election authority. The end result is that the voter believes
he has submitted a vote for some candidate, while the election
authority receives a legitimate-looking vote for another candidate.

\subsection{PDF Ballot Alteration}

The goal of our attack is to modify a PDF file such that it reflects a
different choice than the one made by the voter. We achieve this by
directly modifying the data within the PDF file to change the
vote. 

In a PDF form with radio buttons, each button has an object identifier
of one or more characters. The selected button is indicated in the PDF
with a string of the form ``\texttt{<{}< /V /ID}'', where
``\texttt{ID}'' is the identifier. If we want to change a vote for the
candidate with identifier ``\texttt{ID1}'' to a vote for the candidate
with identifier ``\texttt{ID2}'', it suffices to replace the string
``\texttt{<{}< /V /ID1}'' with the string ``\texttt{<{}< /V /ID2}'' in
the PDF file. In the event that the two identifiers do not have the
same length, the process is only slightly more complicated because we
need to maintain the length of the file (more specifically, of each
packet within the file) to carry out the attack transparently as the
file is transferred over the Internet.

Consider the case where one candidate (Candidate A) has identifier
``\texttt{Aardvark}'' and the other (Candidate Z) has identifier
``\texttt{Zebra}''. If we want to change a vote for Candidate A into a
vote for Candidate Z, we replace the string ``\texttt{<{}< /V
  /Aardvark}'' with the string ``\texttt{<{}< /V
  /Zebra\textvisiblespace\textvisiblespace\textvisiblespace}''; the 3
spaces at the end of our replacement string pad it to the same length
as the original string. These extra spaces are ignored by PDF readers;
though they might be noticed in a close inspection of the file, the
result is still valid PDF and could conceivably have been generated
when the voter initially filled out the form.

On the other hand, if we want to change a vote for Candidate Z into a
vote for Candidate A we must perform multiple string replacements to
change Candidate A's identifier.  First, we pick a new identifier to
represent Candidate A with the same length as (or a shorter length
than) the identifier for Candidate Z; in this case, perhaps we choose
``\texttt{Aard}''. Next, we replace the string ``\texttt{<{}< /V
  /Zebra}'' with the string ``\texttt{<{}< /V
  /Aard\textvisiblespace}'', and also replace two occurrences of
``\texttt{<{}< /Aardvark}'' with ``\texttt{<{}<
  /Aard\textvisiblespace\textvisiblespace\textvisiblespace\textvisiblespace}''. As
before, the extra spaces pad the strings to the same length as in the
original file and are ignored by PDF readers.

While a shortening of the chosen identifier is difficult to notice, a
lengthening of the chosen identifier (and the resulting wholesale
identifier change required to make it fit in the same number of
characters) is easy to spot. However, in any given PDF form, the
identifiers are most likely already the same length. For example, the
default identifiers assigned to buttons in a group when creating a
form in Adobe Acrobat Professional are ``\texttt{Choice1}'',
``\texttt{Choice2}'', \etc With a ballot form containing all
equal-length identifiers, as would be generated by a typical PDF form
generator, it would be effectively impossible to detect this sort of
tampering.

\subsection{Implementation}

The attack is carried out by modifying one of the TCP packets of the
email attachment somewhere along its route between the voter's email
client and the election authority. The modification replaces the byte
sequences inside the PDF file that indicate a selected form element,
as described above.

Since the PDF file is transmitted as an email attachment, the actual
strings replaced are Base64 encoded versions of the strings within the
PDF file. Base64, described in RFC 3548,~\cite{RFC3548} is the
standard encoding mechanism for sending binary files as email
attachments. It converts every three bytes of the binary file to four
6-bit fields. Each of these fields can take one of $2^6=64$ possible
values, and can thus be sent as a single printable character chosen
from a 64-character alphabet. 

Using our previous ``candidates'', we see that the Base64 encodings of
``\texttt{<{}< /V /Aardvark}'' and ``\texttt{<{}< /V
  /Zebra\textvisiblespace\textvisiblespace\textvisiblespace}'' are
``\texttt{PDwgL1YgL0FhcmR2YXJr}'' and
``\texttt{PDwgL1YgL1plYnJhICAg}'', respectively. Depending on the
offset in the file at which the target string appears, the encoding
actually present in the PDF file might be different from that computed
for the string in isolation. If the first character of the target
string is the first in an encoded group of 3 characters, direct
encodings like the ones shown above are correct (note that each of our
target strings has a length, 15, that is a multiple of 3); however, if
the first character of the target string is the second or third in an
encoded group, or if the target string has a length that is not a
multiple of 3, the encoding depends on the previous or subsequent
characters in the file. We use knowledge of the actual structure of
the ballot under attack to compile suitable encoded strings into our
attack code; however, a more robust implementation that does not
depend on known character offsets in the Base64 encoding is easily
achievable.

Our attack code monitors connections on standard email submission
ports, replacing the first encoded string with the second when
encountered. The replacement is performed by modifying a single packet
(per ballot) at the TCP protocol level, and cannot be detected in real
time unless the connection is actively monitored at both ends to
ensure that packet contents are transferred unmodified.

For this demonstration, we chose to attack an off-the-shelf home
wireless router purchased at a local big-box electronics
retailer. Nearly all such routers on the market today are based on
embedded versions of the Linux operating system and therefore, in
accordance with the GNU General Public License, the source code for
their firmware is freely available. We downloaded a complete
distribution of our router's firmware source code and made a small
modification to the kernel code that handles transmission of packets
on network devices. The modification has fewer than 50 lines of code,
of which about 1/3 comprise a simple string search and replace
algorithm. We then built new firmware that, to any observer who does
not perform detailed analysis of its traffic handling or a detailed
inspection of the compiled code, is completely indistinguishable from
the manufacturer's official firmware. It behaves identically to the
manufacturer's firmware except in two respects: first, TCP connections
to ports 25 and 587 (the standard email submission ports) are slower
than on the manufacturer's firmware; second, a particular sequence of
bytes sent to these ports is replaced with a different sequence of
bytes.

To ensure that our attack had minimal impact on router performance, we
ran several benchmarks to gauge its effect. On TCP connections to
ports 25 and 587, we observed a slowdown of approximately 20\%. This
would definitely be a noticeable performance hit, but all
commonly-used email clients send mail asynchronously in the background
and end users typically do not monitor the speed of their outgoing
email. Thus, we believe that it would either not be noticed in
practice, or would be attributed to general network or server
performance issues. On all other TCP connections, and with UDP-based
protocols, there was effectively no impact on performance; in a few
cases performance \emph{increased} slightly with our modified
firmware, probably due to compiler optimizations or other effects of
building the firmware in a different environment than the
manufacturer. Thus, for the applications where the end user actually
cares about Internet performance---web browsing, streaming video, file
downloads---there is effectively no change in behavior.

In order to actually install the modified firmware on an end user's
wireless router, we can take advantage of one of many security
vulnerabilities. One common vulnerability is that most home routers
ship with Universal Plug and Play activated; there are well-documented
ways to take advantage of this to run arbitrary code on several
Linux-based routers.~\cite{UPnP} Some home routers are also vulnerable
to ``magic packets'' sent to particular ports that enable
administrator-level backdoors, while others have remote administration
enabled by default with typically weak default passwords.

Some wireless routers, including the one we purchased, have a built-in
firmware update mechanism that notifies users when new firmware is
available and downloads it for them. On the particular router we
purchased, the firmware update mechanism has effectively no security
protections; it simply connects to an FTP server at a specified
hostname, downloads a firmware information file, determines whether
the firmware is new, and then downloads the firmware from the same FTP
server. While the router does compute an MD5 checksum of the
downloaded firmware and compare it to the checksum in the firmware
information file, the fact that both the checksum and the firmware
come from the same source means that DNS-based spoofing of the
firmware upgrade server can easily fool the router into installing our
modified firmware. The same is true for firmware update mechanisms
from other manufacturers; for example, it was recently publicized that
ASUS RT Series routers are vulnerable to man-in-the-middle attacks
against their software update mechanism for much the same
reason.~\cite{AsusMITM}

Another, more brute-force, technique is to simply ``war drive''
through neighborhoods where a campaign wants to change the vote,
taking advantage of the default weak passwords on wireless networks
and wireless router administrative interfaces that the vast majority
of end users never change. Finally, social engineering attacks, either
via DNS spoofing or via authentic-looking email notifications, are
another promising avenue for placing modified firmware on users' home
routers.

Once the modified firmware is installed, regardless of the method
used, it is straightforward for it to (1) ensure that it remains
installed until after the user has voted (in the absence of a hardware
reset or other drastic action by the user), by hiding firmware update
notifications and ignoring explicit update requests, and (2) eliminate
evidence of tampering after the election, or after the user has voted,
by replacing itself with ``good'' firmware downloaded directly from
the manufacturer's FTP or web site. We did not implement this
functionality, but a real attacker would certainly do so.

\section{Possible Mitigation Strategies}

There are several ways to mitigate the effect of this attack, but all
are impractical because of the difficulty in patching the large
existing population of home routers, the fact that they would require
unreasonable amounts of technical knowledge from voters, or inherent
properties of the current electronic mail and Internet infrastructure.

\paragraph{Encrypted PDF}

Perhaps the most obvious way to prevent the type of attack we have
described here is to encrypt the PDF file before emailing it. The PDF
standard provides a built-in password protection system that, when
properly used, gives a good level of cryptographic security. However,
it is unlikely that voters would use the mechanism properly, and some
PDF readers are incapable of using it at all. There are also clear
issues with password distribution, especially if we already have
control of users' home Internet routers; it is essentially as easy to
include a man-in-the-middle attack against the election authority in
the router firmware as it is to transparently change PDF files in
transit to email servers. For this same reason, requiring that the PDF
files be submitted to a secure website also does not prevent vote
tampering.

\paragraph{Encrypted SMTP}

An encrypted connection to the SMTP server would prevent our attack on
end user home routers from changing the PDF file in transit (unless
our firmware also performs a man-in-the-middle attack against the SMTP
server). However, this mitigation would do nothing to prevent the same
attack from being carried out as the email passes through a backbone
router somewhere between the end user's SMTP server and the election
authority's SMTP server, as the vast majority of server-to-server SMTP
connections are unencrypted. The discussion of exploits that can be
carried out against backbone routers is beyond our scope here, but
vulnerabilities that would allow us to carry out a similar attack to
the one described here clearly exist; for example, many routers ship
with GNU Bash installed and are therefore vulnerable to the
recently-disclosed Bash environment variable command injection flaw.

\paragraph{More Secure Firmware Update Mechanism}

One measure that could help, by eliminating one avenue of injection
for the corrupted firmware, is to tighten the security of routers'
firmware update mechanisms. This could be done in the standard way,
with security certificates and SSL connections. However, this would
leave open a number of other avenues of firmware installation,
including security vulnerabilities such as those we have discussed
above, and would have no impact on most of the existing base of
installed routers.

\section{Conclusion}

Attacks on home networking devices have already been carried out at
large scale. For example, in March 2012 over 4.5 million DSL modems in
Brazil were found to have been compromised through a combination of
malicious DNS servers and automated exploit
scripts.~\cite{BrazilExploit} The goal of that particular compromise
was to obtain customers' banking credentials for financial gain, and
it was discovered in part because the illegitimate use of those
credentials alerted customers to the security breach. Our attack, by
contrast, would raise no immediate ``red flags'', as votes would be
changed transparently in transit; moreover, because our modified
firmware could erase evidence of its own existence after doing its
work, it would be exceedingly difficult to find conclusive evidence of
the vote tampering after the fact.

The overall conclusion is inescapable: unencrypted PDF ballots sent
via electronic mail can be altered transparently, potentially with no
obvious sign of alteration, and certainly with no way to determine
where on the network any alterations took place or the extent to which
votes have been corrupted.

\bibliographystyle{acm}
\bibliography{overview}


% \pagebreak
% \appendix
% \section{Source Code}
% \label{section:source}

% \definecolor{keywordcolor}{rgb}{0.5,0,0.33}
% \definecolor{identifiercolor}{rgb}{0,0,0.75}
% \definecolor{commentcolor}{rgb}{0.3,0.3,0.3} 

% \lstset{language={[ANSI]C},
%   basicstyle={\scriptsize\ttfamily\mdseries},
%   keywordstyle={\color{keywordcolor}},
%   identifierstyle={\color{identifiercolor}},
%   commentstyle={\color{commentcolor}}
% }

% The following is the source code for our attack on Linux-based home
% routers. It patches a critical piece of the Linux network stack that
% handles the transmission of packets on hardware devices. While we
% patched a router with firmware based on Linux 2.6.31, the same patch
% would work equally well on any other recent Linux-based firmware. 

% Our code could, with minimal effort, be implemented as a netfilter
% module instead. It could also be extended to Base64 decode and
% re-encode packets on the fly as necessary, rather than depending on a
% priori knowledge of the offset (modulo 3) of the string to replace
% in the PDF ballot.

% \lstinputlisting{votehack-for-paper.c}

\end{document}
